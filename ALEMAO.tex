\vspace{2.5cm}

\section{I}

\noindent{}Als Gregor Samsa eines Morgens aus unruhigen Träumen erwachte, fand er
sich in seinem Bett zu einem ungeheueren Ungeziefer verwandelt. Er lag
auf seinem panzerartig harten Rücken und sah, wenn er den Kopf ein wenig
hob, seinen gewölbten, braunen, von bogenförmigen Versteifungen
geteilten Bauch, auf dessen Höhe sich die Bettdecke, zum gänzlichen
Niedergleiten bereit, kaum noch erhalten konnte. Seine vielen, im
Vergleich zu seinem sonstigen Umfang kläglich dünnen Beine flimmerten
ihm hilflos vor den Augen.

»Was ist mit mir geschehen?« dachte er. Es war kein Traum. Sein Zimmer,
ein richtiges, nur etwas zu kleines Menschenzimmer, lag ruhig zwischen
den vier wohlbekannten Wänden. Über dem Tisch, auf dem eine
auseinandergepackte Musterkollektion von Tuchwaren ausgebreitet war ---
Samsa war Reisender ---, hing das Bild, das er vor kurzem aus einer
illustrierten Zeitschrift ausgeschnitten und in einem hübschen,
vergoldeten Rahmen untergebracht hatte. Es stellte eine Dame dar, die,
mit einem Pelzhut und einer Pelzboa versehen, aufrecht dasaß und einen
schweren Pelzmuff, in dem ihr ganzer Unterarm verschwunden war, dem
Beschauer entgegenhob.

Gregors Blick richtete sich dann zum Fenster, und das trübe Wetter ---
man hörte Regentropfen auf das Fensterblech aufschlagen --- machte ihn
ganz melancholisch. »Wie wäre es, wenn ich noch ein wenig
weiterschliefe und alle Narrheiten vergäße,« dachte er, aber das war
gänzlich undurchführbar, denn er war gewöhnt, auf der rechten Seite zu
schlafen, konnte sich aber in seinem gegenwärtigen Zustand nicht in
diese Lage bringen. Mit welcher Kraft er sich auch auf die rechte Seite
warf, immer wieder schaukelte er in die Rückenlage zurück. Er versuchte
es wohl hundertmal, schloß die Augen, um die zappelnden Beine nicht
sehen zu müssen, und ließ erst ab, als er in der Seite einen noch nie
gefühlten, leichten, dumpfen Schmerz zu fühlen begann.

»Ach Gott,« dachte er, »was für einen anstrengenden Beruf habe ich
gewählt! Tag aus, Tag ein auf der Reise. Die geschäftlichen Aufregungen
sind viel größer, als im eigentlichen Geschäft zu Hause, und außerdem
ist mir noch diese Plage des Reisens auferlegt, die Sorgen um die
Zuganschlüsse, das unregelmäßige, schlechte Essen, ein immer
wechselnder, nie andauernder, nie herzlich werdender menschlicher
Verkehr. Der Teufel soll das alles holen!« Er fühlte ein leichtes Jucken
oben auf dem Bauch; schob sich auf dem Rücken langsam näher zum
Bettpfosten, um den Kopf besser heben zu können; fand die juckende
Stelle, die mit lauter kleinen weißen Pünktchen besetzt war, die er
nicht zu beurteilen verstand; und wollte mit einem Bein die Stelle
betasten, zog es aber gleich zurück, denn bei der Berührung umwehten ihn
Kälteschauer.

Er glitt wieder in seine frühere Lage zurück. »Dies frühzeitige
Aufstehen«, dachte er, »macht einen ganz blödsinnig. Der Mensch muß
seinen Schlaf haben. Andere Reisende leben wie Haremsfrauen. Wenn ich
zum Beispiel im Laufe des Vormittags ins Gasthaus zurückgehe, um die
erlangten Aufträge zu überschreiben, sitzen diese Herren erst beim
Frühstück. Das sollte ich bei meinem Chef versuchen; ich würde auf der
Stelle hinausfliegen. Wer weiß übrigens, ob das nicht sehr gut für mich
wäre. Wenn ich mich nicht wegen meiner Eltern zurückhielte, ich hätte
längst gekündigt, ich wäre vor den Chef hingetreten und hätte ihm meine
Meinung von Grund des Herzens aus gesagt. Vom Pult hätte er fallen
müssen! Es ist auch eine sonderbare Art, sich auf das Pult zu setzen und
von der Höhe herab mit dem Angestellten zu reden, der überdies wegen der
Schwerhörigkeit des Chefs ganz nahe herantreten muß. Nun, die Hoffnung
ist noch nicht gänzlich aufgegeben, habe ich einmal das Geld beisammen,
um die Schuld der Eltern an ihn abzuzahlen --- es dürfte noch fünf bis
sechs Jahre dauern ---, mache ich die Sache unbedingt. Dann wird der
große Schnitt gemacht. Vorläufig allerdings muß ich aufstehen, denn mein
Zug fährt um fünf.«

Und er sah zur Weckuhr hinüber, die auf dem Kasten tickte. »Himmlischer
Vater!« dachte er, Es war halb sieben Uhr, und die Zeiger gingen ruhig
vorwärts, es war sogar halb vorüber, es näherte sich schon dreiviertel.
Sollte der Wecker nicht geläutet haben? Man sah vom Bett aus, daß er auf
vier Uhr richtig eingestellt war; gewiß hatte er auch geläutet. Ja, aber
war es möglich, dieses möbelerschütternde Läuten ruhig zu verschlafen?
Nun, ruhig hatte er ja nicht geschlafen, aber wahrscheinlich desto
fester. Was aber sollte er jetzt tun? Der nächste Zug ging um sieben
Uhr; um den einzuholen, hätte er sich unsinnig beeilen müssen, und die
Kollektion war noch nicht eingepackt, und er selbst fühlte sich durchaus
nicht besonders frisch und beweglich. Und selbst wenn er den Zug
einholte, ein Donnerwetter des Chefs war nicht zu vermeiden, denn der
Geschäftsdiener hatte beim Fünfuhrzug gewartet und die Meldung von
seiner Versäumnis längst erstattet. Es war eine Kreatur des Chefs, ohne
Rückgrat und Verstand. Wie nun, wenn er sich krank meldete? Das wäre
aber äußerst peinlich und verdächtig, denn Gregor war während seines
fünfjährigen Dienstes noch nicht einmal krank gewesen. Gewiß würde der
Chef mit dem Krankenkassenarzt kommen, würde den Eltern wegen des faulen
Sohnes Vorwürfe machen und alle Einwände durch den Hinweis auf den
Krankenkassenarzt abschneiden, für den es ja überhaupt nur ganz gesunde,
aber arbeitsscheue Menschen gibt. Und hätte er übrigens in diesem Falle
so ganz unrecht? Gregor fühlte sich tatsächlich, abgesehen von einer\est\
nach dem langen Schlaf wirklich überflüssigen Schläfrigkeit, ganz wohl
und hatte sogar einen besonders kräftigen Hunger.

Als er dies alles in größter Eile überlegte, ohne sich entschließen zu
können, das Bett zu verlassen --- gerade schlug der Wecker dreiviertel
sieben --- klopfte es vorsichtig an die Tür am Kopfende seines Bettes.
»Gregor,« rief es --- es war die Mutter ---, »es ist dreiviertel sieben.
Wolltest du nicht wegfahren?« Die sanfte Stimme! Gregor erschrak, als er
seine antwortende Stimme hörte, die wohl unverkennbar seine frühere war,
in die sich aber, wie von unten her, ein nicht zu unterdrückendes,
schmerzliches Piepsen mischte, das die Worte förmlich nur im ersten
Augenblick in ihrer Deutlichkeit beließ, um sie im Nachklang derart zu
zerstören, daß man nicht wußte, ob man recht gehört hatte. Gregor hatte
ausführlich antworten und alles erklären wollen, beschränkte sich aber
bei diesen Umständen darauf, zu sagen: »Ja, ja, danke, Mutter, ich stehe
schon auf.« Infolge der Holztür war die Veränderung in Gregors Stimme
draußen wohl nicht zu merken, denn die Mutter beruhigte sich mit dieser
Erklärung und schlürfte davon. Aber durch das kleine Gespräch waren die
anderen Familienmitglieder darauf aufmerksam geworden, daß Gregor wider
Erwarten noch zu Hause war, und schon klopfte an der einen Seitentür der
Vater, schwach, aber mit der Faust. »Gregor, Gregor,« rief er, »was ist
denn?« Und nach einer kleinen Weile mahnte er nochmals mit tieferer
Stimme: »Gregor! Gregor!« An der anderen Seitentür aber klagte\est\ leise die
Schwester: »Gregor? Ist dir nicht wohl? Brauchst du etwas?« Nach beiden
Seiten hin antwortete Gregor: »Bin schon fertig,« und bemühte sich,
durch die sorgfältigste Aussprache und durch Einschaltung von langen
Pausen zwischen den einzelnen Worten seiner Stimme alles Auffallende zu
nehmen. Der Vater kehrte auch zu seinem Frühstück zurück, die Schwester
aber flüsterte: »Gregor, mach auf, ich beschwöre dich.« Gregor aber
dachte gar nicht daran aufzumachen, sondern lobte die vom Reisen her
übernommene Vorsicht, auch zu Hause alle Türen während der Nacht zu
versperren.

Zunächst wollte er ruhig und ungestört aufstehen, sich anziehen und vor
allem frühstücken, und dann erst das Weitere überlegen, denn, das merkte
er wohl, im Bett würde er mit dem Nachdenken zu keinem vernünftigen Ende
kommen. Er erinnerte sich, schon öfters im Bett irgendeinen vielleicht
durch ungeschicktes Liegen erzeugten, leichten Schmerz empfunden zu
haben, der sich dann beim Aufstehen als reine Einbildung herausstellte,
und er war gespannt, wie sich seine heutigen Vorstellungen allmählich
auflösen würden. Daß die Veränderung der Stimme nichts anderes war als
der Vorbote einer tüchtigen Verkühlung, einer Berufskrankheit der
Reisenden, daran zweifelte er nicht im geringsten.

Die Decke abzuwerfen war ganz einfach; er brauchte sich nur ein wenig
aufzublasen und sie fiel von selbst. Aber weiterhin wurde es schwierig,
besonders weil er so ungemein breit war. Er hätte Arme und Hände\est\
gebraucht, um sich aufzurichten; statt dessen aber hatte er nur die
vielen Beinchen, die ununterbrochen in der verschiedensten Bewegung
waren und die er überdies nicht beherrschen konnte. Wollte er eines
einmal einknicken, so war es das erste, daß er sich streckte; und gelang
es ihm endlich, mit diesem Bein das auszuführen, was er wollte, so
arbeiteten inzwischen alle anderen, wie freigelassen, in höchster,
schmerzlicher Aufregung. »Nur sich nicht im Bett unnütz aufhalten,«
sagte sich Gregor.

Zuerst wollte er mit dem unteren Teil seines Körpers aus dem Bett
hinauskommen, aber dieser untere Teil, den er übrigens noch nicht
gesehen hatte und von dem er sich auch keine rechte Vorstellung machen
konnte, erwies sich als zu schwer beweglich; es ging so langsam; und als
er schließlich, fast wild geworden, mit gesammelter Kraft, ohne
Rücksicht sich vorwärtsstieß, hatte er die Richtung falsch gewählt,
schlug an den unteren Bettpfosten heftig an, und der brennende Schmerz,
den er empfand, belehrte ihn, daß gerade der untere Teil seines Körpers
augenblicklich vielleicht der empfindlichste war.

Er versuchte es daher, zuerst den Oberkörper aus dem Bett zu bekommen,
und drehte vorsichtig den Kopf dem Bettrand zu. Dies gelang auch leicht,
und trotz ihrer Breite und Schwere folgte schließlich die Körpermasse
langsam der Wendung des Kopfes. Aber als er den Kopf endlich außerhalb
des Bettes in der freien Luft hielt, bekam er Angst, weiter auf diese
Weise vorzurücken, denn wenn er sich schließlich so fallen ließ, mußte
geradezu ein Wunder geschehen wenn der Kopf nicht verletzt werden
sollte. Und die Besinnung durfte er gerade jetzt um keinen Preis
verlieren; lieber wollte er im Bett bleiben.

Aber als er wieder nach gleicher Mühe aufseufzend so dalag wie früher,
und wieder seine Beinchen womöglich noch ärger gegeneinander kämpfen sah
und keine Möglichkeit fand, in diese Willkür Ruhe und Ordnung zu
bringen, sagte er sich wieder, daß er unmöglich im Bett bleiben könne
und daß es das Vernünftigste sei, alles zu opfern, wenn auch nur die
kleinste Hoffnung bestünde, sich dadurch vom Bett zu befreien.
Gleichzeitig aber vergaß er nicht, sich zwischendurch daran zu erinnern,
daß viel besser als verzweifelte Entschlüsse ruhige und ruhigste
Überlegung sei. In solchen Augenblicken richtete er die Augen möglichst
scharf auf das Fenster, aber leider war aus dem Anblick des
Morgennebels, der sogar die andere Seite der engen Straße verhüllte,
wenig Zuversicht und Munterkeit zu holen. »Schon sieben Uhr,« sagte er
sich beim neuerlichen Schlagen des Weckers, »schon sieben Uhr und noch
immer ein solcher Nebel.« Und ein Weilchen lang lag er ruhig mit
schwachem Atem, als erwarte er vielleicht von der völligen Stille die
Wiederkehr der wirklichen und selbstverständlichen Verhältnisse.

Dann aber sagte er sich: »Ehe es einviertel acht schlägt, muß ich
unbedingt das Bett vollständig verlassen haben. Im übrigen wird auch bis
dahin jemand aus dem Geschäft kommen, um nach mir zu fragen, denn das
Geschäft wird vor sieben Uhr geöffnet.« Und er machte sich nun daran,
den Körper in seiner ganzen Länge vollständig gleichmäßig aus dem Bett
hinauszuschaukeln. Wenn er sich auf diese Weise aus dem Bett fallen
ließ, blieb der Kopf, den er beim Fall scharf heben wollte,
voraussichtlich unverletzt. Der Rücken schien hart zu sein; dem würde
wohl bei dem Fall auf den Teppich nichts geschehen. Das größte Bedenken
machte ihm die Rücksicht auf den lauten Krach, den es geben müßte und
der wahrscheinlich hinter allen Türen wenn nicht Schrecken, so doch
Besorgnisse erregen würde. Das mußte aber gewagt werden.

Als Gregor schon zur Hälfte aus dem Bette ragte --- die neue Methode war
mehr ein Spiel als eine Anstrengung, er brauchte immer nur ruckweise zu
schaukeln ---, fiel ihm ein, wie einfach alles wäre, wenn man ihm zu
Hilfe käme. Zwei starke Leute --- er dachte an seinen Vater und das
Dienstmädchen --- hätten vollständig genügt; sie hätten ihre Arme nur
unter seinen gewölbten Rücken schieben, ihn so aus dem Bett schälen,
sich mit der Last niederbeugen und dann bloß vorsichtig dulden müssen,
daß er den Überschwung auf dem Fußboden vollzog, wo dann die Beinchen
hoffentlich einen Sinn bekommen würden. Nun, ganz abgesehen davon, daß
die Türen versperrt waren, hätte er wirklich um Hilfe rufen sollen?
Trotz aller Not konnte er bei diesem Gedanken ein Lächeln nicht
unterdrücken.

Schon war er so weit, daß er bei stärkerem Schaukeln kaum das
Gleichgewicht noch erhielt, und sehr bald mußte er sich nun endgültig
entscheiden, denn es war in fünf Minuten einviertel acht, --- als es an
der Wohnungstür läutete. »Das ist jemand aus dem Geschäft,« sagte er
sich und erstarrte fast, während seine Beinchen nur desto eiliger
tanzten. Einen Augenblick blieb alles still. »Sie öffnen nicht,« sagte
sich Gregor, befangen in irgendeiner unsinnigen Hoffnung. Aber dann ging
natürlich wie immer das Dienstmädchen festen Schrittes zur Tür und
öffnete. Gregor brauchte nur das erste Grußwort des Besuchers zu hören
und wußte schon, wer es war --- der Prokurist selbst. Warum war nur
Gregor dazu verurteilt, bei einer Firma zu dienen, wo man bei der
kleinsten Versäumnis gleich den größten Verdacht faßte? Waren denn alle
Angestellten samt und sonders Lumpen, gab es denn unter ihnen keinen
treuen ergebenen Menschen, den, wenn er auch nur ein paar Morgenstunden
für das Geschäft nicht ausgenützt hatte, vor Gewissensbissen närrisch
wurde und geradezu nicht imstande war, das Bett zu verlassen? Genügte es
wirklich nicht, einen Lehrjungen nachfragen zu lassen --- wenn überhaupt
diese Fragerei nötig war ---, mußte da der Prokurist selbst kommen, und
mußte dadurch der ganzen unschuldigen Familie gezeigt werden, daß die
Untersuchung dieser verdächtigen Angelegenheit nur dem Verstand des
Prokuristen anvertraut werden konnte? Und mehr infolge der Erregung, in
welche Gregor durch diese Überlegungen versetzt wurde, als infolge eines
richtigen Entschlusses, schwang er sich mit aller Macht aus dem Bett. Es
gab einen lauten Schlag, aber ein eigentlicher Krach war es nicht. Ein
wenig wurde der Fall durch den Teppich abgeschwächt, auch war der Rücken
elastischer, als Gregor gedacht hatte, daher kam der nicht gar so
auffallende dumpfe Klang. Nur den Kopf hatte er nicht vorsichtig genug
gehalten und ihn angeschlagen; er drehte ihn und rieb ihn an dem Teppich
vor Ärger und Schmerz.

»Da drin ist etwas gefallen,« sagte der Prokurist im Nebenzimmer links.
Gregor suchte sich vorzustellen, ob nicht auch einmal dem Prokuristen
etwas Ähnliches passieren könnte, wie heute ihm; die Möglichkeit dessen
mußte man doch eigentlich zugeben. Aber wie zur rohen Antwort auf diese
Frage machte jetzt der Prokurist im Nebenzimmer ein paar bestimmte
Schritte und ließ seine Lackstiefel knarren. Aus dem Nebenzimmer rechts
flüsterte die Schwester, um Gregor zu verständigen: »Gregor, der
Prokurist ist da.« »Ich weiß,« sagte Gregor vor sich hin; aber so laut,
daß es die Schwester hätte hören können, wagte er die Stimme nicht zu
erheben.

»Gregor,« sagte nun der Vater aus dem Nebenzimmer links, »der Herr
Prokurist ist gekommen und erkundigt sich, warum du nicht mit dem
Frühzug weggefahren bist. Wir wissen nicht, was wir ihm sagen sollen.
Übrigens will er auch mit dir persönlich sprechen. Also bitte mach die
Tür auf. Er wird die Unordnung im Zimmer zu entschuldigen schon die Güte
haben.« »Guten Morgen, Herr Samsa,« rief der Prokurist freundlich
dazwischen. »Ihm ist nicht wohl,« sagte die Mutter zum Prokuristen,
während der Vater noch an der Tür redete, »ihm ist nicht wohl, glauben
Sie mir, Herr Prokurist. Wie würde denn Gregor sonst einen Zug
versäumen! Der Junge hat ja nichts im Kopf als das Geschäft. Ich ärgere
mich schon fast, daß er abends niemals ausgeht; jetzt war er doch acht
Tage in der Stadt, aber jeden Abend war er zu Hause. Da sitzt er bei uns
am Tisch und liest still die Zeitung oder studiert Fahrpläne. Es ist
schon eine Zerstreuung für ihn, wenn er sich mit Laubsägearbeiten
beschäftigt. Da hat er zum Beispiel im Laufe von zwei, drei Abenden
einen kleinen Rahmen geschnitzt; Sie werden staunen, wie hübsch er ist;
er hängt drin im Zimmer; Sie werden ihn gleich sehen, wenn Gregor
aufmacht. Ich bin übrigens glücklich, daß Sie da sind, Herr Prokurist;
wir allein hätten Gregor nicht dazu gebracht, die Tür zu öffnen; er ist
so hartnäckig; und bestimmt ist ihm nicht wohl, trotzdem er es am Morgen
geleugnet hat.« »Ich komme gleich,« sagte Gregor langsam und bedächtig
und rührte sich nicht, um kein Wort der Gespräche zu verlieren. »Anders,
gnädige Frau, kann ich es mir auch nicht erklären,« sagte der Prokurist,
»hoffentlich ist es nichts Ernstes. Wenn ich auch andererseits sagen
muß, daß wir Geschäftsleute --- wie man will, leider oder
glücklicherweise --- ein leichtes Unwohlsein sehr oft aus geschäftlichen
Rücksichten einfach überwinden müssen.« »Also kann der Herr Prokurist
schon zu dir hinein?« fragte der ungeduldige Vater und klopfte wiederum
an die Tür. »Nein,« sagte Gregor. Im Nebenzimmer links trat eine
peinliche Stille ein, im Nebenzimmer rechts begann die Schwester zu
schluchzen.

Warum ging denn die Schwester nicht zu den anderen? Sie war wohl erst
jetzt aus dem Bett aufgestanden und hatte noch gar nicht angefangen sich
anzuziehen. Und warum weinte sie denn? Weil er nicht aufstand und den
Prokuristen nicht hereinließ, weil er in Gefahr war, den Posten zu
verlieren und weil dann der Chef die Eltern mit den alten Forderungen
wieder verfolgen würde? Das waren doch vorläufig wohl unnötige Sorgen.
Noch war Gregor hier und dachte nicht im geringsten daran, seine Familie
zu verlassen. Augenblicklich lag er wohl da auf dem Teppich, und
niemand, der seinen Zustand gekannt hätte, hätte im Ernst von ihm
verlangt, daß er den Prokuristen hereinlasse. Aber wegen dieser kleinen
Unhöflichkeit, für die sich ja später leicht eine passende Ausrede
finden würde, konnte Gregor doch nicht gut sofort weggeschickt werden.
Und Gregor schien es, daß es viel vernünftiger wäre, ihn jetzt in Ruhe
zu lassen, statt ihn mit Weinen und Zureden zu stören. Aber es war eben
die Ungewißheit, welche die anderen bedrängte und ihr Benehmen
entschuldigte.

»Herr Samsa,« rief nun der Prokurist mit erhobener Stimme, »was ist denn
los? Sie verbarrikadieren sich da in Ihrem Zimmer, antworten bloß mit ja
und nein, machen Ihren Eltern schwere, unnötige Sorgen und versäumen ---
dies nur nebenbei erwähnt --- Ihre geschäftlichen Pflichten in einer
eigentlich unerhörten Weise. Ich spreche hier im Namen Ihrer Eltern und
Ihres Chefs und bitte Sie ganz ernsthaft um eine augenblickliche,
deutliche Erklärung. Ich staune, ich staune. Ich glaubte Sie als einen
ruhigen, vernünftigen Menschen zu kennen, und nun scheinen Sie plötzlich
anfangen zu wollen, mit sonderbaren Launen zu paradieren. Der Chef
deutete mir zwar heute früh eine mögliche Erklärung für Ihre Versäumnis
an --- sie betraf das Ihnen seit kurzem anvertraute Inkasso ---, aber ich
legte wahrhaftig fast mein Ehrenwort dafür ein, daß diese Erklärung
nicht zutreffen könne. Nun aber sehe ich hier Ihren unbegreiflichen
Starrsinn und verliere ganz und gar jede Lust, mich auch nur im
geringsten für Sie einzusetzen. Und Ihre Stellung ist durchaus nicht die
festeste. Ich hatte ursprünglich die Absicht, Ihnen das alles unter vier
Augen zu sagen, aber da Sie mich hier nutzlos meine Zeit versäumen
lassen, weiß ich nicht, warum es nicht auch Ihre Herren Eltern erfahren
sollen. Ihre Leistungen in der letzten Zeit waren also sehr
unbefriedigend; es ist zwar nicht die Jahreszeit, um besondere Geschäfte
zu machen, das erkennen wir an; aber eine Jahreszeit, um keine Geschäfte
zu machen, gibt es überhaupt nicht, Herr Samsa, darf es nicht geben.«

»Aber Herr Prokurist,« rief Gregor außer sich und vergaß in der
Aufregung alles andere, »ich mache ja sofort, augenblicklich auf. Ein
leichtes Unwohlsein, ein Schwindelanfall, haben mich verhindert
aufzustehen. Ich liege noch jetzt im Bett. Jetzt bin ich aber schon
wieder ganz frisch. Eben steige ich aus dem Bett. Nur einen kleinen
Augenblick Geduld! Es geht noch nicht so gut, wie ich dachte. Es ist mir
aber schon wohl. Wie das nur einen Menschen so überfallen kann! Noch
gestern abend war mir ganz gut, meine Eltern wissen es ja, oder besser,
schon gestern abend hatte ich eine kleine Vorahnung. Man hätte es mir
ansehen müssen. Warum habe ich es nur im Geschäfte nicht gemeldet! Aber
man denkt eben immer, daß man die Krankheit ohne Zuhausebleiben
überstehen wird. Herr Prokurist! Schonen Sie meine Eltern! Für alle die
Vorwürfe, die Sie mir jetzt machen, ist ja kein Grund; man hat mir ja
davon auch kein Wort gesagt. Sie haben vielleicht die letzten Aufträge,
die ich geschickt habe, nicht gelesen. Übrigens, noch mit dem Achtuhrzug
fahre ich auf die Reise, die paar Stunden Ruhe haben mich gekräftigt.
Halten Sie sich nur nicht auf, Herr Prokurist; ich bin gleich selbst im
Geschäft, und haben Sie die Güte, das zu sagen und mich dem Herrn Chef
zu empfehlen!«

Und während Gregor dies alles hastig ausstieß und kaum wußte, was er
sprach, hatte er sich leicht, wohl infolge der im Bett bereits erlangten
Übung, dem Kasten genähert und versuchte nun, an ihm sich aufzurichten.
Er wollte tatsächlich die Tür aufmachen, tatsächlich sich sehen lassen
und mit dem Prokuristen sprechen; er war begierig zu erfahren, was die
anderen, die jetzt so nach ihm verlangten, bei seinem Anblick sagen
würden. Würden sie erschrecken, dann hatte Gregor keine Verantwortung
mehr und konnte ruhig sein. Würden sie aber alles ruhig hinnehmen, dann
hatte auch er keinen Grund sich aufzuregen, und konnte, wenn er sich
beeilte, um acht Uhr tatsächlich auf dem Bahnhof sein. Zuerst glitt er
nun einigemale von dem glatten Kasten ab, aber endlich gab er sich
einen letzten Schwung und stand aufrecht da; auf die Schmerzen im
Unterleib achtete er gar nicht mehr, so sehr sie auch brannten. Nun ließ
er sich gegen die Rücklehne eines nahen Stuhles fallen, an deren Rändern
er sich mit seinen Beinchen festhielt. Damit hatte er aber auch die
Herrschaft über sich erlangt und verstummte, denn nun konnte er den
Prokuristen anhören.

»Haben Sie auch nur ein Wort verstanden?« fragte der Prokurist die
Eltern, »er macht sich doch wohl nicht einen Narren aus uns?« »Um Gottes
willen,« rief die Mutter schon unter Weinen, »er ist vielleicht schwer
krank, und wir quälen ihn. Grete! Grete!« schrie sie dann. »Mutter?«
rief die Schwester von der anderen Seite. Sie verständigten sich durch
Gregors Zimmer. »Du mußt augenblicklich zum Arzt. Gregor ist krank.
Rasch um den Arzt. Hast du Gregor jetzt reden hören?« »Das war eine
Tierstimme,« sagte der Prokurist, auffallend leise gegenüber dem
Schreien der Mutter. »Anna! Anna!« rief der Vater durch das Vorzimmer in
die Küche und klatschte in die Hände, »sofort einen Schlosser holen!«
Und schon liefen die zwei Mädchen mit rauschenden Röcken durch das
Vorzimmer --- wie hatte sich die Schwester denn so schnell angezogen? ---
und rissen die Wohnungstüre auf. Man hörte gar nicht die Türe
zuschlagen; sie hatten sie wohl offen gelassen, wie es in Wohnungen zu
sein pflegt, in denen ein großes Unglück geschehen ist.

Gregor war aber viel ruhiger geworden. Man verstand zwar also seine
Worte nicht mehr, trotzdem sie ihm genug klar, klarer als früher,
vorgekommen waren, vielleicht infolge der Gewöhnung des Ohres. Aber
immerhin glaubte man nun schon daran, daß es mit ihm nicht ganz in
Ordnung war, und war bereit, ihm zu helfen. Die Zuversicht und
Sicherheit, womit die ersten Anordnungen getroffen worden waren, taten
ihm wohl. Er fühlte sich wieder einbezogen in den menschlichen Kreis und
erhoffte von beiden, vom Arzt und vom Schlosser, ohne sie eigentlich
genau zu scheiden, großartige und überraschende Leistungen. Um für die
sich nähernden entscheidenden Besprechungen eine möglichst klare Stimme
zu bekommen, hustete er ein wenig ab, allerdings bemüht, dies ganz
gedämpft zu tun, da möglicherweise auch schon dieses Geräusch anders als
menschlicher Husten klang, was er selbst zu entscheiden sich nicht mehr
getraute. Im Nebenzimmer war es inzwischen ganz still geworden.
Vielleicht saßen die Eltern mit dem Prokuristen beim Tisch und
tuschelten, vielleicht lehnten alle an der Türe und horchten.

Gregor schob sich langsam mit dem Sessel zur Tür hin, ließ ihn dort los,
warf sich gegen die Tür, hielt sich an ihr aufrecht --- die Ballen seiner
Beinchen hatten ein wenig Klebstoff --- und ruhte sich dort einen
Augenblick lang von der Anstrengung aus. Dann aber machte er sich daran,
mit dem Mund den Schlüssel im Schloß umzudrehen. Es schien leider, daß
er keine eigentlichen Zähne hatte, --- womit sollte er gleich den
Schlüssel fassen? --- aber dafür waren die Kiefer freilich sehr stark,
mit ihrer Hilfe brachte er auch wirklich den Schlüssel in Bewegung und
achtete nicht darauf, daß er sich zweifellos irgendeinen Schaden
zufügte, denn eine braune Flüssigkeit kam ihm aus dem Mund, floß über
den Schlüssel und tropfte auf den Boden. »Hören Sie nur,« sagte der
Prokurist im Nebenzimmer, »er dreht den Schlüssel um.« Das war für
Gregor eine große Aufmunterung; aber alle hätten ihm zurufen sollen,
auch der Vater und die Mutter: »Frisch, Gregor,« hätten sie rufen
sollen, »immer nur heran, fest an das Schloß heran!« Und in der
Vorstellung, daß alle seine Bemühungen mit Spannung verfolgten, verbiß
er sich mit allem, was er an Kraft aufbringen konnte, besinnungslos in
den Schlüssel. Je nach dem Fortschreiten der Drehung des Schlüssels
umtanzte er das Schloß, hielt sich jetzt nur noch mit dem Munde
aufrecht, und je nach Bedarf hing er sich an den Schlüssel oder drückte
ihn dann wieder nieder mit der ganzen Last seines Körpers. Der hellere
Klang des endlich zurückschnappenden Schlosses erweckte Gregor förmlich.
Aufatmend sagte er sich: »Ich habe also den Schlosser nicht gebraucht,«
und legte den Kopf auf die Klinke, um die Türe gänzlich zu öffnen.

Da er die Türe auf diese Weise öffnen mußte, war sie eigentlich schon
recht weit geöffnet, und er selbst noch nicht zu sehen. Er mußte sich
erst langsam um den einen Türflügel herumdrehen, und zwar sehr
vorsichtig, wenn er nicht gerade vor dem Eintritt ins Zimmer plump auf
den Rücken fallen wollte. Er war noch mit jener schwierigen Bewegung
beschäftigt und hatte nicht Zeit, auf anderes zu achten, da hörte er
schon den Prokuristen ein lautes »Oh!« ausstoßen --- es klang, wie wenn
der Wind saust --- und nun sah er ihn auch, wie er, der der Nächste an
der Türe war, die Hand gegen den offenen Mund drückte und langsam
zurückwich, als vertreibe ihn eine unsichtbare, gleichmäßig fortwirkende
Kraft. Die Mutter --- sie stand hier trotz der Anwesenheit des
Prokuristen mit von der Nacht her noch aufgelösten, hoch sich
sträubenden Haaren --- sah zuerst mit gefalteten Händen den Vater an,
ging dann zwei Schritte zu Gregor hin und fiel inmitten ihrer rings um
sie herum sich ausbreitenden Röcke nieder, das Gesicht ganz unauffindbar
zu ihrer Brust gesenkt. Der Vater ballte mit feindseligem Ausdruck die
Faust, als wolle er Gregor in sein Zimmer zurückstoßen, sah sich dann
unsicher im Wohnzimmer um, beschattete dann mit den Händen die Augen und
weinte, daß sich seine mächtige Brust schüttelte.

Gregor trat nun gar nicht in das Zimmer, sondern lehnte sich von innen
an den festgeriegelten Türflügel, so daß sein Leib nur zur Hälfte und
darüber der seitlich geneigte Kopf zu sehen war, mit dem er zu den
anderen hinüberlugte. Es war inzwischen viel heller geworden; klar stand
auf der anderen Straßenseite ein Ausschnitt des gegenüberliegenden,
endlosen, grauschwarzen Hauses --- es war ein Krankenhaus --- mit seinen
hart die Front durchbrechenden regelmäßigen Fenstern; der Regen fiel
noch nieder, aber nur mit großen, einzeln sichtbaren und förmlich auch
einzelnweise auf die Erde hinuntergeworfenen Tropfen. Das
Frühstücksgeschirr stand in überreicher Zahl auf dem Tisch, denn für den
Vater war das Frühstück die wichtigste Mahlzeit des Tages, die er bei
der Lektüre verschiedener Zeitungen stundenlang hinzog. Gerade an der
gegenüberliegenden Wand hing eine Photographie Gregors aus seiner
Militärzeit, die ihn als Leutnant darstellte, wie er, die Hand am Degen,
sorglos lächelnd, Respekt für seine Haltung und Uniform verlangte. Die
Tür zum Vorzimmer war geöffnet, und man sah, da auch die Wohnungstür
offen war, auf den Vorplatz der Wohnung hinaus und auf den Beginn der
abwärts führenden Treppe.

»Nun,« sagte Gregor und war sich dessen wohl bewußt, daß er der einzige
war, der die Ruhe bewahrt hatte, »ich werde mich gleich anziehen, die
Kollektion zusammenpacken und wegfahren. Wollt ihr, wollt ihr mich
wegfahren lassen? Nun, Herr Prokurist, Sie sehen, ich bin nicht
starrköpfig und ich arbeite gern; das Reisen ist beschwerlich, aber ich
könnte ohne das Reisen nicht leben. Wohin gehen Sie denn, Herr
Prokurist? Ins Geschäft? Ja? Werden Sie alles wahrheitsgetreu berichten?
Man kann im Augenblick unfähig sein zu arbeiten, aber dann ist gerade
der richtige Zeitpunkt, sich an die früheren Leistungen zu erinnern und
zu bedenken, daß man später, nach Beseitigung des Hindernisses, gewiß
desto fleißiger und gesammelter arbeiten wird. Ich bin ja dem Herrn Chef
so sehr verpflichtet, das wissen Sie doch recht gut. Andererseits habe
ich die Sorge um meine Eltern und die Schwester. Ich bin in der Klemme,
ich werde mich aber auch wieder herausarbeiten. Machen Sie es mir aber
nicht schwieriger, als es schon ist. Halten Sie im Geschäft meine
Partei! Man liebt den Reisenden nicht, ich weiß. Man denkt, er verdient
ein Heidengeld und führt dabei ein schönes Leben. Man hat eben keine
besondere Veranlassung, dieses Vorurteil besser zu durchdenken. Sie
aber, Herr Prokurist, Sie haben einen besseren Überblick über die
Verhältnisse, als das sonstige Personal, ja sogar, ganz im Vertrauen
gesagt, einen besseren Überblick, als der Herr Chef selbst, der in
seiner Eigenschaft als Unternehmer sich in seinem Urteil leicht
zuungunsten eines Angestellten beirren läßt. Sie wissen auch sehr wohl,
daß der Reisende, der fast das ganze Jahr außerhalb des Geschäftes ist,
so leicht ein Opfer von Klatschereien, Zufälligkeiten und grundlosen
Beschwerden werden kann, gegen die sich zu wehren ihm ganz unmöglich
ist, da er von ihnen meistens gar nichts erfährt und nur dann, wenn er
erschöpft eine Reise beendet hat, zu Hause die schlimmen, auf ihre
Ursachen hin nicht mehr zu durchschauenden Folgen am eigenen Leibe zu
spüren bekommt. Herr Prokurist, gehen Sie nicht weg, ohne mir ein Wort
gesagt zu haben, das mir zeigt, daß Sie mir wenigstens zu einem kleinen
Teil recht geben!«

Aber der Prokurist hatte sich schon bei den ersten Worten Gregors
abgewendet, und nur über die zuckende Schulter hinweg sah er mit
aufgeworfenen Lippen nach Gregor zurück. Und während Gregors Rede stand
er keinen Augenblick still, sondern verzog sich, ohne Gregor aus den
Augen zu lassen, gegen die Tür, aber ganz allmählich, als bestehe ein
geheimes Verbot, das Zimmer zu verlassen. Schon war er im Vorzimmer, und
nach der plötzlichen Bewegung, mit der er zum letztenmal den Fuß aus dem
Wohnzimmer zog, hätte man glauben können, er habe sich soeben die Sohle
verbrannt. Im Vorzimmer aber streckte er die rechte Hand weit von sich
zur Treppe hin, als warte dort auf ihn eine geradezu überirdische
Erlösung.

Gregor sah ein, daß er den Prokuristen in dieser Stimmung auf keinen
Fall weggehen lassen dürfe, wenn dadurch seine Stellung im Geschäft
nicht aufs äußerste gefährdet werden sollte. Die Eltern verstanden das
alles nicht so gut; sie hatten sich in den langen Jahren die Überzeugung
gebildet, daß Gregor in diesem Geschäft für sein Leben versorgt war, und
hatten außerdem jetzt mit den augenblicklichen Sorgen so viel zu tun,
daß ihnen jede Voraussicht abhanden gekommen war. Aber Gregor hatte
diese Voraussicht. Der Prokurist mußte gehalten, beruhigt, überzeugt und
schließlich gewonnen werden; die Zukunft Gregors und seiner Familie hing
doch davon ab! Wäre doch die Schwester hier gewesen! Sie war klug; sie
hatte schon geweint, als Gregor noch ruhig auf dem Rücken lag. Und gewiß
hätte der Prokurist, dieser Damenfreund, sich von ihr lenken lassen;
sie hätte die Wohnungstür zugemacht und ihm im Vorzimmer den Schrecken
ausgeredet. Aber die Schwester war eben nicht da, Gregor selbst mußte
handeln. Und ohne daran zu denken, daß er seine gegenwärtigen
Fähigkeiten, sich zu bewegen, noch gar nicht kannte, ohne auch daran zu
denken, daß seine Rede möglicher- ja wahrscheinlicherweise wieder nicht
verstanden worden war, verließ er den Türflügel; schob sich durch die
Öffnung; wollte zum Prokuristen hingehen, der sich schon am Geländer des
Vorplatzes lächerlicherweise mit beiden Händen festhielt; fiel aber
sofort, nach einem Halt suchend, mit einem kleinen Schrei auf seine
vielen Beinchen nieder. Kaum war das geschehen, fühlte er zum erstenmal
an diesem Morgen ein körperliches Wohlbehagen; die Beinchen hatten
festen Boden unter sich; sie gehorchten vollkommen, wie er zu seiner
Freude merkte; strebten sogar darnach, ihn fortzutragen, wohin er
wollte; und schon glaubte er, die endgültige Besserung alles Leidens
stehe unmittelbar bevor. Aber im gleichen Augenblick, als er da
schaukelnd vor verhaltener Bewegung, gar nicht weit von seiner Mutter
entfernt, ihr gerade gegenüber auf dem Boden lag, sprang diese, die doch
so ganz in sich versunken schien, mit einemmale in die Höhe, die Arme
weit ausgestreckt, die Finger gespreizt, rief: »Hilfe, um Gottes willen
Hilfe!«, hielt den Kopf geneigt, als wolle sie Gregor besser sehen, lief
aber, im Widerspruch dazu, sinnlos zurück; hatte vergessen, daß hinter
ihr der gedeckte Tisch stand; setzte sich, als sie bei ihm angekommen
war, wie in Zerstreutheit, eilig auf ihn, und schien gar nicht zu
merken, daß neben ihr aus der umgeworfenen großen Kanne der Kaffee in
vollem Strome auf den Teppich sich ergoß.

»Mutter, Mutter,« sagte Gregor leise und sah zu ihr hinauf. Der
Prokurist war ihm für einen Augenblick ganz aus dem Sinn gekommen;
dagegen konnte er sich nicht versagen, im Anblick des fließenden Kaffees
mehrmals mit den Kiefern ins Leere zu schnappen. Darüber schrie die
Mutter neuerdings auf, flüchtete vom Tisch und fiel dem ihr
entgegeneilenden Vater in die Arme. Aber Gregor hatte jetzt keine Zeit
für seine Eltern; der Prokurist war schon auf der Treppe; das Kinn auf
dem Geländer, sah er noch zum letzten Male zurück. Gregor nahm einen
Anlauf, um ihn möglichst sicher einzuholen; der Prokurist mußte etwas
ahnen, denn er machte einen Sprung über mehrere Stufen und verschwand;
»Huh!« aber schrie er noch, es klang durchs ganze Treppenhaus. Leider
schien nun auch diese Flucht des Prokuristen den Vater, der bisher
verhältnismäßig gefaßt gewesen war, völlig zu verwirren, denn statt
selbst dem Prokuristen nachzulaufen oder wenigstens Gregor in der
Verfolgung nicht zu hindern, packte er mit der Rechten den Stock des
Prokuristen, den dieser mit Hut und Überzieher auf einem Sessel
zurückgelassen hatte, holte mit der Linken eine große Zeitung vom Tisch
und machte sich unter Füßestampfen daran, Gregor durch Schwenken des
Stockes und der Zeitung in sein Zimmer zurückzutreiben. Kein Bitten
Gregors half, kein Bitten wurde auch verstanden, er mochte den Kopf noch
so demütig drehen, der Vater stampfte nur stärker mit den Füßen. Drüben
hatte die Mutter trotz des kühlen Wetters ein Fenster aufgerissen, und
hinausgelehnt drückte sie ihr Gesicht weit außerhalb des Fensters in
ihre Hände. Zwischen Gasse und Treppenhaus entstand eine starke Zugluft,
die Fenstervorhänge flogen auf, die Zeitungen auf dem Tische rauschten,
einzelne Blätter wehten über den Boden hin. Unerbittlich drängte der
Vater und stieß Zischlaute aus, wie ein Wilder. Nun hatte aber Gregor
noch gar keine Übung im Rückwärtsgehen, es ging wirklich sehr langsam.
Wenn sich Gregor nur hätte umdrehen dürfen, er wäre gleich in seinem
Zimmer gewesen, aber er fürchtete sich, den Vater durch die zeitraubende
Umdrehung ungeduldig zu machen, und jeden Augenblick drohte ihm doch von
dem Stock in des Vaters Hand der tödliche Schlag auf den Rücken oder auf
den Kopf. Endlich aber blieb Gregor doch nichts anderes übrig, denn er
merkte mit Entsetzen, daß er im Rückwärtsgehen nicht einmal die Richtung
einzuhalten verstand; und so begann er, unter unaufhörlichen ängstlichen
Seitenblicken nach dem Vater, sich nach Möglichkeit rasch, in
Wirklichkeit aber doch nur sehr langsam umzudrehen. Vielleicht merkte
der Vater seinen guten Willen, denn er störte ihn hierbei nicht, sondern
dirigierte sogar hie und da die Drehbewegung von der Ferne mit der
Spitze seines Stockes. Wenn nur nicht dieses unerträgliche Zischen des
Vaters gewesen wäre! Gregor verlor darüber ganz den Kopf. Er war schon
fast ganz umgedreht, als er sich, immer auf dieses Zischen horchend,
sogar irrte und sich wieder ein Stück zurückdrehte. Als er aber endlich
glücklich mit dem Kopf vor der Türöffnung war, zeigte es sich, daß sein
Körper zu breit war, um ohne weiteres durchzukommen. Dem Vater fiel es
natürlich in seiner gegenwärtigen Verfassung auch nicht entfernt ein,
etwa den anderen Türflügel zu öffnen, um für Gregor einen genügenden
Durchgang zu schaffen. Seine fixe Idee war bloß, daß Gregor so rasch als
möglich in sein Zimmer müsse. Niemals hätte er auch die umständlichen
Vorbereitungen gestattet, die Gregor brauchte, um sich aufzurichten und
vielleicht auf diese Weise durch die Tür zu kommen. Vielleicht trieb er,
als gäbe es kein Hindernis, Gregor jetzt unter besonderem Lärm
vorwärts; es klang schon hinter Gregor gar nicht mehr wie die Stimme
bloß eines einzigen Vaters; nun gab es wirklich keinen Spaß mehr, und
Gregor drängte sich --- geschehe was wolle --- in die Tür. Die eine Seite
seines Körpers hob sich, er lag schief in der Türöffnung, seine eine
Flanke war ganz wundgerieben, an der weißen Tür blieben häßliche Flecke,
bald steckte er fest und hätte sich allein nicht mehr rühren können, die
Beinchen auf der einen Seite hingen zitternd oben in der Luft, die auf
der anderen waren schmerzhaft zu Boden gedrückt --- da gab ihm der Vater
von hinten einen jetzt wahrhaftig erlösenden starken Stoß, und er flog,
heftig blutend, weit in sein Zimmer hinein. Die Tür wurde noch mit dem
Stock zugeschlagen, dann war es endlich still.

\pagebreak

\vspace*{2.5cm}

\section{II}

\noindent{}Erst in der Abenddämmerung erwachte Gregor aus seinem schweren
ohnmachtähnlichen Schlaf. Er wäre gewiß nicht viel später auch ohne
Störung erwacht, denn er fühlte sich genügend ausgeruht und
ausgeschlafen, doch schien es ihm, als hätte ihn ein flüchtiger Schritt
und ein vorsichtiges Schließen der zum Vorzimmer führenden Tür geweckt.
Der Schein der elektrischen Straßenbahn lag bleich hier und da auf der
Zimmerdecke und auf den höheren Teilen der Möbel, aber unten bei Gregor
war es finster. Langsam schob er sich, noch ungeschickt mit seinen
Fühlern tastend, die er jetzt erst schätzen lernte, zur Türe hin, um
nachzusehen, was dort geschehen war. Seine linke Seite schien eine
einzige lange, unangenehm spannende Narbe, und er mußte auf seinen zwei
Beinreihen regelrecht hinken. Ein Beinchen war übrigens im Laufe der
vormittägigen Vorfälle schwer verletzt worden --- es war fast ein
Wunder, daß nur eines verletzt worden war --- und schleppte leblos nach.

Erst bei der Tür merkte er, was ihn dorthin eigentlich gelockt hatte; es
war der Geruch von etwas\est\ Eßbarem gewesen. Denn dort stand ein Napf mit
süßer Milch gefüllt, in der kleine Schnitte von Weißbrot schwammen. Fast
hätte er vor Freude gelacht, denn er hatte noch größeren Hunger als am
Morgen, und gleich tauchte er seinen Kopf fast bis über die Augen in die
Milch hinein. Aber bald zog er ihn enttäuscht wieder zurück; nicht nur,
daß ihm das Essen wegen seiner heiklen linken Seite Schwierigkeiten
machte --- und er konnte nur essen, wenn der ganze Körper schnaufend
mitarbeitete ---, so schmeckte ihm überdies die Milch, die sonst sein
Lieblingsgetränk war und die ihm gewiß die Schwester deshalb
hereingestellt hatte, gar nicht, ja er wandte sich fast mit Widerwillen
von dem Napf ab und kroch in die Zimmermitte zurück.

Im Wohnzimmer war, wie Gregor durch die Türspalte sah, das Gas
angezündet, aber während sonst zu dieser Tageszeit der Vater seine
nachmittags erscheinende Zeitung der Mutter und manchmal auch der
Schwester mit erhobener Stimme vorzulesen pflegte, hörte man jetzt
keinen Laut. Nun vielleicht war dieses Vorlesen, von dem ihm die
Schwester immer erzählte und schrieb, in der letzten Zeit überhaupt aus
der Übung gekommen. Aber auch ringsherum war es so still, trotzdem doch
gewiß die Wohnung nicht leer war. »Was für ein stilles Leben die Familie
doch führte,« sagte sich Gregor und fühlte, während er starr vor sich
ins Dunkle sah, einen großen Stolz darüber, daß er seinen Eltern und
seiner Schwester ein solches Leben in einer so schönen Wohnung hatte
verschaffen können. Wie aber, wenn jetzt alle Ruhe, aller Wohlstand,
alle Zufriedenheit ein Ende mit Schrecken nehmen sollte? Um sich nicht
in solche Gedanken zu verlieren, setzte sich Gregor lieber in Bewegung
und kroch im Zimmer auf und ab.

Einmal während des langen Abends wurde die eine Seitentüre und einmal
die andere bis zu einer kleinen Spalte geöffnet und rasch wieder
geschlossen; jemand hatte wohl das Bedürfnis hereinzukommen, aber auch
wieder zu viele Bedenken. Gregor machte nun unmittelbar bei der
Wohnzimmertür Halt, entschlossen, den zögernden Besucher doch irgendwie
hereinzubringen oder doch wenigstens zu erfahren, wer es sei; aber nun
wurde die Tür nicht mehr geöffnet und Gregor wartete vergebens. Früh,
als die Türen versperrt waren, hatten alle zu ihm hereinkommen wollen,
jetzt, da er die eine Tür geöffnet hatte und die anderen offenbar
während des Tages geöffnet worden waren, kam keiner mehr, und die
Schlüssel steckten nun auch von außen.

Spät erst in der Nacht wurde das Licht im Wohnzimmer ausgelöscht, und
nun war leicht festzustellen, daß die Eltern und die Schwester so lange
wachgeblieben waren, denn wie man genau hören konnte, entfernten sich
jetzt alle drei auf den Fußspitzen. Nun kam gewiß bis zum Morgen niemand
mehr zu Gregor herein; er hatte also eine lange Zeit, um ungestört zu
überlegen, wie er sein Leben jetzt neu ordnen sollte. Aber das hohe
freie Zimmer, in dem er gezwungen war, flach auf dem Boden zu liegen,
ängstigte ihn, ohne daß er die Ursache herausfinden konnte, denn es war
ja sein seit fünf Jahren von ihm bewohntes Zimmer --- und mit einer halb
unbewußten Wendung und nicht ohne eine leichte Scham eilte er unter das
Kanapee, wo er sich, trotzdem sein Rücken ein wenig gedrückt wurde und
trotzdem er den Kopf nicht mehr erheben konnte, gleich sehr behaglich
fühlte und nur bedauerte, daß sein Körper zu breit war, um vollständig
unter dem Kanapee untergebracht zu werden.

Dort blieb er die ganze Nacht, die er zum Teil im Halbschlaf, aus dem
ihn der Hunger immer wieder aufschreckte, verbrachte, zum Teil aber in
Sorgen und undeutlichen Hoffnungen, die aber alle zu dem Schlusse
führten, daß er sich vorläufig ruhig verhalten und durch Geduld und
größte Rücksichtnahme der Familie die Unannehmlichkeiten erträglich
machen müsse, die er ihr in seinem gegenwärtigen Zustand nun einmal zu
verursachen gezwungen war.

Schon am frühen Morgen, es war fast noch Nacht, hatte Gregor
Gelegenheit, die Kraft seiner eben gefaßten Entschlüsse zu prüfen, denn
vom Vorzimmer her öffnete die Schwester, fast völlig angezogen, die Tür
und sah mit Spannung herein. Sie fand ihn nicht gleich, aber als sie ihn
unter dem Kanapee bemerkte --- Gott, er mußte doch irgendwo sein, er
hatte doch nicht wegfliegen können --- erschrak sie so sehr, daß sie,
ohne sich beherrschen zu können, die Tür von außen wieder zuschlug. Aber
als bereue sie ihr Benehmen, öffnete sie die Tür sofort wieder und trat,
als sei sie bei einem Schwerkranken oder gar bei einem Fremden, auf den
Fußspitzen herein. Gregor hatte den Kopf bis knapp zum Rande des
Kanapees vorgeschoben und beobachtete sie. Ob sie wohl bemerken würde,
daß er die Milch stehen gelassen hatte, und zwar keineswegs aus Mangel
an Hunger, und ob sie eine andere Speise hereinbringen würde, die ihm
besser entsprach? Täte sie es nicht von selbst, er wollte lieber
verhungern, als sie darauf aufmerksam machen, trotzdem es ihn eigentlich
ungeheuer drängte, unterm Kanapee vorzuschießen, sich der Schwester zu
Füßen zu werfen und sie um irgend etwas Gutes zum Essen zu bitten. Aber
die Schwester bemerkte sofort mit Verwunderung den noch vollen Napf, aus
dem nur ein wenig Milch ringsherum verschüttet war, sie hob ihn gleich
auf, zwar nicht mit den bloßen Händen, sondern mit einem Fetzen, und
trug ihn hinaus. Gregor war äußerst neugierig, was sie zum Ersatze
bringen würde, und er machte sich die verschiedensten Gedanken darüber.
Niemals aber hätte er erraten können, was die Schwester in ihrer Güte
wirklich tat. Sie brachte ihm, um seinen Geschmack zu prüfen, eine ganze
Auswahl, alles auf einer alten Zeitung ausgebreitet. Da war altes
halbverfaultes Gemüse; Knochen vom Nachtmahl her, die von festgewordener
weißer Sauce umgeben waren; ein paar Rosinen und Mandeln; ein Käse, den
Gregor vor zwei Tagen für ungenießbar erklärt hatte; ein trockenes Brot,
ein mit Butter beschmiertes Brot und ein mit Butter beschmiertes und
gesalzenes Brot. Außerdem stellte sie zu dem allen noch den
wahrscheinlich ein für allemal für Gregor bestimmten Napf, in den sie
Wasser gegossen hatte. Und aus Zartgefühl, da sie wußte, daß Gregor vor
ihr nicht essen würde, entfernte sie sich eiligst und drehte sogar den
Schlüssel um, damit nur Gregor merken könne, daß er es sich so behaglich
machen dürfe, wie er wolle. Gregors Beinchen schwirrten, als es jetzt
zum Essen ging. Seine Wunden mußten übrigens auch schon vollständig
geheilt sein, er fühlte keine Behinderung mehr, er staunte darüber und
dachte daran, wie er vor mehr als einem Monat sich mit dem Messer ganz
wenig in den Finger geschnitten, und wie ihm diese Wunde noch vorgestern
genug wehgetan hatte. »Sollte ich jetzt weniger Feingefühl haben?«
dachte er und saugte schon gierig an dem Käse, zu dem es ihn vor allen
anderen Speisen sofort und nachdrücklich gezogen hatte. Rasch
hintereinander und mit vor Befriedigung tränenden Augen verzehrte er den
Käse, das Gemüse und die Sauce; die frischen Speisen dagegen schmeckten
ihm nicht, er konnte nicht einmal ihren Geruch vertragen und schleppte
sogar die Sachen, die er essen wollte, ein Stückchen weiter weg. Er war
schon längst mit allem fertig und lag nur noch faul auf der gleichen
Stelle, als die Schwester zum Zeichen, daß er sich zurückziehen solle,
langsam den Schlüssel umdrehte. Das schreckte ihn sofort auf, trotzdem
er schon fast schlummerte, und er eilte wieder unter das Kanapee. Aber
es kostete ihn große Selbstüberwindung, auch nur die kurze Zeit, während
welcher die Schwester im Zimmer war, unter dem Kanapee zu bleiben, denn
von dem reichlichen Essen hatte sich sein Leib ein wenig gerundet, und
er konnte dort in der Enge kaum atmen. Unter kleinen Erstickungsanfällen
sah er mit etwas hervorgequollenen Augen zu, wie die nichtsahnende
Schwester mit einem Besen nicht nur die Überbleibsel zusammenkehrte,
sondern selbst die von Gregor gar nicht berührten Speisen, als seien
also auch diese nicht mehr zu gebrauchen, und wie sie alles hastig in
einen Kübel schüttete, den sie mit einem Holzdeckel schloß, worauf sie
alles hinaustrug. Kaum hatte sie sich umgedreht, zog sich schon Gregor
unter dem Kanapee hervor und streckte und blähte sich.

Auf diese Weise bekam nun Gregor täglich sein Essen, einmal am Morgen,
wenn die Eltern und das Dienstmädchen noch schliefen, das zweitemal nach
dem allgemeinen Mittagessen, denn dann schliefen die Eltern gleichfalls
noch ein Weilchen, und das Dienstmädchen wurde von der Schwester mit
irgendeiner Besorgung weggeschickt. Gewiß wollten auch sie nicht, daß
Gregor verhungere, aber vielleicht hätten sie es nicht ertragen können,
von seinem Essen mehr als durch Hörensagen zu erfahren, vielleicht
wollte die Schwester ihnen auch eine möglicherweise nur kleine Trauer
ersparen, denn tatsächlich litten sie ja gerade genug.

Mit welchen Ausreden man an jenem ersten Vormittag den Arzt und den
Schlosser wieder aus der Wohnung geschafft hatte, konnte Gregor gar
nicht erfahren, denn da er nicht verstanden wurde, dachte niemand daran,
auch die Schwester nicht, daß er die anderen verstehen könne, und so
mußte er sich, wenn die Schwester in seinem Zimmer war, damit begnügen,
nur hier und da ihre Seufzer und Anrufe der Heiligen zu hören. Erst
später, als sie sich ein wenig an alles gewöhnt hatte --- von
vollständiger Gewöhnung konnte natürlich niemals die Rede sein ---,
erhaschte Gregor manchmal eine Bemerkung, die freundlich gemeint war
oder so gedeutet werden konnte. »Heute hat es ihm aber geschmeckt,«
sagte sie, wenn Gregor unter dem Essen tüchtig aufgeräumt hatte, während
sie im gegenteiligen Fall, der sich allmählich immer häufiger
wiederholte, fast traurig zu sagen pflegte: »Nun ist wieder alles
stehengeblieben.«

Während aber Gregor unmittelbar keine Neuigkeit erfahren konnte,
erhorchte er manches aus den Nebenzimmern, und wo er nun einmal Stimmen
hörte, lief er gleich zu der betreffenden Tür und drückte sich mit
ganzem Leib an sie. Besonders in der ersten Zeit gab es kein Gespräch,
das nicht irgendwie wenn auch nur im geheimen, von ihm handelte. Zwei
Tage lang waren bei allen Mahlzeiten Beratungen darüber zu hören, wie
man sich jetzt verhalten solle; aber auch zwischen den Mahlzeiten sprach
man über das gleiche Thema, denn immer waren zumindest zwei
Familienmitglieder zu Hause, da wohl niemand allein zu Hause bleiben
wollte und man die Wohnung doch auf keinen Fall gänzlich verlassen
konnte. Auch hatte das Dienstmädchen gleich am ersten Tag --- es war
nicht ganz klar, was und wieviel sie von dem Vorgefallenen wußte ---
kniefällig die Mutter gebeten, sie sofort zu entlassen, und als sie sich
eine Viertelstunde danach verabschiedete, dankte sie für die Entlassung
unter Tränen, wie für die größte Wohltat, die man ihr hier erwiesen
hatte, und gab, ohne daß man es von ihr verlangte, einen fürchterlichen
Schwur ab, niemandem auch nur das geringste zu verraten.

Nun mußte die Schwester im Verein mit der Mutter auch kochen; allerdings
machte das nicht viel Mühe, denn man aß fast nichts. Immer wieder hörte
Gregor, wie der eine den anderen vergebens zum Essen aufforderte und
keine andere Antwort bekam, als: »Danke ich habe genug« oder etwas
Ähnliches. Getrunken wurde vielleicht auch nichts. Öfters fragte die
Schwester den Vater, ob er Bier haben wolle, und herzlich erbot sie
sich, es selbst zu holen, und als der Vater schwieg, sagte sie, um ihm
jedes Bedenken zu nehmen, sie könne auch die Hausmeisterin darum
schicken, aber dann sagte der Vater schließlich ein großes »Nein«, und
es wurde nicht mehr davon gesprochen.

Schon im Laufe des ersten Tages legte der Vater die ganzen
Vermögensverhältnisse und Aussichten sowohl der Mutter als auch der
Schwester dar. Hie und da stand er vom Tische auf und holte aus seiner
kleinen Wertheimkassa, die er aus dem vor fünf Jahren erfolgten
Zusammenbruch seines Geschäftes gerettet hatte, irgendeinen Beleg oder
irgendein Vormerkbuch. Man hörte, wie er das komplizierte Schloß
aufsperrte und nach Entnahme des Gesuchten wieder verschloß. Diese
Erklärungen des Vaters waren zum Teil das erste Erfreuliche, was Gregor
seit seiner Gefangenschaft zu hören bekam. Er war der Meinung gewesen,
daß dem Vater von jenem Geschäft her nicht das Geringste übriggeblieben
war, zumindest hatte ihm der Vater nichts Gegenteiliges gesagt, und
Gregor allerdings hatte ihn auch nicht darum gefragt. Gregors Sorge war
damals nur gewesen, alles daranzusetzen, um die Familie das
geschäftliche Unglück, das alle in eine vollständige Hoffnungslosigkeit
gebracht hatte, möglichst rasch vergessen zu lassen. Und so hatte er
damals mit ganz besonderem Feuer zu arbeiten angefangen und war fast
über Nacht aus einem kleinen Kommis ein Reisender geworden, der
natürlich ganz andere Möglichkeiten des Geldverdienens hatte, und dessen
Arbeitserfolge sich sofort in Form der Provision zu Bargeld
verwandelten, das der erstaunten und beglückten Familie zu Hause auf den
Tisch gelegt werden konnte. Es waren schöne Zeiten gewesen, und niemals
nachher hatten sie sich, wenigstens in diesem Glanze, wiederholt,
trotzdem Gregor später so viel Geld verdiente, daß er den Aufwand der
ganzen Familie zu tragen imstande war und auch trug. Man hatte sich eben
daran gewöhnt, sowohl die Familie, als auch Gregor, man nahm das Geld
dankbar an, er lieferte es gern ab, aber eine besondere Wärme wollte
sich nicht mehr ergeben. Nur die Schwester war Gregor doch noch nahe
geblieben, und es war sein geheimer Plan, sie, die zum Unterschied von
Gregor Musik sehr liebte und rührend Violine zu spielen verstand,
nächstes Jahr, ohne Rücksicht auf die großen Kosten, die das verursachen
mußte, und die man schon auf andere Weise hereinbringen würde, auf das
Konservatorium zu schicken. Öfters während der kurzen Aufenthalte
Gregors in der Stadt wurde in den Gesprächen mit der Schwester das
Konservatorium erwähnt, aber immer nur als schöner Traum, an dessen
Verwirklichung nicht zu denken war, und die Eltern hörten nicht einmal
diese unschuldigen Erwähnungen gern; aber Gregor dachte sehr bestimmt
daran und beabsichtigte, es am Weihnachtsabend feierlich zu erklären.

Solche in seinem gegenwärtigen Zustand ganz nutzlose Gedanken gingen ihm
durch den Kopf, während er dort aufrecht an der Türe klebte und horchte.
Manchmal konnte er vor allgemeiner Müdigkeit gar nicht mehr zuhören und
ließ den Kopf nachlässig gegen die Tür schlagen, hielt ihn aber sofort
wieder fest, denn selbst das kleine Geräusch, das er damit verursacht
hatte, war nebenan gehört worden und hatte alle verstummen lassen. »Was
er nur wieder treibt,« sagte der Vater nach einer Weile, offenbar zur
Türe hingewendet, und dann erst wurde das unterbrochene Gespräch
allmählich wieder aufgenommen.

Gregor erfuhr nun zur Genüge --- denn der Vater pflegte sich in seinen
Erklärungen öfters zu wiederholen, teils, weil er selbst sich mit diesen
Dingen schon lange nicht beschäftigt hatte, teils auch, weil die Mutter
nicht alles gleich beim erstenmal verstand ---, daß trotz allen Unglücks
ein allerdings ganz kleines Vermögen aus der alten Zeit noch vorhanden
war, das die nicht angerührten Zinsen in der Zwischenzeit ein wenig
hatten anwachsen lassen. Außerdem aber war das Geld, das Gregor
allmonatlich nach Hause gebracht hatte --- er selbst hatte nur ein paar
Gulden für sich behalten ---, nicht vollständig aufgebraucht worden und
hatte sich zu einem kleinen Kapital angesammelt. Gregor, hinter seiner
Türe, nickte eifrig, erfreut über diese unerwartete Vorsicht und
Sparsamkeit. Eigentlich hätte er ja mit diesen überschüssigen Geldern
die Schuld des Vaters gegenüber dem Chef weiter abgetragen haben können,
und jener Tag, an dem er diesen Posten hätte loswerden können, wäre weit
näher gewesen, aber jetzt war es zweifellos besser so, wie es der Vater
eingerichtet hatte.

Nun genügte dieses Geld aber ganz und gar nicht, um die Familie etwa von
den Zinsen leben zu lassen; es genügte vielleicht, um die Familie ein,
höchstens zwei Jahre zu erhalten, mehr war es nicht. Es war also bloß
eine Summe, die man eigentlich nicht angreifen durfte, und die für den
Notfall zurückgelegt werden mußte; das Geld zum Leben aber mußte man
verdienen. Nun war aber der Vater ein zwar gesunder, aber alter Mann,
der schon fünf Jahre nichts gearbeitet hatte und sich jedenfalls nicht
viel zutrauen durfte; er hatte in diesen fünf Jahren, welche die ersten
Ferien seines mühevollen und doch erfolglosen Lebens waren, viel Fett
angesetzt und war dadurch recht schwerfällig geworden. Und die alte
Mutter sollte nun vielleicht Geld verdienen, die an Asthma litt, der
eine Wanderung durch die Wohnung schon Anstrengung verursachte, und die
jeden zweiten Tag in Atembeschwerden auf dem Sofa beim offenen Fenster
verbrachte? Und die Schwester sollte Geld verdienen, die noch ein Kind
war mit ihren siebzehn Jahren, und der ihre bisherige Lebensweise so
sehr zu gönnen war, die daraus bestanden hatte, sich nett zu kleiden,
lange zu schlafen, in der Wirtschaft mitzuhelfen, an ein paar
bescheidenen Vergnügungen sich zu beteiligen und vor allem Violine zu
spielen? Wenn die Rede auf diese Notwendigkeit des Geldverdienens kam,
ließ zuerst immer Gregor die Türe los und warf sich auf das neben der
Tür befindliche kühle Ledersofa, denn ihm war ganz heiß vor Beschämung
und Trauer.

Oft lag er dort die ganzen langen Nächte über, schlief keinen Augenblick
und scharrte nur stundenlang auf dem Leder. Oder er scheute nicht die
große Mühe, einen Sessel zum Fenster zu schieben, dann die
Fensterbrüstung hinaufzukriechen und, in den Sessel gestemmt, sich ans
Fenster zu lehnen, offenbar nur in irgendeiner Erinnerung an das
Befreiende, das früher für ihn darin gelegen war, aus dem Fenster zu
schauen. Denn tatsächlich sah er von Tag zu Tag die auch nur ein wenig
entfernten Dinge immer undeutlicher; das gegenüberliegende Krankenhaus,
dessen nur allzu häufigen Anblick er früher verflucht hatte, bekam er
überhaupt nicht mehr zu Gesicht, und wenn er nicht genau gewußt hätte,
daß er in der stillen, aber völlig städtischen Charlottenstraße wohnte,
hätte er glauben können, von seinem Fenster aus in eine Einöde zu
schauen in welcher der graue Himmel und die graue Erde ununterscheidbar
sich vereinigten. Nur zweimal hatte die aufmerksame Schwester sehen
müssen, daß der Sessel beim Fenster stand, als sie schon jedesmal,
nachdem sie das Zimmer aufgeräumt hatte, den Sessel wieder genau zum
Fenster hinschob, ja sogar von nun ab den inneren Fensterflügel offen
ließ.

Hätte Gregor nur mit der Schwester sprechen und ihr für alles danken
können, was sie für ihn machen mußte, er hätte ihre Dienste leichter
ertragen; so aber litt er darunter. Die Schwester suchte freilich die
Peinlichkeit des Ganzen möglichst zu verwischen, und je längere Zeit
verging, desto besser gelang es ihr natürlich auch, aber auch Gregor
durchschaute mit der Zeit alles viel genauer. Schon ihr Eintritt war für
ihn schrecklich. Kaum war sie eingetreten, lief sie, ohne sich Zeit zu
nehmen, die Türe zu schließen, so sehr sie sonst darauf achtete, jedem
den Anblick von Gregors Zimmer zu ersparen, geradewegs zum Fenster und
riß es, als ersticke sie fast, mit hastigen Händen auf, blieb auch,
selbst wenn es noch so kalt war, ein Weilchen beim Fenster und atmete
tief. Mit diesem Laufen und Lärmen erschreckte sie Gregor täglich
zweimal; die ganze Zeit über zitterte er unter dem Kanapee und wußte
doch sehr gut, daß sie ihn gewiß gerne damit verschont hätte, wenn es
ihr nur möglich gewesen wäre, sich in einem Zimmer, in dem sich Gregor
befand, bei geschlossenem Fenster aufzuhalten.

Einmal, es war wohl schon ein Monat seit Gregors Verwandlung vergangen,
und es war doch schon für die Schwester kein besonderer Grund mehr, über
Gregors Aussehen in Erstaunen zu geraten, kam sie ein wenig früher als
sonst und traf Gregor noch an, wie er, unbeweglich und so recht zum
Erschrecken aufgestellt, aus dem Fenster schaute. Es wäre für Gregor
nicht unerwartet gewesen, wenn sie nicht eingetreten wäre, da er sie
durch seine Stellung verhinderte, sofort das Fenster zu öffnen, aber sie
trat nicht nur nicht ein, sie fuhr sogar zurück und schloß die Tür; ein
Fremder hätte geradezu denken können, Gregor habe ihr aufgelauert und
habe sie beißen wollen. Gregor versteckte sich natürlich sofort unter
dem Kanapee, aber er mußte bis zum Mittag warten, ehe die Schwester
wiederkam, und sie schien viel unruhiger als sonst. Er erkannte daraus,
daß ihr sein Anblick noch immer unerträglich war und ihr auch weiterhin
unerträglich bleiben müsse, und daß sie sich wohl sehr überwinden mußte,
vor dem Anblick auch nur der kleinen Partie seines Körpers nicht
davonzulaufen, mit der er unter dem Kanapee hervorragte. Um ihr auch
diesen Anblick zu ersparen, trug er eines Tages auf seinem Rücken --- er
brauchte zu dieser Arbeit vier Stunden --- das Leintuch auf das Kanapee
und ordnete es in einer solchen Weise an, daß er nun gänzlich verdeckt
war, und daß die Schwester, selbst wenn sie sich bückte, ihn nicht sehen
konnte. Wäre dieses Leintuch ihrer Meinung nach nicht nötig gewesen,
dann hätte sie es ja entfernen können, denn daß es nicht zum Vergnügen
Gregors gehören konnte, sich so ganz und gar abzusperren, war doch klar
genug, aber sie ließ das Leintuch, so wie es war, und Gregor glaubte
sogar einen dankbaren Blick erhascht zu haben, als er einmal mit dem
Kopf vorsichtig das Leintuch ein wenig lüftete, um nachzusehen, wie die
Schwester die neue Einrichtung aufnahm.

In den ersten vierzehn Tagen konnten es die Eltern nicht über sich
bringen, zu ihm hereinzukommen, und er hörte oft, wie sie die jetzige
Arbeit der Schwester völlig anerkannten, während sie sich bisher häufig
über die Schwester geärgert hatten, weil sie ihnen als ein etwas
nutzloses Mädchen erschienen war. Nun aber warteten oft beide, der Vater
und die Mutter, vor Gregors Zimmer, während die Schwester dort
aufräumte, und kaum war sie herausgekommen, mußte sie ganz genau
erzählen, wie es in dem Zimmer aussah, was Gregor gegessen hatte, wie er
sich diesmal benommen hatte, und ob vielleicht eine kleine Besserung zu
bemerken war. Die Mutter übrigens wollte verhältnismäßig bald Gregor
besuchen, aber der Vater und die Schwester hielten sie zuerst mit
Vernunftgründen zurück, denen Gregor sehr aufmerksam zuhörte, und die er
vollständig billigte. Später aber mußte man sie mit Gewalt zurückhalten,
und wenn sie dann rief: »Laßt mich doch zu Gregor, er ist ja mein
unglücklicher Sohn! Begreift ihr es denn nicht, daß ich zu ihm muß?«,
dann dachte Gregor, daß es vielleicht doch gut wäre, wenn die Mutter
hereinkäme, nicht jeden Tag natürlich, aber vielleicht einmal in der
Woche; sie verstand doch alles viel besser als die Schwester, die trotz
all ihrem Mute doch nur ein Kind war und im letzten Grunde vielleicht
nur aus kindlichem Leichtsinn eine so schwere Aufgabe übernommen hatte.

Der Wunsch Gregors, die Mutter zu sehen, ging bald in Erfüllung. Während
des Tages wollte Gregor schon aus Rücksicht auf seine Eltern sich nicht
beim Fenster zeigen, kriechen konnte er aber auf den paar Quadratmetern
des Fußbodens auch nicht viel, das ruhige Liegen ertrug er schon während
der Nacht schwer, das Essen machte ihm bald nicht mehr das geringste
Vergnügen, und so nahm er zur Zerstreuung die Gewohnheit an, kreuz und
quer über Wände und Plafond zu kriechen. Besonders oben an der Decke
hing er gern; es war ganz anders, als das Liegen auf dem Fußboden; man
atmete freier; ein leichtes Schwingen ging durch den Körper, und in der
fast glücklichen Zerstreutheit, in der sich Gregor dort oben befand,
konnte es geschehen, daß er zu seiner eigenen Überraschung sich losließ
und auf den Boden klatschte. Aber nun hatte er natürlich seinen Körper
ganz anders in der Gewalt als früher und beschädigte sich selbst bei
einem so großen Falle nicht. Die Schwester nun bemerkte sofort die neue
Unterhaltung, die Gregor für sich gefunden hatte --- er hinterließ ja
auch beim Kriechen hie und da Spuren seines Klebstoffes ---, und da
setzte sie es sich in den Kopf, Gregor das Kriechen in größtem Ausmaße
zu ermöglichen und die Möbel, die es verhinderten, also vor allem den
Kasten und den Schreibtisch, wegzuschaffen. Nun war sie aber nicht
imstande, dies allein zu tun; den Vater wagte sie nicht um Hilfe zu
bitten; das Dienstmädchen hätte ihr ganz gewiß nicht geholfen, denn
dieses etwa sechzehnjährige Mädchen harrte zwar tapfer seit Entlassung
der früheren Köchin aus, hatte aber um die Vergünstigung gebeten, die
Küche unaufhörlich versperrt halten zu dürfen und nur auf besonderen
Anruf öffnen zu müssen; so blieb der Schwester also nichts übrig, als
einmal in Abwesenheit des Vaters die Mutter zu holen. Mit Ausrufen
erregter Freude kam die Mutter auch heran, verstummte aber an der Tür
vor Gregors Zimmer. Zuerst sah natürlich die Schwester nach, ob alles im
Zimmer in Ordnung war; dann erst ließ sie die Mutter eintreten. Gregor
hatte in größter Eile das Leintuch noch tiefer und mehr in Falten
gezogen, das Ganze sah wirklich nur wie ein zufällig über das Kanapee
geworfenes Leintuch aus. Gregor unterließ auch diesmal, unter dem
Leintuch zu spionieren; er verzichtete darauf, die Mutter schon diesmal
zu sehen, und war nur froh, daß sie nun doch gekommen war. »Komm nur,
man sieht ihn nicht,« sagte die Schwester, und offenbar führte sie die
Mutter an der Hand. Gregor hörte nun, wie die zwei schwachen Frauen den
immerhin schweren alten Kasten von seinem Platze rückten, und wie die
Schwester immerfort den größten Teil der Arbeit für sich beanspruchte,
ohne auf die Warnungen der Mutter zu hören, welche fürchtete, daß sie
sich überanstrengen werde. Es dauerte sehr lange. Wohl nach schon
viertelstündiger Arbeit sagte die Mutter, man solle den Kasten doch
lieber hier lassen, denn erstens sei er zu schwer, sie würden vor
Ankunft des Vaters nicht fertig werden und mit dem Kasten in der Mitte
des Zimmers Gregor jeden Weg verrammeln, zweitens aber sei es doch gar
nicht sicher, daß Gregor mit der Entfernung der Möbel ein Gefallen
geschehe. Ihr scheine das Gegenteil der Fall zu sein; ihr bedrücke der
Anblick der leeren Wand geradezu das Herz; und warum solle nicht auch
Gregor diese Empfindung haben, da er doch an die Zimmermöbel längst
gewöhnt sei und sich deshalb im leeren Zimmer verlassen fühlen werde.
»Und ist es dann nicht so,« schloß die Mutter ganz leise, wie sie
überhaupt fast flüsterte, als wolle sie vermeiden, daß Gregor, dessen
genauen Aufenthalt sie ja nicht kannte, auch nur den Klang der Stimme
höre, denn daß er die Worte nicht verstand, davon war sie überzeugt,
»und ist es nicht so, als ob wir durch die Entfernung der Möbel zeigten,
daß wir jede Hoffnung auf Besserung aufgeben und ihn rücksichtslos sich
selbst überlassen? Ich glaube, es wäre das beste, wir suchen das Zimmer
genau in dem Zustand zu erhalten, in dem es früher war, damit Gregor,
wenn er wieder zu uns zurückkommt, alles unverändert findet und um so
leichter die Zwischenzeit vergessen kann.«

Beim Anhören dieser Worte der Mutter erkannte Gregor, daß der Mangel
jeder unmittelbaren menschlichen Ansprache, verbunden mit dem
einförmigen Leben inmitten der Familie, im Laufe dieser zwei Monate
seinen Verstand hatte verwirren müssen, denn anders konnte er es sich
nicht erklären, daß er ernsthaft darnach hatte verlangen können, daß
sein Zimmer ausgeleert würde. Hatte er wirklich Lust, das warme, mit
ererbten Möbeln gemütlich ausgestattete Zimmer in eine Höhle verwandeln
zu lassen, in der er dann freilich nach allen Richtungen ungestört würde
kriechen können, jedoch auch unter gleichzeitigem, schnellen, gänzlichen
Vergessen seiner menschlichen Vergangenheit? War er doch jetzt schon
nahe daran, zu vergessen, und nur die seit langem nicht gehörte Stimme
der Mutter hatte ihn aufgerüttelt. Nichts sollte entfernt werden, alles
mußte bleiben, die guten Einwirkungen der Möbel auf seinen Zustand
konnte er nicht entbehren; und wenn die Möbel ihn hinderten, das
sinnlose Herumkriechen zu betreiben, so war es kein Schaden, sondern ein
großer Vorteil.

Aber die Schwester war leider anderer Meinung; sie hatte sich,
allerdings nicht ganz unberechtigt, angewöhnt, bei Besprechung der
Angelegenheiten Gregors als besonders Sachverständige gegenüber den
Eltern aufzutreten, und so war auch jetzt der Rat der Mutter für die
Schwester Grund genug, auf der Entfernung nicht nur des Kastens und des
Schreibtisches, an die sie zuerst allein gedacht hatte, sondern auf der
Entfernung sämtlicher Möbel, mit Ausnahme des unentbehrlichen Kanapees,
zu bestehen. Es war natürlich nicht nur kindlicher Trotz und das in der
letzten Zeit so unerwartet und schwer erworbene Selbstvertrauen, das sie
zu dieser Forderung bestimmte; sie hatte doch auch tatsächlich
beobachtet, daß Gregor viel Raum zum Kriechen brauchte, dagegen die
Möbel, soweit man sehen konnte, nicht im geringsten benützte. Vielleicht
aber spielte auch der schwärmerische Sinn der Mädchen ihres Alters mit,
der bei jeder Gelegenheit seine Befriedigung sucht, und durch den Grete
jetzt sich dazu verlocken ließ, die Lage Gregors noch
schreckenerregender machen zu wollen, um dann noch mehr als bis jetzt
für ihn leisten zu können. Denn in einem Raum, in dem Gregor ganz allein
die leeren Wände beherrschte, würde wohl kein Mensch außer Grete jemals
einzutreten sich getrauen.

Und so ließ sie sich von ihrem Entschlusse durch die Mutter nicht
abbringen, die auch in diesem Zimmer vor lauter Unruhe unsicher schien,
bald verstummte und der Schwester nach Kräften beim Hinausschaffen des
Kastens half. Nun, den Kasten konnte Gregor im Notfall noch entbehren,
aber schon der Schreibtisch mußte bleiben. Und kaum hatten die Frauen
mit dem Kasten, an dem sie sich ächzend drückten, das Zimmer verlassen,
als Gregor den Kopf unter dem Kanapee hervorstieß, um zu sehen, wie er
vorsichtig und möglichst rücksichtsvoll eingreifen könnte. Aber zum
Unglück war es gerade die Mutter, welche zuerst zurückkehrte, während
Grete im Nebenzimmer den Kasten umfangen hielt und ihn allein hin und
her schwang, ohne ihn natürlich von der Stelle zu bringen. Die Mutter
aber war Gregors Anblick nicht gewöhnt, er hätte sie krank machen
können, und so eilte Gregor erschrocken im Rückwärtslauf bis an das
andere Ende des Kanapees, konnte es aber nicht mehr verhindern, daß das
Leintuch vorne ein wenig sich bewegte. Das genügte, um die Mutter
aufmerksam zu machen. Sie stockte, stand einen Augenblick still und ging
dann zu Grete zurück.

Trotzdem sich Gregor immer wieder sagte, daß ja nichts Außergewöhnliches
geschehe, sondern nur ein paar Möbel umgestellt würden, wirkte doch, wie
er sich bald eingestehen mußte, dieses Hin- und Hergehen der Frauen,
ihre kleinen Zurufe, das Kratzen der Möbel auf dem Boden, wie ein
großer, von allen Seiten genährter Trubel auf ihn, und er mußte sich, so
fest er Kopf und Beine an sich zog und den Leib bis an den Boden
drückte, unweigerlich sagen, daß er das Ganze nicht lange aushalten
werde. Sie räumten ihm sein Zimmer aus; nahmen ihm alles, was ihm lieb
war; den Kasten, in dem die Laubsäge und andere Werkzeuge lagen, hatten
sie schon hinausgetragen; lockerten jetzt den schon im Boden fest
eingegrabenen Schreibtisch, an dem er als Handelsakademiker, als
Bürgerschüler, ja sogar schon als Volksschüler seine Aufgaben
geschrieben hatte, --- da hatte er wirklich keine Zeit mehr, die guten
Absichten zu prüfen, welche die zwei Frauen hatten, deren Existenz er
übrigens fast vergessen hatte, denn vor Erschöpfung arbeiteten sie schon
stumm, und man hörte nur das schwere Tappen ihrer Füße.

Und so brach er denn hervor --- die Frauen stützten sich gerade im
Nebenzimmer an den Schreibtisch, um ein wenig zu verschnaufen ---,
wechselte viermal die Richtung des Laufes, er wußte wirklich nicht, was
er zuerst retten sollte, da sah er an der im übrigen schon leeren Wand
auffallend das Bild der in lauter Pelzwerk gekleideten Dame hängen,
kroch eilends hinauf und preßte sich an das Glas, das ihn festhielt und
seinem heißen Bauch wohltat. Dieses Bild wenigstens, das Gregor jetzt
ganz verdeckte, würde nun gewiß niemand wegnehmen. Er verdrehte den Kopf
nach der Tür des Wohnzimmers, um die Frauen bei ihrer Rückkehr zu
beobachten.

Sie hatten sich nicht viel Ruhe gegönnt und kamen schon wieder; Grete
hatte den Arm um die Mutter gelegt und trug sie fast. »Also was nehmen
wir jetzt?« sagte Grete und sah sich um, Da kreuzten sich ihre Blicke
mit denen Gregors an der Wand. Wohl nur infolge der Gegenwart der Mutter
behielt sie ihre Fassung, beugte ihr Gesicht zur Mutter, um diese vom
Herumschauen abzuhalten, und sagte, allerdings zitternd und unüberlegt:
»Komm, wollen wir nicht lieber auf einen Augenblick noch ins Wohnzimmer
zurückgehen?« Die Absicht Gretes war für Gregor klar, sie wollte die
Mutter in Sicherheit bringen und dann ihn von der Wand hinunterjagen.
Nun, sie konnte es ja immerhin versuchen! Er saß auf seinem Bild und
gab es nicht her. Lieber würde er Grete ins Gesicht springen.

Aber Gretes Worte hatten die Mutter erst recht beunruhigt, sie trat zur
Seite, erblickte den riesigen braunen Fleck auf der geblümten Tapete,
rief, ehe ihr eigentlich zum Bewußtsein kam, daß das Gregor war, was sie
sah, mit schreiender, rauher Stimme: »Ach Gott, ach Gott!« und fiel mit
ausgebreiteten Armen, als gebe sie alles auf, über das Kanapee hin und
rührte sich nicht. »Du, Gregor!« rief die Schwester mit erhobener Faust
und eindringlichen Blicken. Es waren seit der Verwandlung die ersten
Worte, die sie unmittelbar an ihn gerichtet hatte. Sie lief ins
Nebenzimmer, um irgendeine Essenz zu holen, mit der sie die Mutter aus
ihrer Ohnmacht wecken könnte; Gregor wollte auch helfen --- zur Rettung
des Bildes war noch Zeit ---; er klebte aber fest an dem Glas und mußte
sich mit Gewalt losreißen; er lief dann auch ins Nebenzimmer, als könne
er der Schwester irgendeinen Rat geben, wie in früherer Zeit; mußte aber
dann untätig hinter ihr stehen; während sie in verschiedenen Fläschchen
kramte, erschreckte sie noch, als sie sich umdrehte; eine Flasche fiel
auf den Boden und zerbrach; ein Splitter verletzte Gregor im Gesicht,
irgendeine ätzende Medizin umfloß ihn; Grete nahm nun, ohne sich länger
aufzuhalten, so viele Fläschchen, als sie nur halten konnte, und rannte
mit ihnen zur Mutter hinein; die Tür schlug sie mit dem Fuße zu. Gregor
war nun von der Mutter abgeschlossen, die durch seine Schuld vielleicht
dem Tode nahe war; die Tür durfte er nicht öffnen, wollte er die
Schwester, die bei der Mutter bleiben mußte, nicht verjagen; er hatte
jetzt nichts zu tun, als zu warten; und von Selbstvorwürfen und
Besorgnis bedrängt, begann er zu kriechen, überkroch alles, Wände,
Möbel und Zimmerdecke und fiel endlich in seiner Verzweiflung, als sich
das ganze Zimmer schon um ihn zu drehen anfing, mitten auf den großen
Tisch.

Es verging eine kleine Weile, Gregor lag matt da, ringsherum war es
still, vielleicht war das ein gutes Zeichen. Da läutete es. Das Mädchen
war natürlich in ihrer Küche eingesperrt und Grete mußte daher öffnen
gehen. Der Vater war gekommen. »Was ist\est\ geschehen?« waren seine ersten
Worte; Gretes Aussehen hatte ihm wohl alles verraten. Grete antwortete
mit dumpfer Stimme, offenbar drückte sie ihr Gesicht an des Vaters
Brust: »Die Mutter war ohnmächtig, aber es geht ihr schon besser. Gregor
ist ausgebrochen.« »Ich habe es ja erwartet,« sagte der Vater, »ich habe
es euch ja immer gesagt, aber ihr Frauen wollt nicht hören.« Gregor war
es klar, daß der Vater Gretes allzukurze Mitteilung schlecht gedeutet
hatte und annahm, daß Gregor sich irgendeine Gewalttat habe zuschulden
kommen lassen. Deshalb mußte Gregor den Vater jetzt zu besänftigen
suchen, denn ihn aufzuklären hatte er weder Zeit noch Möglichkeit. Und
so flüchtete er sich zur Tür seines Zimmers und drückte sich an sie,
damit der Vater beim Eintritt vom Vorzimmer her gleich sehen könne, daß
Gregor die beste Absicht habe, sofort in sein Zimmer zurückzukehren, und
daß es nicht nötig sei, ihn zurückzutreiben, sondern daß man nur die Tür
zu öffnen brauchte, und gleich werde er verschwinden.

Aber der Vater war nicht in der Stimmung, solche Feinheiten zu bemerken.
»Ah!« rief er gleich beim Eintritt in einem Tone, als sei er
gleichzeitig wütend und froh. Gregor zog den Kopf von der Tür zurück und
hob ihn gegen den Vater. So hatte er sich den Vater wirklich nicht
vorgestellt, wie er jetzt dastand; allerdings hatte er in der letzten
Zeit über dem neuartigen Herumkriechen versäumt, sich so wie früher um
die Vorgänge in der übrigen Wohnung zu kümmern, und hätte eigentlich
darauf gefaßt sein müssen, veränderte Verhältnisse anzutreffen.
Trotzdem, trotzdem, war das noch der Vater? Der gleiche Mann, der müde
im Bett vergraben lag, wenn früher Gregor zu einer Geschäftsreise
ausgerückt war; der ihn an Abenden der Heimkehr im Schlafrock im
Lehnstuhl empfangen hatte; gar nicht recht imstande war, aufzustehen,
sondern zum Zeichen der Freude nur die Arme gehoben hatte, und der bei
den seltenen gemeinsamen Spaziergängen an ein paar Sonntagen im Jahr und
an den höchsten Feiertagen zwischen Gregor und der Mutter, die schon an
und für sich langsam gingen, immer noch ein wenig langsamer, in seinen
alten Mantel eingepackt, mit stets vorsichtig aufgesetztem Krückstock
sich vorwärts arbeitete und, wenn er etwas sagen wollte, fast immer
stillstand und seine Begleitung um sich versammelte? Nun aber war er
doch gut aufgerichtet; in eine straffe blaue Uniform mit Goldknöpfen
gekleidet, wie sie Diener der Bankinstitute tragen; über dem hohen
steifen Kragen des Rockes entwickelte sich sein starkes Doppelkinn;
unter den buschigen Augenbrauen drang der Blick der schwarzen Augen
frisch und aufmerksam hervor; das sonst zerzauste weiße Haar war zu
einer peinlich genauen, leuchtenden Scheitelfrisur niedergekämmt. Er
warf seine Mütze, auf der ein Goldmonogramm, wahrscheinlich das einer
Bank, angebracht war, über das ganze Zimmer im Bogen auf das Kanapee hin
und ging, die Enden seines langen Uniformrockes zurückgeschlagen, die
Hände in den Hosentaschen, mit verbissenem Gesicht auf Gregor zu. Er
wußte wohl selbst nicht, was er vorhatte; immerhin hob er die Füße
ungewöhnlich hoch, und Gregor staunte über die Riesengröße seiner
Stiefelsohlen. Doch hielt er sich dabei nicht auf, er wußte ja noch vom
ersten Tage seines neuen Lebens her, daß der Vater ihm gegenüber nur die
größte Strenge für angebracht ansah. Und so lief er vor dem Vater her,
stockte, wenn der Vater stehen blieb, und eilte schon wieder vorwärts,
wenn sich der Vater nur rührte. So machten sie mehrmals die Runde um das
Zimmer, ohne daß sich etwas Entscheidendes ereignete, ja ohne daß das
Ganze infolge seines langsamen Tempos den Anschein einer Verfolgung
gehabt hätte. Deshalb blieb auch Gregor vorläufig auf dem Fußboden,
zumal er fürchtete, der Vater könnte eine Flucht auf die Wände oder den
Plafond für besondere Bosheit halten. Allerdings mußte sich Gregor
sagen, daß er sogar dieses Laufen nicht lange aushalten würde, denn
während der Vater einen Schritt machte, mußte er eine Unzahl von
Bewegungen ausführen. Atemnot begann sich schon bemerkbar zu machen, wie
er ja auch in seiner früheren Zeit keine ganz vertrauenswürdige Lunge
besessen hatte. Als er nun so dahintorkelte, um alle Kräfte für den Lauf
zu sammeln, kaum die Augen offenhielt; in seiner Stumpfheit an eine
andere Rettung als durch Laufen gar nicht dachte; und fast schon
vergessen hatte, daß ihm die Wände freistanden, die hier allerdings mit
sorgfältig geschnitzten Möbeln voll Zacken und Spitzen verstellt waren
--- da flog knapp neben ihm, leicht geschleudert, irgend etwas nieder und
rollte vor ihm her. Es war ein Apfel; gleich flog ihm ein zweiter nach;\est\
Gregor blieb vor Schrecken stehen; ein Weiterlaufen war nutzlos, denn
der Vater hatte sich entschlossen, ihn zu bombardieren. Aus der
Obstschale auf der Kredenz hatte er sich die Taschen gefüllt und warf
nun, ohne vorläufig scharf zu zielen, Apfel für Apfel. Diese kleinen
roten Äpfel rollten wie elektrisiert auf dem Boden herum und stießen
aneinander. Ein schwach geworfener Apfel streifte Gregors Rücken, glitt
aber unschädlich ab. Ein ihm sofort nachfliegender drang dagegen
förmlich in Gregors Rücken ein; Gregor wollte sich weiterschleppen, als
könne der überraschende unglaubliche Schmerz mit dem Ortswechsel
vergehen; doch fühlte er sich wie festgenagelt und streckte sich in
vollständiger Verwirrung aller Sinne. Nur mit dem letzten Blick sah er
noch, wie die Tür seines Zimmers aufgerissen wurde, und vor der
schreienden Schwester die Mutter hervoreilte, im Hemd, denn die
Schwester hatte sie entkleidet, um ihr in der Ohnmacht Atemfreiheit zu
verschaffen, wie dann die Mutter auf den Vater zulief und ihr auf dem
Weg die aufgebundenen Röcke einer nach dem anderen zu Boden glitten, und
wie sie stolpernd über die Röcke auf den Vater eindrang und ihn
umarmend, in gänzlicher Vereinigung mit ihm --- nun versagte aber Gregors
Sehkraft schon --- die Hände an des Vaters Hinterkopf um Schonung von
Gregors Leben bat.

\pagebreak

\vspace*{2.5cm}

\section{III}

\noindent{}Die schwere Verwundung Gregors, an der er über einen Monat litt --- der
Apfel blieb, da ihn niemand zu entfernen wagte, als sichtbares Andenken
im Fleische sitzen ---, schien selbst den Vater daran erinnert zu haben,
daß Gregor trotz seiner gegenwärtigen traurigen und ekelhaften Gestalt
ein Familienglied war, das man nicht wie einen Feind behandeln durfte,
sondern dem gegenüber es das Gebot der Familienpflicht war, den
Widerwillen hinunterzuschlucken und zu dulden, nichts als dulden.

Und wenn nun auch Gregor durch seine Wunde an Beweglichkeit
wahrscheinlich für immer verloren hatte und vorläufig zur Durchquerung
seines Zimmers wie ein alter Invalide lange, lange Minuten brauchte ---
an das Kriechen in der Höhe war nicht zu denken ---, so bekam er für
diese Verschlimmerung seines Zustandes einen seiner Meinung nach
vollständig genügenden Ersatz dadurch, daß immer gegen Abend die
Wohnzimmertür, die er schon ein bis zwei Stunden vorher scharf zu
beobachten pflegte, geöffnet wurde, so daß er, im Dunkel seines Zimmers
liegend, vom Wohnzimmer aus unsichtbar, die ganze Familie beim
beleuchteten Tische sehen und ihre Reden, gewissermaßen mit allgemeiner
Erlaubnis, also ganz anders als früher, anhören durfte.

Freilich waren es nicht mehr die lebhaften Unterhaltungen der früheren
Zeiten, an die Gregor in den kleinen Hotelzimmern stets mit einigem
Verlangen gedacht hatte, wenn er sich müde in das feuchte Bettzeug hatte
werfen müssen. Es ging jetzt meist nur sehr still zu.

Der Vater schlief
bald nach dem Nachtessen in seinem Sessel ein; die Mutter und Schwester
ermahnten einander zur Stille; die Mutter nähte, weit über das Licht
vorgebeugt, feine Wäsche für ein Modengeschäft; die Schwester, die eine
Stellung als Verkäuferin angenommen hatte, lernte am Abend Stenographie
und Französisch, um vielleicht später einmal einen besseren Posten zu
erreichen. Manchmal wachte der Vater auf, und als wisse er gar nicht,
daß er geschlafen habe, sagte er zur Mutter: »Wie lange du heute schon
wieder nähst!« und schlief sofort wieder ein, während Mutter und
Schwester einander müde zulächelten.

Mit einer Art Eigensinn weigerte sich der Vater, auch zu Hause seine
Dieneruniform abzulegen; und während der Schlafrock nutzlos am
Kleiderhaken hing, schlummerte der Vater vollständig angezogen auf
seinem Platz, als sei er immer zu seinem Dienste bereit und warte auch
hier auf die Stimme des Vorgesetzten. Infolgedessen verlor die gleich
anfangs nicht neue Uniform trotz aller Sorgfalt von Mutter und Schwester
an Reinlichkeit, und Gregor sah oft ganze Abende lang auf dieses über
und über fleckige, mit seinen stets geputzten Goldknöpfen leuchtende
Kleid, in dem der alte Mann höchst unbequem und doch ruhig schlief.

Sobald die Uhr zehn schlug, suchte die Mutter durch leise Zusprache den
Vater zu wecken und dann zu überreden, ins Bett zu gehen, denn hier war
es doch kein richtiger Schlaf und diesen hatte der Vater, der um sechs
Uhr seinen Dienst antreten mußte, äußerst nötig. Aber in dem Eigensinn,
der ihn, seitdem er Diener war, ergriffen hatte, bestand er immer
darauf, noch länger bei Tisch zu bleiben, trotzdem er regelmäßig
einschlief, und war dann überdies nur mit der größten Mühe zu bewegen,
den Sessel mit dem Bett zu vertauschen. Da mochten Mutter und Schwester
mit kleinen Ermahnungen noch so sehr auf ihn eindringen,
viertelstundenlang schüttelte er langsam den Kopf, hielt die Augen
geschlossen und stand nicht auf. Die Mutter zupfte ihn am Ärmel, sagte
ihm Schmeichelworte ins Ohr, die Schwester verließ ihre Aufgabe, um der
Mutter zu helfen, aber beim Vater verfing das nicht. Er versank nur noch
tiefer in seinen Sessel. Erst bis ihn die Frauen unter den Achseln
faßten, schlug er die Augen auf, sah abwechselnd die Mutter und die
Schwester an und pflegte zu sagen: »Das ist ein Leben. Das ist die Ruhe
meiner alten Tage.« Und auf die beiden Frauen gestützt, erhob er sich,
umständlich, als sei er für sich selbst die größte Last, ließ sich von
den Frauen bis zur Türe führen, winkte ihnen dort ab und ging nun
selbständig weiter, während die Mutter ihr Nähzeug, die Schwester ihre
Feder eiligst hinwarfen, um hinter dem Vater zu laufen und ihm weiter
behilflich zu sein.

Wer hatte in dieser abgearbeiteten und übermüdeten Familie Zeit, sich um
Gregor mehr zu kümmern, als unbedingt nötig war? Der Haushalt wurde
immer mehr eingeschränkt; das Dienstmädchen wurde nun doch entlassen;
eine riesige knochige Bedienerin mit weißem, den Kopf umflatterndem Haar
kam des Morgens und des Abends, um die schwerste Arbeit zu leisten;
alles andere besorgte die Mutter neben ihrer vielen Näharbeit. Es
geschah sogar, daß verschiedene Familienschmuckstücke, welche früher die
Mutter und die Schwester überglücklich bei Unterhaltungen und
Feierlichkeiten getragen hatten, verkauft wurden, wie Gregor am Abend
aus der allgemeinen Besprechung der erzielten Preise erfuhr. Die größte
Klage war aber stets, daß man diese für die gegenwärtigen Verhältnisse
allzugroße Wohnung nicht verlassen konnte, da es nicht auszudenken war,
wie man Gregor übersiedeln sollte. Aber Gregor sah wohl ein, daß es
nicht nur die Rücksicht auf ihn war, welche eine Übersiedlung
verhinderte, denn ihn hätte man doch in einer passenden Kiste mit ein
paar Luftlöchern leicht transportieren können; was die Familie
hauptsächlich vom Wohnungswechsel abhielt, war vielmehr die völlige
Hoffnungslosigkeit und der Gedanke daran, daß sie mit einem Unglück
geschlagen war, wie niemand sonst im ganzen Verwandten- und
Bekanntenkreis. Was die Welt von armen Leuten verlangt, erfüllten sie
bis zum äußersten, der Vater holte den kleinen Bankbeamten das
Frühstück, die Mutter opferte sich für die Wäsche fremder Leute, die
Schwester lief nach dem Befehl der Kunden hinter dem Pulte hin und her,
aber weiter reichten die Kräfte der Familie schon nicht. Und die Wunde
im Rücken fing Gregor wie neu zu schmerzen an, wenn Mutter und
Schwester, nachdem sie den Vater zu Bett gebracht hatten, nun
zurückkehrten, die Arbeit liegen ließen, nahe zusammenrückten, schon
Wange an Wange saßen; wenn jetzt die Mutter, auf Gregors Zimmer zeigend,
sagte: »Mach' dort die Tür zu, Grete,« und wenn nun Gregor wieder im
Dunkel war, während nebenan die Frauen ihre Tränen vermischten oder gar
tränenlos den Tisch anstarrten.

Die Nächte und Tage verbrachte Gregor fast ganz ohne Schlaf. Manchmal
dachte er daran, beim nächsten Öffnen der Tür die Angelegenheiten der
Familie ganz so wie früher wieder in die Hand zu nehmen; in seinen
Gedanken erschienen wieder nach langer Zeit der Chef und der Prokurist,
die Kommis und die Lehrjungen, der so begriffsstützige Hausknecht, zwei
drei Freunde aus anderen Geschäften, ein Stubenmädchen aus einem Hotel
in der Provinz, eine liebe, flüchtige Erinnerung, eine Kassiererin aus
einem Hutgeschäft, um die er sich ernsthaft, aber zu langsam beworben
hatte --- sie alle erschienen untermischt mit Fremden oder schon
Vergessenen, aber statt ihm und seiner Familie zu helfen, waren sie
sämtlich unzugänglich, und er war froh, wenn sie verschwanden. Dann\est\ aber
war er wieder gar nicht in der Laune, sich um seine Familie zu sorgen,
bloß Wut über die schlechte Wartung erfüllte ihn, und trotzdem er sich
nichts vorstellen konnte, worauf er Appetit gehabt hätte, machte er doch
Pläne, wie er in die Speisekammer gelangen könnte, um dort zu nehmen,
was ihm, auch wenn er keinen Hunger hatte, immerhin gebührte. Ohne jetzt
mehr nachzudenken, womit man Gregor einen besonderen Gefallen machen
könnte, schob die Schwester eiligst, ehe sie morgens und mittags ins
Geschäft lief, mit dem Fuß irgendeine beliebige Speise in Gregors Zimmer
hinein, um sie am Abend, gleichgültig dagegen, ob die Speise vielleicht
nur gekostet oder --- der häufigste Fall --- gänzlich unberührt war, mit
einem Schwenken des Besens hinauszukehren. Das Aufräumen des Zimmers,
das sie nun immer abends besorgte, konnte gar nicht mehr schneller getan
sein. Schmutzstreifen zogen sich die Wände entlang, hie und da lagen
Knäuel von Staub und Unrat. In der ersten Zeit stellte sich Gregor bei
der Ankunft der Schwester in derartige besonders bezeichnende Winkel, um
ihr durch diese Stellung gewissermaßen einen Vorwurf zu machen. Aber er
hätte wohl wochenlang dort bleiben können, ohne daß sich die Schwester
gebessert hätte; sie sah ja den Schmutz genau so wie er, aber sie hatte
sich eben entschlossen, ihn zu lassen. Dabei wachte sie mit einer an ihr
ganz neuen Empfindlichkeit, die überhaupt die ganze Familie ergriffen
hatte, darüber, daß das Aufräumen von Gregors Zimmer ihr vorbehalten
blieb. Einmal hatte die Mutter Gregors Zimmer einer großen Reinigung
unterzogen, die ihr nur nach Verbrauch einiger Kübel Wasser gelungen war
--- die viele Feuchtigkeit kränkte allerdings Gregor auch und er lag
breit, verbittert und unbeweglich auf dem Kanapee ---, aber die Strafe
blieb für die Mutter nicht aus. Denn kaum hatte am Abend die Schwester
die Veränderung in Gregors Zimmer bemerkt, als sie, aufs höchste
beleidigt, ins Wohnzimmer lief und, trotz der beschwörend erhobenen
Hände der Mutter, in einen Weinkrampf ausbrach, dem die Eltern --- der
Vater war natürlich aus seinem Sessel aufgeschreckt worden --- zuerst
erstaunt und hilflos zusahen; bis auch sie sich zu rühren anfingen; der
Vater rechts der Mutter Vorwürfe machte, daß sie Gregors Zimmer nicht
der Schwester zur Reinigung überließ; links dagegen die Schwester
anschrie, sie werde niemals mehr Gregors Zimmer reinigen dürfen; während
die Mutter den Vater, der sich vor Erregung nicht mehr kannte, ins
Schlafzimmer zu schleppen suchte; die Schwester, von Schluchzen
geschüttelt, mit ihren kleinen Fäusten den Tisch bearbeitete; und Gregor
laut vor Wut darüber zischte, daß es keinem einfiel, die Tür zu
schließen und ihm diesen Anblick und Lärm zu ersparen.

Aber selbst wenn die Schwester, erschöpft von ihrer Berufsarbeit, dessen
überdrüssig geworden war, für Gregor, wie früher, zu sorgen, so hätte
noch keineswegs die Mutter für sie eintreten müssen und Gregor hätte
doch nicht vernachlässigt zu werden brauchen. Denn nun war die
Bedienerin da. Diese alte Witwe, die in ihrem langen Leben mit Hilfe
ihres starken Knochenbaues das Ärgste überstanden haben mochte, hatte
keinen eigentlichen Abscheu vor Gregor. Ohne irgendwie neugierig zu
sein, hatte sie zufällig einmal die Tür von Gregors Zimmer aufgemacht
und war im Anblick Gregors, der, gänzlich überrascht, trotzdem ihn
niemand jagte, hin- und herzulaufen begann, die Hände im Schoß gefaltet
staunend stehen geblieben. Seitdem versäumte sie nicht, stets flüchtig
morgens und abends die Tür ein wenig zu öffnen und zu Gregor
hineinzuschauen. Anfangs rief sie ihn auch zu sich herbei, mit Worten,
die sie wahrscheinlich für freundlich hielt, wie »Komm mal herüber,
alter Mistkäfer!« oder »Seht mal den alten Mistkäfer!« Auf solche
Ansprachen antwortete Gregor mit nichts, sondern blieb unbeweglich auf
seinem Platz, als sei die Tür gar nicht geöffnet worden. Hätte man doch
dieser Bedienerin, statt sie nach ihrer Laune ihn nutzlos stören zu
lassen, lieber den Befehl gegeben, sein Zimmer täglich zu reinigen!
Einmal am frühen Morgen --- ein heftiger Regen, vielleicht schon ein
Zeichen des kommenden Frühjahrs, schlug an die Scheiben --- war Gregor,
als die Bedienerin mit ihren Redensarten wieder begann, derartig
erbittert, daß er, wie zum Angriff, allerdings langsam und hinfällig,
sich gegen sie wendete. Die Bedienerin aber, statt sich zu fürchten, hob
bloß einen in der Nähe der Tür befindlichen Stuhl hoch empor, und wie
sie mit groß geöffnetem Munde dastand, war ihre Absicht klar, den Mund
erst zu schließen, wenn der Sessel in ihrer Hand auf Gregors Rücken
niederschlagen würde. »Also weiter geht es nicht?« fragte sie, als
Gregor sich wieder umdrehte, und stellte den Sessel ruhig in die Ecke
zurück.

Gregor aß nun fast gar nichts mehr. Nur wenn er zufällig an der
vorbereiteten Speise vorüberkam, nahm er zum Spiel einen Bissen in den
Mund, hielt ihn dort stundenlang und spie ihn dann meist wieder aus.
Zuerst dachte er, es sei die Trauer über den Zustand seines Zimmers, die
ihn vom Essen abhalte, aber gerade mit den Veränderungen des Zimmers
söhnte er sich sehr bald aus. Man hatte sich angewöhnt, Dinge, die man
anderswo nicht unterbringen konnte, in dieses Zimmer hineinzustellen,
und solcher Dinge gab es nun viele, da man ein Zimmer der Wohnung an
drei Zimmerherren vermietet hatte. Diese ernsten Herren, --- alle drei
hatten Vollbärte, wie Gregor einmal durch eine Türspalte feststellte ---
waren peinlich auf Ordnung, nicht nur in ihrem Zimmer, sondern, da sie
sich nun einmal hier eingemietet hatten, in der ganzen Wirtschaft, also
insbesondere in der Küche, bedacht. Unnützen oder gar schmutzigen Kram
ertrugen sie nicht. Überdies hatten sie zum größten Teil ihre eigenen
Einrichtungsstücke mitgebracht. Aus diesem Grunde waren viele Dinge
überflüssig geworden, die zwar nicht verkäuflich waren, die man aber
auch nicht wegwerfen wollte. Alle diese wanderten in Gregors Zimmer.
Ebenso auch die Aschenkiste und die Abfallkiste aus der Küche. Was nur
im Augenblick unbrauchbar war, schleuderte die Bedienerin, die es immer
sehr eilig hatte, einfach in Gregors Zimmer; Gregor sah\est\ glücklicherweise
meist nur den betreffenden Gegenstand und die Hand, die ihn hielt. Die
Bedienerin hatte vielleicht die Absicht, bei Zeit und Gelegenheit die
Dinge wieder zu holen oder alle insgesamt mit einemmal hinauszuwerfen,
tatsächlich aber blieben sie dort liegen, wohin sie durch den ersten
Wurf gekommen waren, wenn nicht Gregor sich durch das Rumpelzeug wand
und es in Bewegung brachte, zuerst gezwungen, weil kein sonstiger Platz
zum Kriechen frei war, später aber mit wachsendem Vergnügen, obwohl er
nach solchen Wanderungen, zum Sterben müde und traurig, wieder
stundenlang sich nicht rührte.

Da die Zimmerherren manchmal auch ihr Abendessen zu Hause im gemeinsamen
Wohnzimmer einnahmen, blieb die Wohnzimmertür an manchen Abenden
geschlossen, aber Gregor verzichtete ganz leicht auf das Öffnen der Tür,
hatte er doch schon manche Abende, an denen sie geöffnet war, nicht
ausgenützt, sondern war, ohne daß es die Familie merkte, im dunkelsten
Winkel seines Zimmers gelegen. Einmal aber hatte die Bedienerin die Tür
zum Wohnzimmer ein wenig offen gelassen, und sie blieb so offen, auch
als die Zimmerherren am Abend eintraten und Licht gemacht wurde. Sie
setzten sich oben an den Tisch, wo in früheren Zeiten der Vater, die
Mutter und Gregor gesessen hatten, entfalteten die Servietten und nahmen
Messer und Gabel in die Hand. Sofort erschien in der Tür die Mutter mit
einer Schüssel Fleisch und knapp hinter ihr die Schwester mit einer
Schüssel hochgeschichteter Kartoffeln. Das Essen dampfte mit starkem
Rauch. Die Zimmerherren beugten sich über die vor sie hingestellten
Schüsseln, als wollten sie sie vor dem Essen prüfen, und tatsächlich
zerschnitt der, welcher in der Mitte saß und den anderen zwei als
Autorität zu gelten schien, ein Stück Fleisch noch auf der Schüssel,
offenbar um festzustellen, ob es mürbe genug sei und ob es nicht etwa in
die Küche zurückgeschickt werden solle. Er war befriedigt, und Mutter
und Schwester, die gespannt zugesehen hatten, begannen aufatmend zu
lächeln.

Die Familie selbst aß in der Küche. Trotzdem kam der Vater, ehe er in
die Küche ging, in dieses Zimmer herein und machte mit einer einzigen
Verbeugung, die Kappe in der Hand, einen Rundgang um den Tisch. Die
Zimmerherren erhoben sich sämtlich und murmelten etwas in ihre Bärte.
Als sie dann allein waren, aßen sie fast unter vollkommenem
Stillschweigen. Sonderbar schien es Gregor, daß man aus allen
mannigfachen Geräuschen des Essens immer wieder ihre kauenden Zähne
heraushörte, als ob damit Gregor gezeigt werden sollte, daß man Zähne
brauche, um zu essen, und daß man auch mit den schönsten zahnlosen
Kiefern nichts ausrichten könne. »Ich habe ja Appetit,« sagte sich
Gregor sorgenvoll, »aber nicht auf diese Dinge. Wie sich diese
Zimmerherren nähren, und ich komme um!«

Gerade an diesem Abend --- Gregor erinnerte sich nicht, während der
ganzen Zeit die Violine gehört zu haben --- ertönte sie von der Küche
her. Die Zimmerherren hatten schon ihr Nachtmahl beendet, der mittlere
hatte eine Zeitung hervorgezogen, den zwei anderen je ein Blatt gegeben,
und nun lasen sie zurückgelehnt und rauchten. Als die Violine zu spielen
begann, wurden sie aufmerksam, erhoben sich und gingen auf den
Fußspitzen zur Vorzimmertür, in der sie aneinandergedrängt stehen
blieben. Man mußte sie von der Küche aus gehört haben, denn der Vater
rief: »Ist den Herren das Spiel vielleicht unangenehm? Es kann sofort
eingestellt werden.« »Im Gegenteil,« sagte der mittlere der Herren,
»möchte das Fräulein nicht zu uns hereinkommen und hier im Zimmer
spielen, wo es doch viel bequemer und gemütlicher ist?« »O bitte,« rief
der Vater, als sei er der Violinspieler. Die Herren traten ins Zimmer
zurück und warteten. Bald kam der Vater mit dem Notenpult, die Mutter
mit den Noten und die Schwester mit der Violine. Die Schwester bereitete
alles ruhig zum Spiele vor; die Eltern, die niemals früher Zimmer
vermietet hatten und deshalb die Höflichkeit gegen die Zimmerherren
übertrieben, wagten gar nicht, sich auf ihre eigenen Sessel zu setzen;
der Vater lehnte an der Tür, die rechte Hand zwischen zwei Knöpfe des
geschlossenen Livreerockes gesteckt; die Mutter aber erhielt von einem
Herrn einen Sessel angeboten und saß, da sie den Sessel dort ließ, wohin
ihn der Herr zufällig gestellt hatte, abseits in einem Winkel.

Die Schwester begann zu spielen; Vater und Mutter verfolgten, jeder von
seiner Seite, aufmerksam die Bewegungen ihrer Hände. Gregor hatte, von
dem Spiele angezogen, sich ein wenig weiter vorgewagt und war schon mit
dem Kopf im Wohnzimmer. Er wunderte sich kaum darüber, daß er in letzter
Zeit so wenig Rücksicht auf die andern nahm; früher war diese
Rücksichtnahme sein Stolz gewesen. Und dabei hätte er gerade jetzt mehr
Grund gehabt, sich zu verstecken, denn infolge des Staubes, der in
seinem Zimmer überall lag und bei der kleinsten Bewegung umherflog, war
auch er ganz staubbedeckt; Fäden, Haare, Speiseüberreste schleppte er
auf seinem Rücken und an den Seiten mit sich herum; seine
Gleichgültigkeit gegen alles war viel zu groß, als daß er sich, wie
früher mehrmals während des Tages, auf den Rücken gelegt und am Teppich
gescheuert hätte. Und trotz dieses Zustandes hatte er keine Scheu, ein
Stück auf dem makellosen Fußboden des Wohnzimmers vorzurücken.

Allerdings achtete auch niemand auf ihn. Die Familie war gänzlich vom
Violinspiel in Anspruch genommen; die Zimmerherren dagegen, die
zunächst, die Hände in den Hosentaschen, viel zu nahe hinter dem
Notenpult der Schwester sich aufgestellt hatten, so daß sie alle in die
Noten hätte sehen können, was sicher die Schwester stören mußte, zogen
sich bald unter halblauten Gesprächen mit gesenkten Köpfen zum Fenster
zurück, wo sie, vom Vater besorgt beobachtet, auch blieben. Es hatte nun
wirklich den überdeutlichen Anschein, als wären sie in ihrer Annahme,
ein schönes oder unterhaltendes Violinspiel zu hören, enttäuscht, hätten
die ganze Vorführung satt und ließen sich nur aus Höflichkeit noch in
ihrer Ruhe stören. Besonders die Art, wie sie alle aus Nase und Mund den
Rauch ihrer Zigarren in die Höhe bliesen, ließ auf große Nervosität
schließen. Und doch spielte die Schwester so schön. Ihr Gesicht war zur
Seite geneigt, prüfend und traurig folgten ihre Blicke den Notenzeilen.
Gregor kroch noch ein Stück vorwärts und hielt den Kopf eng an den
Boden, um möglicherweise ihren Blicken begegnen zu können. War er ein
Tier, da ihn Musik so ergriff? Ihm war, als zeige sich ihm der Weg zu
der ersehnten unbekannten Nahrung. Er war entschlossen, bis zur
Schwester vorzudringen, sie am Rock zu zupfen und ihr dadurch
anzudeuten, sie möge doch mit ihrer Violine in sein Zimmer kommen, denn
niemand lohnte hier das Spiel so, wie er es lohnen wollte. Er wollte sie
nicht mehr aus seinem Zimmer lassen, wenigstens nicht, solange er lebte;
seine Schreckgestalt sollte ihm zum erstenmal nützlich werden; an allen
Türen seines Zimmers wollte er gleichzeitig sein und den Angreifern
entgegenfauchen; die Schwester aber sollte nicht gezwungen, sondern
freiwillig bei ihm bleiben; sie sollte neben ihm auf dem Kanapee sitzen,
das Ohr zu ihm herunterneigen, und er wollte ihr dann anvertrauen, daß
er die feste Absicht gehabt habe, sie auf das Konservatorium zu
schicken, und daß er dies, wenn nicht das Unglück dazwischen gekommen
wäre, vergangene Weihnachten --- Weihnachten war doch wohl schon vorüber?
--- allen gesagt hätte, ohne sich um irgendwelche Widerreden zu kümmern.
Nach dieser Erklärung würde die Schwester in Tränen der Rührung
ausbrechen, und Gregor würde sich bis zu ihrer Achsel erheben und ihren
Hals küssen, den sie, seitdem sie ins Geschäft ging, frei ohne Band oder
Kragen trug.

»Herr Samsa!« rief der mittlere Herr dem Vater zu und zeigte, ohne ein
weiteres Wort zu verlieren, mit dem Zeigefinger auf den langsam sich
vorwärtsbewegenden Gregor. Die Violine verstummte, der mittlere
Zimmerherr lächelte erst einmal kopfschüttelnd seinen Freunden zu und
sah dann wieder auf Gregor hin. Der Vater schien es für nötiger zu
halten, statt Gregor zu vertreiben, vorerst die Zimmerherren zu
beruhigen, trotzdem diese gar nicht aufgeregt waren und Gregor sie mehr
als das Violinspiel zu unterhalten schien. Er eilte zu ihnen und suchte
sie mit ausgebreiteten Armen in ihr Zimmer zu drängen und gleichzeitig
mit seinem Körper ihnen den Ausblick auf Gregor zu nehmen. Sie wurden
nun tatsächlich ein wenig böse, man wußte nicht mehr, ob über das
Benehmen des Vaters oder über die ihnen jetzt aufgehende Erkenntnis,
ohne es zu wissen, einen solchen Zimmernachbar wie Gregor besessen zu
haben. Sie verlangten vom Vater Erklärungen, hoben ihrerseits die Arme,
zupften unruhig an ihren Bärten und wichen nur langsam gegen ihr Zimmer
zurück. Inzwischen hatte die Schwester die Verlorenheit, in die sie nach
dem plötzlich abgebrochenen Spiel verfallen war, überwunden, hatte sich,
nachdem sie eine Zeitlang in den lässig hängenden Händen Violine und
Bogen gehalten und weiter, als spiele sie noch, in die Noten gesehen
hatte, mit einem Male aufgerafft, hatte das Instrument auf den Schoß der
Mutter gelegt, die in Atembeschwerden mit heftig arbeitenden Lungen noch
auf ihrem Sessel saß, und war in das Nebenzimmer gelaufen, dem sich die
Zimmerherren unter dem Drängen des Vaters schon schneller näherten. Man
sah, wie unter den geübten Händen der Schwester die Decken und Polster
in den Betten in die Höhe flogen und sich ordneten. Noch ehe die Herren
das Zimmer erreicht hatten, war sie mit dem Aufbetten fertig und
schlüpfte heraus. Der Vater schien wieder von seinem Eigensinn derartig
ergriffen, daß er jeden Respekt vergaß, den er seinen Mietern immerhin
schuldete. Er drängte nur und drängte, bis schon in der Tür des Zimmers
der mittlere der Herren donnernd mit dem Fuß aufstampfte und dadurch den
Vater zum Stehen brachte. »Ich erkläre hiermit,« sagte er, hob die Hand
und suchte mit den Blicken auch die Mutter und die Schwester, »daß ich
mit Rücksicht auf die in dieser Wohnung und Familie herrschenden
widerlichen Verhältnisse« --- hierbei spie er kurz entschlossen auf den
Boden --- »mein Zimmer augenblicklich kündige. Ich werde natürlich auch
für die Tage, die ich hier gewohnt habe, nicht das Geringste bezahlen,
dagegen werde ich es mir noch überlegen, ob ich nicht mit irgendwelchen
--- glauben Sie mir --- sehr leicht zu begründenden Forderungen gegen Sie
auftreten werde.« Er schwieg und sah gerade vor sich hin, als erwarte er
etwas. Tatsächlich fielen sofort seine zwei Freunde mit den Worten ein:
»Auch wir kündigen augenblicklich.« Darauf faßte er die Türklinke und
schloß mit einem Krach die Tür.

Der Vater wankte mit tastenden Händen zu seinem Sessel und ließ sich
hineinfallen; es sah aus, als strecke er sich zu seinem gewöhnlichen
Abendschläfchen, aber das starke Nicken seines wie haltlosen Kopfes
zeigte, daß er ganz und gar nicht schlief. Gregor war die ganze Zeit
still auf dem Platz gelegen, auf dem ihn die Zimmerherren ertappt
hatten. Die Enttäuschung über das Mißlingen seines Planes, vielleicht
aber auch die durch das viele Hungern verursachte Schwäche machten es
ihm unmöglich, sich zu bewegen. Er fürchtete mit einer gewissen
Bestimmtheit schon für den nächsten Augenblick einen allgemeinen über
ihn sich entladenden Zusammensturz und wartete. Nicht einmal die Violine
schreckte ihn auf, die, unter den zitternden Fingern der Mutter hervor,
ihr vom Schoße fiel und einen hallenden Ton von sich gab.

»Liebe Eltern,« sagte die Schwester und schlug zur Einleitung mit der
Hand auf den Tisch, »so geht es nicht weiter. Wenn ihr das vielleicht
nicht einsehet, ich sehe es ein. Ich will vor diesem Untier nicht den
Namen meines Bruders aussprechen und sage daher bloß: wir müssen
versuchen es loszuwerden. Wir haben das Menschenmögliche versucht, es zu
pflegen und zu dulden, ich glaube, es kann uns niemand den geringsten
Vorwurf machen.«

»Sie hat tausendmal recht,« sagte der Vater für sich. Die Mutter, die
noch immer nicht genug Atem finden konnte, fing mit einem irrsinnigen
Ausdruck der Augen dumpf in die vorgehaltene Hand zu husten an.

Die Schwester eilte zur Mutter und hielt ihr die Stirn. Der Vater schien
durch die Worte der Schwester auf bestimmtere Gedanken gebracht zu sein,
hatte sich aufrecht gesetzt, spielte mit seiner Dienermütze zwischen den
Tellern, die noch vom Nachtmahl der Zimmerherren her auf dem Tische
standen, und sah bisweilen auf den stillen Gregor hin.

»Wir müssen es loszuwerden suchen,« sagte die Schwester nun
ausschließlich zum Vater, denn die Mutter hörte in ihrem Husten nichts,
»es bringt euch noch beide um, ich sehe es kommen. Wenn man schon so
schwer arbeiten muß, wie wir alle, kann man nicht noch zu Hause diese
ewige Quälerei ertragen. Ich kann es auch nicht mehr.« Und sie brach so
heftig in Weinen aus, daß ihre Tränen auf das Gesicht der Mutter
niederflossen, von dem sie sie mit mechanischen Handbewegungen wischte.

»Kind,« sagte der Vater mitleidig und mit auffallendem Verständnis, »was
sollen wir aber tun?«

Die Schwester zuckte nur die Achseln zum Zeichen der Ratlosigkeit, die
sie nun während des Weinens im Gegensatz zu ihrer früheren Sicherheit
ergriffen hatte.

»Wenn er uns verstünde,« sagte der Vater halb fragend; die Schwester
schüttelte aus dem Weinen heraus heftig die Hand zum Zeichen, daß daran
nicht zu denken sei.

»Wenn er uns verstünde,« wiederholte der Vater und nahm durch Schließen
der Augen die Überzeugung der Schwester von der Unmöglichkeit dessen in
sich auf, »dann wäre vielleicht ein Übereinkommen mit ihm möglich. Aber
so ---«

»Weg muß es,« rief die Schwester, »das ist das einzige Mittel, Vater. Du
mußt bloß den Gedanken loszuwerden suchen, daß es Gregor ist. Daß wir es
so lange geglaubt haben, das ist ja unser eigentliches Unglück. Aber wie
kann es denn Gregor sein? Wenn es Gregor wäre, er hätte längst
eingesehen, daß ein Zusammenleben von Menschen mit einem solchen Tier
nicht möglich ist, und wäre freiwillig fortgegangen. Wir hätten dann
keinen Bruder, aber könnten weiter leben und sein Andenken in Ehren
halten. So aber verfolgt uns dieses Tier, vertreibt die Zimmerherren,
will offenbar die ganze Wohnung einnehmen und uns auf der Gasse
übernachten lassen. Sieh nur, Vater,« schrie sie plötzlich auf, »er
fängt schon wieder an!« Und in einem für Gregor gänzlich
unverständlichen Schrecken verließ die Schwester sogar die Mutter, stieß
sich förmlich von ihrem Sessel ab, als wollte sie lieber die Mutter
opfern, als in Gregors Nähe bleiben, und eilte hinter den Vater, der,
lediglich durch ihr Benehmen erregt, auch aufstand und die Arme wie zum
Schutze der Schwester vor ihr halb erhob.

Aber Gregor fiel es doch gar nicht ein, irgend jemandem und gar seiner
Schwester Angst machen zu wollen. Er hatte bloß angefangen sich
umzudrehen, um in sein Zimmer zurückzuwandern, und das nahm sich
allerdings auffallend aus, da er infolge seines leidenden Zustandes bei
den schwierigen Umdrehungen mit seinem Kopfe nachhelfen mußte, den er
hierbei viele Male hob und gegen den Boden schlug. Er hielt inne und sah
sich um. Seine gute Absicht schien erkannt worden zu sein; es war nur
ein augenblicklicher Schrecken gewesen. Nun sahen ihn alle schweigend
und traurig an. Die Mutter lag, die Beine ausgestreckt und
aneinandergedrückt, in ihrem Sessel, die Augen fielen ihr vor Ermattung
fast zu; der Vater und die Schwester saßen nebeneinander, die Schwester
hatte ihre Hand um des Vaters Hals gelegt.

»Nun darf ich mich schon vielleicht umdrehen,« dachte Gregor und begann
seine Arbeit wieder. Er konnte das Schnaufen der Anstrengung nicht
unterdrücken und mußte auch hie und da ausruhen. Im übrigen drängte ihn
auch niemand, es war alles ihm selbst überlassen. Als er die Umdrehung
vollendet hatte, fing er sofort an, geradeaus zurückzuwandern. Er
staunte über die große Entfernung, die ihn von seinem Zimmer trennte,
und begriff gar nicht, wie er bei seiner Schwäche vor kurzer Zeit den
gleichen Weg, fast ohne es zu merken, zurückgelegt hatte. Immerfort nur
auf rasches Kriechen bedacht, achtete er kaum darauf, daß kein Wort,
kein Ausruf seiner Familie ihn störte. Erst als er schon in der Tür war,
wendete er den Kopf, nicht, vollständig, denn er fühlte den Hals steif
werden, immerhin sah er noch, daß sich hinter ihm nichts verändert
hatte, nur die Schwester war aufgestanden. Sein letzter Blick streifte
die Mutter, die nun völlig eingeschlafen war.

Kaum war er innerhalb seines Zimmers, wurde die Tür eiligst zugedrückt,
festgeriegelt und versperrt. Über den plötzlichen Lärm hinter sich
erschrak Gregor so, daß ihm die Beinchen einknickten. Es war die
Schwester, die sich so beeilt hatte. Aufrecht war sie schon da
gestanden und hatte gewartet, leichtfüßig war sie dann
vorwärtsgesprungen, Gregor hatte sie gar nicht kommen hören, und ein
»Endlich!« rief sie den Eltern zu, während sie den Schlüssel im Schloß
umdrehte.

»Und jetzt?« fragte sich Gregor und sah sich im Dunkeln um. Er machte
bald die Entdeckung, daß er sich nun überhaupt nicht mehr rühren konnte.
Er wunderte sich darüber nicht, eher kam es ihm unnatürlich vor, daß er
sich bis jetzt tatsächlich mit diesen dünnen Beinchen hatte fortbewegen
können. Im übrigen fühlte er sich verhältnismäßig behaglich. Er hatte
zwar Schmerzen im ganzen Leib, aber ihm war, als würden sie allmählich
schwächer und schwächer und würden schließlich ganz vergehen. Den
verfaulten Apfel in seinem Rücken und die entzündete Umgebung, die ganz
von weichem Staub bedeckt war, spürte er schon kaum. An seine Familie
dachte er mit Rührung und Liebe zurück. Seine Meinung darüber, daß er
verschwinden müsse, war womöglich noch entschiedener, als die seiner
Schwester. In diesem Zustand leeren und friedlichen Nachdenkens blieb
er, bis die Turmuhr die dritte Morgenstunde schlug. Den Anfang des
allgemeinen Hellerwerdens draußen vor dem Fenster erlebte er noch. Dann
sank sein Kopf ohne seinen Willen gänzlich nieder, und aus seinen
Nüstern strömte sein letzter Atem schwach hervor.

Als am frühen Morgen die Bedienerin kam --- vor lauter Kraft und Eile
schlug sie, wie oft man sie auch schon gebeten hatte, das zu vermeiden,
alle Türen derartig zu, daß in der ganzen Wohnung von ihrem Kommen an
kein ruhiger Schlaf mehr möglich war ---, fand sie bei ihrem gewöhnlichen
kurzen Besuch bei Gregor zuerst nichts Besonderes. Sie dachte, er liege
absichtlich so unbeweglich da und spiele den Beleidigten; sie traute
ihm allen möglichen Verstand zu. Weil sie zufällig den langen Besen in
der Hand hielt, suchte sie mit ihm Gregor von der Tür aus zu kitzeln.
Als sich auch da kein Erfolg zeigte, wurde sie ärgerlich und stieß ein
wenig in Gregor hinein, und erst als sie ihn ohne jeden Widerstand von
seinem Platze geschoben hatte, wurde sie aufmerksam. Als sie bald den
wahren Sachverhalt erkannte, machte sie große Augen, pfiff vor sich hin,
hielt sich aber nicht lange auf, sondern riß die Tür des Schlafzimmers
auf und rief mit lauter Stimme in das Dunkel hinein: »Sehen Sie nur mal
an, es ist krepiert; da liegt es, ganz und gar krepiert!«

Das Ehepaar Samsa saß im Ehebett aufrecht da und hatte zu tun, den
Schrecken über die Bedienerin zu verwinden, ehe es dazu kam, ihre
Meldung aufzufassen. Dann aber stiegen Herr und Frau Samsa, jeder auf
seiner Seite, eiligst aus dem Bett, Herr Samsa warf die Decke über seine
Schultern, Frau Samsa kam nur im Nachthemd hervor; so traten sie in
Gregors Zimmer. Inzwischen hatte sich\est\ auch die Tür des Wohnzimmers
geöffnet, in dem Grete seit dem Einzug der Zimmerherren schlief; sie war
völlig angezogen, als hätte sie gar nicht geschlafen, auch ihr bleiches
Gesicht schien das zu beweisen. »Tot?« sagte Frau Samsa und sah fragend
zur Bedienerin auf, trotzdem sie doch alles selbst prüfen und sogar ohne
Prüfung erkennen konnte. »Das will ich meinen,« sagte die Bedienerin und
stieß zum Beweis Gregors Leiche mit dem Besen noch ein großes Stück
seitwärts. Frau Samsa machte eine Bewegung, als wolle sie den Besen
zurückhalten, tat es aber nicht. »Nun,« sagte Herr Samsa, »jetzt können
wir Gott danken.« Er bekreuzte sich, und die drei Frauen folgten seinem
Beispiel. Grete, die kein Auge von der Leiche wendete, sagte: »Seht
nur, wie mager er war. Er hat ja auch schon so lange Zeit nichts
gegessen. So wie die Speisen hereinkamen, sind sie wieder
hinausgekommen.« Tatsächlich war Gregors Körper vollständig flach und
trocken, man erkannte das eigentlich erst jetzt, da er nicht mehr von
den Beinchen gehoben war und auch sonst nichts den Blick ablenkte.

»Komm, Grete, auf ein Weilchen zu uns herein,« sagte Frau Samsa mit
einem wehmütigen Lächeln, und Grete ging, nicht ohne nach der Leiche
zurückzusehen, hinter den Eltern in das Schlafzimmer. Die Bedienerin
schloß die Tür und öffnete gänzlich das Fenster. Trotz des frühen
Morgens war der frischen Luft schon etwas Lauigkeit beigemischt. Es war
eben schon Ende März.

Aus ihrem Zimmer traten die drei Zimmerherren und sahen sich erstaunt
nach ihrem Frühstück um; man hatte sie vergessen. »Wo ist das
Frühstück?« fragte der mittlere der Herren mürrisch die Bedienerin.
Diese aber legte den Finger an den Mund und winkte dann hastig und
schweigend den Herren zu, sie möchten in Gregors Zimmer kommen. Sie
kamen auch und standen dann, die Hände in den Taschen ihrer etwas
abgenützten Röckchen, in dem nun schon ganz hellen Zimmer um Gregors
Leiche herum.

Da öffnete sich die Tür des Schlafzimmers, und Herr Samsa erschien in
seiner Livree, an einem Arm seine Frau, am anderen seine Tochter. Alle
waren ein wenig verweint; Grete drückte bisweilen ihr Gesicht an den Arm
des Vaters.

»Verlassen Sie sofort meine Wohnung!« sagte Herr Samsa und zeigte auf
die Tür, ohne die Frauen von sich zu lassen. »Wie meinen Sie das?« sagte
der mittlere der Herren etwas bestürzt und lächelte süßlich. Die zwei
anderen hielten die Hände auf dem Rücken und rieben sie ununterbrochen
aneinander, wie in freudiger Erwartung eines großen Streites, der aber
für sie günstig ausfallen mußte. »Ich meine es genau so, wie ich es
sage,« antwortete Herr Samsa und ging in einer Linie mit seinen zwei
Begleiterinnen auf den Zimmerherrn zu. Dieser stand zuerst still da und
sah zu Boden, als ob sich die Dinge in seinem Kopf zu einer neuen
Ordnung zusammenstellten. »Dann gehen wir also,« sagte er dann und sah
zu Herrn Samsa auf,\est\ als verlange er in einer plötzlich ihn überkommenden
Demut sogar für diesen Entschluß eine neue Genehmigung. Herr Samsa
nickte ihm bloß mehrmals kurz mit großen Augen zu. Daraufhin ging der
Herr tatsächlich sofort mit langen Schritten ins Vorzimmer; seine beiden
Freunde hatten schon ein Weilchen lang mit ganz ruhigen Händen
aufgehorcht und hüpften ihm jetzt geradezu nach, wie in Angst, Herr
Samsa könnte vor ihnen ins Vorzimmer eintreten und die Verbindung mit
ihrem Führer stören. Im Vorzimmer nahmen alle drei die Hüte vom
Kleiderrechen, zogen ihre Stöcke aus dem Stockbehälter, verbeugten sich
stumm und verließen die Wohnung. In einem, wie sich zeigte, gänzlich
unbegründeten Mißtrauen trat Herr Samsa mit den zwei Frauen auf den
Vorplatz hinaus; an das Geländer gelehnt, sahen sie zu, wie die drei
Herren zwar langsam, aber ständig die lange Treppe hinunterstiegen, in
jedem Stockwerk in einer bestimmten Biegung des Treppenhauses
verschwanden und nach ein paar Augenblicken wieder hervorkamen; je
tiefer sie gelangten, desto mehr verlor sich das Interesse der Familie
Samsa für sie, und als ihnen entgegen und dann hoch über sie hinweg ein
Fleischergeselle mit der Trage auf dem Kopf in stolzer Haltung
heraufstieg, verließ bald Herr Samsa mit den Frauen das Geländer, und
alle kehrten, wie erleichtert, in ihre Wohnung zurück.

\pagebreak

Sie beschlossen, den heutigen Tag zum Ausruhen und Spazierengehen zu
verwenden; sie hatten diese Arbeitsunterbrechung nicht nur verdient, sie
brauchten sie sogar unbedingt. Und so setzten sie sich zum Tisch und
schrieben drei Entschuldigungsbriefe, Herr Samsa an seine Direktion,
Frau Samsa an ihren Auftraggeber, und Grete an ihren Prinzipal. Während
des Schreibens kam die Bedienerin herein, um zu sagen, daß sie fortgehe,
denn ihre Morgenarbeit war beendet. Die drei Schreibenden nickten zuerst
bloß, ohne aufzuschauen, erst als die Bedienerin sich immer noch nicht
entfernen wollte, sah man ärgerlich auf. »Nun?« fragte Herr Samsa. Die
Bedienerin stand lächelnd in der Tür, als habe sie der Familie ein
großes Glück zu melden, werde es aber nur dann tun, wenn sie gründlich
ausgefragt werde. Die fast aufrechte kleine Straußfeder auf ihrem Hut,
über die sich Herr Samsa schon während ihrer ganzen Dienstzeit ärgerte,
schwankte leicht nach allen Richtungen. »Also was wollen Sie
eigentlich?« fragte Frau Samsa, vor welcher die Bedienerin noch am
meisten Respekt hatte. »Ja,« antwortete die Bedienerin und konnte vor
freundlichem Lachen nicht gleich weiter reden, »also darüber, wie das
Zeug von nebenan weggeschafft werden soll, müssen Sie sich keine Sorge
machen. Es ist schon in Ordnung.« Frau Samsa und Grete beugten sich zu
ihren Briefen nieder, als wollten sie weiterschreiben; Herr Samsa,
welcher merkte, daß die Bedienerin nun alles ausführlich zu beschreiben
anfangen wollte, wehrte dies mit ausgestreckter Hand entschieden ab. Da
sie aber nicht erzählen durfte, erinnerte sie sich an die große Eile,
die sie hatte, rief offenbar beleidigt: »Adjes allseits,« drehte sich
wild um und verließ unter fürchterlichem Türezuschlagen die Wohnung.

»Abends wird sie entlassen,« sagte Herr Samsa, bekam aber weder von
seiner Frau noch von seiner Tochter eine Antwort, denn die Bedienerin
schien ihre kaum gewonnene Ruhe wieder gestört zu haben. Sie erhoben
sich, gingen zum Fenster und blieben dort, sich umschlungen haltend.
Herr Samsa drehte sich in seinem Sessel nach ihnen um und beobachtete
sie still ein Weilchen. Dann rief er: »Also kommt doch her. Laßt schon
endlich die alten Sachen. Und nehmt auch ein wenig Rücksicht auf mich.«
Gleich folgten ihm die Frauen, eilten zu ihm, liebkosten ihn und
beendeten rasch ihre Briefe.

Dann verließen alle drei gemeinschaftlich die Wohnung, was sie schon
seit Monaten nicht getan hatten, und fuhren mit der Elektrischen ins
Freie vor die Stadt. Der Wagen, in dem sie allein saßen, war ganz von
warmer Sonne durchschienen. Sie besprachen, bequem auf ihren Sitzen
zurückgelehnt, die Aussichten für die Zukunft, und es fand sich, daß
diese bei näherer Betrachtung durchaus nicht schlecht waren, denn aller
drei Anstellungen waren, worüber sie einander eigentlich noch gar nicht
ausgefragt hatten, überaus günstig und besonders für später
vielversprechend. Die größte augenblickliche Besserung der Lage mußte
sich natürlich leicht durch einen Wohnungswechsel ergeben; sie\est\ wollten
nun eine kleinere und billigere, aber besser gelegene und überhaupt
praktischere Wohnung nehmen, als es die jetzige, noch von Gregor
ausgesuchte war. Während sie sich so unterhielten, fiel es Herrn und
Frau Samsa im Anblick ihrer immer lebhafter werdenden Tochter fast
gleichzeitig ein, wie sie in der letzten Zeit trotz aller Pflege, die
ihre Wangen bleich gemacht hatte, zu einem schönen und üppigen Mädchen
aufgeblüht war. Stiller werdend und fast unbewußt durch Blicke sich
verständigend, dachten sie daran, daß es nun Zeit sein werde, auch einen
braven Mann für sie zu suchen. Und es war ihnen wie eine Bestätigung
ihrer neuen Träume und guten Absichten, als am Ziele ihrer Fahrt die
Tochter als erste sich erhob und ihren jungen Körper dehnte.