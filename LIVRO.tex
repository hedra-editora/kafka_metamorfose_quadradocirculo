\documentclass[semcabeco,showtrims,12pt,conselho,spreadimages]{memoir}

\usepackage[largepost]{hedraoptions} %% << %%%%%%%%%%%%%%%%
\usepackage[baruch]{hedrastyles}
\usepackage[xetex,chicagofootnotes]{tipografia}
\usepackage[standart,compontinhos]{toc}
\usepackage{hedraextra}
\usepackage{penalidades}
\usepackage{graficos}
\usepackage{hedralogo}
\usepackage{hifensextras}
\usepackage{fichatecnica}
\usepackage[standart]{aparatos}
\usepackage{tabelas}
\usepackage{versos}
\usepackage{gitrevisioninfo}
\usepackage{parallel}


\newcommand{\forceindent}{\leavevmode{\parindent=1,4em\indent}}

\linespread{1.15}

\usepackage{endnotes}
\renewcommand{\notesname}{Notas}

\newcommand{\paragraphbr}[1]{\vfill\pagebreak\paragraph{#1}}
\renewcommand\theparagraph{\Roman{paragraph}}
\newcommand{\quebra}{\vfil\pagebreak}
\newcommand{\est}{\vspace{10cm}}
\newenvironment{mesma}%
   {\par\samepage}%
   {\par\pagebreak[0]}

%\counterwithin*{endnote}{part}
%\counterwithin*{endnote}{chapter}

\let\latexchapter\chapter
\makeatletter
\renewcommand\enoteheading{%
  \setcounter{secnumdepth}{-2}
  \latexchapter*{\notesname\markboth{NOTAS}{}}
  \mbox{}\par\vskip-\baselineskip
  \let\@afterindentfalse\@afterindenttrue
}
\makeatother
%\usepackage{fancyhdr}
%\pagestyle{fancy}
%\setlength{\headheight}{9mm}
%\fancyhf{}
%\fancyhead[R]{\thepage}
%\renewcommand{\headrulewidth}{0pt}

%\lhead[\fancyplain{}]{}
%\chead[\fancyplain{}]{}
%\rhead[\fancyplain{}]{\cnvt{\thepage} -- \thepage}

%\newcommand*{\cnvt}[1]{\the\numexpr#1-1\relax}

%\fancypagestyle{chapter}{
%\pagestyle{fancy}
%\setlength{\headheight}{5mm}
%\fancyhf{}
%\fancyhead[R]{\thepage}
%\renewcommand{\headrulewidth}{0pt}}


\usepackage{footmisc}

\renewcommand*\footnoterule{}% tira a barrinha da footnote
%\fancyhf[RO]{\cnvt{\thepage} -- \thepage}
%\fancyfoot{}
%\renewcommand{\headrulewidth}{0pt}
%\renewcommand{\footrulewidth}{0pt}}

\usepackage{fontspec}

%\usepackage{Formular}
\newfontfamily\Formular{Formular-Regular}[
BoldFont = Formular-Bold.otf,
ItalicFont = Formular-Italic.otf]
%\newcommand{\Formular}[1]{#1}

\newfontfamily\Brabo{FS Brabo Pro Regular}

%--------------------------------------------ALTERAR DISTÃNCIA ENTRE TÍTULO DO SUMÁRIO E CAPÍTULOS
%\addtocontents{toc}{\vskip-15pt}
%--------------------------------------------
\usepackage{afterpage}

\newcommand\blankpage{%
    \null
    \thispagestyle{empty}%
    \addtocounter{page}{0}%
    \newpage}

%\usepackage{imakeidx} 
%\makeindex[program=xindy, options=-C utf8 -L portuguese]
%\newcommand\gobbleone[1]{}
%\newcommand*{\seeonly}[2]{\ (\emph{\seename} #1)}
%\newcommand*{\also}[2]{\emph{cf.} #1}
%\newcommand{\Also}[2]{\emph{See also} #1}
%\renewcommand\indexname{Índice onomástico}
%\makeindex[intoc]

\setcounter{tocdepth}{0}
\setcounter{secnumdepth}{-2} 
%\linespread{1.08}
\usepackage{commands}

\usepackage{setspace}

\makeatletter
\newenvironment{Parskip}{%
   \par
   \parskip=0.3\baselineskip \advance\parskip by 0pt plus 2pt
   \parindent=\z@
   \def\@listI{\leftmargin\leftmargini
      \topsep\z@ \parsep\parskip \itemsep\z@}
   \let\@listi\@listI
   \@listi
   \def\@listii{\leftmargin\leftmarginii
      \labelwidth\leftmarginii\advance\labelwidth-\labelsep
      \topsep\z@ \parsep\parskip \itemsep\z@}
   \def\@listiii{\leftmargin\leftmarginiii
       \labelwidth\leftmarginiii\advance\labelwidth-\labelsep
       \topsep\z@ \parsep\parskip \itemsep\z@}
   \partopsep=\z@
}{\par}
\makeatother

\makeatletter
\newenvironment{myParskip}{%
   \par
   \parskip=0.2\baselineskip \advance\parskip by 0pt plus 2pt
   \parindent=\z@
   \def\@listI{\leftmargin\leftmargini
      \topsep\z@ \parsep\parskip \itemsep\z@}
   \let\@listi\@listI
   \@listi
   \def\@listii{\leftmargin\leftmarginii
      \labelwidth\leftmarginii\advance\labelwidth-\labelsep
      \topsep\z@ \parsep\parskip \itemsep\z@}
   \def\@listiii{\leftmargin\leftmarginiii
       \labelwidth\leftmarginiii\advance\labelwidth-\labelsep
       \topsep\z@ \parsep\parskip \itemsep\z@}
   \partopsep=\z@
}{\par}
\makeatother

\newcommand{\mystar}{{\fontfamily{lmr}\selectfont$\star$}}

%\makeatletter
%\renewcommand{\@chapapp}{}% Not necessary...
%\newenvironment{chapquote}[2][2em]
%  {\setlength{\@tempdima}{#1}%
%   \def\chapquote@author{#2}%
%   \parshape 1 \@tempdima \dimexpr\textwidth-2\@tempdima\relax%
%   \itshape}
%  {\par\scriptsize\hfill-- \chapquote@author\hspace*{\@tempdima}\par\bigskip}
%\makeatother

%\newcommand\Chapter[2]{\chapter
%  [#1\hfil\hbox{}\protect\linebreak{\itshape#1}]%
%  {#1\\[2ex]\Large\itshape#2}%
%}

\begin{document}


\chapter*{Introdução}
\addcontentsline{toc}{chapter}{Introdução, \emph{por Celso Donizete Cruz}
\medskip}
%\hedramarkboth{Introdução}{Celso Donizete Cruz}

\textsc{Eis uma nova} tradução brasileira da obra mais conhecida de Franz Kafka.
A princípio quis chamá"-la de \textit{A transformação}, modo de
recuperar a repetição sonora do substantivo alemão do título original,
“Verwandlung”, que ecoa na forma verbal “verwandelt” (“transformado”),
no fim da primeira frase da narrativa, considerada por muitos a
sentença de abertura mais célebre de toda a literatura.\footnote{ “Als 
Gregor Samsa eines Morgens aus unruhigen
Träumen erwachte, fande er sich in seinem Bett zu einem ungeheuren
Ungeziefer \textit{verwandelt}”. (A fim de manter a mesma
correspondência, Modesto Carone, responsável pela primeira tradução
brasileira diretamente do alemão, na década de 1980,
propõe solução inversa: não mexe no título, mas traduz o verbo por
“metamorfoseado”, opção desde então seguida por algumas traduções
posteriores.)} Dizem que o escritor argentino Jorge Luis Borges também
criticava o título consagrado nas traduções, argumentando que a língua
alemã possui a palavra “Metamorphose”, e Kafka a adotaria se sua
intenção fosse de fato privilegiar em sua narrativa a mutação
biológica, o que não é o caso. Na recepção em espanhol do século \textsc{xxi}, em
consonância com o reparo, propõe"-se uma nova tradução do título,
\textit{La Transformación}, nas edições da Editorial Funambulista, de
Madri, e da Debolsillo, de Barcelona, ambas de 2005. Em língua inglesa,
já no final do século \textsc{xx} surgia uma proposta conciliatória, \textit{The
transformation (Metamorphosis)}, na edição da Penguin Classics, de
1995.

Os exemplos não são muitos, afinal, e houve também argumentos contrários
à adoção de um novo título, todos no fundo receosos de afrontar
gratuitamente a tradição (em português, tradução e tradição é que dão
um trocadilho revelador das condições do campo). A experiência poderia
ser desastrosa em mais de um sentido. Poderia levar a perder leitores
interessados na obra, porém em busca do título tradicional, o que
autoriza quando muito um parêntese, como na edição inglesa. A
mudança de título poderia além do mais ser vista como tática meramente
novidadeira, sem maiores implicações para a fruição da obra. Ou quem
sabe angariasse para a tradução a pecha de enganadora, dando título
desconhecido a uma obra mais do que famosa (fosse entendida a proposta
só como brincadeira, e já estaria melhor). Mantenha"-se então
\textit{A metamorfose}, a tradução consagrada do título em língua
portuguesa. Não há por que polemizar, a questão é mesmo menor. O que
importa vem depois do título, com ou sem eco, e aí logo se percebe que
o foco não está na metamorfose, mas nas transformações que ela
acarreta.

No Brasil, \textit{A metamorfose} vem funcionando como o carro"-chefe
da recepção de Kafka, sobretudo de sua recepção popular. De 1956 a
2002, contam"-se no país pelo menos 21 edições diferentes da
obra.\footnote{ Cf. Celso Cruz. \textit{Metamorfoses de Kafka}, São
Paulo: Annablume, 2007.} É o livro que fisga o leitor e lhe abre as
portas para o universo kafkiano.\footnote{ Alguns colegas reivindicam
essa primazia para \textit{O processo}, às vezes até para \textit{O
castelo}, o que pode até acontecer entre o público mais intelectual.
Porém, o número de edições e traduções de \textit{A metamorfose} é bem
maior, o que confirma sua extensa popularização. \textit{O processo}
exige mais do leitor, e \textit{O castelo} ainda mais, daí a
dificuldade dessas obras de atingir o grande público. Elas tendem a
atrair o interesse desse público no caminho que \textit{A metamorfose}
pavimenta.} A mesma coisa deve se dar em outros países. A história do
homem que se transforma em inseto tem um forte apelo, nunca deixando de
inspirar novos lançamentos, cuja sucessão reafirma o notável sucesso da
obra entre leitores dos mais distintos estratos culturais. Difusão sem
dúvida louvável, quando se pensa no teor crítico do discurso kafkiano e
em seu poder desalienante. Contudo, vendo o que já se fez para sua
divulgação, pode"-se supor também um leitor leigo, seduzido pelo
título e por algumas capas, a julgar que se trata de uma história de
terror cujo protagonista é um homem que vira uma barata gigante e
ameaça sua família. Tal leitor não estará absolutamente errado, só que
se acompanhar a narrativa há de topar com um terror estranho e
inesperado, por vezes mais engraçado que aflitivo (pode achar que
levou gato por lebre: queria \textit{A metamorfose}, e recebeu
\textit{A transformação}). A hipótese não é de todo descabida, ainda
que seja difícil um leitor se aproximar da obra assim tão
desavisadamente. O adjetivo derivado do nome de seu autor é presença
certa nos dicionários, além do que Kafka e os títulos de suas obras
mais famosas já são verbetes obrigatórios das enciclopédias. Se o
leitor vai ao livro, é porque em geral soube de antemão alguma coisa.

Soube no mínimo do grande prestígio do escritor, um dos nossos maiores
ícones literários. Embora tenha escrito no começo do século \textsc{xx}, e
alcançado a glória póstuma após a metade desse mesmo século,
rapidamente ganhou posição ao lado dos clássicos imortais da literatura
de todos os tempos. O adjetivo “kafkiano” ultrapassou os círculos do
pensamento literário, vindo a servir para designar determinadas
situações de nossa vida prática. Tão entranhado assim ficou em nossa
cultura, que figura ao lado de outros adjetivos literários, como dantesco, quixotesco,
homérico. A uma tal altura no Olimpo das letras, não será difícil ao
leitor divisá"-lo ao adentrar o pátio principal da literatura do
Ocidente. Mas o que justifica tamanho destaque? De onde virá a força
que lhe assegura de saída um lugar no panteão dos gênios indisputáveis?
Na resposta a essas questões, há a considerar o que Kafka fez, e o que
dele foi feito.

\textit{A metamorfose} é um bom exemplo para tanto. Em parte por ser sua
obra mais popular e ter sido publicada com o autor ainda vivo. Não
foi muita coisa que ele deixou vir a público enquanto vivia.\footnote{
Há opinião diversa, como a de Osman Durrani, no \textit{\mbox{The Cambridge}
Companion to Kafka}, que procura desfazer o mito do autor tímido, avesso à
publicação de sua obra. Em relação ao mito, até que Kafka publicou
bastante. In \textit{The Cambridge
Companion to Kafka}, Org. Julian Preece, Cambridge University Press,
2002.} Três pequenos livros de narrativas curtas:
\textit{Betrachtung} (\textit{Contemplação}), de 1913; \textit{Ein
Landartz} (\textit{Um médico rural}), de 1919; e \textit{Ein
Hungerkünstler} (\textit{Um artista da fome}), publicado no ano de sua
morte, 1924. Três narrativas médias: \textit{Der Heizer} (\textit{O
foguista}), de 1913; \textit{Das Urteil} (\textit{O julgamento} ou
\textit{O veredito}), de 1916; e \textit{In der Strafkolonie}
(\textit{Na colônia penal}), de 1919. Além de \textit{Die Verwandlung}
(\textit{A transformação [metamorfose]}), uma narrativa
longa, que saiu inicialmente na revista \textit{Weiße Blätter}
(\textit{Folhas Brancas}) em 1915, depois em livro, em 1916, e alcançou
uma segunda edição em 1918. Em conjunto, os escritos publicados em vida
não ultrapassam quinhentas páginas.\footnote{ São 447, numa contagem
mais recente, incluindo textos não literários. Cf. Osman Durrani,
“Editions, translations, adaptations”, in \textit{The Cambridge
Companion to Kafka. Op. cit}. p. 208.} A economia narrativa e o rigor no acabamento
apresentam"-se desde já como parte do projeto literário de Kafka. De
fato, apenas essas obras talvez fossem suficientes para garantir sua
posição entre os grandes mestres. As principais características de sua
ficção estão praticamente todas presentes. Só não se saberia então que
o material publicado era apenas parte do edifício.

Ficou mais do que notável uma das últimas vontades de Kafka, a de que,
após a sua morte, seu espólio literário fosse destruído. O amigo Max
Brod, incumbido pelo autor da realização dessa vontade, evidentemente
não cumpriu a promessa. Esse episódio biográfico já deu o que falar e é
um dos que contribuem para a construção de uma visão romantizada da vida de
Kafka. Pode"-se imaginar o escritor em seu leito de morte, vencido
pela tuberculose, entre acessos de tosse e escarros de sangue,
encarecendo o amigo com a tarefa inglória; na cena seguinte o amigo, ao
abrir o baú, surpreso e maravilhado com a quantidade e a qualidade do
tesouro que encontra; no final feliz, o tesouro partilhado com os
próximos e os pósteros\ldots{} Deve ter sido mais ou menos isso o que
aconteceu, descontada a dose de má ficção. O já citado Jorge Luis
Borges é um dos que referem o episódio,\footnote{ Num conhecido prólogo
publicado no Brasil na abertura de uma edição da Ediouro de \textit{A
metamorfose}, tradução de Torrieri Guimarães, de 1998, coleção
“Biblioteca de Babel”, dedicada à literatura fantástica, homônima porém
não a mesma dirigida por Borges e Bioy Casares na Argentina. A
propósito de Borges, ainda, também se acredita que tenha traduzido
\textit{A metamorfose}, fato entretanto desmentido por Fernando
Sorrentino, em ``\,`La Metamorfosis’ que Borges jamás tradujo”, \textit{La
Nación}, Buenos Aires, 9 de marzo de 1997 (disponível \textit{on-line} em
<http://www.sololiteratura.com/sor/sorrenelkafkiano.htm>, acesso em
30/05/2008, com o título “El kafkiano caso de la Verwandlung que Borges
jamás tradujo”).} evocando para efeito de comparação o caso de Virgílio
(outro escritor cujo similar último desejo também não se realizou) e a
seguir argumentando que se essa fosse realmente a vontade desses
autores, eles mesmos se encarregariam de riscar o fósforo. Ironias à
parte, o pedido não atendido de Kafka pode ser a tradução sincera de
sua dúvida quanto ao valor de suas páginas inacabadas. O rigor de seus
critérios de acabamento aumenta na medida da desproporção entre o muito
que escreveu e o pouco que publicou. \textit{A metamorfose}, todavia,
não deixa dúvidas, pois passou pelo crivo do autor, não sofreu as
interferências da organização e edição póstumas de Max Brod, não
padecendo assim da desconfiança, algo desmedida, diga"-se, quanto à
autenticidade de alguns trechos de sua produção literária divulgada
\textit{post mortem}. Por isso vem a ser mesmo o livro ideal para um
contato inicial preciso com a mais pura ficção kafkiana.\footnote{
Entretanto, não se quer dizer que o que veio após sua morte deva ser
descartado, longe disso. Inclusive, acontece uma coisa interessante, a
partir da recepção das obras do espólio. O inacabado e o fragmentário
próprio desses papéis cuja redação não foi retomada, ou que não foram
revistos para publicação, são incorporados como matrizes da expressão
literária de Kafka, e revertem sobre suas produções anteriores.
Cumpre"-se de certa forma a perspicaz observação, de novo de Borges,
agora em “Kafka e seus precursores”, de que os autores que influenciam
Kafka só vêm a surgir depois de sua morte.}

Proponho a distinção entre as narrativas póstumas e as publicadas em
vida apenas como tentativa de destacar o cuidado do autor com seus
escritos, sua consciência literária, seu senso crítico apurado, seu
compromisso vital com a literatura. Trata"-se de um homem de letras,
que frequentou espaços sociais comuns a intelectuais e artistas, que
tinha uma visão particular da literatura, estava informado das
novidades de seu tempo, e certamente manifestaria suas opiniões em
encontros com os amigos nos cafés de Praga. Não corresponderia
unicamente à imagem do escritor desconhecido, enclausurado, sombrio,
gênio incompreendido e maldito --- visões românticas tantas vezes
propagadas nas biografias. Max Brod, conhecendo o amigo, por certo
estaria consciente de seu alto valor literário, e de antemão calcularia
a importância do que o aguardava no baú. Kafka não foi afinal o
escritor anônimo descoberto da noite para o dia, infelizmente quando
era tarde demais e já não podia desfrutar da fama. Não viu o sucesso de
nenhuma das obras que publicou, é certo, porém é igualmente correto que
alcançou de imediato com elas o reconhecimento de seus pares em Praga,
despertando reações positivas também em alguns círculos literários da
Alemanha.\footnote{ Luiz Costa Lima comprova em \textit{Limites da voz:
Kafka} (Rocco, 1993) que o escritor “não foi um
ignorado”, e que sua “recepção inteligente” soube lhe destacar o valor,
além de ser em alguns casos muito feliz na caracterização de suas
peculiaridades.} O seu talento de primeira grandeza não era popular, mas
foi notado. Consta que arrebatou em 1915 a terceira edição do Prêmio
Theodor Fontane de Arte e Literatura, instituído na Alemanha, embora
tenha sido uma vitória indireta: o vencedor oficial, o dramaturgo
alemão Carl Sternheim, repassou depois a premiação a Kafka. Episódio
emblemático de uma recepção restrita --- que tem o autor como escritor dos
escritores, conhecido apenas em pequenos círculos literários, condição
que após a sua morte sua obra superaria totalmente, chegando ao coração
das massas, o que é até espantoso, em face do desconforto inevitável
provocado por sua leitura.

Franz Kafka nasceu em 1883 em Praga, capital da então Boêmia, hoje
República Tcheca. À época, a Boêmia fazia parte do Império
Austro"-Húngaro, e seu idioma administrativo oficial era o alemão. A
submissão compulsória ao império obviamente não retirava aos tchecos o
sentimento de pertença à cultura de sua região, e logo os movimentos
nacionalistas desta e de outras regiões submetidas iriam dissolver o
império. Imagine"-se a aversão pelo imperialismo, e a desconfiança
para com todos que parecessem mais fiéis ao império do que à Boêmia.
Este em parte devia ser o caso de Kafka que, apesar de seu local de
nascimento, não possuía identificação muito evidente com a cultura
tcheca. Era filho de pais judeus emigrados da Áustria, praticamente sem
laços afetivos ou nacionalistas com a Boêmia. Sua família fazia parte
da comunidade judaica de Praga e ao mesmo tempo flertava com os
oficiais alemães, tanto é que colocaram Kafka para estudar numa escola
alemã. Seu caso, evidentemente, não seria único, contudo não será
também de admirar a crise identitária e o sentimento de perseguição
decorrentes da situação. Kafka era tcheco, mas escreveu em alemão e
acabou órfão das duas culturas. Um dos maiores nomes da literatura
alemã de todos os tempos não era alemão. E Praga em sua obra é nada
mais que a sombra de um cenário ocasional. Judeu, mas desgarrado e
descrente, tampouco pode"-se dizer que encontrasse sua identidade em
meio à comunidade judaica. Exilado das três pátrias, seria hoje cidadão
do mundo\ldots{} Mas a Europa era outra, e Kafka a viu antes, durante e logo
depois da Primeira Guerra. Foi testemunha desse acontecimento
traumático, que literalmente expôs as entranhas de uma sociedade
pretensamente racional. A condição marginal lhe possibilitaria observar
essa sociedade sem comprometimentos patrióticos. O que tinha para
dizer, e deixou por escrito, não se dirigia especificamente à cultura
tcheca, alemã ou judaica. A nenhuma das três em particular, mas a todas
a um só tempo --- ao humano em cada uma delas.

Sua posição à parte no conturbado cenário europeu de então deu"-lhe uma
compreensão inusitada dos problemas do homem de seu tempo, o homem
contemporâneo, este que veio a ser o que ainda hoje somos. Sua obra
resulta dessa compreensão, um dos motivos elementares da importância a
ela atribuída. Kafka parece ter dito uma vez que concebia a literatura
como uma “expedição à verdade”.\footnote{ “Dichtung ist immer nur eine
Expedition nach der Wahrheit”, frase atribuída a Kafka por Gustav
Janouch em seu livro \textit{Conversas com Kafka}.} Essa concepção
acentua outro tanto o interesse pelo que deixou. Seus textos literários
são nesse sentido uma contribuição à filosofia (em sentido lato), que
de direito se ocupa dos problemas da verdade. A literatura kafkiana
demonstra que a reserva filosófica não impede a progressão do método
literário na exploração de um mesmo território. A diferença é que, onde
a filosofia explica, a literatura mostra. Não se recorre às premissas
que permitirão a dedução de uma situação absurda na qual o ser humano,
“barateado”, reduz"-se à condição de inseto. Não se aciona o
pensamento lógico \textit{stricto sensu}. O absurdo é maior e mais
impactante com a eclosão inexplicável do inseto humano no seio de uma
típica família pequeno"-burguesa. O fenômeno é incomum, e visível
apenas pelas lentes literárias. Mas não desperta nenhuma dúvida nas
personagens, que em nenhum momento questionam a impossibilidade do
fato. Note"-se que não se trata de metáfora, o inseto está lá em toda
sua concretude, para quem quiser ver. Age como inseto: tem dificuldades
para se mover, não possui dentes, rasteja pelas paredes e pelo teto,
se alimenta de restos e, apesar de ainda raciocinar como humano e de
entender a língua dos humanos, estes não só não entendem o que ele
fala, como o julgam (com exceção talvez da faxineira) incapaz de
compreendê"-los. Chama"-se aqui a atenção para o irreal que afinal
aparece como a condição para que se enxergue a realidade, e aí temos um
método de desalienação. Didática de Kafka: os contrassensos não são
discutidos, são vistos, e é o impacto do que se vê que perturba o
entendimento sossegado do leitor.

Vale falar de um propósito na dedicação extrema de Kafka à literatura. A
julgar pelo que relatou em diários e cartas, sua vida só adquiria
sentido em função da literatura. Acredito que seja possível confiar na
sinceridade desses escritos pessoais, embora a relação da biografia do
autor (em boa parte inspirada por esses mesmos escritos) com as obras
que deixou dê margem a interpretações muitas vezes equivocadas ou
ingênuas. Não é que tenha retratado episódios de sua vida pessoal.
Estes, no máximo, iriam lhe servir de inspiração. A literatura, como a
concebia, seria mais uma forma de flagrar as contradições da cultura
ocidental no princípio do século \textsc{xx}, de um modo eficaz, no entanto nada
confortável nem óbvio. Seria uma tentativa de entender o que acontece
com os humanos numa sociedade cada vez menos humanizada, se é que algum
dia houvesse sido mais\ldots{} Escrever lhe era vital, provavelmente porque
o punha em contato com a \textit{verdadeira} vida. A procura da
verdade, se por um lado enfeixa suas produções na confluência da
literatura com a filosofia --- e não por acaso os maiores filósofos do
século se dispuseram a interpretá"-lo ---, por outro lado leva a
classificá"-las como realistas. De um realismo que não se reduz à
descrição pitoresca da superfície do real, antes corresponde à
percepção objetiva da realidade. Com efeito, seu realismo é de tipo
expressionista, à medida que dá vazão a uma realidade desfigurada pela
percepção interna do sujeito. Entretanto, o propósito de objetivar essa
realidade impede a expressão puramente subjetiva. Como explica 
Luis Costa Lima (\textit{op}. \textit{cit}., pp. 65--66), a ficção
de Kafka pressupõe uma mediação, “um meio interposto entre a
subjetividade e o mundo externo, que permita a objetivação daquela”:
“Sua questão é converter as tematizações pessoais de próprias ao espaço
interno em capazes de se mover no externo; i.e., transformá"-las de
fantasmas em objetos, cujos traços mostrariam a si e a seu tempo”. Se
bem entendo a lição, diviso uma metodologia nessa busca de conversão do
interior em exterior, de “fantasmas em objetos”, da subjetividade em
objetividade, enfim. Só faz sentido falar em método quando se quer
atingir um objetivo, no caso \textit{mostrar} “a si e a seu tempo”, o
que vem a ser a confirmação de uma intenção realista.

Aqui se cai de chofre na \textit{selva selvaggia} da fortuna crítica.
Missão impossível não recorrer ao paradoxo na descrição da
singularidade do autor. O subjetivo objetivo, a ação que é inação, o
estranho familiar\ldots{} Nomeia"-se pela contradição uma obra que se
realiza no limite, sempre na dúvida entre o que \textit{é} e o que
\textit{não é}. Tal indecisão retira as bases de qualquer juízo crítico
absoluto. E o mistério sempre se mantém um mistério, mesmo depois de
aberto com as diferentes chaves forjadas pela crítica. Ora, não será
demasiado supor que era esse precisamente o ponto visado pela
literatura de Kafka, a apresentação de situações numa perspectiva
ambígua, trágica e cômica ao mesmo tempo, próxima e distante, real e
fantástica (termos e contratermos se sucedem\ldots{}). Toda representação
kafkiana sustenta"-se na evocação de sua face contrária. O caso de Gregor
Samsa, personagem principal de \textit{A metamorfose}, é mais uma vez
exemplar. Ele só toma consciência de sua alienação ao ser alienado de
sua forma humana. Não é o fato de se transformar em inseto o que o
aliena, isso só lhe revela sua real alienação. Já era inseto quando
ainda era humano, se ainda é humano quando já é inseto? 
Se para descobrir sua humanidade é
preciso que a perca, a metamorfose é a condição de sua consciência. A
exposição do humano é levada ao extremo com a oposição do inseto. No
choque dos opostos é possível viver uma verdade.

Não devia mesmo ser fácil ao autor sustentar tal ponto de vista. Kafka
sempre se queixou da falta de espaço e tempo para se dedicar à
literatura como gostaria e, de acordo com seus critérios, deveria. A
biografia em quadrinhos de Robert Crumb e David Zane
Mairowitz\footnote{ \textit{Kafka de Crumb}, trad. José Gradel, Rio de
Janeiro: Relume"-Dumará, 2006.} retrata enfaticamente a ausência de
privacidade na casa dos pais, onde residiu durante quase toda a sua
vida, e também o grau de concentração exigido em sua prática literária.
No traço de Crumb, o escritor entra em transe ao escrever, os olhos
esbugalhados, como se transportado para um outro plano existencial. O
transe é ainda ambíguo, pois significa a necessidade tanto de superar
um entorno desfavorável à prática (situação do sujeito) quanto de
aceder ao plano de perseguição da verdade (condição do objeto). Não
admira que o esforço exaurisse o autor, solicitando"-lhe uma
disposição que somente teria se pudesse abandonar o trabalho e demais
compromissos sociais. Kafka se dizia um fraco. Para poder escrever, se
viu obrigado a abdicar de possíveis casamentos e a buscar a solidão.
Ainda assim, o mínimo exigido de vida social já lhe parecia muito e
roubava"-lhe as forças de que necessitava para completar suas obras.
Em que pese a fantasia \textit{underground}, a representação proposta
por Crumb sintetiza os apuros do escritor, que se sentia hábil e capaz
apenas para o trabalho literário que sua vida lhe dificultava exercer.
O transe místico é ainda o simulacro de sua obsessão com a literatura,
à qual sacrificava a vida.

Nesse ponto tocamos a esfera do mito. Não é fácil acreditar que um autor
se dispusesse a tanto, nem que a literatura exija pacto tão radical.
Mas fato é que o próprio Kafka cultivou a ideia do escritor abnegado. A
divulgação de suas obras póstumas também pôs em circulação seus
escritos pessoais, e é nesses que se acham declarações do autor sobre
seu envolvimento com a criação literária. Essas declarações alimentam o
mito. De acordo com elas, o escritor dedica"-se à literatura como a um
sacerdócio. A literatura seria sua religião mais cara, se de fato lhe
revelasse a verdade. Daí a enxergá"-lo como profeta é um passo. Isso
sem contar a simpatia despertada por sua situação pessoal precária.
Note"-se, porém, que Kafka fala de uma \textit{expedição} à verdade,
não de uma \textit{revelação}. Uma expedição é uma viagem, uma
aventura, e só se vive uma verdadeira aventura quando não se pode
prever o final. Para abandonar o mito, é preciso compreender o
compromisso com a verdade do ponto de vista ético, não religioso. Os
resultados das expedições nunca são conclusivos. Mas o rigoroso relato
do percurso é a prova da dedicação e da fidelidade ao compromisso
assumido. A literatura é, assim, seu instrumento de busca da verdade,
ao encontro da qual não é necessário ir com a alma pura dos crentes
inocentes. Também não é preciso deixar ao cinismo o papel principal.
Parece haver em Kafka, como nos grandes autores, um compromisso com a
sinceridade. A tortura ou o transe da criação podem, sim, se associar antes ao
rigor do que à mística. Essa reivindicação, contudo, no fundo também
obedece a imperativos associados a uma visão específica da arte,
entendida aqui mais como construção e cálculo do que como magia e
inspiração. Reclama"-se um escritor consciente de sua proposta
literária, antes que um “médium” da expressão de forças superiores.

Entretanto, a preferência parece tender ao místico. 
Com a popularização de sua recepção, Franz Kafka
passa a habitar uma outra dimensão, e se transforma em um personagem da
mitologia moderna, cujos círculos inevitavelmente se misturam à
mitologia dos tempos imemoriais. E faz pouco mais de oitenta anos que
veio a falecer. Morria quando nossos pais ou avós nasciam, há duas
gerações apenas. Como esse intervalo é relativamente pequeno, ainda é
possível testar, com base em documentação histórica, a verdade de
alguns relatos biográficos.\footnote{ Cf., por exemplo, Anthony
Northey, “Myths and realities in Kafka biography”, in \textit{The
Cambridge Companion to Kafka}. \textit{Op. cit}.} Mas a própria disputa pela verdade biográfica tende a confirmar
o mito. Algumas correções não perturbam a imagem geral, pelo contrário.
A voracidade do mito traga qualquer migalha de veracidade histórica.

Creio que o que acontece com a recepção de Kafka no Brasil repete em
escala doméstica, ressalvado o atraso, o movimento internacional de sua
popularização. De início, sua leitura é privilégio de pequenos círculos,
mas logo suas produções vêm a ser difundidas para todos os estratos
sociais. \textit{A metamorfose} é sua obra mais divulgada, porta de
entrada de sua recepção, como observado. Sua primeira tradução
brasileira, de Brenno Silveira, data de 1956, e foi feita a partir do inglês. É só
a partir dos anos 1960 que o autor passa a ser popularizado, já
então como clássico. Em muito concorre para sua popularização a
história do homem que se transforma em um inseto. Na época de sua
primeira publicação em livro, em 1916, Kafka instou para que o inseto
de modo algum fosse sugerido na capa. Logo na primeira edição
brasileira, contudo, já aparecem justapostos, de costas um para o
outro, os perfis de um homem e de uma barata, esta mais detalhada que
aquele. A tônica no inseto descortina uma estratégia de difusão que
potencializa em demasia certos apelos popularescos da história
original, e tende a preservar o mito. \textit{A metamorfose} é
normalmente divulgada como clássico da literatura moderna ou universal.
Só por isso já deveria ser lida. Mas a presença do homem"-inseto é um
incentivo a mais, considerada a curiosidade que o fantástico e o
sobrenatural em geral despertam no público. Em nome desse apelo é que
se coloca a metamorfose do homem em primeiro plano, quando na verdade a
narrativa trata é das transformações de seu entorno em face de uma
situação totalmente inesperada. Todas as nuanças possíveis decorrentes
desse evento inicial são exploradas. A metamorfose desperta reações em
cadeia, e são essas reações que a narrativa de Kafka acompanha. Daí, na
verdade pouco importa o inseto, basta frisar sua inadequação e a
repulsa que ele provoca. Se em lugar do inseto houver uma massa amorfa
e gosmenta, nada muda, a não ser a forma de suas pegadas.

Por isso a sugestão de deslocar o foco. Trocar a metamorfose pela
transformação. As reações ao asco são mais interessantes que o objeto
asqueroso. Aí é que está o humor, um humor não autorizado pelo horror
da situação descrita e que entretanto comparece como possível matriz do
modo de narração. O distanciamento narrativo é máximo, mesmo que
tenhamos acesso direto aos pensamentos das personagens. O narrador é
onisciente e não se compromete. Mantém a objetividade ainda que o
evento narrado seja o maior dos absurdos. Faz questão de chamar a
atenção para detalhes periféricos das situações principais, e tais
detalhes acabam sendo reveladores das reais motivações das personagens.
Ora, o absurdo, o inesperado, o grotesco, o ínfimo que se revela
fundamental, essas ocorrências são comuns ao reino do cômico, isso sem
falar que o distanciamento é a condição da comédia, pois são poucos os
que acham graça quando são os objetos de derrisão. Não admira pois que,
conforme reza a lenda, Kafka tenha chegado às gargalhadas ao ler a
narrativa em primeira mão para os amigos. Acredito que falte uma pitada
maior desse humor nas edições e traduções brasileiras, o qual todavia
está presente nas ilustrações do pintor Walter Levy para a primeira
edição brasileira de 1956. Esse veio interpretativo ficou meio esquecido nas
várias edições posteriores. A tônica foi mais para o horror ou para o
trágico, que fazem justiça à obra, mas não a esgotam.

Constata"-se, por mais incrível que pareça, apesar de toda a avalanche
interpretativa a que o autor esteve e continua sujeito, a existência de
espaços ainda a explorar, afora a necessidade de revisão de algumas
ideias prontas herdadas de recepções passadas. O maior desafio da
crítica kafkiana talvez seja escapar aos mitos e às múltiplas
interpretações preestabelecidas de sua obra. De qualquer modo, todo
clássico acaba se impondo por si só quando nos dispomos à sua leitura.
Uma nova tradução é só mais uma proposta de interpretação, sempre
possível porque o contexto de recepção nunca é estável. O clássico
atravessa as gerações, tendo sempre o que dizer a cada uma delas. Há
portanto sempre uma oportunidade de renovação a comprovar o seu vigor
atemporal.


\begin{Parallel}[p]{}{} 
\ParallelLText{\selectlanguage{ngerman} \vspace{2.5cm}

\section{I}

\noindent{}Als Gregor Samsa eines Morgens aus unruhigen Träumen erwachte, fand er
sich in seinem Bett zu einem ungeheueren Ungeziefer verwandelt. Er lag
auf seinem panzerartig harten Rücken und sah, wenn er den Kopf ein wenig
hob, seinen gewölbten, braunen, von bogenförmigen Versteifungen
geteilten Bauch, auf dessen Höhe sich die Bettdecke, zum gänzlichen
Niedergleiten bereit, kaum noch erhalten konnte. Seine vielen, im
Vergleich zu seinem sonstigen Umfang kläglich dünnen Beine flimmerten
ihm hilflos vor den Augen.

»Was ist mit mir geschehen?« dachte er. Es war kein Traum. Sein Zimmer,
ein richtiges, nur etwas zu kleines Menschenzimmer, lag ruhig zwischen
den vier wohlbekannten Wänden. Über dem Tisch, auf dem eine
auseinandergepackte Musterkollektion von Tuchwaren ausgebreitet war ---
Samsa war Reisender ---, hing das Bild, das er vor kurzem aus einer
illustrierten Zeitschrift ausgeschnitten und in einem hübschen,
vergoldeten Rahmen untergebracht hatte. Es stellte eine Dame dar, die,
mit einem Pelzhut und einer Pelzboa versehen, aufrecht dasaß und einen
schweren Pelzmuff, in dem ihr ganzer Unterarm verschwunden war, dem
Beschauer entgegenhob.

Gregors Blick richtete sich dann zum Fenster, und das trübe Wetter ---
man hörte Regentropfen auf das Fensterblech aufschlagen --- machte ihn
ganz melancholisch. »Wie wäre es, wenn ich noch ein wenig
weiterschliefe und alle Narrheiten vergäße,« dachte er, aber das war
gänzlich undurchführbar, denn er war gewöhnt, auf der rechten Seite zu
schlafen, konnte sich aber in seinem gegenwärtigen Zustand nicht in
diese Lage bringen. Mit welcher Kraft er sich auch auf die rechte Seite
warf, immer wieder schaukelte er in die Rückenlage zurück. Er versuchte
es wohl hundertmal, schloß die Augen, um die zappelnden Beine nicht
sehen zu müssen, und ließ erst ab, als er in der Seite einen noch nie
gefühlten, leichten, dumpfen Schmerz zu fühlen begann.

»Ach Gott,« dachte er, »was für einen anstrengenden Beruf habe ich
gewählt! Tag aus, Tag ein auf der Reise. Die geschäftlichen Aufregungen
sind viel größer, als im eigentlichen Geschäft zu Hause, und außerdem
ist mir noch diese Plage des Reisens auferlegt, die Sorgen um die
Zuganschlüsse, das unregelmäßige, schlechte Essen, ein immer
wechselnder, nie andauernder, nie herzlich werdender menschlicher
Verkehr. Der Teufel soll das alles holen!« Er fühlte ein leichtes Jucken
oben auf dem Bauch; schob sich auf dem Rücken langsam näher zum
Bettpfosten, um den Kopf besser heben zu können; fand die juckende
Stelle, die mit lauter kleinen weißen Pünktchen besetzt war, die er
nicht zu beurteilen verstand; und wollte mit einem Bein die Stelle
betasten, zog es aber gleich zurück, denn bei der Berührung umwehten ihn
Kälteschauer.

Er glitt wieder in seine frühere Lage zurück. »Dies frühzeitige
Aufstehen«, dachte er, »macht einen ganz blödsinnig. Der Mensch muß
seinen Schlaf haben. Andere Reisende leben wie Haremsfrauen. Wenn ich
zum Beispiel im Laufe des Vormittags ins Gasthaus zurückgehe, um die
erlangten Aufträge zu überschreiben, sitzen diese Herren erst beim
Frühstück. Das sollte ich bei meinem Chef versuchen; ich würde auf der
Stelle hinausfliegen. Wer weiß übrigens, ob das nicht sehr gut für mich
wäre. Wenn ich mich nicht wegen meiner Eltern zurückhielte, ich hätte
längst gekündigt, ich wäre vor den Chef hingetreten und hätte ihm meine
Meinung von Grund des Herzens aus gesagt. Vom Pult hätte er fallen
müssen! Es ist auch eine sonderbare Art, sich auf das Pult zu setzen und
von der Höhe herab mit dem Angestellten zu reden, der überdies wegen der
Schwerhörigkeit des Chefs ganz nahe herantreten muß. Nun, die Hoffnung
ist noch nicht gänzlich aufgegeben, habe ich einmal das Geld beisammen,
um die Schuld der Eltern an ihn abzuzahlen --- es dürfte noch fünf bis
sechs Jahre dauern ---, mache ich die Sache unbedingt. Dann wird der
große Schnitt gemacht. Vorläufig allerdings muß ich aufstehen, denn mein
Zug fährt um fünf.«

Und er sah zur Weckuhr hinüber, die auf dem Kasten tickte. »Himmlischer
Vater!« dachte er, Es war halb sieben Uhr, und die Zeiger gingen ruhig
vorwärts, es war sogar halb vorüber, es näherte sich schon dreiviertel.
Sollte der Wecker nicht geläutet haben? Man sah vom Bett aus, daß er auf
vier Uhr richtig eingestellt war; gewiß hatte er auch geläutet. Ja, aber
war es möglich, dieses möbelerschütternde Läuten ruhig zu verschlafen?
Nun, ruhig hatte er ja nicht geschlafen, aber wahrscheinlich desto
fester. Was aber sollte er jetzt tun? Der nächste Zug ging um sieben
Uhr; um den einzuholen, hätte er sich unsinnig beeilen müssen, und die
Kollektion war noch nicht eingepackt, und er selbst fühlte sich durchaus
nicht besonders frisch und beweglich. Und selbst wenn er den Zug
einholte, ein Donnerwetter des Chefs war nicht zu vermeiden, denn der
Geschäftsdiener hatte beim Fünfuhrzug gewartet und die Meldung von
seiner Versäumnis längst erstattet. Es war eine Kreatur des Chefs, ohne
Rückgrat und Verstand. Wie nun, wenn er sich krank meldete? Das wäre
aber äußerst peinlich und verdächtig, denn Gregor war während seines
fünfjährigen Dienstes noch nicht einmal krank gewesen. Gewiß würde der
Chef mit dem Krankenkassenarzt kommen, würde den Eltern wegen des faulen
Sohnes Vorwürfe machen und alle Einwände durch den Hinweis auf den
Krankenkassenarzt abschneiden, für den es ja überhaupt nur ganz gesunde,
aber arbeitsscheue Menschen gibt. Und hätte er übrigens in diesem Falle
so ganz unrecht? Gregor fühlte sich tatsächlich, abgesehen von einer\est\
nach dem langen Schlaf wirklich überflüssigen Schläfrigkeit, ganz wohl
und hatte sogar einen besonders kräftigen Hunger.

Als er dies alles in größter Eile überlegte, ohne sich entschließen zu
können, das Bett zu verlassen --- gerade schlug der Wecker dreiviertel
sieben --- klopfte es vorsichtig an die Tür am Kopfende seines Bettes.
»Gregor,« rief es --- es war die Mutter ---, »es ist dreiviertel sieben.
Wolltest du nicht wegfahren?« Die sanfte Stimme! Gregor erschrak, als er
seine antwortende Stimme hörte, die wohl unverkennbar seine frühere war,
in die sich aber, wie von unten her, ein nicht zu unterdrückendes,
schmerzliches Piepsen mischte, das die Worte förmlich nur im ersten
Augenblick in ihrer Deutlichkeit beließ, um sie im Nachklang derart zu
zerstören, daß man nicht wußte, ob man recht gehört hatte. Gregor hatte
ausführlich antworten und alles erklären wollen, beschränkte sich aber
bei diesen Umständen darauf, zu sagen: »Ja, ja, danke, Mutter, ich stehe
schon auf.« Infolge der Holztür war die Veränderung in Gregors Stimme
draußen wohl nicht zu merken, denn die Mutter beruhigte sich mit dieser
Erklärung und schlürfte davon. Aber durch das kleine Gespräch waren die
anderen Familienmitglieder darauf aufmerksam geworden, daß Gregor wider
Erwarten noch zu Hause war, und schon klopfte an der einen Seitentür der
Vater, schwach, aber mit der Faust. »Gregor, Gregor,« rief er, »was ist
denn?« Und nach einer kleinen Weile mahnte er nochmals mit tieferer
Stimme: »Gregor! Gregor!« An der anderen Seitentür aber klagte\est\ leise die
Schwester: »Gregor? Ist dir nicht wohl? Brauchst du etwas?« Nach beiden
Seiten hin antwortete Gregor: »Bin schon fertig,« und bemühte sich,
durch die sorgfältigste Aussprache und durch Einschaltung von langen
Pausen zwischen den einzelnen Worten seiner Stimme alles Auffallende zu
nehmen. Der Vater kehrte auch zu seinem Frühstück zurück, die Schwester
aber flüsterte: »Gregor, mach auf, ich beschwöre dich.« Gregor aber
dachte gar nicht daran aufzumachen, sondern lobte die vom Reisen her
übernommene Vorsicht, auch zu Hause alle Türen während der Nacht zu
versperren.

Zunächst wollte er ruhig und ungestört aufstehen, sich anziehen und vor
allem frühstücken, und dann erst das Weitere überlegen, denn, das merkte
er wohl, im Bett würde er mit dem Nachdenken zu keinem vernünftigen Ende
kommen. Er erinnerte sich, schon öfters im Bett irgendeinen vielleicht
durch ungeschicktes Liegen erzeugten, leichten Schmerz empfunden zu
haben, der sich dann beim Aufstehen als reine Einbildung herausstellte,
und er war gespannt, wie sich seine heutigen Vorstellungen allmählich
auflösen würden. Daß die Veränderung der Stimme nichts anderes war als
der Vorbote einer tüchtigen Verkühlung, einer Berufskrankheit der
Reisenden, daran zweifelte er nicht im geringsten.

Die Decke abzuwerfen war ganz einfach; er brauchte sich nur ein wenig
aufzublasen und sie fiel von selbst. Aber weiterhin wurde es schwierig,
besonders weil er so ungemein breit war. Er hätte Arme und Hände\est\
gebraucht, um sich aufzurichten; statt dessen aber hatte er nur die
vielen Beinchen, die ununterbrochen in der verschiedensten Bewegung
waren und die er überdies nicht beherrschen konnte. Wollte er eines
einmal einknicken, so war es das erste, daß er sich streckte; und gelang
es ihm endlich, mit diesem Bein das auszuführen, was er wollte, so
arbeiteten inzwischen alle anderen, wie freigelassen, in höchster,
schmerzlicher Aufregung. »Nur sich nicht im Bett unnütz aufhalten,«
sagte sich Gregor.

Zuerst wollte er mit dem unteren Teil seines Körpers aus dem Bett
hinauskommen, aber dieser untere Teil, den er übrigens noch nicht
gesehen hatte und von dem er sich auch keine rechte Vorstellung machen
konnte, erwies sich als zu schwer beweglich; es ging so langsam; und als
er schließlich, fast wild geworden, mit gesammelter Kraft, ohne
Rücksicht sich vorwärtsstieß, hatte er die Richtung falsch gewählt,
schlug an den unteren Bettpfosten heftig an, und der brennende Schmerz,
den er empfand, belehrte ihn, daß gerade der untere Teil seines Körpers
augenblicklich vielleicht der empfindlichste war.

Er versuchte es daher, zuerst den Oberkörper aus dem Bett zu bekommen,
und drehte vorsichtig den Kopf dem Bettrand zu. Dies gelang auch leicht,
und trotz ihrer Breite und Schwere folgte schließlich die Körpermasse
langsam der Wendung des Kopfes. Aber als er den Kopf endlich außerhalb
des Bettes in der freien Luft hielt, bekam er Angst, weiter auf diese
Weise vorzurücken, denn wenn er sich schließlich so fallen ließ, mußte
geradezu ein Wunder geschehen wenn der Kopf nicht verletzt werden
sollte. Und die Besinnung durfte er gerade jetzt um keinen Preis
verlieren; lieber wollte er im Bett bleiben.

Aber als er wieder nach gleicher Mühe aufseufzend so dalag wie früher,
und wieder seine Beinchen womöglich noch ärger gegeneinander kämpfen sah
und keine Möglichkeit fand, in diese Willkür Ruhe und Ordnung zu
bringen, sagte er sich wieder, daß er unmöglich im Bett bleiben könne
und daß es das Vernünftigste sei, alles zu opfern, wenn auch nur die
kleinste Hoffnung bestünde, sich dadurch vom Bett zu befreien.
Gleichzeitig aber vergaß er nicht, sich zwischendurch daran zu erinnern,
daß viel besser als verzweifelte Entschlüsse ruhige und ruhigste
Überlegung sei. In solchen Augenblicken richtete er die Augen möglichst
scharf auf das Fenster, aber leider war aus dem Anblick des
Morgennebels, der sogar die andere Seite der engen Straße verhüllte,
wenig Zuversicht und Munterkeit zu holen. »Schon sieben Uhr,« sagte er
sich beim neuerlichen Schlagen des Weckers, »schon sieben Uhr und noch
immer ein solcher Nebel.« Und ein Weilchen lang lag er ruhig mit
schwachem Atem, als erwarte er vielleicht von der völligen Stille die
Wiederkehr der wirklichen und selbstverständlichen Verhältnisse.

Dann aber sagte er sich: »Ehe es einviertel acht schlägt, muß ich
unbedingt das Bett vollständig verlassen haben. Im übrigen wird auch bis
dahin jemand aus dem Geschäft kommen, um nach mir zu fragen, denn das
Geschäft wird vor sieben Uhr geöffnet.« Und er machte sich nun daran,
den Körper in seiner ganzen Länge vollständig gleichmäßig aus dem Bett
hinauszuschaukeln. Wenn er sich auf diese Weise aus dem Bett fallen
ließ, blieb der Kopf, den er beim Fall scharf heben wollte,
voraussichtlich unverletzt. Der Rücken schien hart zu sein; dem würde
wohl bei dem Fall auf den Teppich nichts geschehen. Das größte Bedenken
machte ihm die Rücksicht auf den lauten Krach, den es geben müßte und
der wahrscheinlich hinter allen Türen wenn nicht Schrecken, so doch
Besorgnisse erregen würde. Das mußte aber gewagt werden.

Als Gregor schon zur Hälfte aus dem Bette ragte --- die neue Methode war
mehr ein Spiel als eine Anstrengung, er brauchte immer nur ruckweise zu
schaukeln ---, fiel ihm ein, wie einfach alles wäre, wenn man ihm zu
Hilfe käme. Zwei starke Leute --- er dachte an seinen Vater und das
Dienstmädchen --- hätten vollständig genügt; sie hätten ihre Arme nur
unter seinen gewölbten Rücken schieben, ihn so aus dem Bett schälen,
sich mit der Last niederbeugen und dann bloß vorsichtig dulden müssen,
daß er den Überschwung auf dem Fußboden vollzog, wo dann die Beinchen
hoffentlich einen Sinn bekommen würden. Nun, ganz abgesehen davon, daß
die Türen versperrt waren, hätte er wirklich um Hilfe rufen sollen?
Trotz aller Not konnte er bei diesem Gedanken ein Lächeln nicht
unterdrücken.

Schon war er so weit, daß er bei stärkerem Schaukeln kaum das
Gleichgewicht noch erhielt, und sehr bald mußte er sich nun endgültig
entscheiden, denn es war in fünf Minuten einviertel acht, --- als es an
der Wohnungstür läutete. »Das ist jemand aus dem Geschäft,« sagte er
sich und erstarrte fast, während seine Beinchen nur desto eiliger
tanzten. Einen Augenblick blieb alles still. »Sie öffnen nicht,« sagte
sich Gregor, befangen in irgendeiner unsinnigen Hoffnung. Aber dann ging
natürlich wie immer das Dienstmädchen festen Schrittes zur Tür und
öffnete. Gregor brauchte nur das erste Grußwort des Besuchers zu hören
und wußte schon, wer es war --- der Prokurist selbst. Warum war nur
Gregor dazu verurteilt, bei einer Firma zu dienen, wo man bei der
kleinsten Versäumnis gleich den größten Verdacht faßte? Waren denn alle
Angestellten samt und sonders Lumpen, gab es denn unter ihnen keinen
treuen ergebenen Menschen, den, wenn er auch nur ein paar Morgenstunden
für das Geschäft nicht ausgenützt hatte, vor Gewissensbissen närrisch
wurde und geradezu nicht imstande war, das Bett zu verlassen? Genügte es
wirklich nicht, einen Lehrjungen nachfragen zu lassen --- wenn überhaupt
diese Fragerei nötig war ---, mußte da der Prokurist selbst kommen, und
mußte dadurch der ganzen unschuldigen Familie gezeigt werden, daß die
Untersuchung dieser verdächtigen Angelegenheit nur dem Verstand des
Prokuristen anvertraut werden konnte? Und mehr infolge der Erregung, in
welche Gregor durch diese Überlegungen versetzt wurde, als infolge eines
richtigen Entschlusses, schwang er sich mit aller Macht aus dem Bett. Es
gab einen lauten Schlag, aber ein eigentlicher Krach war es nicht. Ein
wenig wurde der Fall durch den Teppich abgeschwächt, auch war der Rücken
elastischer, als Gregor gedacht hatte, daher kam der nicht gar so
auffallende dumpfe Klang. Nur den Kopf hatte er nicht vorsichtig genug
gehalten und ihn angeschlagen; er drehte ihn und rieb ihn an dem Teppich
vor Ärger und Schmerz.

»Da drin ist etwas gefallen,« sagte der Prokurist im Nebenzimmer links.
Gregor suchte sich vorzustellen, ob nicht auch einmal dem Prokuristen
etwas Ähnliches passieren könnte, wie heute ihm; die Möglichkeit dessen
mußte man doch eigentlich zugeben. Aber wie zur rohen Antwort auf diese
Frage machte jetzt der Prokurist im Nebenzimmer ein paar bestimmte
Schritte und ließ seine Lackstiefel knarren. Aus dem Nebenzimmer rechts
flüsterte die Schwester, um Gregor zu verständigen: »Gregor, der
Prokurist ist da.« »Ich weiß,« sagte Gregor vor sich hin; aber so laut,
daß es die Schwester hätte hören können, wagte er die Stimme nicht zu
erheben.

»Gregor,« sagte nun der Vater aus dem Nebenzimmer links, »der Herr
Prokurist ist gekommen und erkundigt sich, warum du nicht mit dem
Frühzug weggefahren bist. Wir wissen nicht, was wir ihm sagen sollen.
Übrigens will er auch mit dir persönlich sprechen. Also bitte mach die
Tür auf. Er wird die Unordnung im Zimmer zu entschuldigen schon die Güte
haben.« »Guten Morgen, Herr Samsa,« rief der Prokurist freundlich
dazwischen. »Ihm ist nicht wohl,« sagte die Mutter zum Prokuristen,
während der Vater noch an der Tür redete, »ihm ist nicht wohl, glauben
Sie mir, Herr Prokurist. Wie würde denn Gregor sonst einen Zug
versäumen! Der Junge hat ja nichts im Kopf als das Geschäft. Ich ärgere
mich schon fast, daß er abends niemals ausgeht; jetzt war er doch acht
Tage in der Stadt, aber jeden Abend war er zu Hause. Da sitzt er bei uns
am Tisch und liest still die Zeitung oder studiert Fahrpläne. Es ist
schon eine Zerstreuung für ihn, wenn er sich mit Laubsägearbeiten
beschäftigt. Da hat er zum Beispiel im Laufe von zwei, drei Abenden
einen kleinen Rahmen geschnitzt; Sie werden staunen, wie hübsch er ist;
er hängt drin im Zimmer; Sie werden ihn gleich sehen, wenn Gregor
aufmacht. Ich bin übrigens glücklich, daß Sie da sind, Herr Prokurist;
wir allein hätten Gregor nicht dazu gebracht, die Tür zu öffnen; er ist
so hartnäckig; und bestimmt ist ihm nicht wohl, trotzdem er es am Morgen
geleugnet hat.« »Ich komme gleich,« sagte Gregor langsam und bedächtig
und rührte sich nicht, um kein Wort der Gespräche zu verlieren. »Anders,
gnädige Frau, kann ich es mir auch nicht erklären,« sagte der Prokurist,
»hoffentlich ist es nichts Ernstes. Wenn ich auch andererseits sagen
muß, daß wir Geschäftsleute --- wie man will, leider oder
glücklicherweise --- ein leichtes Unwohlsein sehr oft aus geschäftlichen
Rücksichten einfach überwinden müssen.« »Also kann der Herr Prokurist
schon zu dir hinein?« fragte der ungeduldige Vater und klopfte wiederum
an die Tür. »Nein,« sagte Gregor. Im Nebenzimmer links trat eine
peinliche Stille ein, im Nebenzimmer rechts begann die Schwester zu
schluchzen.

Warum ging denn die Schwester nicht zu den anderen? Sie war wohl erst
jetzt aus dem Bett aufgestanden und hatte noch gar nicht angefangen sich
anzuziehen. Und warum weinte sie denn? Weil er nicht aufstand und den
Prokuristen nicht hereinließ, weil er in Gefahr war, den Posten zu
verlieren und weil dann der Chef die Eltern mit den alten Forderungen
wieder verfolgen würde? Das waren doch vorläufig wohl unnötige Sorgen.
Noch war Gregor hier und dachte nicht im geringsten daran, seine Familie
zu verlassen. Augenblicklich lag er wohl da auf dem Teppich, und
niemand, der seinen Zustand gekannt hätte, hätte im Ernst von ihm
verlangt, daß er den Prokuristen hereinlasse. Aber wegen dieser kleinen
Unhöflichkeit, für die sich ja später leicht eine passende Ausrede
finden würde, konnte Gregor doch nicht gut sofort weggeschickt werden.
Und Gregor schien es, daß es viel vernünftiger wäre, ihn jetzt in Ruhe
zu lassen, statt ihn mit Weinen und Zureden zu stören. Aber es war eben
die Ungewißheit, welche die anderen bedrängte und ihr Benehmen
entschuldigte.

»Herr Samsa,« rief nun der Prokurist mit erhobener Stimme, »was ist denn
los? Sie verbarrikadieren sich da in Ihrem Zimmer, antworten bloß mit ja
und nein, machen Ihren Eltern schwere, unnötige Sorgen und versäumen ---
dies nur nebenbei erwähnt --- Ihre geschäftlichen Pflichten in einer
eigentlich unerhörten Weise. Ich spreche hier im Namen Ihrer Eltern und
Ihres Chefs und bitte Sie ganz ernsthaft um eine augenblickliche,
deutliche Erklärung. Ich staune, ich staune. Ich glaubte Sie als einen
ruhigen, vernünftigen Menschen zu kennen, und nun scheinen Sie plötzlich
anfangen zu wollen, mit sonderbaren Launen zu paradieren. Der Chef
deutete mir zwar heute früh eine mögliche Erklärung für Ihre Versäumnis
an --- sie betraf das Ihnen seit kurzem anvertraute Inkasso ---, aber ich
legte wahrhaftig fast mein Ehrenwort dafür ein, daß diese Erklärung
nicht zutreffen könne. Nun aber sehe ich hier Ihren unbegreiflichen
Starrsinn und verliere ganz und gar jede Lust, mich auch nur im
geringsten für Sie einzusetzen. Und Ihre Stellung ist durchaus nicht die
festeste. Ich hatte ursprünglich die Absicht, Ihnen das alles unter vier
Augen zu sagen, aber da Sie mich hier nutzlos meine Zeit versäumen
lassen, weiß ich nicht, warum es nicht auch Ihre Herren Eltern erfahren
sollen. Ihre Leistungen in der letzten Zeit waren also sehr
unbefriedigend; es ist zwar nicht die Jahreszeit, um besondere Geschäfte
zu machen, das erkennen wir an; aber eine Jahreszeit, um keine Geschäfte
zu machen, gibt es überhaupt nicht, Herr Samsa, darf es nicht geben.«

»Aber Herr Prokurist,« rief Gregor außer sich und vergaß in der
Aufregung alles andere, »ich mache ja sofort, augenblicklich auf. Ein
leichtes Unwohlsein, ein Schwindelanfall, haben mich verhindert
aufzustehen. Ich liege noch jetzt im Bett. Jetzt bin ich aber schon
wieder ganz frisch. Eben steige ich aus dem Bett. Nur einen kleinen
Augenblick Geduld! Es geht noch nicht so gut, wie ich dachte. Es ist mir
aber schon wohl. Wie das nur einen Menschen so überfallen kann! Noch
gestern abend war mir ganz gut, meine Eltern wissen es ja, oder besser,
schon gestern abend hatte ich eine kleine Vorahnung. Man hätte es mir
ansehen müssen. Warum habe ich es nur im Geschäfte nicht gemeldet! Aber
man denkt eben immer, daß man die Krankheit ohne Zuhausebleiben
überstehen wird. Herr Prokurist! Schonen Sie meine Eltern! Für alle die
Vorwürfe, die Sie mir jetzt machen, ist ja kein Grund; man hat mir ja
davon auch kein Wort gesagt. Sie haben vielleicht die letzten Aufträge,
die ich geschickt habe, nicht gelesen. Übrigens, noch mit dem Achtuhrzug
fahre ich auf die Reise, die paar Stunden Ruhe haben mich gekräftigt.
Halten Sie sich nur nicht auf, Herr Prokurist; ich bin gleich selbst im
Geschäft, und haben Sie die Güte, das zu sagen und mich dem Herrn Chef
zu empfehlen!«

Und während Gregor dies alles hastig ausstieß und kaum wußte, was er
sprach, hatte er sich leicht, wohl infolge der im Bett bereits erlangten
Übung, dem Kasten genähert und versuchte nun, an ihm sich aufzurichten.
Er wollte tatsächlich die Tür aufmachen, tatsächlich sich sehen lassen
und mit dem Prokuristen sprechen; er war begierig zu erfahren, was die
anderen, die jetzt so nach ihm verlangten, bei seinem Anblick sagen
würden. Würden sie erschrecken, dann hatte Gregor keine Verantwortung
mehr und konnte ruhig sein. Würden sie aber alles ruhig hinnehmen, dann
hatte auch er keinen Grund sich aufzuregen, und konnte, wenn er sich
beeilte, um acht Uhr tatsächlich auf dem Bahnhof sein. Zuerst glitt er
nun einigemale von dem glatten Kasten ab, aber endlich gab er sich
einen letzten Schwung und stand aufrecht da; auf die Schmerzen im
Unterleib achtete er gar nicht mehr, so sehr sie auch brannten. Nun ließ
er sich gegen die Rücklehne eines nahen Stuhles fallen, an deren Rändern
er sich mit seinen Beinchen festhielt. Damit hatte er aber auch die
Herrschaft über sich erlangt und verstummte, denn nun konnte er den
Prokuristen anhören.

»Haben Sie auch nur ein Wort verstanden?« fragte der Prokurist die
Eltern, »er macht sich doch wohl nicht einen Narren aus uns?« »Um Gottes
willen,« rief die Mutter schon unter Weinen, »er ist vielleicht schwer
krank, und wir quälen ihn. Grete! Grete!« schrie sie dann. »Mutter?«
rief die Schwester von der anderen Seite. Sie verständigten sich durch
Gregors Zimmer. »Du mußt augenblicklich zum Arzt. Gregor ist krank.
Rasch um den Arzt. Hast du Gregor jetzt reden hören?« »Das war eine
Tierstimme,« sagte der Prokurist, auffallend leise gegenüber dem
Schreien der Mutter. »Anna! Anna!« rief der Vater durch das Vorzimmer in
die Küche und klatschte in die Hände, »sofort einen Schlosser holen!«
Und schon liefen die zwei Mädchen mit rauschenden Röcken durch das
Vorzimmer --- wie hatte sich die Schwester denn so schnell angezogen? ---
und rissen die Wohnungstüre auf. Man hörte gar nicht die Türe
zuschlagen; sie hatten sie wohl offen gelassen, wie es in Wohnungen zu
sein pflegt, in denen ein großes Unglück geschehen ist.

Gregor war aber viel ruhiger geworden. Man verstand zwar also seine
Worte nicht mehr, trotzdem sie ihm genug klar, klarer als früher,
vorgekommen waren, vielleicht infolge der Gewöhnung des Ohres. Aber
immerhin glaubte man nun schon daran, daß es mit ihm nicht ganz in
Ordnung war, und war bereit, ihm zu helfen. Die Zuversicht und
Sicherheit, womit die ersten Anordnungen getroffen worden waren, taten
ihm wohl. Er fühlte sich wieder einbezogen in den menschlichen Kreis und
erhoffte von beiden, vom Arzt und vom Schlosser, ohne sie eigentlich
genau zu scheiden, großartige und überraschende Leistungen. Um für die
sich nähernden entscheidenden Besprechungen eine möglichst klare Stimme
zu bekommen, hustete er ein wenig ab, allerdings bemüht, dies ganz
gedämpft zu tun, da möglicherweise auch schon dieses Geräusch anders als
menschlicher Husten klang, was er selbst zu entscheiden sich nicht mehr
getraute. Im Nebenzimmer war es inzwischen ganz still geworden.
Vielleicht saßen die Eltern mit dem Prokuristen beim Tisch und
tuschelten, vielleicht lehnten alle an der Türe und horchten.

Gregor schob sich langsam mit dem Sessel zur Tür hin, ließ ihn dort los,
warf sich gegen die Tür, hielt sich an ihr aufrecht --- die Ballen seiner
Beinchen hatten ein wenig Klebstoff --- und ruhte sich dort einen
Augenblick lang von der Anstrengung aus. Dann aber machte er sich daran,
mit dem Mund den Schlüssel im Schloß umzudrehen. Es schien leider, daß
er keine eigentlichen Zähne hatte, --- womit sollte er gleich den
Schlüssel fassen? --- aber dafür waren die Kiefer freilich sehr stark,
mit ihrer Hilfe brachte er auch wirklich den Schlüssel in Bewegung und
achtete nicht darauf, daß er sich zweifellos irgendeinen Schaden
zufügte, denn eine braune Flüssigkeit kam ihm aus dem Mund, floß über
den Schlüssel und tropfte auf den Boden. »Hören Sie nur,« sagte der
Prokurist im Nebenzimmer, »er dreht den Schlüssel um.« Das war für
Gregor eine große Aufmunterung; aber alle hätten ihm zurufen sollen,
auch der Vater und die Mutter: »Frisch, Gregor,« hätten sie rufen
sollen, »immer nur heran, fest an das Schloß heran!« Und in der
Vorstellung, daß alle seine Bemühungen mit Spannung verfolgten, verbiß
er sich mit allem, was er an Kraft aufbringen konnte, besinnungslos in
den Schlüssel. Je nach dem Fortschreiten der Drehung des Schlüssels
umtanzte er das Schloß, hielt sich jetzt nur noch mit dem Munde
aufrecht, und je nach Bedarf hing er sich an den Schlüssel oder drückte
ihn dann wieder nieder mit der ganzen Last seines Körpers. Der hellere
Klang des endlich zurückschnappenden Schlosses erweckte Gregor förmlich.
Aufatmend sagte er sich: »Ich habe also den Schlosser nicht gebraucht,«
und legte den Kopf auf die Klinke, um die Türe gänzlich zu öffnen.

Da er die Türe auf diese Weise öffnen mußte, war sie eigentlich schon
recht weit geöffnet, und er selbst noch nicht zu sehen. Er mußte sich
erst langsam um den einen Türflügel herumdrehen, und zwar sehr
vorsichtig, wenn er nicht gerade vor dem Eintritt ins Zimmer plump auf
den Rücken fallen wollte. Er war noch mit jener schwierigen Bewegung
beschäftigt und hatte nicht Zeit, auf anderes zu achten, da hörte er
schon den Prokuristen ein lautes »Oh!« ausstoßen --- es klang, wie wenn
der Wind saust --- und nun sah er ihn auch, wie er, der der Nächste an
der Türe war, die Hand gegen den offenen Mund drückte und langsam
zurückwich, als vertreibe ihn eine unsichtbare, gleichmäßig fortwirkende
Kraft. Die Mutter --- sie stand hier trotz der Anwesenheit des
Prokuristen mit von der Nacht her noch aufgelösten, hoch sich
sträubenden Haaren --- sah zuerst mit gefalteten Händen den Vater an,
ging dann zwei Schritte zu Gregor hin und fiel inmitten ihrer rings um
sie herum sich ausbreitenden Röcke nieder, das Gesicht ganz unauffindbar
zu ihrer Brust gesenkt. Der Vater ballte mit feindseligem Ausdruck die
Faust, als wolle er Gregor in sein Zimmer zurückstoßen, sah sich dann
unsicher im Wohnzimmer um, beschattete dann mit den Händen die Augen und
weinte, daß sich seine mächtige Brust schüttelte.

Gregor trat nun gar nicht in das Zimmer, sondern lehnte sich von innen
an den festgeriegelten Türflügel, so daß sein Leib nur zur Hälfte und
darüber der seitlich geneigte Kopf zu sehen war, mit dem er zu den
anderen hinüberlugte. Es war inzwischen viel heller geworden; klar stand
auf der anderen Straßenseite ein Ausschnitt des gegenüberliegenden,
endlosen, grauschwarzen Hauses --- es war ein Krankenhaus --- mit seinen
hart die Front durchbrechenden regelmäßigen Fenstern; der Regen fiel
noch nieder, aber nur mit großen, einzeln sichtbaren und förmlich auch
einzelnweise auf die Erde hinuntergeworfenen Tropfen. Das
Frühstücksgeschirr stand in überreicher Zahl auf dem Tisch, denn für den
Vater war das Frühstück die wichtigste Mahlzeit des Tages, die er bei
der Lektüre verschiedener Zeitungen stundenlang hinzog. Gerade an der
gegenüberliegenden Wand hing eine Photographie Gregors aus seiner
Militärzeit, die ihn als Leutnant darstellte, wie er, die Hand am Degen,
sorglos lächelnd, Respekt für seine Haltung und Uniform verlangte. Die
Tür zum Vorzimmer war geöffnet, und man sah, da auch die Wohnungstür
offen war, auf den Vorplatz der Wohnung hinaus und auf den Beginn der
abwärts führenden Treppe.

»Nun,« sagte Gregor und war sich dessen wohl bewußt, daß er der einzige
war, der die Ruhe bewahrt hatte, »ich werde mich gleich anziehen, die
Kollektion zusammenpacken und wegfahren. Wollt ihr, wollt ihr mich
wegfahren lassen? Nun, Herr Prokurist, Sie sehen, ich bin nicht
starrköpfig und ich arbeite gern; das Reisen ist beschwerlich, aber ich
könnte ohne das Reisen nicht leben. Wohin gehen Sie denn, Herr
Prokurist? Ins Geschäft? Ja? Werden Sie alles wahrheitsgetreu berichten?
Man kann im Augenblick unfähig sein zu arbeiten, aber dann ist gerade
der richtige Zeitpunkt, sich an die früheren Leistungen zu erinnern und
zu bedenken, daß man später, nach Beseitigung des Hindernisses, gewiß
desto fleißiger und gesammelter arbeiten wird. Ich bin ja dem Herrn Chef
so sehr verpflichtet, das wissen Sie doch recht gut. Andererseits habe
ich die Sorge um meine Eltern und die Schwester. Ich bin in der Klemme,
ich werde mich aber auch wieder herausarbeiten. Machen Sie es mir aber
nicht schwieriger, als es schon ist. Halten Sie im Geschäft meine
Partei! Man liebt den Reisenden nicht, ich weiß. Man denkt, er verdient
ein Heidengeld und führt dabei ein schönes Leben. Man hat eben keine
besondere Veranlassung, dieses Vorurteil besser zu durchdenken. Sie
aber, Herr Prokurist, Sie haben einen besseren Überblick über die
Verhältnisse, als das sonstige Personal, ja sogar, ganz im Vertrauen
gesagt, einen besseren Überblick, als der Herr Chef selbst, der in
seiner Eigenschaft als Unternehmer sich in seinem Urteil leicht
zuungunsten eines Angestellten beirren läßt. Sie wissen auch sehr wohl,
daß der Reisende, der fast das ganze Jahr außerhalb des Geschäftes ist,
so leicht ein Opfer von Klatschereien, Zufälligkeiten und grundlosen
Beschwerden werden kann, gegen die sich zu wehren ihm ganz unmöglich
ist, da er von ihnen meistens gar nichts erfährt und nur dann, wenn er
erschöpft eine Reise beendet hat, zu Hause die schlimmen, auf ihre
Ursachen hin nicht mehr zu durchschauenden Folgen am eigenen Leibe zu
spüren bekommt. Herr Prokurist, gehen Sie nicht weg, ohne mir ein Wort
gesagt zu haben, das mir zeigt, daß Sie mir wenigstens zu einem kleinen
Teil recht geben!«

Aber der Prokurist hatte sich schon bei den ersten Worten Gregors
abgewendet, und nur über die zuckende Schulter hinweg sah er mit
aufgeworfenen Lippen nach Gregor zurück. Und während Gregors Rede stand
er keinen Augenblick still, sondern verzog sich, ohne Gregor aus den
Augen zu lassen, gegen die Tür, aber ganz allmählich, als bestehe ein
geheimes Verbot, das Zimmer zu verlassen. Schon war er im Vorzimmer, und
nach der plötzlichen Bewegung, mit der er zum letztenmal den Fuß aus dem
Wohnzimmer zog, hätte man glauben können, er habe sich soeben die Sohle
verbrannt. Im Vorzimmer aber streckte er die rechte Hand weit von sich
zur Treppe hin, als warte dort auf ihn eine geradezu überirdische
Erlösung.

Gregor sah ein, daß er den Prokuristen in dieser Stimmung auf keinen
Fall weggehen lassen dürfe, wenn dadurch seine Stellung im Geschäft
nicht aufs äußerste gefährdet werden sollte. Die Eltern verstanden das
alles nicht so gut; sie hatten sich in den langen Jahren die Überzeugung
gebildet, daß Gregor in diesem Geschäft für sein Leben versorgt war, und
hatten außerdem jetzt mit den augenblicklichen Sorgen so viel zu tun,
daß ihnen jede Voraussicht abhanden gekommen war. Aber Gregor hatte
diese Voraussicht. Der Prokurist mußte gehalten, beruhigt, überzeugt und
schließlich gewonnen werden; die Zukunft Gregors und seiner Familie hing
doch davon ab! Wäre doch die Schwester hier gewesen! Sie war klug; sie
hatte schon geweint, als Gregor noch ruhig auf dem Rücken lag. Und gewiß
hätte der Prokurist, dieser Damenfreund, sich von ihr lenken lassen;
sie hätte die Wohnungstür zugemacht und ihm im Vorzimmer den Schrecken
ausgeredet. Aber die Schwester war eben nicht da, Gregor selbst mußte
handeln. Und ohne daran zu denken, daß er seine gegenwärtigen
Fähigkeiten, sich zu bewegen, noch gar nicht kannte, ohne auch daran zu
denken, daß seine Rede möglicher- ja wahrscheinlicherweise wieder nicht
verstanden worden war, verließ er den Türflügel; schob sich durch die
Öffnung; wollte zum Prokuristen hingehen, der sich schon am Geländer des
Vorplatzes lächerlicherweise mit beiden Händen festhielt; fiel aber
sofort, nach einem Halt suchend, mit einem kleinen Schrei auf seine
vielen Beinchen nieder. Kaum war das geschehen, fühlte er zum erstenmal
an diesem Morgen ein körperliches Wohlbehagen; die Beinchen hatten
festen Boden unter sich; sie gehorchten vollkommen, wie er zu seiner
Freude merkte; strebten sogar darnach, ihn fortzutragen, wohin er
wollte; und schon glaubte er, die endgültige Besserung alles Leidens
stehe unmittelbar bevor. Aber im gleichen Augenblick, als er da
schaukelnd vor verhaltener Bewegung, gar nicht weit von seiner Mutter
entfernt, ihr gerade gegenüber auf dem Boden lag, sprang diese, die doch
so ganz in sich versunken schien, mit einemmale in die Höhe, die Arme
weit ausgestreckt, die Finger gespreizt, rief: »Hilfe, um Gottes willen
Hilfe!«, hielt den Kopf geneigt, als wolle sie Gregor besser sehen, lief
aber, im Widerspruch dazu, sinnlos zurück; hatte vergessen, daß hinter
ihr der gedeckte Tisch stand; setzte sich, als sie bei ihm angekommen
war, wie in Zerstreutheit, eilig auf ihn, und schien gar nicht zu
merken, daß neben ihr aus der umgeworfenen großen Kanne der Kaffee in
vollem Strome auf den Teppich sich ergoß.

»Mutter, Mutter,« sagte Gregor leise und sah zu ihr hinauf. Der
Prokurist war ihm für einen Augenblick ganz aus dem Sinn gekommen;
dagegen konnte er sich nicht versagen, im Anblick des fließenden Kaffees
mehrmals mit den Kiefern ins Leere zu schnappen. Darüber schrie die
Mutter neuerdings auf, flüchtete vom Tisch und fiel dem ihr
entgegeneilenden Vater in die Arme. Aber Gregor hatte jetzt keine Zeit
für seine Eltern; der Prokurist war schon auf der Treppe; das Kinn auf
dem Geländer, sah er noch zum letzten Male zurück. Gregor nahm einen
Anlauf, um ihn möglichst sicher einzuholen; der Prokurist mußte etwas
ahnen, denn er machte einen Sprung über mehrere Stufen und verschwand;
»Huh!« aber schrie er noch, es klang durchs ganze Treppenhaus. Leider
schien nun auch diese Flucht des Prokuristen den Vater, der bisher
verhältnismäßig gefaßt gewesen war, völlig zu verwirren, denn statt
selbst dem Prokuristen nachzulaufen oder wenigstens Gregor in der
Verfolgung nicht zu hindern, packte er mit der Rechten den Stock des
Prokuristen, den dieser mit Hut und Überzieher auf einem Sessel
zurückgelassen hatte, holte mit der Linken eine große Zeitung vom Tisch
und machte sich unter Füßestampfen daran, Gregor durch Schwenken des
Stockes und der Zeitung in sein Zimmer zurückzutreiben. Kein Bitten
Gregors half, kein Bitten wurde auch verstanden, er mochte den Kopf noch
so demütig drehen, der Vater stampfte nur stärker mit den Füßen. Drüben
hatte die Mutter trotz des kühlen Wetters ein Fenster aufgerissen, und
hinausgelehnt drückte sie ihr Gesicht weit außerhalb des Fensters in
ihre Hände. Zwischen Gasse und Treppenhaus entstand eine starke Zugluft,
die Fenstervorhänge flogen auf, die Zeitungen auf dem Tische rauschten,
einzelne Blätter wehten über den Boden hin. Unerbittlich drängte der
Vater und stieß Zischlaute aus, wie ein Wilder. Nun hatte aber Gregor
noch gar keine Übung im Rückwärtsgehen, es ging wirklich sehr langsam.
Wenn sich Gregor nur hätte umdrehen dürfen, er wäre gleich in seinem
Zimmer gewesen, aber er fürchtete sich, den Vater durch die zeitraubende
Umdrehung ungeduldig zu machen, und jeden Augenblick drohte ihm doch von
dem Stock in des Vaters Hand der tödliche Schlag auf den Rücken oder auf
den Kopf. Endlich aber blieb Gregor doch nichts anderes übrig, denn er
merkte mit Entsetzen, daß er im Rückwärtsgehen nicht einmal die Richtung
einzuhalten verstand; und so begann er, unter unaufhörlichen ängstlichen
Seitenblicken nach dem Vater, sich nach Möglichkeit rasch, in
Wirklichkeit aber doch nur sehr langsam umzudrehen. Vielleicht merkte
der Vater seinen guten Willen, denn er störte ihn hierbei nicht, sondern
dirigierte sogar hie und da die Drehbewegung von der Ferne mit der
Spitze seines Stockes. Wenn nur nicht dieses unerträgliche Zischen des
Vaters gewesen wäre! Gregor verlor darüber ganz den Kopf. Er war schon
fast ganz umgedreht, als er sich, immer auf dieses Zischen horchend,
sogar irrte und sich wieder ein Stück zurückdrehte. Als er aber endlich
glücklich mit dem Kopf vor der Türöffnung war, zeigte es sich, daß sein
Körper zu breit war, um ohne weiteres durchzukommen. Dem Vater fiel es
natürlich in seiner gegenwärtigen Verfassung auch nicht entfernt ein,
etwa den anderen Türflügel zu öffnen, um für Gregor einen genügenden
Durchgang zu schaffen. Seine fixe Idee war bloß, daß Gregor so rasch als
möglich in sein Zimmer müsse. Niemals hätte er auch die umständlichen
Vorbereitungen gestattet, die Gregor brauchte, um sich aufzurichten und
vielleicht auf diese Weise durch die Tür zu kommen. Vielleicht trieb er,
als gäbe es kein Hindernis, Gregor jetzt unter besonderem Lärm
vorwärts; es klang schon hinter Gregor gar nicht mehr wie die Stimme
bloß eines einzigen Vaters; nun gab es wirklich keinen Spaß mehr, und
Gregor drängte sich --- geschehe was wolle --- in die Tür. Die eine Seite
seines Körpers hob sich, er lag schief in der Türöffnung, seine eine
Flanke war ganz wundgerieben, an der weißen Tür blieben häßliche Flecke,
bald steckte er fest und hätte sich allein nicht mehr rühren können, die
Beinchen auf der einen Seite hingen zitternd oben in der Luft, die auf
der anderen waren schmerzhaft zu Boden gedrückt --- da gab ihm der Vater
von hinten einen jetzt wahrhaftig erlösenden starken Stoß, und er flog,
heftig blutend, weit in sein Zimmer hinein. Die Tür wurde noch mit dem
Stock zugeschlagen, dann war es endlich still.

\pagebreak

\vspace*{2.5cm}

\section{II}

\noindent{}Erst in der Abenddämmerung erwachte Gregor aus seinem schweren
ohnmachtähnlichen Schlaf. Er wäre gewiß nicht viel später auch ohne
Störung erwacht, denn er fühlte sich genügend ausgeruht und
ausgeschlafen, doch schien es ihm, als hätte ihn ein flüchtiger Schritt
und ein vorsichtiges Schließen der zum Vorzimmer führenden Tür geweckt.
Der Schein der elektrischen Straßenbahn lag bleich hier und da auf der
Zimmerdecke und auf den höheren Teilen der Möbel, aber unten bei Gregor
war es finster. Langsam schob er sich, noch ungeschickt mit seinen
Fühlern tastend, die er jetzt erst schätzen lernte, zur Türe hin, um
nachzusehen, was dort geschehen war. Seine linke Seite schien eine
einzige lange, unangenehm spannende Narbe, und er mußte auf seinen zwei
Beinreihen regelrecht hinken. Ein Beinchen war übrigens im Laufe der
vormittägigen Vorfälle schwer verletzt worden --- es war fast ein
Wunder, daß nur eines verletzt worden war --- und schleppte leblos nach.

Erst bei der Tür merkte er, was ihn dorthin eigentlich gelockt hatte; es
war der Geruch von etwas\est\ Eßbarem gewesen. Denn dort stand ein Napf mit
süßer Milch gefüllt, in der kleine Schnitte von Weißbrot schwammen. Fast
hätte er vor Freude gelacht, denn er hatte noch größeren Hunger als am
Morgen, und gleich tauchte er seinen Kopf fast bis über die Augen in die
Milch hinein. Aber bald zog er ihn enttäuscht wieder zurück; nicht nur,
daß ihm das Essen wegen seiner heiklen linken Seite Schwierigkeiten
machte --- und er konnte nur essen, wenn der ganze Körper schnaufend
mitarbeitete ---, so schmeckte ihm überdies die Milch, die sonst sein
Lieblingsgetränk war und die ihm gewiß die Schwester deshalb
hereingestellt hatte, gar nicht, ja er wandte sich fast mit Widerwillen
von dem Napf ab und kroch in die Zimmermitte zurück.

Im Wohnzimmer war, wie Gregor durch die Türspalte sah, das Gas
angezündet, aber während sonst zu dieser Tageszeit der Vater seine
nachmittags erscheinende Zeitung der Mutter und manchmal auch der
Schwester mit erhobener Stimme vorzulesen pflegte, hörte man jetzt
keinen Laut. Nun vielleicht war dieses Vorlesen, von dem ihm die
Schwester immer erzählte und schrieb, in der letzten Zeit überhaupt aus
der Übung gekommen. Aber auch ringsherum war es so still, trotzdem doch
gewiß die Wohnung nicht leer war. »Was für ein stilles Leben die Familie
doch führte,« sagte sich Gregor und fühlte, während er starr vor sich
ins Dunkle sah, einen großen Stolz darüber, daß er seinen Eltern und
seiner Schwester ein solches Leben in einer so schönen Wohnung hatte
verschaffen können. Wie aber, wenn jetzt alle Ruhe, aller Wohlstand,
alle Zufriedenheit ein Ende mit Schrecken nehmen sollte? Um sich nicht
in solche Gedanken zu verlieren, setzte sich Gregor lieber in Bewegung
und kroch im Zimmer auf und ab.

Einmal während des langen Abends wurde die eine Seitentüre und einmal
die andere bis zu einer kleinen Spalte geöffnet und rasch wieder
geschlossen; jemand hatte wohl das Bedürfnis hereinzukommen, aber auch
wieder zu viele Bedenken. Gregor machte nun unmittelbar bei der
Wohnzimmertür Halt, entschlossen, den zögernden Besucher doch irgendwie
hereinzubringen oder doch wenigstens zu erfahren, wer es sei; aber nun
wurde die Tür nicht mehr geöffnet und Gregor wartete vergebens. Früh,
als die Türen versperrt waren, hatten alle zu ihm hereinkommen wollen,
jetzt, da er die eine Tür geöffnet hatte und die anderen offenbar
während des Tages geöffnet worden waren, kam keiner mehr, und die
Schlüssel steckten nun auch von außen.

Spät erst in der Nacht wurde das Licht im Wohnzimmer ausgelöscht, und
nun war leicht festzustellen, daß die Eltern und die Schwester so lange
wachgeblieben waren, denn wie man genau hören konnte, entfernten sich
jetzt alle drei auf den Fußspitzen. Nun kam gewiß bis zum Morgen niemand
mehr zu Gregor herein; er hatte also eine lange Zeit, um ungestört zu
überlegen, wie er sein Leben jetzt neu ordnen sollte. Aber das hohe
freie Zimmer, in dem er gezwungen war, flach auf dem Boden zu liegen,
ängstigte ihn, ohne daß er die Ursache herausfinden konnte, denn es war
ja sein seit fünf Jahren von ihm bewohntes Zimmer --- und mit einer halb
unbewußten Wendung und nicht ohne eine leichte Scham eilte er unter das
Kanapee, wo er sich, trotzdem sein Rücken ein wenig gedrückt wurde und
trotzdem er den Kopf nicht mehr erheben konnte, gleich sehr behaglich
fühlte und nur bedauerte, daß sein Körper zu breit war, um vollständig
unter dem Kanapee untergebracht zu werden.

Dort blieb er die ganze Nacht, die er zum Teil im Halbschlaf, aus dem
ihn der Hunger immer wieder aufschreckte, verbrachte, zum Teil aber in
Sorgen und undeutlichen Hoffnungen, die aber alle zu dem Schlusse
führten, daß er sich vorläufig ruhig verhalten und durch Geduld und
größte Rücksichtnahme der Familie die Unannehmlichkeiten erträglich
machen müsse, die er ihr in seinem gegenwärtigen Zustand nun einmal zu
verursachen gezwungen war.

Schon am frühen Morgen, es war fast noch Nacht, hatte Gregor
Gelegenheit, die Kraft seiner eben gefaßten Entschlüsse zu prüfen, denn
vom Vorzimmer her öffnete die Schwester, fast völlig angezogen, die Tür
und sah mit Spannung herein. Sie fand ihn nicht gleich, aber als sie ihn
unter dem Kanapee bemerkte --- Gott, er mußte doch irgendwo sein, er
hatte doch nicht wegfliegen können --- erschrak sie so sehr, daß sie,
ohne sich beherrschen zu können, die Tür von außen wieder zuschlug. Aber
als bereue sie ihr Benehmen, öffnete sie die Tür sofort wieder und trat,
als sei sie bei einem Schwerkranken oder gar bei einem Fremden, auf den
Fußspitzen herein. Gregor hatte den Kopf bis knapp zum Rande des
Kanapees vorgeschoben und beobachtete sie. Ob sie wohl bemerken würde,
daß er die Milch stehen gelassen hatte, und zwar keineswegs aus Mangel
an Hunger, und ob sie eine andere Speise hereinbringen würde, die ihm
besser entsprach? Täte sie es nicht von selbst, er wollte lieber
verhungern, als sie darauf aufmerksam machen, trotzdem es ihn eigentlich
ungeheuer drängte, unterm Kanapee vorzuschießen, sich der Schwester zu
Füßen zu werfen und sie um irgend etwas Gutes zum Essen zu bitten. Aber
die Schwester bemerkte sofort mit Verwunderung den noch vollen Napf, aus
dem nur ein wenig Milch ringsherum verschüttet war, sie hob ihn gleich
auf, zwar nicht mit den bloßen Händen, sondern mit einem Fetzen, und
trug ihn hinaus. Gregor war äußerst neugierig, was sie zum Ersatze
bringen würde, und er machte sich die verschiedensten Gedanken darüber.
Niemals aber hätte er erraten können, was die Schwester in ihrer Güte
wirklich tat. Sie brachte ihm, um seinen Geschmack zu prüfen, eine ganze
Auswahl, alles auf einer alten Zeitung ausgebreitet. Da war altes
halbverfaultes Gemüse; Knochen vom Nachtmahl her, die von festgewordener
weißer Sauce umgeben waren; ein paar Rosinen und Mandeln; ein Käse, den
Gregor vor zwei Tagen für ungenießbar erklärt hatte; ein trockenes Brot,
ein mit Butter beschmiertes Brot und ein mit Butter beschmiertes und
gesalzenes Brot. Außerdem stellte sie zu dem allen noch den
wahrscheinlich ein für allemal für Gregor bestimmten Napf, in den sie
Wasser gegossen hatte. Und aus Zartgefühl, da sie wußte, daß Gregor vor
ihr nicht essen würde, entfernte sie sich eiligst und drehte sogar den
Schlüssel um, damit nur Gregor merken könne, daß er es sich so behaglich
machen dürfe, wie er wolle. Gregors Beinchen schwirrten, als es jetzt
zum Essen ging. Seine Wunden mußten übrigens auch schon vollständig
geheilt sein, er fühlte keine Behinderung mehr, er staunte darüber und
dachte daran, wie er vor mehr als einem Monat sich mit dem Messer ganz
wenig in den Finger geschnitten, und wie ihm diese Wunde noch vorgestern
genug wehgetan hatte. »Sollte ich jetzt weniger Feingefühl haben?«
dachte er und saugte schon gierig an dem Käse, zu dem es ihn vor allen
anderen Speisen sofort und nachdrücklich gezogen hatte. Rasch
hintereinander und mit vor Befriedigung tränenden Augen verzehrte er den
Käse, das Gemüse und die Sauce; die frischen Speisen dagegen schmeckten
ihm nicht, er konnte nicht einmal ihren Geruch vertragen und schleppte
sogar die Sachen, die er essen wollte, ein Stückchen weiter weg. Er war
schon längst mit allem fertig und lag nur noch faul auf der gleichen
Stelle, als die Schwester zum Zeichen, daß er sich zurückziehen solle,
langsam den Schlüssel umdrehte. Das schreckte ihn sofort auf, trotzdem
er schon fast schlummerte, und er eilte wieder unter das Kanapee. Aber
es kostete ihn große Selbstüberwindung, auch nur die kurze Zeit, während
welcher die Schwester im Zimmer war, unter dem Kanapee zu bleiben, denn
von dem reichlichen Essen hatte sich sein Leib ein wenig gerundet, und
er konnte dort in der Enge kaum atmen. Unter kleinen Erstickungsanfällen
sah er mit etwas hervorgequollenen Augen zu, wie die nichtsahnende
Schwester mit einem Besen nicht nur die Überbleibsel zusammenkehrte,
sondern selbst die von Gregor gar nicht berührten Speisen, als seien
also auch diese nicht mehr zu gebrauchen, und wie sie alles hastig in
einen Kübel schüttete, den sie mit einem Holzdeckel schloß, worauf sie
alles hinaustrug. Kaum hatte sie sich umgedreht, zog sich schon Gregor
unter dem Kanapee hervor und streckte und blähte sich.

Auf diese Weise bekam nun Gregor täglich sein Essen, einmal am Morgen,
wenn die Eltern und das Dienstmädchen noch schliefen, das zweitemal nach
dem allgemeinen Mittagessen, denn dann schliefen die Eltern gleichfalls
noch ein Weilchen, und das Dienstmädchen wurde von der Schwester mit
irgendeiner Besorgung weggeschickt. Gewiß wollten auch sie nicht, daß
Gregor verhungere, aber vielleicht hätten sie es nicht ertragen können,
von seinem Essen mehr als durch Hörensagen zu erfahren, vielleicht
wollte die Schwester ihnen auch eine möglicherweise nur kleine Trauer
ersparen, denn tatsächlich litten sie ja gerade genug.

Mit welchen Ausreden man an jenem ersten Vormittag den Arzt und den
Schlosser wieder aus der Wohnung geschafft hatte, konnte Gregor gar
nicht erfahren, denn da er nicht verstanden wurde, dachte niemand daran,
auch die Schwester nicht, daß er die anderen verstehen könne, und so
mußte er sich, wenn die Schwester in seinem Zimmer war, damit begnügen,
nur hier und da ihre Seufzer und Anrufe der Heiligen zu hören. Erst
später, als sie sich ein wenig an alles gewöhnt hatte --- von
vollständiger Gewöhnung konnte natürlich niemals die Rede sein ---,
erhaschte Gregor manchmal eine Bemerkung, die freundlich gemeint war
oder so gedeutet werden konnte. »Heute hat es ihm aber geschmeckt,«
sagte sie, wenn Gregor unter dem Essen tüchtig aufgeräumt hatte, während
sie im gegenteiligen Fall, der sich allmählich immer häufiger
wiederholte, fast traurig zu sagen pflegte: »Nun ist wieder alles
stehengeblieben.«

Während aber Gregor unmittelbar keine Neuigkeit erfahren konnte,
erhorchte er manches aus den Nebenzimmern, und wo er nun einmal Stimmen
hörte, lief er gleich zu der betreffenden Tür und drückte sich mit
ganzem Leib an sie. Besonders in der ersten Zeit gab es kein Gespräch,
das nicht irgendwie wenn auch nur im geheimen, von ihm handelte. Zwei
Tage lang waren bei allen Mahlzeiten Beratungen darüber zu hören, wie
man sich jetzt verhalten solle; aber auch zwischen den Mahlzeiten sprach
man über das gleiche Thema, denn immer waren zumindest zwei
Familienmitglieder zu Hause, da wohl niemand allein zu Hause bleiben
wollte und man die Wohnung doch auf keinen Fall gänzlich verlassen
konnte. Auch hatte das Dienstmädchen gleich am ersten Tag --- es war
nicht ganz klar, was und wieviel sie von dem Vorgefallenen wußte ---
kniefällig die Mutter gebeten, sie sofort zu entlassen, und als sie sich
eine Viertelstunde danach verabschiedete, dankte sie für die Entlassung
unter Tränen, wie für die größte Wohltat, die man ihr hier erwiesen
hatte, und gab, ohne daß man es von ihr verlangte, einen fürchterlichen
Schwur ab, niemandem auch nur das geringste zu verraten.

Nun mußte die Schwester im Verein mit der Mutter auch kochen; allerdings
machte das nicht viel Mühe, denn man aß fast nichts. Immer wieder hörte
Gregor, wie der eine den anderen vergebens zum Essen aufforderte und
keine andere Antwort bekam, als: »Danke ich habe genug« oder etwas
Ähnliches. Getrunken wurde vielleicht auch nichts. Öfters fragte die
Schwester den Vater, ob er Bier haben wolle, und herzlich erbot sie
sich, es selbst zu holen, und als der Vater schwieg, sagte sie, um ihm
jedes Bedenken zu nehmen, sie könne auch die Hausmeisterin darum
schicken, aber dann sagte der Vater schließlich ein großes »Nein«, und
es wurde nicht mehr davon gesprochen.

Schon im Laufe des ersten Tages legte der Vater die ganzen
Vermögensverhältnisse und Aussichten sowohl der Mutter als auch der
Schwester dar. Hie und da stand er vom Tische auf und holte aus seiner
kleinen Wertheimkassa, die er aus dem vor fünf Jahren erfolgten
Zusammenbruch seines Geschäftes gerettet hatte, irgendeinen Beleg oder
irgendein Vormerkbuch. Man hörte, wie er das komplizierte Schloß
aufsperrte und nach Entnahme des Gesuchten wieder verschloß. Diese
Erklärungen des Vaters waren zum Teil das erste Erfreuliche, was Gregor
seit seiner Gefangenschaft zu hören bekam. Er war der Meinung gewesen,
daß dem Vater von jenem Geschäft her nicht das Geringste übriggeblieben
war, zumindest hatte ihm der Vater nichts Gegenteiliges gesagt, und
Gregor allerdings hatte ihn auch nicht darum gefragt. Gregors Sorge war
damals nur gewesen, alles daranzusetzen, um die Familie das
geschäftliche Unglück, das alle in eine vollständige Hoffnungslosigkeit
gebracht hatte, möglichst rasch vergessen zu lassen. Und so hatte er
damals mit ganz besonderem Feuer zu arbeiten angefangen und war fast
über Nacht aus einem kleinen Kommis ein Reisender geworden, der
natürlich ganz andere Möglichkeiten des Geldverdienens hatte, und dessen
Arbeitserfolge sich sofort in Form der Provision zu Bargeld
verwandelten, das der erstaunten und beglückten Familie zu Hause auf den
Tisch gelegt werden konnte. Es waren schöne Zeiten gewesen, und niemals
nachher hatten sie sich, wenigstens in diesem Glanze, wiederholt,
trotzdem Gregor später so viel Geld verdiente, daß er den Aufwand der
ganzen Familie zu tragen imstande war und auch trug. Man hatte sich eben
daran gewöhnt, sowohl die Familie, als auch Gregor, man nahm das Geld
dankbar an, er lieferte es gern ab, aber eine besondere Wärme wollte
sich nicht mehr ergeben. Nur die Schwester war Gregor doch noch nahe
geblieben, und es war sein geheimer Plan, sie, die zum Unterschied von
Gregor Musik sehr liebte und rührend Violine zu spielen verstand,
nächstes Jahr, ohne Rücksicht auf die großen Kosten, die das verursachen
mußte, und die man schon auf andere Weise hereinbringen würde, auf das
Konservatorium zu schicken. Öfters während der kurzen Aufenthalte
Gregors in der Stadt wurde in den Gesprächen mit der Schwester das
Konservatorium erwähnt, aber immer nur als schöner Traum, an dessen
Verwirklichung nicht zu denken war, und die Eltern hörten nicht einmal
diese unschuldigen Erwähnungen gern; aber Gregor dachte sehr bestimmt
daran und beabsichtigte, es am Weihnachtsabend feierlich zu erklären.

Solche in seinem gegenwärtigen Zustand ganz nutzlose Gedanken gingen ihm
durch den Kopf, während er dort aufrecht an der Türe klebte und horchte.
Manchmal konnte er vor allgemeiner Müdigkeit gar nicht mehr zuhören und
ließ den Kopf nachlässig gegen die Tür schlagen, hielt ihn aber sofort
wieder fest, denn selbst das kleine Geräusch, das er damit verursacht
hatte, war nebenan gehört worden und hatte alle verstummen lassen. »Was
er nur wieder treibt,« sagte der Vater nach einer Weile, offenbar zur
Türe hingewendet, und dann erst wurde das unterbrochene Gespräch
allmählich wieder aufgenommen.

Gregor erfuhr nun zur Genüge --- denn der Vater pflegte sich in seinen
Erklärungen öfters zu wiederholen, teils, weil er selbst sich mit diesen
Dingen schon lange nicht beschäftigt hatte, teils auch, weil die Mutter
nicht alles gleich beim erstenmal verstand ---, daß trotz allen Unglücks
ein allerdings ganz kleines Vermögen aus der alten Zeit noch vorhanden
war, das die nicht angerührten Zinsen in der Zwischenzeit ein wenig
hatten anwachsen lassen. Außerdem aber war das Geld, das Gregor
allmonatlich nach Hause gebracht hatte --- er selbst hatte nur ein paar
Gulden für sich behalten ---, nicht vollständig aufgebraucht worden und
hatte sich zu einem kleinen Kapital angesammelt. Gregor, hinter seiner
Türe, nickte eifrig, erfreut über diese unerwartete Vorsicht und
Sparsamkeit. Eigentlich hätte er ja mit diesen überschüssigen Geldern
die Schuld des Vaters gegenüber dem Chef weiter abgetragen haben können,
und jener Tag, an dem er diesen Posten hätte loswerden können, wäre weit
näher gewesen, aber jetzt war es zweifellos besser so, wie es der Vater
eingerichtet hatte.

Nun genügte dieses Geld aber ganz und gar nicht, um die Familie etwa von
den Zinsen leben zu lassen; es genügte vielleicht, um die Familie ein,
höchstens zwei Jahre zu erhalten, mehr war es nicht. Es war also bloß
eine Summe, die man eigentlich nicht angreifen durfte, und die für den
Notfall zurückgelegt werden mußte; das Geld zum Leben aber mußte man
verdienen. Nun war aber der Vater ein zwar gesunder, aber alter Mann,
der schon fünf Jahre nichts gearbeitet hatte und sich jedenfalls nicht
viel zutrauen durfte; er hatte in diesen fünf Jahren, welche die ersten
Ferien seines mühevollen und doch erfolglosen Lebens waren, viel Fett
angesetzt und war dadurch recht schwerfällig geworden. Und die alte
Mutter sollte nun vielleicht Geld verdienen, die an Asthma litt, der
eine Wanderung durch die Wohnung schon Anstrengung verursachte, und die
jeden zweiten Tag in Atembeschwerden auf dem Sofa beim offenen Fenster
verbrachte? Und die Schwester sollte Geld verdienen, die noch ein Kind
war mit ihren siebzehn Jahren, und der ihre bisherige Lebensweise so
sehr zu gönnen war, die daraus bestanden hatte, sich nett zu kleiden,
lange zu schlafen, in der Wirtschaft mitzuhelfen, an ein paar
bescheidenen Vergnügungen sich zu beteiligen und vor allem Violine zu
spielen? Wenn die Rede auf diese Notwendigkeit des Geldverdienens kam,
ließ zuerst immer Gregor die Türe los und warf sich auf das neben der
Tür befindliche kühle Ledersofa, denn ihm war ganz heiß vor Beschämung
und Trauer.

Oft lag er dort die ganzen langen Nächte über, schlief keinen Augenblick
und scharrte nur stundenlang auf dem Leder. Oder er scheute nicht die
große Mühe, einen Sessel zum Fenster zu schieben, dann die
Fensterbrüstung hinaufzukriechen und, in den Sessel gestemmt, sich ans
Fenster zu lehnen, offenbar nur in irgendeiner Erinnerung an das
Befreiende, das früher für ihn darin gelegen war, aus dem Fenster zu
schauen. Denn tatsächlich sah er von Tag zu Tag die auch nur ein wenig
entfernten Dinge immer undeutlicher; das gegenüberliegende Krankenhaus,
dessen nur allzu häufigen Anblick er früher verflucht hatte, bekam er
überhaupt nicht mehr zu Gesicht, und wenn er nicht genau gewußt hätte,
daß er in der stillen, aber völlig städtischen Charlottenstraße wohnte,
hätte er glauben können, von seinem Fenster aus in eine Einöde zu
schauen in welcher der graue Himmel und die graue Erde ununterscheidbar
sich vereinigten. Nur zweimal hatte die aufmerksame Schwester sehen
müssen, daß der Sessel beim Fenster stand, als sie schon jedesmal,
nachdem sie das Zimmer aufgeräumt hatte, den Sessel wieder genau zum
Fenster hinschob, ja sogar von nun ab den inneren Fensterflügel offen
ließ.

Hätte Gregor nur mit der Schwester sprechen und ihr für alles danken
können, was sie für ihn machen mußte, er hätte ihre Dienste leichter
ertragen; so aber litt er darunter. Die Schwester suchte freilich die
Peinlichkeit des Ganzen möglichst zu verwischen, und je längere Zeit
verging, desto besser gelang es ihr natürlich auch, aber auch Gregor
durchschaute mit der Zeit alles viel genauer. Schon ihr Eintritt war für
ihn schrecklich. Kaum war sie eingetreten, lief sie, ohne sich Zeit zu
nehmen, die Türe zu schließen, so sehr sie sonst darauf achtete, jedem
den Anblick von Gregors Zimmer zu ersparen, geradewegs zum Fenster und
riß es, als ersticke sie fast, mit hastigen Händen auf, blieb auch,
selbst wenn es noch so kalt war, ein Weilchen beim Fenster und atmete
tief. Mit diesem Laufen und Lärmen erschreckte sie Gregor täglich
zweimal; die ganze Zeit über zitterte er unter dem Kanapee und wußte
doch sehr gut, daß sie ihn gewiß gerne damit verschont hätte, wenn es
ihr nur möglich gewesen wäre, sich in einem Zimmer, in dem sich Gregor
befand, bei geschlossenem Fenster aufzuhalten.

Einmal, es war wohl schon ein Monat seit Gregors Verwandlung vergangen,
und es war doch schon für die Schwester kein besonderer Grund mehr, über
Gregors Aussehen in Erstaunen zu geraten, kam sie ein wenig früher als
sonst und traf Gregor noch an, wie er, unbeweglich und so recht zum
Erschrecken aufgestellt, aus dem Fenster schaute. Es wäre für Gregor
nicht unerwartet gewesen, wenn sie nicht eingetreten wäre, da er sie
durch seine Stellung verhinderte, sofort das Fenster zu öffnen, aber sie
trat nicht nur nicht ein, sie fuhr sogar zurück und schloß die Tür; ein
Fremder hätte geradezu denken können, Gregor habe ihr aufgelauert und
habe sie beißen wollen. Gregor versteckte sich natürlich sofort unter
dem Kanapee, aber er mußte bis zum Mittag warten, ehe die Schwester
wiederkam, und sie schien viel unruhiger als sonst. Er erkannte daraus,
daß ihr sein Anblick noch immer unerträglich war und ihr auch weiterhin
unerträglich bleiben müsse, und daß sie sich wohl sehr überwinden mußte,
vor dem Anblick auch nur der kleinen Partie seines Körpers nicht
davonzulaufen, mit der er unter dem Kanapee hervorragte. Um ihr auch
diesen Anblick zu ersparen, trug er eines Tages auf seinem Rücken --- er
brauchte zu dieser Arbeit vier Stunden --- das Leintuch auf das Kanapee
und ordnete es in einer solchen Weise an, daß er nun gänzlich verdeckt
war, und daß die Schwester, selbst wenn sie sich bückte, ihn nicht sehen
konnte. Wäre dieses Leintuch ihrer Meinung nach nicht nötig gewesen,
dann hätte sie es ja entfernen können, denn daß es nicht zum Vergnügen
Gregors gehören konnte, sich so ganz und gar abzusperren, war doch klar
genug, aber sie ließ das Leintuch, so wie es war, und Gregor glaubte
sogar einen dankbaren Blick erhascht zu haben, als er einmal mit dem
Kopf vorsichtig das Leintuch ein wenig lüftete, um nachzusehen, wie die
Schwester die neue Einrichtung aufnahm.

In den ersten vierzehn Tagen konnten es die Eltern nicht über sich
bringen, zu ihm hereinzukommen, und er hörte oft, wie sie die jetzige
Arbeit der Schwester völlig anerkannten, während sie sich bisher häufig
über die Schwester geärgert hatten, weil sie ihnen als ein etwas
nutzloses Mädchen erschienen war. Nun aber warteten oft beide, der Vater
und die Mutter, vor Gregors Zimmer, während die Schwester dort
aufräumte, und kaum war sie herausgekommen, mußte sie ganz genau
erzählen, wie es in dem Zimmer aussah, was Gregor gegessen hatte, wie er
sich diesmal benommen hatte, und ob vielleicht eine kleine Besserung zu
bemerken war. Die Mutter übrigens wollte verhältnismäßig bald Gregor
besuchen, aber der Vater und die Schwester hielten sie zuerst mit
Vernunftgründen zurück, denen Gregor sehr aufmerksam zuhörte, und die er
vollständig billigte. Später aber mußte man sie mit Gewalt zurückhalten,
und wenn sie dann rief: »Laßt mich doch zu Gregor, er ist ja mein
unglücklicher Sohn! Begreift ihr es denn nicht, daß ich zu ihm muß?«,
dann dachte Gregor, daß es vielleicht doch gut wäre, wenn die Mutter
hereinkäme, nicht jeden Tag natürlich, aber vielleicht einmal in der
Woche; sie verstand doch alles viel besser als die Schwester, die trotz
all ihrem Mute doch nur ein Kind war und im letzten Grunde vielleicht
nur aus kindlichem Leichtsinn eine so schwere Aufgabe übernommen hatte.

Der Wunsch Gregors, die Mutter zu sehen, ging bald in Erfüllung. Während
des Tages wollte Gregor schon aus Rücksicht auf seine Eltern sich nicht
beim Fenster zeigen, kriechen konnte er aber auf den paar Quadratmetern
des Fußbodens auch nicht viel, das ruhige Liegen ertrug er schon während
der Nacht schwer, das Essen machte ihm bald nicht mehr das geringste
Vergnügen, und so nahm er zur Zerstreuung die Gewohnheit an, kreuz und
quer über Wände und Plafond zu kriechen. Besonders oben an der Decke
hing er gern; es war ganz anders, als das Liegen auf dem Fußboden; man
atmete freier; ein leichtes Schwingen ging durch den Körper, und in der
fast glücklichen Zerstreutheit, in der sich Gregor dort oben befand,
konnte es geschehen, daß er zu seiner eigenen Überraschung sich losließ
und auf den Boden klatschte. Aber nun hatte er natürlich seinen Körper
ganz anders in der Gewalt als früher und beschädigte sich selbst bei
einem so großen Falle nicht. Die Schwester nun bemerkte sofort die neue
Unterhaltung, die Gregor für sich gefunden hatte --- er hinterließ ja
auch beim Kriechen hie und da Spuren seines Klebstoffes ---, und da
setzte sie es sich in den Kopf, Gregor das Kriechen in größtem Ausmaße
zu ermöglichen und die Möbel, die es verhinderten, also vor allem den
Kasten und den Schreibtisch, wegzuschaffen. Nun war sie aber nicht
imstande, dies allein zu tun; den Vater wagte sie nicht um Hilfe zu
bitten; das Dienstmädchen hätte ihr ganz gewiß nicht geholfen, denn
dieses etwa sechzehnjährige Mädchen harrte zwar tapfer seit Entlassung
der früheren Köchin aus, hatte aber um die Vergünstigung gebeten, die
Küche unaufhörlich versperrt halten zu dürfen und nur auf besonderen
Anruf öffnen zu müssen; so blieb der Schwester also nichts übrig, als
einmal in Abwesenheit des Vaters die Mutter zu holen. Mit Ausrufen
erregter Freude kam die Mutter auch heran, verstummte aber an der Tür
vor Gregors Zimmer. Zuerst sah natürlich die Schwester nach, ob alles im
Zimmer in Ordnung war; dann erst ließ sie die Mutter eintreten. Gregor
hatte in größter Eile das Leintuch noch tiefer und mehr in Falten
gezogen, das Ganze sah wirklich nur wie ein zufällig über das Kanapee
geworfenes Leintuch aus. Gregor unterließ auch diesmal, unter dem
Leintuch zu spionieren; er verzichtete darauf, die Mutter schon diesmal
zu sehen, und war nur froh, daß sie nun doch gekommen war. »Komm nur,
man sieht ihn nicht,« sagte die Schwester, und offenbar führte sie die
Mutter an der Hand. Gregor hörte nun, wie die zwei schwachen Frauen den
immerhin schweren alten Kasten von seinem Platze rückten, und wie die
Schwester immerfort den größten Teil der Arbeit für sich beanspruchte,
ohne auf die Warnungen der Mutter zu hören, welche fürchtete, daß sie
sich überanstrengen werde. Es dauerte sehr lange. Wohl nach schon
viertelstündiger Arbeit sagte die Mutter, man solle den Kasten doch
lieber hier lassen, denn erstens sei er zu schwer, sie würden vor
Ankunft des Vaters nicht fertig werden und mit dem Kasten in der Mitte
des Zimmers Gregor jeden Weg verrammeln, zweitens aber sei es doch gar
nicht sicher, daß Gregor mit der Entfernung der Möbel ein Gefallen
geschehe. Ihr scheine das Gegenteil der Fall zu sein; ihr bedrücke der
Anblick der leeren Wand geradezu das Herz; und warum solle nicht auch
Gregor diese Empfindung haben, da er doch an die Zimmermöbel längst
gewöhnt sei und sich deshalb im leeren Zimmer verlassen fühlen werde.
»Und ist es dann nicht so,« schloß die Mutter ganz leise, wie sie
überhaupt fast flüsterte, als wolle sie vermeiden, daß Gregor, dessen
genauen Aufenthalt sie ja nicht kannte, auch nur den Klang der Stimme
höre, denn daß er die Worte nicht verstand, davon war sie überzeugt,
»und ist es nicht so, als ob wir durch die Entfernung der Möbel zeigten,
daß wir jede Hoffnung auf Besserung aufgeben und ihn rücksichtslos sich
selbst überlassen? Ich glaube, es wäre das beste, wir suchen das Zimmer
genau in dem Zustand zu erhalten, in dem es früher war, damit Gregor,
wenn er wieder zu uns zurückkommt, alles unverändert findet und um so
leichter die Zwischenzeit vergessen kann.«

Beim Anhören dieser Worte der Mutter erkannte Gregor, daß der Mangel
jeder unmittelbaren menschlichen Ansprache, verbunden mit dem
einförmigen Leben inmitten der Familie, im Laufe dieser zwei Monate
seinen Verstand hatte verwirren müssen, denn anders konnte er es sich
nicht erklären, daß er ernsthaft darnach hatte verlangen können, daß
sein Zimmer ausgeleert würde. Hatte er wirklich Lust, das warme, mit
ererbten Möbeln gemütlich ausgestattete Zimmer in eine Höhle verwandeln
zu lassen, in der er dann freilich nach allen Richtungen ungestört würde
kriechen können, jedoch auch unter gleichzeitigem, schnellen, gänzlichen
Vergessen seiner menschlichen Vergangenheit? War er doch jetzt schon
nahe daran, zu vergessen, und nur die seit langem nicht gehörte Stimme
der Mutter hatte ihn aufgerüttelt. Nichts sollte entfernt werden, alles
mußte bleiben, die guten Einwirkungen der Möbel auf seinen Zustand
konnte er nicht entbehren; und wenn die Möbel ihn hinderten, das
sinnlose Herumkriechen zu betreiben, so war es kein Schaden, sondern ein
großer Vorteil.

Aber die Schwester war leider anderer Meinung; sie hatte sich,
allerdings nicht ganz unberechtigt, angewöhnt, bei Besprechung der
Angelegenheiten Gregors als besonders Sachverständige gegenüber den
Eltern aufzutreten, und so war auch jetzt der Rat der Mutter für die
Schwester Grund genug, auf der Entfernung nicht nur des Kastens und des
Schreibtisches, an die sie zuerst allein gedacht hatte, sondern auf der
Entfernung sämtlicher Möbel, mit Ausnahme des unentbehrlichen Kanapees,
zu bestehen. Es war natürlich nicht nur kindlicher Trotz und das in der
letzten Zeit so unerwartet und schwer erworbene Selbstvertrauen, das sie
zu dieser Forderung bestimmte; sie hatte doch auch tatsächlich
beobachtet, daß Gregor viel Raum zum Kriechen brauchte, dagegen die
Möbel, soweit man sehen konnte, nicht im geringsten benützte. Vielleicht
aber spielte auch der schwärmerische Sinn der Mädchen ihres Alters mit,
der bei jeder Gelegenheit seine Befriedigung sucht, und durch den Grete
jetzt sich dazu verlocken ließ, die Lage Gregors noch
schreckenerregender machen zu wollen, um dann noch mehr als bis jetzt
für ihn leisten zu können. Denn in einem Raum, in dem Gregor ganz allein
die leeren Wände beherrschte, würde wohl kein Mensch außer Grete jemals
einzutreten sich getrauen.

Und so ließ sie sich von ihrem Entschlusse durch die Mutter nicht
abbringen, die auch in diesem Zimmer vor lauter Unruhe unsicher schien,
bald verstummte und der Schwester nach Kräften beim Hinausschaffen des
Kastens half. Nun, den Kasten konnte Gregor im Notfall noch entbehren,
aber schon der Schreibtisch mußte bleiben. Und kaum hatten die Frauen
mit dem Kasten, an dem sie sich ächzend drückten, das Zimmer verlassen,
als Gregor den Kopf unter dem Kanapee hervorstieß, um zu sehen, wie er
vorsichtig und möglichst rücksichtsvoll eingreifen könnte. Aber zum
Unglück war es gerade die Mutter, welche zuerst zurückkehrte, während
Grete im Nebenzimmer den Kasten umfangen hielt und ihn allein hin und
her schwang, ohne ihn natürlich von der Stelle zu bringen. Die Mutter
aber war Gregors Anblick nicht gewöhnt, er hätte sie krank machen
können, und so eilte Gregor erschrocken im Rückwärtslauf bis an das
andere Ende des Kanapees, konnte es aber nicht mehr verhindern, daß das
Leintuch vorne ein wenig sich bewegte. Das genügte, um die Mutter
aufmerksam zu machen. Sie stockte, stand einen Augenblick still und ging
dann zu Grete zurück.

Trotzdem sich Gregor immer wieder sagte, daß ja nichts Außergewöhnliches
geschehe, sondern nur ein paar Möbel umgestellt würden, wirkte doch, wie
er sich bald eingestehen mußte, dieses Hin- und Hergehen der Frauen,
ihre kleinen Zurufe, das Kratzen der Möbel auf dem Boden, wie ein
großer, von allen Seiten genährter Trubel auf ihn, und er mußte sich, so
fest er Kopf und Beine an sich zog und den Leib bis an den Boden
drückte, unweigerlich sagen, daß er das Ganze nicht lange aushalten
werde. Sie räumten ihm sein Zimmer aus; nahmen ihm alles, was ihm lieb
war; den Kasten, in dem die Laubsäge und andere Werkzeuge lagen, hatten
sie schon hinausgetragen; lockerten jetzt den schon im Boden fest
eingegrabenen Schreibtisch, an dem er als Handelsakademiker, als
Bürgerschüler, ja sogar schon als Volksschüler seine Aufgaben
geschrieben hatte, --- da hatte er wirklich keine Zeit mehr, die guten
Absichten zu prüfen, welche die zwei Frauen hatten, deren Existenz er
übrigens fast vergessen hatte, denn vor Erschöpfung arbeiteten sie schon
stumm, und man hörte nur das schwere Tappen ihrer Füße.

Und so brach er denn hervor --- die Frauen stützten sich gerade im
Nebenzimmer an den Schreibtisch, um ein wenig zu verschnaufen ---,
wechselte viermal die Richtung des Laufes, er wußte wirklich nicht, was
er zuerst retten sollte, da sah er an der im übrigen schon leeren Wand
auffallend das Bild der in lauter Pelzwerk gekleideten Dame hängen,
kroch eilends hinauf und preßte sich an das Glas, das ihn festhielt und
seinem heißen Bauch wohltat. Dieses Bild wenigstens, das Gregor jetzt
ganz verdeckte, würde nun gewiß niemand wegnehmen. Er verdrehte den Kopf
nach der Tür des Wohnzimmers, um die Frauen bei ihrer Rückkehr zu
beobachten.

Sie hatten sich nicht viel Ruhe gegönnt und kamen schon wieder; Grete
hatte den Arm um die Mutter gelegt und trug sie fast. »Also was nehmen
wir jetzt?« sagte Grete und sah sich um, Da kreuzten sich ihre Blicke
mit denen Gregors an der Wand. Wohl nur infolge der Gegenwart der Mutter
behielt sie ihre Fassung, beugte ihr Gesicht zur Mutter, um diese vom
Herumschauen abzuhalten, und sagte, allerdings zitternd und unüberlegt:
»Komm, wollen wir nicht lieber auf einen Augenblick noch ins Wohnzimmer
zurückgehen?« Die Absicht Gretes war für Gregor klar, sie wollte die
Mutter in Sicherheit bringen und dann ihn von der Wand hinunterjagen.
Nun, sie konnte es ja immerhin versuchen! Er saß auf seinem Bild und
gab es nicht her. Lieber würde er Grete ins Gesicht springen.

Aber Gretes Worte hatten die Mutter erst recht beunruhigt, sie trat zur
Seite, erblickte den riesigen braunen Fleck auf der geblümten Tapete,
rief, ehe ihr eigentlich zum Bewußtsein kam, daß das Gregor war, was sie
sah, mit schreiender, rauher Stimme: »Ach Gott, ach Gott!« und fiel mit
ausgebreiteten Armen, als gebe sie alles auf, über das Kanapee hin und
rührte sich nicht. »Du, Gregor!« rief die Schwester mit erhobener Faust
und eindringlichen Blicken. Es waren seit der Verwandlung die ersten
Worte, die sie unmittelbar an ihn gerichtet hatte. Sie lief ins
Nebenzimmer, um irgendeine Essenz zu holen, mit der sie die Mutter aus
ihrer Ohnmacht wecken könnte; Gregor wollte auch helfen --- zur Rettung
des Bildes war noch Zeit ---; er klebte aber fest an dem Glas und mußte
sich mit Gewalt losreißen; er lief dann auch ins Nebenzimmer, als könne
er der Schwester irgendeinen Rat geben, wie in früherer Zeit; mußte aber
dann untätig hinter ihr stehen; während sie in verschiedenen Fläschchen
kramte, erschreckte sie noch, als sie sich umdrehte; eine Flasche fiel
auf den Boden und zerbrach; ein Splitter verletzte Gregor im Gesicht,
irgendeine ätzende Medizin umfloß ihn; Grete nahm nun, ohne sich länger
aufzuhalten, so viele Fläschchen, als sie nur halten konnte, und rannte
mit ihnen zur Mutter hinein; die Tür schlug sie mit dem Fuße zu. Gregor
war nun von der Mutter abgeschlossen, die durch seine Schuld vielleicht
dem Tode nahe war; die Tür durfte er nicht öffnen, wollte er die
Schwester, die bei der Mutter bleiben mußte, nicht verjagen; er hatte
jetzt nichts zu tun, als zu warten; und von Selbstvorwürfen und
Besorgnis bedrängt, begann er zu kriechen, überkroch alles, Wände,
Möbel und Zimmerdecke und fiel endlich in seiner Verzweiflung, als sich
das ganze Zimmer schon um ihn zu drehen anfing, mitten auf den großen
Tisch.

Es verging eine kleine Weile, Gregor lag matt da, ringsherum war es
still, vielleicht war das ein gutes Zeichen. Da läutete es. Das Mädchen
war natürlich in ihrer Küche eingesperrt und Grete mußte daher öffnen
gehen. Der Vater war gekommen. »Was ist\est\ geschehen?« waren seine ersten
Worte; Gretes Aussehen hatte ihm wohl alles verraten. Grete antwortete
mit dumpfer Stimme, offenbar drückte sie ihr Gesicht an des Vaters
Brust: »Die Mutter war ohnmächtig, aber es geht ihr schon besser. Gregor
ist ausgebrochen.« »Ich habe es ja erwartet,« sagte der Vater, »ich habe
es euch ja immer gesagt, aber ihr Frauen wollt nicht hören.« Gregor war
es klar, daß der Vater Gretes allzukurze Mitteilung schlecht gedeutet
hatte und annahm, daß Gregor sich irgendeine Gewalttat habe zuschulden
kommen lassen. Deshalb mußte Gregor den Vater jetzt zu besänftigen
suchen, denn ihn aufzuklären hatte er weder Zeit noch Möglichkeit. Und
so flüchtete er sich zur Tür seines Zimmers und drückte sich an sie,
damit der Vater beim Eintritt vom Vorzimmer her gleich sehen könne, daß
Gregor die beste Absicht habe, sofort in sein Zimmer zurückzukehren, und
daß es nicht nötig sei, ihn zurückzutreiben, sondern daß man nur die Tür
zu öffnen brauchte, und gleich werde er verschwinden.

Aber der Vater war nicht in der Stimmung, solche Feinheiten zu bemerken.
»Ah!« rief er gleich beim Eintritt in einem Tone, als sei er
gleichzeitig wütend und froh. Gregor zog den Kopf von der Tür zurück und
hob ihn gegen den Vater. So hatte er sich den Vater wirklich nicht
vorgestellt, wie er jetzt dastand; allerdings hatte er in der letzten
Zeit über dem neuartigen Herumkriechen versäumt, sich so wie früher um
die Vorgänge in der übrigen Wohnung zu kümmern, und hätte eigentlich
darauf gefaßt sein müssen, veränderte Verhältnisse anzutreffen.
Trotzdem, trotzdem, war das noch der Vater? Der gleiche Mann, der müde
im Bett vergraben lag, wenn früher Gregor zu einer Geschäftsreise
ausgerückt war; der ihn an Abenden der Heimkehr im Schlafrock im
Lehnstuhl empfangen hatte; gar nicht recht imstande war, aufzustehen,
sondern zum Zeichen der Freude nur die Arme gehoben hatte, und der bei
den seltenen gemeinsamen Spaziergängen an ein paar Sonntagen im Jahr und
an den höchsten Feiertagen zwischen Gregor und der Mutter, die schon an
und für sich langsam gingen, immer noch ein wenig langsamer, in seinen
alten Mantel eingepackt, mit stets vorsichtig aufgesetztem Krückstock
sich vorwärts arbeitete und, wenn er etwas sagen wollte, fast immer
stillstand und seine Begleitung um sich versammelte? Nun aber war er
doch gut aufgerichtet; in eine straffe blaue Uniform mit Goldknöpfen
gekleidet, wie sie Diener der Bankinstitute tragen; über dem hohen
steifen Kragen des Rockes entwickelte sich sein starkes Doppelkinn;
unter den buschigen Augenbrauen drang der Blick der schwarzen Augen
frisch und aufmerksam hervor; das sonst zerzauste weiße Haar war zu
einer peinlich genauen, leuchtenden Scheitelfrisur niedergekämmt. Er
warf seine Mütze, auf der ein Goldmonogramm, wahrscheinlich das einer
Bank, angebracht war, über das ganze Zimmer im Bogen auf das Kanapee hin
und ging, die Enden seines langen Uniformrockes zurückgeschlagen, die
Hände in den Hosentaschen, mit verbissenem Gesicht auf Gregor zu. Er
wußte wohl selbst nicht, was er vorhatte; immerhin hob er die Füße
ungewöhnlich hoch, und Gregor staunte über die Riesengröße seiner
Stiefelsohlen. Doch hielt er sich dabei nicht auf, er wußte ja noch vom
ersten Tage seines neuen Lebens her, daß der Vater ihm gegenüber nur die
größte Strenge für angebracht ansah. Und so lief er vor dem Vater her,
stockte, wenn der Vater stehen blieb, und eilte schon wieder vorwärts,
wenn sich der Vater nur rührte. So machten sie mehrmals die Runde um das
Zimmer, ohne daß sich etwas Entscheidendes ereignete, ja ohne daß das
Ganze infolge seines langsamen Tempos den Anschein einer Verfolgung
gehabt hätte. Deshalb blieb auch Gregor vorläufig auf dem Fußboden,
zumal er fürchtete, der Vater könnte eine Flucht auf die Wände oder den
Plafond für besondere Bosheit halten. Allerdings mußte sich Gregor
sagen, daß er sogar dieses Laufen nicht lange aushalten würde, denn
während der Vater einen Schritt machte, mußte er eine Unzahl von
Bewegungen ausführen. Atemnot begann sich schon bemerkbar zu machen, wie
er ja auch in seiner früheren Zeit keine ganz vertrauenswürdige Lunge
besessen hatte. Als er nun so dahintorkelte, um alle Kräfte für den Lauf
zu sammeln, kaum die Augen offenhielt; in seiner Stumpfheit an eine
andere Rettung als durch Laufen gar nicht dachte; und fast schon
vergessen hatte, daß ihm die Wände freistanden, die hier allerdings mit
sorgfältig geschnitzten Möbeln voll Zacken und Spitzen verstellt waren
--- da flog knapp neben ihm, leicht geschleudert, irgend etwas nieder und
rollte vor ihm her. Es war ein Apfel; gleich flog ihm ein zweiter nach;\est\
Gregor blieb vor Schrecken stehen; ein Weiterlaufen war nutzlos, denn
der Vater hatte sich entschlossen, ihn zu bombardieren. Aus der
Obstschale auf der Kredenz hatte er sich die Taschen gefüllt und warf
nun, ohne vorläufig scharf zu zielen, Apfel für Apfel. Diese kleinen
roten Äpfel rollten wie elektrisiert auf dem Boden herum und stießen
aneinander. Ein schwach geworfener Apfel streifte Gregors Rücken, glitt
aber unschädlich ab. Ein ihm sofort nachfliegender drang dagegen
förmlich in Gregors Rücken ein; Gregor wollte sich weiterschleppen, als
könne der überraschende unglaubliche Schmerz mit dem Ortswechsel
vergehen; doch fühlte er sich wie festgenagelt und streckte sich in
vollständiger Verwirrung aller Sinne. Nur mit dem letzten Blick sah er
noch, wie die Tür seines Zimmers aufgerissen wurde, und vor der
schreienden Schwester die Mutter hervoreilte, im Hemd, denn die
Schwester hatte sie entkleidet, um ihr in der Ohnmacht Atemfreiheit zu
verschaffen, wie dann die Mutter auf den Vater zulief und ihr auf dem
Weg die aufgebundenen Röcke einer nach dem anderen zu Boden glitten, und
wie sie stolpernd über die Röcke auf den Vater eindrang und ihn
umarmend, in gänzlicher Vereinigung mit ihm --- nun versagte aber Gregors
Sehkraft schon --- die Hände an des Vaters Hinterkopf um Schonung von
Gregors Leben bat.

\pagebreak

\vspace*{2.5cm}

\section{III}

\noindent{}Die schwere Verwundung Gregors, an der er über einen Monat litt --- der
Apfel blieb, da ihn niemand zu entfernen wagte, als sichtbares Andenken
im Fleische sitzen ---, schien selbst den Vater daran erinnert zu haben,
daß Gregor trotz seiner gegenwärtigen traurigen und ekelhaften Gestalt
ein Familienglied war, das man nicht wie einen Feind behandeln durfte,
sondern dem gegenüber es das Gebot der Familienpflicht war, den
Widerwillen hinunterzuschlucken und zu dulden, nichts als dulden.

Und wenn nun auch Gregor durch seine Wunde an Beweglichkeit
wahrscheinlich für immer verloren hatte und vorläufig zur Durchquerung
seines Zimmers wie ein alter Invalide lange, lange Minuten brauchte ---
an das Kriechen in der Höhe war nicht zu denken ---, so bekam er für
diese Verschlimmerung seines Zustandes einen seiner Meinung nach
vollständig genügenden Ersatz dadurch, daß immer gegen Abend die
Wohnzimmertür, die er schon ein bis zwei Stunden vorher scharf zu
beobachten pflegte, geöffnet wurde, so daß er, im Dunkel seines Zimmers
liegend, vom Wohnzimmer aus unsichtbar, die ganze Familie beim
beleuchteten Tische sehen und ihre Reden, gewissermaßen mit allgemeiner
Erlaubnis, also ganz anders als früher, anhören durfte.

Freilich waren es nicht mehr die lebhaften Unterhaltungen der früheren
Zeiten, an die Gregor in den kleinen Hotelzimmern stets mit einigem
Verlangen gedacht hatte, wenn er sich müde in das feuchte Bettzeug hatte
werfen müssen. Es ging jetzt meist nur sehr still zu.

Der Vater schlief
bald nach dem Nachtessen in seinem Sessel ein; die Mutter und Schwester
ermahnten einander zur Stille; die Mutter nähte, weit über das Licht
vorgebeugt, feine Wäsche für ein Modengeschäft; die Schwester, die eine
Stellung als Verkäuferin angenommen hatte, lernte am Abend Stenographie
und Französisch, um vielleicht später einmal einen besseren Posten zu
erreichen. Manchmal wachte der Vater auf, und als wisse er gar nicht,
daß er geschlafen habe, sagte er zur Mutter: »Wie lange du heute schon
wieder nähst!« und schlief sofort wieder ein, während Mutter und
Schwester einander müde zulächelten.

Mit einer Art Eigensinn weigerte sich der Vater, auch zu Hause seine
Dieneruniform abzulegen; und während der Schlafrock nutzlos am
Kleiderhaken hing, schlummerte der Vater vollständig angezogen auf
seinem Platz, als sei er immer zu seinem Dienste bereit und warte auch
hier auf die Stimme des Vorgesetzten. Infolgedessen verlor die gleich
anfangs nicht neue Uniform trotz aller Sorgfalt von Mutter und Schwester
an Reinlichkeit, und Gregor sah oft ganze Abende lang auf dieses über
und über fleckige, mit seinen stets geputzten Goldknöpfen leuchtende
Kleid, in dem der alte Mann höchst unbequem und doch ruhig schlief.

Sobald die Uhr zehn schlug, suchte die Mutter durch leise Zusprache den
Vater zu wecken und dann zu überreden, ins Bett zu gehen, denn hier war
es doch kein richtiger Schlaf und diesen hatte der Vater, der um sechs
Uhr seinen Dienst antreten mußte, äußerst nötig. Aber in dem Eigensinn,
der ihn, seitdem er Diener war, ergriffen hatte, bestand er immer
darauf, noch länger bei Tisch zu bleiben, trotzdem er regelmäßig
einschlief, und war dann überdies nur mit der größten Mühe zu bewegen,
den Sessel mit dem Bett zu vertauschen. Da mochten Mutter und Schwester
mit kleinen Ermahnungen noch so sehr auf ihn eindringen,
viertelstundenlang schüttelte er langsam den Kopf, hielt die Augen
geschlossen und stand nicht auf. Die Mutter zupfte ihn am Ärmel, sagte
ihm Schmeichelworte ins Ohr, die Schwester verließ ihre Aufgabe, um der
Mutter zu helfen, aber beim Vater verfing das nicht. Er versank nur noch
tiefer in seinen Sessel. Erst bis ihn die Frauen unter den Achseln
faßten, schlug er die Augen auf, sah abwechselnd die Mutter und die
Schwester an und pflegte zu sagen: »Das ist ein Leben. Das ist die Ruhe
meiner alten Tage.« Und auf die beiden Frauen gestützt, erhob er sich,
umständlich, als sei er für sich selbst die größte Last, ließ sich von
den Frauen bis zur Türe führen, winkte ihnen dort ab und ging nun
selbständig weiter, während die Mutter ihr Nähzeug, die Schwester ihre
Feder eiligst hinwarfen, um hinter dem Vater zu laufen und ihm weiter
behilflich zu sein.

Wer hatte in dieser abgearbeiteten und übermüdeten Familie Zeit, sich um
Gregor mehr zu kümmern, als unbedingt nötig war? Der Haushalt wurde
immer mehr eingeschränkt; das Dienstmädchen wurde nun doch entlassen;
eine riesige knochige Bedienerin mit weißem, den Kopf umflatterndem Haar
kam des Morgens und des Abends, um die schwerste Arbeit zu leisten;
alles andere besorgte die Mutter neben ihrer vielen Näharbeit. Es
geschah sogar, daß verschiedene Familienschmuckstücke, welche früher die
Mutter und die Schwester überglücklich bei Unterhaltungen und
Feierlichkeiten getragen hatten, verkauft wurden, wie Gregor am Abend
aus der allgemeinen Besprechung der erzielten Preise erfuhr. Die größte
Klage war aber stets, daß man diese für die gegenwärtigen Verhältnisse
allzugroße Wohnung nicht verlassen konnte, da es nicht auszudenken war,
wie man Gregor übersiedeln sollte. Aber Gregor sah wohl ein, daß es
nicht nur die Rücksicht auf ihn war, welche eine Übersiedlung
verhinderte, denn ihn hätte man doch in einer passenden Kiste mit ein
paar Luftlöchern leicht transportieren können; was die Familie
hauptsächlich vom Wohnungswechsel abhielt, war vielmehr die völlige
Hoffnungslosigkeit und der Gedanke daran, daß sie mit einem Unglück
geschlagen war, wie niemand sonst im ganzen Verwandten- und
Bekanntenkreis. Was die Welt von armen Leuten verlangt, erfüllten sie
bis zum äußersten, der Vater holte den kleinen Bankbeamten das
Frühstück, die Mutter opferte sich für die Wäsche fremder Leute, die
Schwester lief nach dem Befehl der Kunden hinter dem Pulte hin und her,
aber weiter reichten die Kräfte der Familie schon nicht. Und die Wunde
im Rücken fing Gregor wie neu zu schmerzen an, wenn Mutter und
Schwester, nachdem sie den Vater zu Bett gebracht hatten, nun
zurückkehrten, die Arbeit liegen ließen, nahe zusammenrückten, schon
Wange an Wange saßen; wenn jetzt die Mutter, auf Gregors Zimmer zeigend,
sagte: »Mach' dort die Tür zu, Grete,« und wenn nun Gregor wieder im
Dunkel war, während nebenan die Frauen ihre Tränen vermischten oder gar
tränenlos den Tisch anstarrten.

Die Nächte und Tage verbrachte Gregor fast ganz ohne Schlaf. Manchmal
dachte er daran, beim nächsten Öffnen der Tür die Angelegenheiten der
Familie ganz so wie früher wieder in die Hand zu nehmen; in seinen
Gedanken erschienen wieder nach langer Zeit der Chef und der Prokurist,
die Kommis und die Lehrjungen, der so begriffsstützige Hausknecht, zwei
drei Freunde aus anderen Geschäften, ein Stubenmädchen aus einem Hotel
in der Provinz, eine liebe, flüchtige Erinnerung, eine Kassiererin aus
einem Hutgeschäft, um die er sich ernsthaft, aber zu langsam beworben
hatte --- sie alle erschienen untermischt mit Fremden oder schon
Vergessenen, aber statt ihm und seiner Familie zu helfen, waren sie
sämtlich unzugänglich, und er war froh, wenn sie verschwanden. Dann\est\ aber
war er wieder gar nicht in der Laune, sich um seine Familie zu sorgen,
bloß Wut über die schlechte Wartung erfüllte ihn, und trotzdem er sich
nichts vorstellen konnte, worauf er Appetit gehabt hätte, machte er doch
Pläne, wie er in die Speisekammer gelangen könnte, um dort zu nehmen,
was ihm, auch wenn er keinen Hunger hatte, immerhin gebührte. Ohne jetzt
mehr nachzudenken, womit man Gregor einen besonderen Gefallen machen
könnte, schob die Schwester eiligst, ehe sie morgens und mittags ins
Geschäft lief, mit dem Fuß irgendeine beliebige Speise in Gregors Zimmer
hinein, um sie am Abend, gleichgültig dagegen, ob die Speise vielleicht
nur gekostet oder --- der häufigste Fall --- gänzlich unberührt war, mit
einem Schwenken des Besens hinauszukehren. Das Aufräumen des Zimmers,
das sie nun immer abends besorgte, konnte gar nicht mehr schneller getan
sein. Schmutzstreifen zogen sich die Wände entlang, hie und da lagen
Knäuel von Staub und Unrat. In der ersten Zeit stellte sich Gregor bei
der Ankunft der Schwester in derartige besonders bezeichnende Winkel, um
ihr durch diese Stellung gewissermaßen einen Vorwurf zu machen. Aber er
hätte wohl wochenlang dort bleiben können, ohne daß sich die Schwester
gebessert hätte; sie sah ja den Schmutz genau so wie er, aber sie hatte
sich eben entschlossen, ihn zu lassen. Dabei wachte sie mit einer an ihr
ganz neuen Empfindlichkeit, die überhaupt die ganze Familie ergriffen
hatte, darüber, daß das Aufräumen von Gregors Zimmer ihr vorbehalten
blieb. Einmal hatte die Mutter Gregors Zimmer einer großen Reinigung
unterzogen, die ihr nur nach Verbrauch einiger Kübel Wasser gelungen war
--- die viele Feuchtigkeit kränkte allerdings Gregor auch und er lag
breit, verbittert und unbeweglich auf dem Kanapee ---, aber die Strafe
blieb für die Mutter nicht aus. Denn kaum hatte am Abend die Schwester
die Veränderung in Gregors Zimmer bemerkt, als sie, aufs höchste
beleidigt, ins Wohnzimmer lief und, trotz der beschwörend erhobenen
Hände der Mutter, in einen Weinkrampf ausbrach, dem die Eltern --- der
Vater war natürlich aus seinem Sessel aufgeschreckt worden --- zuerst
erstaunt und hilflos zusahen; bis auch sie sich zu rühren anfingen; der
Vater rechts der Mutter Vorwürfe machte, daß sie Gregors Zimmer nicht
der Schwester zur Reinigung überließ; links dagegen die Schwester
anschrie, sie werde niemals mehr Gregors Zimmer reinigen dürfen; während
die Mutter den Vater, der sich vor Erregung nicht mehr kannte, ins
Schlafzimmer zu schleppen suchte; die Schwester, von Schluchzen
geschüttelt, mit ihren kleinen Fäusten den Tisch bearbeitete; und Gregor
laut vor Wut darüber zischte, daß es keinem einfiel, die Tür zu
schließen und ihm diesen Anblick und Lärm zu ersparen.

Aber selbst wenn die Schwester, erschöpft von ihrer Berufsarbeit, dessen
überdrüssig geworden war, für Gregor, wie früher, zu sorgen, so hätte
noch keineswegs die Mutter für sie eintreten müssen und Gregor hätte
doch nicht vernachlässigt zu werden brauchen. Denn nun war die
Bedienerin da. Diese alte Witwe, die in ihrem langen Leben mit Hilfe
ihres starken Knochenbaues das Ärgste überstanden haben mochte, hatte
keinen eigentlichen Abscheu vor Gregor. Ohne irgendwie neugierig zu
sein, hatte sie zufällig einmal die Tür von Gregors Zimmer aufgemacht
und war im Anblick Gregors, der, gänzlich überrascht, trotzdem ihn
niemand jagte, hin- und herzulaufen begann, die Hände im Schoß gefaltet
staunend stehen geblieben. Seitdem versäumte sie nicht, stets flüchtig
morgens und abends die Tür ein wenig zu öffnen und zu Gregor
hineinzuschauen. Anfangs rief sie ihn auch zu sich herbei, mit Worten,
die sie wahrscheinlich für freundlich hielt, wie »Komm mal herüber,
alter Mistkäfer!« oder »Seht mal den alten Mistkäfer!« Auf solche
Ansprachen antwortete Gregor mit nichts, sondern blieb unbeweglich auf
seinem Platz, als sei die Tür gar nicht geöffnet worden. Hätte man doch
dieser Bedienerin, statt sie nach ihrer Laune ihn nutzlos stören zu
lassen, lieber den Befehl gegeben, sein Zimmer täglich zu reinigen!
Einmal am frühen Morgen --- ein heftiger Regen, vielleicht schon ein
Zeichen des kommenden Frühjahrs, schlug an die Scheiben --- war Gregor,
als die Bedienerin mit ihren Redensarten wieder begann, derartig
erbittert, daß er, wie zum Angriff, allerdings langsam und hinfällig,
sich gegen sie wendete. Die Bedienerin aber, statt sich zu fürchten, hob
bloß einen in der Nähe der Tür befindlichen Stuhl hoch empor, und wie
sie mit groß geöffnetem Munde dastand, war ihre Absicht klar, den Mund
erst zu schließen, wenn der Sessel in ihrer Hand auf Gregors Rücken
niederschlagen würde. »Also weiter geht es nicht?« fragte sie, als
Gregor sich wieder umdrehte, und stellte den Sessel ruhig in die Ecke
zurück.

Gregor aß nun fast gar nichts mehr. Nur wenn er zufällig an der
vorbereiteten Speise vorüberkam, nahm er zum Spiel einen Bissen in den
Mund, hielt ihn dort stundenlang und spie ihn dann meist wieder aus.
Zuerst dachte er, es sei die Trauer über den Zustand seines Zimmers, die
ihn vom Essen abhalte, aber gerade mit den Veränderungen des Zimmers
söhnte er sich sehr bald aus. Man hatte sich angewöhnt, Dinge, die man
anderswo nicht unterbringen konnte, in dieses Zimmer hineinzustellen,
und solcher Dinge gab es nun viele, da man ein Zimmer der Wohnung an
drei Zimmerherren vermietet hatte. Diese ernsten Herren, --- alle drei
hatten Vollbärte, wie Gregor einmal durch eine Türspalte feststellte ---
waren peinlich auf Ordnung, nicht nur in ihrem Zimmer, sondern, da sie
sich nun einmal hier eingemietet hatten, in der ganzen Wirtschaft, also
insbesondere in der Küche, bedacht. Unnützen oder gar schmutzigen Kram
ertrugen sie nicht. Überdies hatten sie zum größten Teil ihre eigenen
Einrichtungsstücke mitgebracht. Aus diesem Grunde waren viele Dinge
überflüssig geworden, die zwar nicht verkäuflich waren, die man aber
auch nicht wegwerfen wollte. Alle diese wanderten in Gregors Zimmer.
Ebenso auch die Aschenkiste und die Abfallkiste aus der Küche. Was nur
im Augenblick unbrauchbar war, schleuderte die Bedienerin, die es immer
sehr eilig hatte, einfach in Gregors Zimmer; Gregor sah\est\ glücklicherweise
meist nur den betreffenden Gegenstand und die Hand, die ihn hielt. Die
Bedienerin hatte vielleicht die Absicht, bei Zeit und Gelegenheit die
Dinge wieder zu holen oder alle insgesamt mit einemmal hinauszuwerfen,
tatsächlich aber blieben sie dort liegen, wohin sie durch den ersten
Wurf gekommen waren, wenn nicht Gregor sich durch das Rumpelzeug wand
und es in Bewegung brachte, zuerst gezwungen, weil kein sonstiger Platz
zum Kriechen frei war, später aber mit wachsendem Vergnügen, obwohl er
nach solchen Wanderungen, zum Sterben müde und traurig, wieder
stundenlang sich nicht rührte.

Da die Zimmerherren manchmal auch ihr Abendessen zu Hause im gemeinsamen
Wohnzimmer einnahmen, blieb die Wohnzimmertür an manchen Abenden
geschlossen, aber Gregor verzichtete ganz leicht auf das Öffnen der Tür,
hatte er doch schon manche Abende, an denen sie geöffnet war, nicht
ausgenützt, sondern war, ohne daß es die Familie merkte, im dunkelsten
Winkel seines Zimmers gelegen. Einmal aber hatte die Bedienerin die Tür
zum Wohnzimmer ein wenig offen gelassen, und sie blieb so offen, auch
als die Zimmerherren am Abend eintraten und Licht gemacht wurde. Sie
setzten sich oben an den Tisch, wo in früheren Zeiten der Vater, die
Mutter und Gregor gesessen hatten, entfalteten die Servietten und nahmen
Messer und Gabel in die Hand. Sofort erschien in der Tür die Mutter mit
einer Schüssel Fleisch und knapp hinter ihr die Schwester mit einer
Schüssel hochgeschichteter Kartoffeln. Das Essen dampfte mit starkem
Rauch. Die Zimmerherren beugten sich über die vor sie hingestellten
Schüsseln, als wollten sie sie vor dem Essen prüfen, und tatsächlich
zerschnitt der, welcher in der Mitte saß und den anderen zwei als
Autorität zu gelten schien, ein Stück Fleisch noch auf der Schüssel,
offenbar um festzustellen, ob es mürbe genug sei und ob es nicht etwa in
die Küche zurückgeschickt werden solle. Er war befriedigt, und Mutter
und Schwester, die gespannt zugesehen hatten, begannen aufatmend zu
lächeln.

Die Familie selbst aß in der Küche. Trotzdem kam der Vater, ehe er in
die Küche ging, in dieses Zimmer herein und machte mit einer einzigen
Verbeugung, die Kappe in der Hand, einen Rundgang um den Tisch. Die
Zimmerherren erhoben sich sämtlich und murmelten etwas in ihre Bärte.
Als sie dann allein waren, aßen sie fast unter vollkommenem
Stillschweigen. Sonderbar schien es Gregor, daß man aus allen
mannigfachen Geräuschen des Essens immer wieder ihre kauenden Zähne
heraushörte, als ob damit Gregor gezeigt werden sollte, daß man Zähne
brauche, um zu essen, und daß man auch mit den schönsten zahnlosen
Kiefern nichts ausrichten könne. »Ich habe ja Appetit,« sagte sich
Gregor sorgenvoll, »aber nicht auf diese Dinge. Wie sich diese
Zimmerherren nähren, und ich komme um!«

Gerade an diesem Abend --- Gregor erinnerte sich nicht, während der
ganzen Zeit die Violine gehört zu haben --- ertönte sie von der Küche
her. Die Zimmerherren hatten schon ihr Nachtmahl beendet, der mittlere
hatte eine Zeitung hervorgezogen, den zwei anderen je ein Blatt gegeben,
und nun lasen sie zurückgelehnt und rauchten. Als die Violine zu spielen
begann, wurden sie aufmerksam, erhoben sich und gingen auf den
Fußspitzen zur Vorzimmertür, in der sie aneinandergedrängt stehen
blieben. Man mußte sie von der Küche aus gehört haben, denn der Vater
rief: »Ist den Herren das Spiel vielleicht unangenehm? Es kann sofort
eingestellt werden.« »Im Gegenteil,« sagte der mittlere der Herren,
»möchte das Fräulein nicht zu uns hereinkommen und hier im Zimmer
spielen, wo es doch viel bequemer und gemütlicher ist?« »O bitte,« rief
der Vater, als sei er der Violinspieler. Die Herren traten ins Zimmer
zurück und warteten. Bald kam der Vater mit dem Notenpult, die Mutter
mit den Noten und die Schwester mit der Violine. Die Schwester bereitete
alles ruhig zum Spiele vor; die Eltern, die niemals früher Zimmer
vermietet hatten und deshalb die Höflichkeit gegen die Zimmerherren
übertrieben, wagten gar nicht, sich auf ihre eigenen Sessel zu setzen;
der Vater lehnte an der Tür, die rechte Hand zwischen zwei Knöpfe des
geschlossenen Livreerockes gesteckt; die Mutter aber erhielt von einem
Herrn einen Sessel angeboten und saß, da sie den Sessel dort ließ, wohin
ihn der Herr zufällig gestellt hatte, abseits in einem Winkel.

Die Schwester begann zu spielen; Vater und Mutter verfolgten, jeder von
seiner Seite, aufmerksam die Bewegungen ihrer Hände. Gregor hatte, von
dem Spiele angezogen, sich ein wenig weiter vorgewagt und war schon mit
dem Kopf im Wohnzimmer. Er wunderte sich kaum darüber, daß er in letzter
Zeit so wenig Rücksicht auf die andern nahm; früher war diese
Rücksichtnahme sein Stolz gewesen. Und dabei hätte er gerade jetzt mehr
Grund gehabt, sich zu verstecken, denn infolge des Staubes, der in
seinem Zimmer überall lag und bei der kleinsten Bewegung umherflog, war
auch er ganz staubbedeckt; Fäden, Haare, Speiseüberreste schleppte er
auf seinem Rücken und an den Seiten mit sich herum; seine
Gleichgültigkeit gegen alles war viel zu groß, als daß er sich, wie
früher mehrmals während des Tages, auf den Rücken gelegt und am Teppich
gescheuert hätte. Und trotz dieses Zustandes hatte er keine Scheu, ein
Stück auf dem makellosen Fußboden des Wohnzimmers vorzurücken.

Allerdings achtete auch niemand auf ihn. Die Familie war gänzlich vom
Violinspiel in Anspruch genommen; die Zimmerherren dagegen, die
zunächst, die Hände in den Hosentaschen, viel zu nahe hinter dem
Notenpult der Schwester sich aufgestellt hatten, so daß sie alle in die
Noten hätte sehen können, was sicher die Schwester stören mußte, zogen
sich bald unter halblauten Gesprächen mit gesenkten Köpfen zum Fenster
zurück, wo sie, vom Vater besorgt beobachtet, auch blieben. Es hatte nun
wirklich den überdeutlichen Anschein, als wären sie in ihrer Annahme,
ein schönes oder unterhaltendes Violinspiel zu hören, enttäuscht, hätten
die ganze Vorführung satt und ließen sich nur aus Höflichkeit noch in
ihrer Ruhe stören. Besonders die Art, wie sie alle aus Nase und Mund den
Rauch ihrer Zigarren in die Höhe bliesen, ließ auf große Nervosität
schließen. Und doch spielte die Schwester so schön. Ihr Gesicht war zur
Seite geneigt, prüfend und traurig folgten ihre Blicke den Notenzeilen.
Gregor kroch noch ein Stück vorwärts und hielt den Kopf eng an den
Boden, um möglicherweise ihren Blicken begegnen zu können. War er ein
Tier, da ihn Musik so ergriff? Ihm war, als zeige sich ihm der Weg zu
der ersehnten unbekannten Nahrung. Er war entschlossen, bis zur
Schwester vorzudringen, sie am Rock zu zupfen und ihr dadurch
anzudeuten, sie möge doch mit ihrer Violine in sein Zimmer kommen, denn
niemand lohnte hier das Spiel so, wie er es lohnen wollte. Er wollte sie
nicht mehr aus seinem Zimmer lassen, wenigstens nicht, solange er lebte;
seine Schreckgestalt sollte ihm zum erstenmal nützlich werden; an allen
Türen seines Zimmers wollte er gleichzeitig sein und den Angreifern
entgegenfauchen; die Schwester aber sollte nicht gezwungen, sondern
freiwillig bei ihm bleiben; sie sollte neben ihm auf dem Kanapee sitzen,
das Ohr zu ihm herunterneigen, und er wollte ihr dann anvertrauen, daß
er die feste Absicht gehabt habe, sie auf das Konservatorium zu
schicken, und daß er dies, wenn nicht das Unglück dazwischen gekommen
wäre, vergangene Weihnachten --- Weihnachten war doch wohl schon vorüber?
--- allen gesagt hätte, ohne sich um irgendwelche Widerreden zu kümmern.
Nach dieser Erklärung würde die Schwester in Tränen der Rührung
ausbrechen, und Gregor würde sich bis zu ihrer Achsel erheben und ihren
Hals küssen, den sie, seitdem sie ins Geschäft ging, frei ohne Band oder
Kragen trug.

»Herr Samsa!« rief der mittlere Herr dem Vater zu und zeigte, ohne ein
weiteres Wort zu verlieren, mit dem Zeigefinger auf den langsam sich
vorwärtsbewegenden Gregor. Die Violine verstummte, der mittlere
Zimmerherr lächelte erst einmal kopfschüttelnd seinen Freunden zu und
sah dann wieder auf Gregor hin. Der Vater schien es für nötiger zu
halten, statt Gregor zu vertreiben, vorerst die Zimmerherren zu
beruhigen, trotzdem diese gar nicht aufgeregt waren und Gregor sie mehr
als das Violinspiel zu unterhalten schien. Er eilte zu ihnen und suchte
sie mit ausgebreiteten Armen in ihr Zimmer zu drängen und gleichzeitig
mit seinem Körper ihnen den Ausblick auf Gregor zu nehmen. Sie wurden
nun tatsächlich ein wenig böse, man wußte nicht mehr, ob über das
Benehmen des Vaters oder über die ihnen jetzt aufgehende Erkenntnis,
ohne es zu wissen, einen solchen Zimmernachbar wie Gregor besessen zu
haben. Sie verlangten vom Vater Erklärungen, hoben ihrerseits die Arme,
zupften unruhig an ihren Bärten und wichen nur langsam gegen ihr Zimmer
zurück. Inzwischen hatte die Schwester die Verlorenheit, in die sie nach
dem plötzlich abgebrochenen Spiel verfallen war, überwunden, hatte sich,
nachdem sie eine Zeitlang in den lässig hängenden Händen Violine und
Bogen gehalten und weiter, als spiele sie noch, in die Noten gesehen
hatte, mit einem Male aufgerafft, hatte das Instrument auf den Schoß der
Mutter gelegt, die in Atembeschwerden mit heftig arbeitenden Lungen noch
auf ihrem Sessel saß, und war in das Nebenzimmer gelaufen, dem sich die
Zimmerherren unter dem Drängen des Vaters schon schneller näherten. Man
sah, wie unter den geübten Händen der Schwester die Decken und Polster
in den Betten in die Höhe flogen und sich ordneten. Noch ehe die Herren
das Zimmer erreicht hatten, war sie mit dem Aufbetten fertig und
schlüpfte heraus. Der Vater schien wieder von seinem Eigensinn derartig
ergriffen, daß er jeden Respekt vergaß, den er seinen Mietern immerhin
schuldete. Er drängte nur und drängte, bis schon in der Tür des Zimmers
der mittlere der Herren donnernd mit dem Fuß aufstampfte und dadurch den
Vater zum Stehen brachte. »Ich erkläre hiermit,« sagte er, hob die Hand
und suchte mit den Blicken auch die Mutter und die Schwester, »daß ich
mit Rücksicht auf die in dieser Wohnung und Familie herrschenden
widerlichen Verhältnisse« --- hierbei spie er kurz entschlossen auf den
Boden --- »mein Zimmer augenblicklich kündige. Ich werde natürlich auch
für die Tage, die ich hier gewohnt habe, nicht das Geringste bezahlen,
dagegen werde ich es mir noch überlegen, ob ich nicht mit irgendwelchen
--- glauben Sie mir --- sehr leicht zu begründenden Forderungen gegen Sie
auftreten werde.« Er schwieg und sah gerade vor sich hin, als erwarte er
etwas. Tatsächlich fielen sofort seine zwei Freunde mit den Worten ein:
»Auch wir kündigen augenblicklich.« Darauf faßte er die Türklinke und
schloß mit einem Krach die Tür.

Der Vater wankte mit tastenden Händen zu seinem Sessel und ließ sich
hineinfallen; es sah aus, als strecke er sich zu seinem gewöhnlichen
Abendschläfchen, aber das starke Nicken seines wie haltlosen Kopfes
zeigte, daß er ganz und gar nicht schlief. Gregor war die ganze Zeit
still auf dem Platz gelegen, auf dem ihn die Zimmerherren ertappt
hatten. Die Enttäuschung über das Mißlingen seines Planes, vielleicht
aber auch die durch das viele Hungern verursachte Schwäche machten es
ihm unmöglich, sich zu bewegen. Er fürchtete mit einer gewissen
Bestimmtheit schon für den nächsten Augenblick einen allgemeinen über
ihn sich entladenden Zusammensturz und wartete. Nicht einmal die Violine
schreckte ihn auf, die, unter den zitternden Fingern der Mutter hervor,
ihr vom Schoße fiel und einen hallenden Ton von sich gab.

»Liebe Eltern,« sagte die Schwester und schlug zur Einleitung mit der
Hand auf den Tisch, »so geht es nicht weiter. Wenn ihr das vielleicht
nicht einsehet, ich sehe es ein. Ich will vor diesem Untier nicht den
Namen meines Bruders aussprechen und sage daher bloß: wir müssen
versuchen es loszuwerden. Wir haben das Menschenmögliche versucht, es zu
pflegen und zu dulden, ich glaube, es kann uns niemand den geringsten
Vorwurf machen.«

»Sie hat tausendmal recht,« sagte der Vater für sich. Die Mutter, die
noch immer nicht genug Atem finden konnte, fing mit einem irrsinnigen
Ausdruck der Augen dumpf in die vorgehaltene Hand zu husten an.

Die Schwester eilte zur Mutter und hielt ihr die Stirn. Der Vater schien
durch die Worte der Schwester auf bestimmtere Gedanken gebracht zu sein,
hatte sich aufrecht gesetzt, spielte mit seiner Dienermütze zwischen den
Tellern, die noch vom Nachtmahl der Zimmerherren her auf dem Tische
standen, und sah bisweilen auf den stillen Gregor hin.

»Wir müssen es loszuwerden suchen,« sagte die Schwester nun
ausschließlich zum Vater, denn die Mutter hörte in ihrem Husten nichts,
»es bringt euch noch beide um, ich sehe es kommen. Wenn man schon so
schwer arbeiten muß, wie wir alle, kann man nicht noch zu Hause diese
ewige Quälerei ertragen. Ich kann es auch nicht mehr.« Und sie brach so
heftig in Weinen aus, daß ihre Tränen auf das Gesicht der Mutter
niederflossen, von dem sie sie mit mechanischen Handbewegungen wischte.

»Kind,« sagte der Vater mitleidig und mit auffallendem Verständnis, »was
sollen wir aber tun?«

Die Schwester zuckte nur die Achseln zum Zeichen der Ratlosigkeit, die
sie nun während des Weinens im Gegensatz zu ihrer früheren Sicherheit
ergriffen hatte.

»Wenn er uns verstünde,« sagte der Vater halb fragend; die Schwester
schüttelte aus dem Weinen heraus heftig die Hand zum Zeichen, daß daran
nicht zu denken sei.

»Wenn er uns verstünde,« wiederholte der Vater und nahm durch Schließen
der Augen die Überzeugung der Schwester von der Unmöglichkeit dessen in
sich auf, »dann wäre vielleicht ein Übereinkommen mit ihm möglich. Aber
so ---«

»Weg muß es,« rief die Schwester, »das ist das einzige Mittel, Vater. Du
mußt bloß den Gedanken loszuwerden suchen, daß es Gregor ist. Daß wir es
so lange geglaubt haben, das ist ja unser eigentliches Unglück. Aber wie
kann es denn Gregor sein? Wenn es Gregor wäre, er hätte längst
eingesehen, daß ein Zusammenleben von Menschen mit einem solchen Tier
nicht möglich ist, und wäre freiwillig fortgegangen. Wir hätten dann
keinen Bruder, aber könnten weiter leben und sein Andenken in Ehren
halten. So aber verfolgt uns dieses Tier, vertreibt die Zimmerherren,
will offenbar die ganze Wohnung einnehmen und uns auf der Gasse
übernachten lassen. Sieh nur, Vater,« schrie sie plötzlich auf, »er
fängt schon wieder an!« Und in einem für Gregor gänzlich
unverständlichen Schrecken verließ die Schwester sogar die Mutter, stieß
sich förmlich von ihrem Sessel ab, als wollte sie lieber die Mutter
opfern, als in Gregors Nähe bleiben, und eilte hinter den Vater, der,
lediglich durch ihr Benehmen erregt, auch aufstand und die Arme wie zum
Schutze der Schwester vor ihr halb erhob.

Aber Gregor fiel es doch gar nicht ein, irgend jemandem und gar seiner
Schwester Angst machen zu wollen. Er hatte bloß angefangen sich
umzudrehen, um in sein Zimmer zurückzuwandern, und das nahm sich
allerdings auffallend aus, da er infolge seines leidenden Zustandes bei
den schwierigen Umdrehungen mit seinem Kopfe nachhelfen mußte, den er
hierbei viele Male hob und gegen den Boden schlug. Er hielt inne und sah
sich um. Seine gute Absicht schien erkannt worden zu sein; es war nur
ein augenblicklicher Schrecken gewesen. Nun sahen ihn alle schweigend
und traurig an. Die Mutter lag, die Beine ausgestreckt und
aneinandergedrückt, in ihrem Sessel, die Augen fielen ihr vor Ermattung
fast zu; der Vater und die Schwester saßen nebeneinander, die Schwester
hatte ihre Hand um des Vaters Hals gelegt.

»Nun darf ich mich schon vielleicht umdrehen,« dachte Gregor und begann
seine Arbeit wieder. Er konnte das Schnaufen der Anstrengung nicht
unterdrücken und mußte auch hie und da ausruhen. Im übrigen drängte ihn
auch niemand, es war alles ihm selbst überlassen. Als er die Umdrehung
vollendet hatte, fing er sofort an, geradeaus zurückzuwandern. Er
staunte über die große Entfernung, die ihn von seinem Zimmer trennte,
und begriff gar nicht, wie er bei seiner Schwäche vor kurzer Zeit den
gleichen Weg, fast ohne es zu merken, zurückgelegt hatte. Immerfort nur
auf rasches Kriechen bedacht, achtete er kaum darauf, daß kein Wort,
kein Ausruf seiner Familie ihn störte. Erst als er schon in der Tür war,
wendete er den Kopf, nicht, vollständig, denn er fühlte den Hals steif
werden, immerhin sah er noch, daß sich hinter ihm nichts verändert
hatte, nur die Schwester war aufgestanden. Sein letzter Blick streifte
die Mutter, die nun völlig eingeschlafen war.

Kaum war er innerhalb seines Zimmers, wurde die Tür eiligst zugedrückt,
festgeriegelt und versperrt. Über den plötzlichen Lärm hinter sich
erschrak Gregor so, daß ihm die Beinchen einknickten. Es war die
Schwester, die sich so beeilt hatte. Aufrecht war sie schon da
gestanden und hatte gewartet, leichtfüßig war sie dann
vorwärtsgesprungen, Gregor hatte sie gar nicht kommen hören, und ein
»Endlich!« rief sie den Eltern zu, während sie den Schlüssel im Schloß
umdrehte.

»Und jetzt?« fragte sich Gregor und sah sich im Dunkeln um. Er machte
bald die Entdeckung, daß er sich nun überhaupt nicht mehr rühren konnte.
Er wunderte sich darüber nicht, eher kam es ihm unnatürlich vor, daß er
sich bis jetzt tatsächlich mit diesen dünnen Beinchen hatte fortbewegen
können. Im übrigen fühlte er sich verhältnismäßig behaglich. Er hatte
zwar Schmerzen im ganzen Leib, aber ihm war, als würden sie allmählich
schwächer und schwächer und würden schließlich ganz vergehen. Den
verfaulten Apfel in seinem Rücken und die entzündete Umgebung, die ganz
von weichem Staub bedeckt war, spürte er schon kaum. An seine Familie
dachte er mit Rührung und Liebe zurück. Seine Meinung darüber, daß er
verschwinden müsse, war womöglich noch entschiedener, als die seiner
Schwester. In diesem Zustand leeren und friedlichen Nachdenkens blieb
er, bis die Turmuhr die dritte Morgenstunde schlug. Den Anfang des
allgemeinen Hellerwerdens draußen vor dem Fenster erlebte er noch. Dann
sank sein Kopf ohne seinen Willen gänzlich nieder, und aus seinen
Nüstern strömte sein letzter Atem schwach hervor.

Als am frühen Morgen die Bedienerin kam --- vor lauter Kraft und Eile
schlug sie, wie oft man sie auch schon gebeten hatte, das zu vermeiden,
alle Türen derartig zu, daß in der ganzen Wohnung von ihrem Kommen an
kein ruhiger Schlaf mehr möglich war ---, fand sie bei ihrem gewöhnlichen
kurzen Besuch bei Gregor zuerst nichts Besonderes. Sie dachte, er liege
absichtlich so unbeweglich da und spiele den Beleidigten; sie traute
ihm allen möglichen Verstand zu. Weil sie zufällig den langen Besen in
der Hand hielt, suchte sie mit ihm Gregor von der Tür aus zu kitzeln.
Als sich auch da kein Erfolg zeigte, wurde sie ärgerlich und stieß ein
wenig in Gregor hinein, und erst als sie ihn ohne jeden Widerstand von
seinem Platze geschoben hatte, wurde sie aufmerksam. Als sie bald den
wahren Sachverhalt erkannte, machte sie große Augen, pfiff vor sich hin,
hielt sich aber nicht lange auf, sondern riß die Tür des Schlafzimmers
auf und rief mit lauter Stimme in das Dunkel hinein: »Sehen Sie nur mal
an, es ist krepiert; da liegt es, ganz und gar krepiert!«

Das Ehepaar Samsa saß im Ehebett aufrecht da und hatte zu tun, den
Schrecken über die Bedienerin zu verwinden, ehe es dazu kam, ihre
Meldung aufzufassen. Dann aber stiegen Herr und Frau Samsa, jeder auf
seiner Seite, eiligst aus dem Bett, Herr Samsa warf die Decke über seine
Schultern, Frau Samsa kam nur im Nachthemd hervor; so traten sie in
Gregors Zimmer. Inzwischen hatte sich\est\ auch die Tür des Wohnzimmers
geöffnet, in dem Grete seit dem Einzug der Zimmerherren schlief; sie war
völlig angezogen, als hätte sie gar nicht geschlafen, auch ihr bleiches
Gesicht schien das zu beweisen. »Tot?« sagte Frau Samsa und sah fragend
zur Bedienerin auf, trotzdem sie doch alles selbst prüfen und sogar ohne
Prüfung erkennen konnte. »Das will ich meinen,« sagte die Bedienerin und
stieß zum Beweis Gregors Leiche mit dem Besen noch ein großes Stück
seitwärts. Frau Samsa machte eine Bewegung, als wolle sie den Besen
zurückhalten, tat es aber nicht. »Nun,« sagte Herr Samsa, »jetzt können
wir Gott danken.« Er bekreuzte sich, und die drei Frauen folgten seinem
Beispiel. Grete, die kein Auge von der Leiche wendete, sagte: »Seht
nur, wie mager er war. Er hat ja auch schon so lange Zeit nichts
gegessen. So wie die Speisen hereinkamen, sind sie wieder
hinausgekommen.« Tatsächlich war Gregors Körper vollständig flach und
trocken, man erkannte das eigentlich erst jetzt, da er nicht mehr von
den Beinchen gehoben war und auch sonst nichts den Blick ablenkte.

»Komm, Grete, auf ein Weilchen zu uns herein,« sagte Frau Samsa mit
einem wehmütigen Lächeln, und Grete ging, nicht ohne nach der Leiche
zurückzusehen, hinter den Eltern in das Schlafzimmer. Die Bedienerin
schloß die Tür und öffnete gänzlich das Fenster. Trotz des frühen
Morgens war der frischen Luft schon etwas Lauigkeit beigemischt. Es war
eben schon Ende März.

Aus ihrem Zimmer traten die drei Zimmerherren und sahen sich erstaunt
nach ihrem Frühstück um; man hatte sie vergessen. »Wo ist das
Frühstück?« fragte der mittlere der Herren mürrisch die Bedienerin.
Diese aber legte den Finger an den Mund und winkte dann hastig und
schweigend den Herren zu, sie möchten in Gregors Zimmer kommen. Sie
kamen auch und standen dann, die Hände in den Taschen ihrer etwas
abgenützten Röckchen, in dem nun schon ganz hellen Zimmer um Gregors
Leiche herum.

Da öffnete sich die Tür des Schlafzimmers, und Herr Samsa erschien in
seiner Livree, an einem Arm seine Frau, am anderen seine Tochter. Alle
waren ein wenig verweint; Grete drückte bisweilen ihr Gesicht an den Arm
des Vaters.

»Verlassen Sie sofort meine Wohnung!« sagte Herr Samsa und zeigte auf
die Tür, ohne die Frauen von sich zu lassen. »Wie meinen Sie das?« sagte
der mittlere der Herren etwas bestürzt und lächelte süßlich. Die zwei
anderen hielten die Hände auf dem Rücken und rieben sie ununterbrochen
aneinander, wie in freudiger Erwartung eines großen Streites, der aber
für sie günstig ausfallen mußte. »Ich meine es genau so, wie ich es
sage,« antwortete Herr Samsa und ging in einer Linie mit seinen zwei
Begleiterinnen auf den Zimmerherrn zu. Dieser stand zuerst still da und
sah zu Boden, als ob sich die Dinge in seinem Kopf zu einer neuen
Ordnung zusammenstellten. »Dann gehen wir also,« sagte er dann und sah
zu Herrn Samsa auf,\est\ als verlange er in einer plötzlich ihn überkommenden
Demut sogar für diesen Entschluß eine neue Genehmigung. Herr Samsa
nickte ihm bloß mehrmals kurz mit großen Augen zu. Daraufhin ging der
Herr tatsächlich sofort mit langen Schritten ins Vorzimmer; seine beiden
Freunde hatten schon ein Weilchen lang mit ganz ruhigen Händen
aufgehorcht und hüpften ihm jetzt geradezu nach, wie in Angst, Herr
Samsa könnte vor ihnen ins Vorzimmer eintreten und die Verbindung mit
ihrem Führer stören. Im Vorzimmer nahmen alle drei die Hüte vom
Kleiderrechen, zogen ihre Stöcke aus dem Stockbehälter, verbeugten sich
stumm und verließen die Wohnung. In einem, wie sich zeigte, gänzlich
unbegründeten Mißtrauen trat Herr Samsa mit den zwei Frauen auf den
Vorplatz hinaus; an das Geländer gelehnt, sahen sie zu, wie die drei
Herren zwar langsam, aber ständig die lange Treppe hinunterstiegen, in
jedem Stockwerk in einer bestimmten Biegung des Treppenhauses
verschwanden und nach ein paar Augenblicken wieder hervorkamen; je
tiefer sie gelangten, desto mehr verlor sich das Interesse der Familie
Samsa für sie, und als ihnen entgegen und dann hoch über sie hinweg ein
Fleischergeselle mit der Trage auf dem Kopf in stolzer Haltung
heraufstieg, verließ bald Herr Samsa mit den Frauen das Geländer, und
alle kehrten, wie erleichtert, in ihre Wohnung zurück.

\pagebreak

Sie beschlossen, den heutigen Tag zum Ausruhen und Spazierengehen zu
verwenden; sie hatten diese Arbeitsunterbrechung nicht nur verdient, sie
brauchten sie sogar unbedingt. Und so setzten sie sich zum Tisch und
schrieben drei Entschuldigungsbriefe, Herr Samsa an seine Direktion,
Frau Samsa an ihren Auftraggeber, und Grete an ihren Prinzipal. Während
des Schreibens kam die Bedienerin herein, um zu sagen, daß sie fortgehe,
denn ihre Morgenarbeit war beendet. Die drei Schreibenden nickten zuerst
bloß, ohne aufzuschauen, erst als die Bedienerin sich immer noch nicht
entfernen wollte, sah man ärgerlich auf. »Nun?« fragte Herr Samsa. Die
Bedienerin stand lächelnd in der Tür, als habe sie der Familie ein
großes Glück zu melden, werde es aber nur dann tun, wenn sie gründlich
ausgefragt werde. Die fast aufrechte kleine Straußfeder auf ihrem Hut,
über die sich Herr Samsa schon während ihrer ganzen Dienstzeit ärgerte,
schwankte leicht nach allen Richtungen. »Also was wollen Sie
eigentlich?« fragte Frau Samsa, vor welcher die Bedienerin noch am
meisten Respekt hatte. »Ja,« antwortete die Bedienerin und konnte vor
freundlichem Lachen nicht gleich weiter reden, »also darüber, wie das
Zeug von nebenan weggeschafft werden soll, müssen Sie sich keine Sorge
machen. Es ist schon in Ordnung.« Frau Samsa und Grete beugten sich zu
ihren Briefen nieder, als wollten sie weiterschreiben; Herr Samsa,
welcher merkte, daß die Bedienerin nun alles ausführlich zu beschreiben
anfangen wollte, wehrte dies mit ausgestreckter Hand entschieden ab. Da
sie aber nicht erzählen durfte, erinnerte sie sich an die große Eile,
die sie hatte, rief offenbar beleidigt: »Adjes allseits,« drehte sich
wild um und verließ unter fürchterlichem Türezuschlagen die Wohnung.

»Abends wird sie entlassen,« sagte Herr Samsa, bekam aber weder von
seiner Frau noch von seiner Tochter eine Antwort, denn die Bedienerin
schien ihre kaum gewonnene Ruhe wieder gestört zu haben. Sie erhoben
sich, gingen zum Fenster und blieben dort, sich umschlungen haltend.
Herr Samsa drehte sich in seinem Sessel nach ihnen um und beobachtete
sie still ein Weilchen. Dann rief er: »Also kommt doch her. Laßt schon
endlich die alten Sachen. Und nehmt auch ein wenig Rücksicht auf mich.«
Gleich folgten ihm die Frauen, eilten zu ihm, liebkosten ihn und
beendeten rasch ihre Briefe.

Dann verließen alle drei gemeinschaftlich die Wohnung, was sie schon
seit Monaten nicht getan hatten, und fuhren mit der Elektrischen ins
Freie vor die Stadt. Der Wagen, in dem sie allein saßen, war ganz von
warmer Sonne durchschienen. Sie besprachen, bequem auf ihren Sitzen
zurückgelehnt, die Aussichten für die Zukunft, und es fand sich, daß
diese bei näherer Betrachtung durchaus nicht schlecht waren, denn aller
drei Anstellungen waren, worüber sie einander eigentlich noch gar nicht
ausgefragt hatten, überaus günstig und besonders für später
vielversprechend. Die größte augenblickliche Besserung der Lage mußte
sich natürlich leicht durch einen Wohnungswechsel ergeben; sie\est\ wollten
nun eine kleinere und billigere, aber besser gelegene und überhaupt
praktischere Wohnung nehmen, als es die jetzige, noch von Gregor
ausgesuchte war. Während sie sich so unterhielten, fiel es Herrn und
Frau Samsa im Anblick ihrer immer lebhafter werdenden Tochter fast
gleichzeitig ein, wie sie in der letzten Zeit trotz aller Pflege, die
ihre Wangen bleich gemacht hatte, zu einem schönen und üppigen Mädchen
aufgeblüht war. Stiller werdend und fast unbewußt durch Blicke sich
verständigend, dachten sie daran, daß es nun Zeit sein werde, auch einen
braven Mann für sie zu suchen. Und es war ihnen wie eine Bestätigung
ihrer neuen Träume und guten Absichten, als am Ziele ihrer Fahrt die
Tochter als erste sich erhob und ihren jungen Körper dehnte.} 
\ParallelRText{ %\addfontfeature{LetterSpace=-4.0}

\part*{A metamorfose}
\addcontentsline{toc}{chapter}{A metamorfose}
%\hedramarkboth{a metamorfose}{kafka}

\sectionitem

Certa manhã, ao despertar de um sonho inquieto, Gregor Samsa descobriu"-se
em sua cama transformado num insuportável inseto. Deitado de costas, duras
como um casco, ele viu, ao erguer um pouco a cabeça, sua barriga arredondada,
pardacenta, repartida por pregas arqueadas, do alto da qual a coberta, já
quase toda caída, escorregava. Diante de seus olhos moviam"-se
desesperadas suas várias pernas, ridiculamente finas em comparação com
suas proporções de antes.

“O que aconteceu comigo?”, pensou. Não era um sonho. Seu quarto, abrigo
humano e normal em tudo, só um tanto quanto pequeno, jazia em silêncio
entre as quatro paredes velhas conhecidas. Acima da mesa, onde se
espalhavam pacotes desembrulhados de amostras de tecidos --- Samsa era
caixeiro"-viajante ---, pendia a ilustração que ele recortara há pouco tempo
de uma revista e havia encaixilhado numa moldura linda, dourada. Era o
retrato de uma dona elegante, sentada, aprumada, ornamentada com um
barrete e uma estola de peles, que elevava na direção do observador um
pesado regalo também de pele, no interior do qual quase todo o seu
antebraço desaparecia.

O olhar de Gregor voltou"-se então para a janela, e o tempo fechado ---
ouviam"-se gotas de chuva batendo no peitoril de metal --- deixou"-o bastante
melancólico. “Como seria bom dormir um pouco mais e esquecer
essas maluquices”, pensou, mas isso era inexequível, pois estava
acostumado a dormir do lado direito, e no seu estado atual não conseguia
ficar nessa posição. Por mais força que fizesse ao se projetar para a
direita, acabava sempre mandado de volta à posição inicial, de costas. Já
havia tentado umas cem vezes, fechava os olhos para não ter que ver a
movimentação das pernas, e só parou quando começou a sentir na
lateral do corpo uma
ligeira dor, surda, nunca antes sentida.

“Deus do céu”, pensou, “que profissão mais desgastante eu fui escolher! É
viajar todo santo dia. A tensão desse comércio é de fato muito maior do
que o trabalho na loja, e além disso a mim me toca ainda esse tormento das
viagens, a preocupação com as conexões dos trens, a comida péssima, sem
hora certa, o contato humano sempre alternado, nunca permanente, nunca
chegando a ser afetuoso. Que isso tudo vá pro inferno!” Sentiu uma
coceirinha na parte de cima, na barriga; empurrou as costas devagar para
junto da armação da cama, a fim de poder erguer melhor a cabeça; divisou a
região que coçava, coberta de minúsculos pontinhos brancos pronunciados,
não chegou a atinar o que fossem; e quis cutucar o local com uma perna,
mas retirou"-a na mesma hora, pois ao contato foi acometido por um
calafrio.

Deslizou de volta à sua posição anterior. “Esse negócio de acordar tão
cedo”, pensou, “deixa a pessoa apalermada. Um homem deve ter direito a
suas horas de sono. Os outros vendedores levam vida de princesa. Quando,
por exemplo, eu volto para a hospedaria no meio da manhã, e vou passar a
limpo os pedidos, só então esses cidadãos se sentam para tomar o café. Vou
eu tentar a mesma coisa com o chefe que eu tenho; iria parar no olho da
rua. Aliás, vai saber se isso não seria mesmo o melhor para mim. Se eu,
por causa dos meus pais, não estivesse de mãos atadas, já tinha pedido as
contas há muito tempo, teria parado bem na frente do chefe e dito o que
penso com absoluta franqueza. Era capaz dele cair da mesa! Essa é outra
mania esquisita do chefe, sentar na mesa e falar com os funcionários
olhando de cima, sem contar que, por causa de sua audição sofrível, a
gente precisa chegar bem pertinho. Mas não perdi de todo as esperanças;
assim que juntar o dinheiro e saldar a dívida que os meus pais têm com ele
--- o que deve durar ainda uns cinco ou seis anos ---, adoto a medida sem
falta. Então as amarras serão todas rompidas. Por enquanto, todavia, eu
tenho de me levantar, porque o meu trem parte às cinco.”

E olhou de lado na direção do despertador, que fazia tique"-taque em cima
do guarda"-roupa. “Minha Nossa Senhora!”, pensou. Eram seis e meia, e os
ponteiros seguiam mansos adiante, era até mais tarde, já estava perto de
quinze para as sete.
Será que o despertador não tinha tocado? Via"-se da cama
que ele havia sido ajustado direitinho para as quatro; na certa, pois,
tocara. É, mas seria possível manter um sono tranquilo com esse barulho
que chegava a estremecer os móveis? Bem, tranquilo é que não fora o seu
sono, entretanto, talvez por conta disso mesmo, teria sido mais profundo.
Mas o que devia fazer agora? O próximo trem partia às sete; para
alcançá"-lo, teria de se apressar feito um louco, e as amostras ainda não
estavam na mala, e ele próprio não se sentia inteiramente descansado e
disposto. E depois, se o alcançasse, não havia mais como evitar um
acesso furioso do chefe, pois o menino da loja teria aguardado o trem das
cinco e há muito já transmitira o informe de sua falta. Esse era uma cria
do chefe, sem fibra e sem discernimento. Não poderia avisar que estava
doente? Isso, porém, seria demasiado constrangedor e suspeito, pois
Gregor, durante os seus cinco anos de serviço, não ficara doente nem uma
vez sequer. Na certa o chefe viria com o médico da previdência,
repreenderia os pais por causa do filho preguiçoso e rejeitaria qualquer
objeção com base no palpite do médico, que era da opinião de que a boa
saúde nunca faltava aos homens, faltava era a disposição para o trabalho.
E, a propósito, nesse caso estaria ele tão errado assim? Na verdade,
descontada uma certa sonolência realmente superficial, devida ao longo
período de sono, Gregor se sentia muito bem e até estava com uma fome um
pouco além do comum.

Enquanto refletia aos atropelos sobre tudo isso, sem encontrar coragem
para deixar a cama --- o despertador marcava exatamente quinze para as sete
---, foram ouvidas leves batidas na porta ao lado da cabeceira. “Gregor”,
chamavam --- era a mãe ---, “já são quinze pras sete. Você não ia
viajar?” Que
voz mais doce! Ao responder, Gregor se assustou com a sua própria, que era
nitidamente a mesma voz, porém agora, como se vindo do fundo, mesclava"-se
a ela um guincho aflitivo, impossível de reprimir, que só num primeiro
momento deixava as palavras soarem inteligíveis, para corrompê"-las em
seguida com seu eco por trás de cada emissão, de uma tal maneira que a
pessoa não sabia se tinha escutado direito. Gregor teve vontade de
responder em detalhes e explicar tudo, entretanto, dadas as
circunstâncias, limitou"-se a dizer: “É, ia; obrigado, mãe; já vou
levantar”. Na certa por causa da madeira da porta, a alteração na voz de
Gregor não foi notada do lado de fora, pois a mãe se satisfez com a
explicação e se afastou arrastando as chinelas. Contudo, a pequena troca
de palavras alertou os outros membros da família para o fato de que
Gregor, contrariando o previsto, ainda estava em casa, e logo o pai batia
em uma das portas laterais, sem força, mas com o punho. “Gregor, Gregor”,
chamou, “o que houve?” E depois de um breve intervalo voltou a exortar,
engrossando a voz: “Gregor! Gregor!”. Na outra porta lateral, por sua vez,
a irmã chamou baixinho, num tom queixoso: “Gregor? Está se sentindo bem?
Precisa de alguma coisa?”. Gregor respondeu para ambos os lados: “Já estou
indo”, e esforçou"-se para que não reparassem em sua voz, pronunciando as
palavras com o máximo de cuidado e intercalando longas pausas de uma
emissão a outra. Também o pai deu meia volta e retornou ao seu café da
manhã, a irmã entretanto sussurrou: “Gregor, abra, eu insisto”. Mas Gregor
não pensava de forma alguma em abrir, pelo contrário, louvava a precaução
adquirida com as viagens, e até em casa mantinha todas as portas trancadas
durante a noite.

Agora o que queria era levantar"-se com calma e sem ser incomodado,
vestir"-se e, acima de tudo, tomar o café da manhã, para só então refletir
sobre o restante, pois ele via muito bem que ali deitado não conseguiria
conduzir seus pensamentos a nenhuma conclusão satisfatória. Lembrou"-se de
já ter sentido várias vezes na cama uma certa dor leve, provável resultado
do sono em posição desajeitada, que logo em seguida, ao se levantar,
revelava"-se pura imaginação, e estava curioso para ver como aos poucos se
dissipariam suas impressões do dia. Que a alteração na voz nada mais fosse
do que o sintoma de um belo resfriado, doença típica da sua profissão,
disso não tinha a menor dúvida.

Afastar a coberta foi muito fácil; bastou a ele inflar"-se um pouco que ela
caiu por si só. Mas a partir daí ficou difícil, principalmente por causa
de sua largura tão excepcional. Ele teria necessitado de braços e mãos
para se pôr de pé; em vez disso, porém, tinha apenas as várias perninhas
que se movimentavam sem trégua para todos os lados, e que além do mais ele
não conseguia dominar. Queria fazer uma delas dobrar, e essa era a
primeira a ficar esticada; conseguia afinal realizar o que queria com
essa, e já todas as outras, nesse meio tempo, trabalhavam por conta
própria, numa agitação maior e mais aflitiva ainda. “Não vai ficar
estendido aí como um inútil”, Gregor disse a si mesmo.

Primeiro quis deixar a cama com a parte de baixo de seu corpo, mas essa
parte, que aliás ele ainda não vira e da qual também não conseguia fazer
uma ideia muito precisa, mostrou"-se muito difícil de mover; avançava tão
devagar; e quando ele afinal, numa fúria quase animalesca, projetou"-se
para frente com todas as forças, sem prestar atenção, calculou a direção, bateu contra a armação dos pés da cama, e a dor
lancinante que sentiu o ensinou que justo a parte de baixo de seu corpo
talvez fosse no momento a mais sensível.

Em vista disso procurou sair primeiro com a parte de cima, e virou a
cabeça com cuidado para a beira da cama. Pareceu fácil e, apesar da
largura e do peso, seu corpanzil acabou por acompanhar aos poucos a
movimentação da cabeça. Porém, quando a sustentou fora da cama, em pleno
ar, ficou com medo de continuar avançando dessa maneira, porque, se no fim
ele se deixasse cair assim, seria um verdadeiro milagre que sua cabeça não
saísse seriamente machucada. E ele não podia de modo algum se arriscar a
perder os sentidos agora; preferia permanecer na cama.

Mas quando, bufando com a repetição do esforço, voltou à mesma posição de
antes, e tornou a enxergar suas perninhas em luta umas contra as outras,
ainda mais agitadas, se isso era possível, e não viu nenhuma chance de
acalmar e organizar aquela anarquia, repetiu consigo mesmo que era
inadmissível continuar deitado e que o mais lógico seria sacrificar tudo,
ainda que fosse ínfima a esperança de que assim conseguisse sair da cama.
Ao mesmo tempo, porém, não se esquecia de lembrar que naquela situação uma
reflexão calma, a mais calma, seria muito melhor do que resoluções
desesperadas. Nesses momentos buscava com os olhos a janela, aguçando ao
máximo o olhar, mas infelizmente muito pouca confiança e inspiração havia
a extrair da visão da neblina matutina, que ocultava até o lado oposto da
rua estreita. “Sete horas já”, ele disse ao ouvir o clique do despertador,
“sete horas já, e ainda uma neblina dessas.” E durante alguns instantes
permaneceu deitado quieto, quase sem respirar, como se esperasse da
quietude total o retorno das condições normais de realidade.

Mas depois falou para si mesmo: “Antes das sete e quinze, é indispensável
que eu tenha deixado a cama de uma vez. Aliás, até lá terá vindo alguém da
loja perguntar por mim, porque ela abre antes das sete”. E decidiu então
sair da cama com o corpo todo, movendo"-o em toda a sua extensão num
balanço calculado e uniforme. Se chegasse a cair da cama dessa forma, era
de supor que conservaria ilesa a cabeça, que pretendia manter firmemente
ereta durante a queda. As costas pareciam duras; na certa nada sofreriam
no choque com o tapete. O maior receio lhe vinha da atenção que o alto
estrondo despertaria e das preocupações, quando não do susto, que deveria
causar atrás de todas as portas. Mas era preciso correr o risco.

Quando já estava suspenso pela metade para fora da cama --- o novo método
era mais uma diversão que um esforço, ele só precisava de pequenos
arrancos ao balançar ---, ocorreu"-lhe como seria simples se alguém viesse em
seu auxílio. Duas pessoas fortes --- pensava em seu pai e na empregada ---
seriam mais do que suficientes; bastava"-lhes enfiar os braços por baixo de
suas costas redondas, removê"-lo da cama nessa posição, inclinar"-se com a
carga e então apenas assisti"-lo com cautela para que completasse o giro no
chão, onde enfim as perninhas oxalá iriam adquirir algum sentido. No
entanto, descontado o fato de que as portas estavam trancadas, ele deveria
mesmo pedir ajuda? Apesar de toda a aflição, não pôde deixar de sorrir a
esse pensamento.

Tanto já se deslocara que a um balanço mais forte mal poderia
manter o equilíbrio, e em muito pouco tempo teria de se decidir de uma
vez, pois dali a cinco minutos seriam sete e quinze --- quando a campainha
do apartamento tocou. “É alguém da loja”, falou para si e quase congelou,
enquanto suas perninhas em resposta dançavam ainda mais depressa. Por um
momento tudo continuou quieto. “Não vão abrir”, disse, agarrando"-se a
alguma esperança absurda. Mas então, como sempre, naturalmente, a
empregada dirigiu"-se com o passo firme até a porta e a abriu. Gregor só
precisou ouvir a primeira palavra do cumprimento da visita e já soube quem
era --- nada menos que o gerente. Por que será que Gregor estava condenado a
ser empregado numa firma onde a menor falta era vista com a maior das
desconfianças? Todos os funcionários eram, portanto, sem exceção, uns
mandriões, não havia entre eles uma só pessoa dedicada e fiel, que, se por
acaso apenas umas poucas horas da manhã não havia utilizado em prol da
loja, estaria se moendo de remorsos e definitivamente sem condições de
deixar a cama? Já não seria suficiente mandar um aprendiz pedir
informações --- se é que um tal interrogatório fosse mesmo necessário ---,
precisava vir o gerente em pessoa, e precisava com isso ser mostrado a
todos os familiares inocentes que a investigação desse caso suspeito só
podia ser confiada ao juízo do gerente? E mais por causa da irritação a
que fora levado através desses pensamentos do que em decorrência de uma
decisão tomada, lançou"-se com todas as forças para fora da cama. Houve um
barulho alto de pancada, mas de fato um estrondo é que não foi. O tapete
amorteceu um pouco a queda, e também as costas tinham mais elasticidade do
que Gregor supunha, daí o som abafado, que nem chamava a atenção tanto
assim. Só não erguera a cabeça com o cuidado necessário, e por isso a
contundira; girou"-a e, cheio de raiva e dor, esfregou"-a no tapete.

“Alguma coisa caiu lá dentro”, disse o gerente no cômodo da esquerda.
Gregor procurou imaginar se ao menos uma vez não poderia se passar com o
gerente algo semelhante ao que lhe acontecera hoje; a possibilidade não
deveria de modo algum ser descartada. Porém, como se fosse uma resposta
seca a essa hipótese, o gerente deu alguns passos firmes e ouviu"-se o
rangido de suas botas de verniz. No quarto da direita, a irmã sussurrou
para adverti"-lo: “Gregor, o gerente está aí”. “Já sei”, disse Gregor de si
para si; mas erguer a voz a uma altura tal que pudesse ser ouvida pela
irmã, isso ele não arriscou.

“Gregor”, agora falava o pai, do cômodo à esquerda, “o senhor gerente veio
até aqui e quer saber por que você não partiu com o primeiro trem. Nós não
sabemos o que dizer a ele. E ele também quer falar com você em particular.
Então faça o favor de abrir a porta. Ele terá a bondade de desculpar a
desarrumação do quarto.” “Bom dia, senhor Samsa”, intrometeu"-se o gerente,
chamando com voz amigável. “Ele não está bem”, a mãe disse ao gerente,
enquanto o pai ainda discursava para a porta, “ele não está bem, acredite,
senhor gerente. Se não, como Gregor iria perder um trem! O rapaz não tem
mais nada na cabeça a não ser a loja. Eu até fico irritada, porque ele
nunca sai à noite; agora mesmo, ele esteve oito dias seguidos na cidade,
mas ficou em casa todas as noites. Ele senta à mesa com a gente, e fica
quieto lendo o jornal ou estudando o horário dos trens. Já é uma grande
distração quando se ocupa com algum trabalho de marcenaria. Agora mesmo,
por exemplo, em duas ou três noites ele acabou de entalhar uma pequena
moldura; o senhor vai ficar admirado de ver como ela ficou bonita; está
pendurada lá dentro, no quarto; o senhor vai ver, assim que Gregor abrir.
Eu, aliás, fico contente que o senhor esteja aqui, senhor gerente; só a
gente não seria capaz de fazer o Gregor abrir a porta; ele é tão teimoso;
e com certeza não está nada bem, apesar de ter dito o contrário hoje de
manhã.” “Já vai”, disse Gregor, calculadamente lento, e não se mexeu, para
não perder nenhuma palavra da conversa. “De outro modo, prezada senhora,
eu também não saberia explicar”, disse o gerente, “tomara que não seja
nada sério. Embora, por outro lado, eu seja também obrigado a dizer que
nós, homens de negócios --- feliz ou infelizmente, como queira ---, muitas
vezes, em atenção às obrigações comerciais, devemos simplesmente ignorar
qualquer indisposição passageira.” “Então, o senhor gerente já pode
entrar?”, perguntou o pai com impaciência, e voltou a bater na porta.
“Não”, disse Gregor. No cômodo da esquerda sobreveio um silêncio
perturbador, no quarto da direita, a irmã começou a soluçar.

Por que será que a irmã não ia se juntar aos outros? Na certa só agora ela
havia saído da cama e ainda não se vestira. E por que chorava? Por que ele
não se levantava e não deixava o gerente entrar, por que se arriscava a
perder o emprego e por que desse jeito o chefe voltaria a perseguir os pais
com as antigas cobranças? Por enquanto, porém, essas eram preocupações de
todo desnecessárias. Gregor ainda estava presente e não tinha a menor
intenção de abandonar sua família. É certo que no momento ele continuava
lá, estendido no tapete, e ninguém que tivesse conhecimento de sua
situação iria lhe pedir a sério que deixasse o gerente entrar. Contudo,
por causa dessa pequena descortesia, para a qual mais tarde seria fácil
achar uma desculpa aceitável, Gregor não poderia ser mandado embora assim,
sumariamente. E lhe parecia muito mais lógico deixá"-lo em paz
neste momento, em vez de perturbá"-lo com choros e exortações. Mas era sem
dúvida a incerteza que afligia os outros e lhes desculpava o
comportamento.

“Senhor Samsa”, chamou então o gerente, em voz alta, “o que se passa? O
senhor fica entrincheirado aí em seu quarto, responde apenas com
monossílabos, deixa, sem necessidade, seus pais gravemente preocupados e ---
isso seja dito só de passagem --- falta às suas obrigações comerciais de uma
maneira realmente nunca vista. Eu falo aqui em nome de seus pais e do seu
chefe e lhe solicito, com toda a seriedade, o favor de uma explicação
clara e imediata. Estou atônito, estupefato. Eu acreditava conhecê"-lo como
um homem pacato, ajuizado, e agora o senhor de repente dá mostras de
querer começar a exibir tais caprichos. É verdade que o chefe hoje de
manhã me insinuou uma possível explicação para a sua negligência --- dizia
respeito à cobrança recentemente confiada ao senhor ---, mas eu interpus a
bem da verdade quase a minha palavra de honra, dizendo que essa explicação
não tinha cabimento. Agora, contudo, que vejo sua obstinação
incompreensível, perco por completo a vontade de intervir o mínimo que
seja a seu favor. E sua posição não é em absoluto das mais garantidas. Eu
tinha a princípio a intenção de lhe dizer isso tudo em particular, porém,
já que o senhor me faz vir aqui desperdiçar inutilmente o meu tempo, não
sei por que também os seus pais não devam ouvir. Seus resultados nos
últimos tempos, aliás, não foram muito satisfatórios; claro que esta não é
a época do ano em que se fecham grandes negócios, nós reconhecemos; mas
uma época do ano em que não se fecha negócio algum, isso terminantemente
não existe, senhor Samsa, não pode existir.”

“Mas, senhor gerente”, Gregor gritou fora de si, esquecendo tudo o mais no
alvoroço, “eu abro agora mesmo, num instante. Um pequeno mal"-estar, uma
tontura, impediu que eu me levantasse. Ainda estou aqui deitado. Mas já me
sinto mais disposto. Acabo mesmo de me levantar da cama. Só um minutinho
de paciência! Ainda não estou tão bem como pensava. Mas já me sinto
melhor. Como um homem pode ser pego assim de surpresa! Ainda ontem à noite
estava tudo bem comigo, meus pais são testemunha, ou melhor, já ontem à
noite eu tive um leve pressentimento. Deviam ter reparado em mim. Por que
não mandei logo avisar na loja?! Mas a gente sempre pensa que vai vencer a
doença sem precisar ficar em casa. Senhor gerente! Poupe os meus pais!
Todas as acusações que o senhor me faz agora, elas não têm fundamento;
também não me disseram uma palavra a esse respeito. Talvez o senhor não
tenha tomado conhecimento dos últimos pedidos que eu despachei. A
propósito, ainda saio para viajar com o trem das oito, essas poucas horas
de descanso me fortaleceram. Não é preciso se demorar mais, senhor
gerente; agora mesmo eu vou para a loja, e o senhor tenha a bondade de
transmitir esse recado e apresentar os meus cumprimentos ao senhor chefe.”

E enquanto Gregor despejava tudo isso às pressas, mal sabendo o que dizia,
havia se aproximado do guarda"-roupa com facilidade, graças à prática
adquirida antes na cama, e procurava se levantar apoiado nele.
Queria muito abrir a porta, queria de fato mostrar"-se e falar com o
gerente; estava ansioso para saber o que iriam dizer quando o vissem, eles
que agora tanto reclamavam sua presença. Se tomassem um susto, Gregor não
precisava justificar mais nada, e podia ficar descansado. E se aceitassem
tudo com calma, então também não havia motivo para se preocupar, e ele
poderia, se se apressasse, estar de fato às oito horas na estação
ferroviária. No começo ele escorregou algumas vezes no guarda"-roupa liso,
mas ao final deu um último arranco e conseguiu se erguer; já não prestava
atenção à dor na parte de baixo de seu corpo, por mais que ardesse.
Deixou"-se depois cair em direção ao encosto de uma cadeira próxima, a cujas
bordas se agarrou com suas perninhas. Só aí voltou a recuperar o domínio
sobre si mesmo e emudeceu, pois agora precisava ouvir o gerente.

“Os senhores entenderam uma única palavra?”, perguntou o gerente aos pais,
“não estará ele nos pregando uma peça?” “Meu Deus do céu”, exclamou a mãe
já começando a chorar, “ele pode estar muito doente, e nós aqui o
atormentando. Grete! Grete!”, gritou então. “Mamãe?”, respondeu a irmã do
outro lado. Elas se comunicavam através do quarto de Gregor. “Você tem que
chamar o médico agora mesmo. Gregor está doente. Corre até o médico. Você
ouviu como ele falou?” “Era uma voz de animal”, disse o gerente num tom
estranhamente baixo, em contraste com os gritos da mãe. “Anna! Anna!”,
chamou o pai batendo palmas da antessala para a cozinha, “vai já buscar um
chaveiro!” Logo as duas moças atravessavam a antessala correndo num frufru
de saias --- como é que a irmã havia se vestido tão rápido? --- e abriam
precipitadas a porta do apartamento. Não se ouviu a porta bater de volta;
decerto a tinham deixado aberta, como é de costume nas casas onde
aconteceu uma grande desgraça.

Gregor, porém, ficou bem mais tranquilo. É fato que suas palavras já não
eram compreendidas, embora tenham lhe parecido claras, mais claras até do
que antes, talvez porque seu ouvido logo se ajustara a elas. Mas, ainda
assim, agora já sabiam que nem tudo estava em ordem com ele, e se
dispunham a ajudá"-lo. Fizeram"-lhe bem a confiança e a firmeza com que as
primeiras providências foram tomadas. Ele se sentiu de novo integrado ao
círculo humano e esperava de ambos, médico e chaveiro, sem distingui"-los
com muita precisão, realizações grandiosas e surpreendentes. A fim de
participar das discussões decisivas que estavam por vir com a voz o mais
clara possível, tossiu limpando a garganta, todavia se esforçou para
abafar o ruído, que provavelmente também teria um som distinto da tosse
humana, algo que ele mesmo já não tinha competência para discernir. No
cômodo do lado, nesse meio tempo, fez"-se completo silêncio. Talvez os pais
tenham ido se sentar à mesa com o gerente, e cochichavam, talvez
estivessem todos pregados à porta, na escuta.

Gregor moveu"-se até lá empurrando a cadeira devagar, depois soltou"-a,
jogou"-se contra a porta, segurou"-se a ela mantendo"-se na vertical --- as
pontas de suas perninhas tinham uma espécie de grude --- e descansou ali um
instante do esforço realizado. A seguir, contudo, foi tentar girar com a
boca a chave na fechadura. Era de lamentar que não tivesse uns dentinhos
de verdade --- com o que mais iria se agarrar à chave? ---, mas em compensação
as mandíbulas eram muito fortes, com certeza; com a ajuda delas de fato
conseguiu movimentar a chave e nem reparou que assim fatalmente infligia a
si mesmo algum ferimento, dado que um líquido marrom saía"-lhe da boca,
escorria pela chave e pingava no chão. “Prestem atenção”, disse o gerente
no cômodo ao lado, “ele está virando a chave.” Para Gregor, esse foi um
grande incentivo; mas todos deveriam apoiá"-lo, inclusive o pai e a mãe:
“Ânimo, Gregor”, deveriam gritar, “não desiste, dá duro na fechadura!”
Então, imaginando que todos acompanhavam seus esforços com interesse,
aferrou"-se à chave sem pensar em mais nada, reunindo todas as forças que
podia. Bailava em torno da fechadura conforme o avanço da volta da chave;
acabou segurando"-se na vertical apenas com a boca, e de acordo com a
necessidade pendurava"-se na chave ou a forçava mais uma vez para baixo com
todo o peso do seu corpo. O claro estalo da fechadura que enfim destravava
tirou"-o do transe. Tomando fôlego, ele disse: “Pois então, nem precisei do
chaveiro”, e deitou a cabeça na maçaneta, para abrir de par em par as
folhas da porta.

Como só podia abri"-las puxando daquela maneira, uma delas já estava
praticamente toda aberta e ele mesmo ainda não podia ser visto. Devia
primeiro contornar devagar essa folha, e sempre com muita cautela, se não
quisesse fazer o papelão de cair de costas bem na entrada do quarto.
Estava ainda ocupado com essa difícil operação e não tinha tempo para
prestar atenção em outra coisa, quando ouviu o gerente soltar um sonoro
“Oh!” --- soava como o sopro do vento --- e então Gregor o enxergou também,
viu como ele, que era o que estava mais próximo da porta, comprimia com a
mão a boca aberta e retrocedia aos poucos, parecia puxado por uma força de
atração contínua, invisível. A mãe --- que apesar da presença do gerente
apresentava"-se com os cabelos ainda soltos, desgrenhados pela noite de
sono --- juntou as mãos e olhou primeiro para o pai, depois deu dois passos
na direção de Gregor e despencou no meio do círculo formado por suas saias
esparramadas, o rosto encoberto pendendo contra o peito. O pai cerrou o
punho com uma expressão ameaçadora, como se quisesse forçar Gregor a
voltar para dentro do quarto, então olhou a sala em torno de si, indeciso,
cobriu os olhos com as mãos e caiu num choro que chegou a sacudir seu
peito forte.

Gregor contudo não havia nem saído do quarto, ainda se esticava de dentro
na direção da folha que permanecia trancada, de modo que se avistavam
apenas metade de seu corpo e acima, inclinada para o lado, a cabeça com a
qual olhava de soslaio para os outros. Havia clareado bastante nesse meio
tempo; no outro lado da rua projetava"-se nítido um recorte do imenso
edifício da frente, cinza escuro --- era um hospital ---, com suas janelas
rigorosamente simétricas rasgando a fachada; a chuva ainda caía, mas eram
apenas gotas grossas, esparsas, e que assim espaçadas atingiam o solo num
ritmo regular. A louça do café espalhava"-se em grande número sobre a mesa,
pois para o pai o café da manhã era a refeição mais importante do dia, e
ele a prolongava horas a fio com a leitura de diferentes jornais. Na
parede oposta pendia uma fotografia de Gregor da época do exército,
posando como tenente, as mãos na espada, sorrindo despreocupado, invocando
respeito por sua postura e sua farda. A porta da antessala estava aberta
e, como a porta da frente também continuava aberta, era possível enxergar
na parte de fora o corredor e o começo das escadas que conduziam para
baixo.

“Bem”, disse Gregor, com a plena consciência de que era o único que havia
mantido a calma, “agora mesmo vou me vestir, empacotar as amostras e
partir. Vocês vão querer, por favor, me deixar partir? Como vê, senhor
gerente, não estou sendo teimoso e trabalho com vontade; as viagens são
incômodas, mas eu não poderia viver sem viajar. Para onde está indo,
senhor gerente? Para a loja? É? O senhor vai reportar tudo direitinho? Uma
pessoa pode estar num momento incapacitada para o trabalho, mas essa é
exatamente a hora certa de recordar suas realizações passadas e de pensar
que depois, afastado o impedimento, com certeza ela virá a trabalhar até
mesmo com mais aplicação e concentração do que antes. O senhor sabe muito
bem o tanto que eu devo ao chefe. E ainda tenho de cuidar dos meus pais e
da minha irmã. Estou na penúria, mas com o trabalho vou conseguir dar a
volta por cima. Não torne as coisas mais difíceis do que já são para mim.
Tome o meu partido na loja! Eu sei que ninguém gosta do caixeiro"-viajante.
Pensam que ele ganha rios de dinheiro e além disso leva uma vida folgada.
Ninguém nem mesmo toma a iniciativa de discutir mais a fundo esse
preconceito. Mas o senhor, senhor gerente, tem uma visão geral da situação
melhor que a dos outros empregados, até mesmo, seja dito em absoluto
segredo, melhor que a do próprio chefe, que em sua posição de patrão às
vezes se deixa levar por juízos equivocados, prejudicando um funcionário.
O senhor também sabe muito bem que o caixeiro"-viajante, que passa quase o
ano inteiro fora da loja, torna"-se facilmente vítima de intrigas,
maledicências e queixas infundadas, das quais é impossível que se defenda,
porque na maioria das vezes ele nem chega a tomar conhecimento delas e é só
quando volta para casa, esgotado após outra viagem, que vem a receber de
corpo presente suas graves consequências, cujas causas originais não tem
mais como descobrir. Senhor gerente, não vá embora sem me dizer uma
palavra demonstrando que ao menos em parte o senhor me dá alguma razão!”

Mas o gerente, logo às primeiras palavras, já havia lhe dado as costas e,
com um esgar de lábios, só ousava olhar em sua direção por cima dos ombros
trêmulos. E durante o discurso de Gregor não permaneceu parado um segundo
sequer, ao contrário, sem perdê"-lo de vista, retrocedeu em direção à
porta, porém muito devagar, como se houvesse uma lei misteriosa que o
proibisse de deixar a sala. Aos poucos chegou à antessala e, a julgar pelo
movimento repentino com que retirou o pé no último passo para fora da
sala, era possível acreditar que tivesse pisado em brasas. Já na
antessala ele esticava a mão direita cada vez mais na direção da escada,
como se lá o aguardasse uma salvação decididamente extraterrena.

Gregor sabia que em hipótese alguma devia deixar o gerente partir naquele
estado de espírito, se não quisesse que sua posição na loja ficasse
comprometida. Os pais não entendiam direito o que acontecia; ao longo dos
anos, haviam formado a convicção de que Gregor estava garantido na loja
até o fim da vida, e agora, ainda por cima, com a urgência da situação,
tinham tanto a fazer que uma conjetura dessas lhes passava despercebida.
Mas a Gregor não passava. Era preciso reter o gerente, apaziguá"-lo,
persuadi"-lo e por fim convencê"-lo; o futuro de Gregor e de sua família
dependia muito disso! Quem dera a irmã estivesse aqui! Ela era
inteligente; chorou por antecipação quando Gregor ainda estava só deitado
quieto, virado de costas. E na certa o gerente, esse conquistador, iria se
deixar levar por ela; que fecharia a porta do apartamento e ali mesmo na
antessala o acalmaria. Mas a irmã nem ao menos estava lá, Gregor mesmo
devia cuidar do assunto. E sem pensar que nada sabia de sua real
capacidade de se movimentar, sem pensar também que era possível, ou
melhor, muito provável que seu discurso mais uma vez não houvesse sido
compreendido, ele largou a folha da porta e se lançou pela abertura;
queria correr para junto do gerente que, de um modo patético, já se
agarrava com ambas as mãos ao corrimão do corredor; porém, no instante
seguinte, procurando algum apoio, Gregor deu um gritinho e caiu por cima
de suas perninhas. Mal isso aconteceu, ele experimentou, pela primeira vez
naquela manhã, algum conforto físico; as perninhas encontraram chão firme
abaixo de si; e obedeciam de pronto, como ele notou com satisfação; até
faziam força para levá"-lo aonde quisesse; e ele logo acreditou que era
iminente a melhora definitiva de todos aqueles incômodos. Porém, no mesmo
momento em que se viu ali no chão, agitado pelo desejo de se movimentar,
não muito afastado de sua mãe, justamente à sua frente, ela, que parecia
tão recolhida dentro de si mesma, de um pulo se levantou, os braços
esticados, apontava com o dedo, gritando: “Socorro, meu Deus do céu,
socorro!”, mantinha a cabeça inclinada, como se quisesse observar Gregor
com mais atenção, entretanto, num gesto contraditório, recuava sem pensar em
mais nada; esquecera que atrás de si a mesa estava posta; quando encostou
nela, como que distraída, sentou"-se logo sobre o tampo; e pareceu nem
reparar que a seu lado, saindo do grande bule que virara, o café derramava
em grandes golfadas sobre o tapete.

“Minha mãe”, Gregor disse baixinho, e olhou na direção dela. Por um
instante, o gerente foi varrido de seus pensamentos; em contrapartida,
ante a visão do café que escorria, não pôde deixar de estalar várias vezes
as mandíbulas, com cobiça. Diante disso, a mãe soltou um novo grito, saiu
correndo da mesa e caiu nos braços do pai, que veio depressa ao seu
encontro. Gregor, contudo, agora não tinha tempo para os pais; o gerente
já descia a escada; o queixo na altura do corrimão, ainda voltava o olhar
pela última vez. Gregor se preparou para correr, na expectativa de alcançá"-lo;
o gerente deve ter pressentido alguma coisa, pois de um salto desceu
vários degraus e desapareceu; mas ainda soltou um grito, “Ahh!”, que ecoou
em toda a escadaria. Por infelicidade, então, parece que a evasão do
gerente também deixou o pai, que até aqui havia se comportado de modo
relativamente calmo, bastante transtornado, pois, em vez de
correr ele mesmo atrás do homem ou de pelo menos permitir que Gregor saísse em seu
encalço, agarrou com a mão direita a bengala do gerente, por este deixada
na cadeira, junto com o chapéu e o sobretudo, apanhou com a mão esquerda
um grosso jornal de cima da mesa e, com passadas pesadas, sacudindo a
bengala e o jornal, pôs"-se a tocar Gregor de volta para o quarto. Nenhuma
súplica lhe foi de valia, nenhuma súplica sequer fora entendida, Gregor
quis baixar a cabeça de modo ainda mais humilde, e o pai só fez bater os
pés com mais força ainda. Do lado oposto, a mãe, apesar do tempo frio,
havia aberto uma janela e, debruçada o mais para fora possível, apertava o
rosto contra as mãos. Da viela, pela escadaria, veio uma forte corrente de
ar, as cortinas esvoaçaram, os jornais farfalharam sobre a mesa, algumas
folhas voaram e foram parar no chão. Implacável, o pai insistia e passou a
silvar como um selvagem. Mas Gregor ainda não tinha a menor prática em
andar para trás, e ia realmente muito devagar. Se ao menos pudesse dar
meia volta, na mesma hora estaria dentro do quarto, mas ele temia
impacientar o pai com uma manobra muito demorada, e a todo instante via"-se
ameaçado pelo golpe fatal da bengala, desferido em suas costas ou na
cabeça. Porém, no fim não restou nenhuma outra alternativa, pois ele notou
assustado que, ao andar para trás, nem uma vez sequer conseguira manter a
direção; e assim, entre olhadelas furtivas, medrosas e incessantes na
direção do pai, começou a se virar, o mais rápido que podia, na
verdade, contudo, ainda muitíssimo lento. Talvez o pai tenha notado a sua boa
vontade, porque não o atrapalhou nessa hora, pelo contrário, até
direcionou a volta aqui e ali, de longe, com a ponta da bengala. Se não
fossem aqueles silvos insuportáveis! Por causa deles, Gregor perdia a
cabeça. Já havia dado quase toda a volta quando, atordoado por aqueles
ruídos incessantes, chegou a se enganar e recuou um bom pedaço no sentido
contrário. Mas quando afinal, contente, viu"-se diante da abertura da
porta, percebeu que seu corpo era muito largo para passar por ela sem
dificuldades. Ao pai, é lógico, naquele estado de ânimo, não ocorria nem
de longe abrir um pouco a outra folha da porta, de modo a deixar espaço
suficiente para a passagem de Gregor. Sua ideia fixa resumia"-se a fazê"-lo
entrar no quarto o mais rápido possível. Jamais toleraria também os
preparativos minuciosos de que precisava para se erguer e desse modo
tentar passar pela porta. Em vez de ajudar, agora fazendo um barulho infernal,
forçava o avanço de Gregor como se não houvesse nenhum obstáculo; soava
como se já não fosse mais a voz de um único pai apenas; com efeito, a
coisa deixou de ser brincadeira, e Gregor --- seja lá o que acontecesse ---
jogou"-se contra a porta. Um dos lados de seu corpo subiu, ele ficou
atravessado na abertura, uma parte de suas costas foi toda esfolada, na
tinta branca da porta restaram manchas repulsivas, logo se viu prensado e
sozinho não teria podido mais se mexer, as perninhas do lado de cima
pendiam vibrando no ar, as do outro lado eram dolorosamente pressionadas
contra o chão --- foi quando o pai, de trás, deu"-lhe um empurrão forte,
desta vez deveras libertador, e ele, sangrando a valer, entrou voando para
dentro do quarto. A porta ainda foi fechada com a bengala e então enfim
tudo ficou quieto. 

\pagebreak
\sectionitem

Só ao crepúsculo Gregor acordou de seu sono pesado, que mais parecia um
desmaio. Com certeza, se não fosse perturbado, também não teria acordado
muito mais tarde, pois sentia que já dormira e descansara o suficiente,
embora tivesse a impressão de haver sido despertado por alguns passos
furtivos e pelo ruído da porta de acesso à antessala, que fora trancada
por precaução. A fraca luz das lâmpadas elétricas da rua iluminava
palidamente alguns pedaços do teto do quarto e a parte de cima dos móveis,
mas Gregor embaixo estava às escuras. Aos poucos, tateando ainda
desajeitado com suas antenas, às quais só agora dava o devido valor,
deslocou"-se até a porta, para ver o que havia ocorrido. Seu lado esquerdo
estava que era uma única cicatriz, comprida, esticada, incômoda, e por
isso ele tinha de andar mancando mesmo com suas duas fileiras de pernas.
Uma perninha, aliás, saíra seriamente machucada dos incidentes da manhã ---
era um milagre que apenas uma tivesse se machucado --- e pendia inerte,
arrastada pelas outras.

Só ao se aproximar da porta é que foi perceber o que o atraíra na verdade;
era o cheiro de alguma coisa comestível. Com efeito, lá estava uma tigela
cheia de leite fresco, no qual algumas migalhas de pão boiavam. Por pouco
não riu de tanta alegria, pois continuava com uma fome ainda maior do que
estava pela manhã, e na mesma hora mergulhou a cabeça no leite, quase até
os olhos. No instante seguinte porém a retirou, desapontado; não apenas
comer lhe era dificultoso, por causa do lado esquerdo avariado --- e ele só
podia comer com a colaboração ofegante de todo o corpo ---, mas também
acima de tudo não lhe apeteceu em absoluto o leite, que antes era sua
bebida favorita, e na certa por isso a irmã lho trouxera, de modo que ele,
meio enojado, deixou de lado a tigela, voltando a se arrastar para o
centro do quarto.

Na sala, como Gregor via pelas frinchas da porta, o gás fora aceso, porém,
enquanto antigamente a essa hora o pai se dedicava à leitura em voz alta
do jornal vespertino, para a mãe e à vezes também para a irmã, agora não
se ouvia ruído algum. Pode ser que essa leitura, que a irmã sempre lhe
descrevia e comentava nas cartas, tivesse saído da rotina nos últimos
tempos. Em todo caso, o silêncio prevalecia, embora com toda a certeza o
apartamento não estivesse vazio. “Mas que vida mais tranquila a família
leva”, disse Gregor com seus botões e, enquanto olhava fixo a escuridão à
sua frente, sentiu um grande orgulho de que pudesse proporcionar a seus
pais e a sua irmã uma vida dessas em um apartamento tão bom. Mas e se
agora todo o sossego, todo o bem"-estar, toda a paz tivessem de chegar a um
terrível fim? Para não se perder em tais pensamentos, Gregor preferiu se
movimentar, e ficou se arrastando no quarto, de um lado para o outro.

Durante a longa noite, as duas portas laterais, primeiro uma, depois a
outra, foram abertas uma frestinha apenas, e fechadas rapidamente em
seguida; alguém sem dúvida sentia necessidade de entrar, mas resolvera
pensar duas vezes. Gregor se deteve então bem na frente da porta da sala,
determinado a trazer a indecisa visita para dentro de alguma maneira, ou
pelo menos disposto a descobrir quem era; a porta porém não voltou a ser
aberta e ele esperou em vão. De manhã, quando as portas estavam trancadas,
todos queriam entrar, agora, depois que ele abrira sozinho uma delas, e as
outras ao que tudo indicava teriam sido abertas no decorrer do dia,
ninguém mais vinha, mesmo com as chaves do lado de fora.

Só bem tarde da noite é que desligaram a luz da sala, e nesse momento foi
fácil comprovar que os pais e a irmã haviam ficado acordados até aquela
hora, pois dava para ouvir direitinho como todos os três se afastavam na
ponta dos pés. Na certa até amanhã ninguém mais viria à procura de Gregor;
ele tinha assim bastante tempo para refletir, sem ser incomodado, sobre o
modo como devia reorganizar sua vida a partir de agora. Contudo, o quarto
alto e despojado, no qual era forçado a permanecer deitado contra o rés do
chão, angustiava"-o, e ele não conseguia descobrir por que, uma vez que era
o mesmo quarto que habitava havia já cinco anos --- então, com uma meia
volta quase involuntária, e não sem certa vergonha, correu para debaixo do
canapé, onde se sentiu muito bem acomodado, apesar de suas costas meio
espremidas e apesar de não poder mais erguer a cabeça, lamentando apenas
que seu corpo fosse tão largo que não coubesse inteiro embaixo do móvel.

Ali ele passou toda a noite, uma parte em cochilos dos quais era seguidas
vezes despertado de súbito pela fome, uma parte tomado por preocupações
e incertas esperanças de que tudo afinal encontraria uma solução, de que o
certo seria agir com calma nesse meio tempo e, com paciência e o máximo de
respeito aos familiares, tentar tornar suportável o desgosto que ele, em
suas atuais circunstâncias, excepcionalmente era obrigado a lhes causar.

Já de manhã bem cedo, madrugadinha ainda, Gregor teve oportunidade de pôr
à prova suas ponderadas decisões, pois vindo da antessala a irmã, vestida
quase dos pés à cabeça, abriu a porta e examinou o interior do quarto com
apreensão. Ela não o descobriu na hora, mas quando o divisou ali embaixo
do canapé --- Deus do céu, ele tinha de estar em algum lugar, não podia sair
voando por aí --- levou um susto tão grande que, sem que pudesse se
controlar, voltou a fechar a porta no mesmo instante. Porém, parecendo
arrependida de sua atitude, tornou a abri"-la em seguida e entrou na ponta
dos pés, como se ali estivesse alguém muito doente ou fosse um estranho.
Gregor estendeu um tantinho a cabeça até a borda do canapé e observou a
irmã. Iria ela notar que ele tinha deixado o leite intacto, lógico que de
modo algum por falta de apetite, e será que traria uma outra comida que
lhe fosse mais adequada? Se ela não tomasse a iniciativa, ele preferia
morrer de fome a ter de chamar sua atenção para isso, embora no fundo
sentisse uma vontade urgente de deixar o canapé, se atirar aos pés da irmã
e pedir a ela algo de bom para comer. Mas a irmã, surpresa, logo notou a
tigela ainda cheia, ao redor da qual havia apenas um pouco de leite
derramado, recolheu"-a no mesmo instante, claro que não com as mãos nuas, e
sim com um pedaço de pano, e a levou para fora. Gregor ficou bastante
curioso para saber o que ela traria em troca, e teceu as mais diversas
suposições a respeito. Nunca, porém, teria adivinhado o que a irmã, em sua
grande bondade, realmente fez. A fim de testar o seu paladar, ela lhe
trouxe várias coisas sortidas, que dispôs em uma folha de jornal velho.
Havia ali legumes passados já meio apodrecidos; ossos da última ceia
recobertos de molho branco ressecado; uma porção de passas e amêndoas; um
queijo que há dois dias Gregor julgara intragável; um pedaço de pão duro,
uma fatia de pão com manteiga e outra fatia com manteiga e sal. Além
disso, à frente de tudo voltou a colocar a tigela, pelo visto destinada de
uma vez por todas a ele, agora cheia de água. E por delicadeza, sabendo
que Gregor não iria comer diante dela, afastou"-se às pressas e chegou a
dar a volta na chave, a fim de que ele notasse que poderia se pôr tão à
vontade quanto quisesse. As perninhas de Gregor zuniram quando ele partiu
na direção da comida. Suas feridas, aliás, já deviam ter sarado de todo,
ele não sentia mais nenhum desconforto, o que o deixou pasmo pois lembrou
como, mais de um mês atrás, havia feito com a faca um cortinho de nada no
dedo, e esse machucado ainda antes de ontem doía um bocado. “Será que
minha sensibilidade diminuiu?”, pensou, e sorveu ávido o queijo, que acima
de todas as outras comidas o atraíra imediata e energicamente. Com
rapidez, um após o outro, vertendo lágrimas de contentamento, ele devorou
o queijo, os legumes e o molho; os alimentos frescos, ao contrário, não
agradavam o seu paladar, ele mal podia suportar"-lhes o cheiro, e teve até
de afastar para o lado as coisas que queria comer. Já dera cabo de tudo há
algum tempo, e continuava largado no mesmo lugar, apenas descansando,
quando a irmã, para sinalizar que ele devia retroceder, deu uma volta na
chave. Isso o despertou de pronto, quando estava a ponto de cochilar, e
ele voltou correndo para debaixo do canapé. Mas precisou de muito
autocontrole para permanecer ali embaixo durante o curto espaço de tempo
em que a irmã esteve no quarto, porque, após a lauta refeição, seu corpo
ficou mais redondo e ele mal conseguia respirar naquele aperto. Entre
breves crises de asfixia, viu com os olhos um pouco saltados a irmã, sem
desconfiar de nada, juntar com uma vassoura não apenas as sobras, mas
inclusive os alimentos em que ele nem sequer chegara a tocar, como se
esses também não pudessem mais ser aproveitados, e a viu ainda despejar
tudo depressa em um balde que cobriu com uma tampa de madeira e levou para
fora do quarto. Mal ela lhe deu as costas, Gregor avançou logo para fora
do canapé e relaxou, voltando a estufar"-se.

Dessa maneira recebia doravante Gregor diariamente sua refeição, a
primeira vez de manhã, quando os pais e a empregada ainda dormiam, a
segunda vez depois que todos almoçavam, pois era o momento em que os dois
dormiam de novo ainda um bocadinho, e a empregada saía, encarregada pela
irmã de uma incumbência qualquer. Com certeza os pais não queriam que
Gregor morresse de fome, porém talvez só suportassem, da experiência de
suas refeições, no máximo ouvir dizer que aconteciam, ou quem sabe a irmã
quisesse lhes poupar mais uma aflição, ainda que pequena, pois era visto
que já sofriam o bastante.

Gregor não chegou a saber com quais desculpas o médico e o chaveiro foram
dispensados naquela manhã inicial, porque, como não o compreendiam,
ninguém presumia, nem mesmo a irmã, que ele pudesse compreender os outros,
e por isso, quando ela estava no quarto, ele era obrigado a se contentar
apenas com os suspiros e os lamentos que de vez em quando escutava. Só
depois que ela se acostumou um pouco com tudo --- uma aceitação completa,
claro, não poderia nunca ser cogitada ---, é que Gregor chegou a flagrar
algumas vezes uma observação mais simpática ou que assim poderia ser
entendida. “Acho que hoje estava bom”, ela falava, quando Gregor havia
devorado com gosto a refeição, ou, caso contrário, o que gradativamente
foi se repetindo com maior frequência, costumava dizer, meio entristecida:
“De novo não tocou em nada”.

Mas, conquanto não conseguisse saber de nenhuma novidade diretamente,
Gregor captava quase tudo o que vinha dos cômodos adjacentes, e assim que
escutava alguma voz corria no ato até a porta respectiva e colava"-se a ela
com o corpo todo. Sobretudo nos primeiros tempos, não havia uma conversa
que, de algum modo, ainda que só às escondidas, não tratasse dele. Dois
dias seguidos, durante todas as refeições, só se ouviram deliberações
sobre como deviam se comportar daí em diante; mas também entre as
refeições falava"-se do mesmo assunto, pois havia sempre no mínimo dois
membros da família em casa, dado que ninguém queria ficar sozinho ali e
não tinham a menor condição de abandonar o apartamento logo de uma vez.
Ainda nos primeiros dias --- não estava muito claro nem o que nem o quanto
ela sabia do incidente --- a empregada pediu de joelhos à mãe que a mandasse
embora de imediato, e quando, quinze minutos depois, deixava o emprego,
agradeceu com lágrimas nos olhos tanto a demissão quanto o imenso favor
que nessa hora lhe prestavam, e jurou de pé junto, sem que isso lhe fosse
solicitado, não revelar o menor detalhe a ninguém.

Então a irmã se viu também obrigada a ajudar a mãe na cozinha; em todo
caso, não era muito o esforço exigido, pois não se comia quase nada.
Gregor ouvia vezes seguidas como um deles incentivava em vão o outro a
comer, sempre obtendo como única resposta: “Obrigado, estou satisfeito”,
ou algo semelhante. Tampouco deviam beber. Em várias ocasiões a irmã
perguntava ao pai se ele não queria cerveja, e se punha com alegria à
disposição para ela mesma ir buscar ou então, ante o silêncio do pai e
para eliminar qualquer suspeita de inconveniência, dizia que podia mandar
a zeladora ir no lugar dela, mas daí o pai respondia com um poderoso “Não”
final, e não se falava mais no assunto.

Já no decorrer desses primeiros dias o pai expôs a ambas, mãe e irmã, suas
reais perspectivas e condições financeiras. Aqui e ali ele deixava a mesa
para ir buscar algum comprovante ou alguma caderneta no pequeno e valioso
cofre que conseguira salvar da bancarrota de seu negócio, ocorrida há
cinco anos. Ouvia"-se ele destravar a complicada fechadura e, depois de
retirar o que buscava, voltar a trancá"-la. Essas explicações do pai, por
um lado, foram as primeiras boas notícias que Gregor chegou a ouvir desde
o seu confinamento. Sempre achara que nada havia sobrado para o pai do
antigo negócio, pelo menos o pai nunca lhe dissera o contrário, e de
qualquer modo Gregor também nunca lhe perguntara nada a respeito. Sua
única preocupação na época havia sido fazer de tudo para que a família
superasse o mais rápido possível o contratempo comercial, que deixara
todos no mais completo desalento. E foi assim que ele começou a trabalhar
com uma disposição fora do comum, e passou da noite para o dia de simples
balconista a caixeiro"-viajante, função que evidentemente lhe abria muitas
outras oportunidades de ganho, e cujos ótimos resultados, sob a forma de
comissões, rápido se transformaram em dinheiro vivo que podia ser
espalhado sobre a mesa em casa, diante da família assombrada e satisfeita.
Foram bons tempos aqueles, que depois nunca mais se repetiram, pelo menos
não com igual esplendor, apesar de Gregor ter recebido mais tarde ainda
muito dinheiro, o que o capacitava a assumir, como assumiu, as despesas de
toda a família. Tanto a família quanto Gregor se acostumaram logo a essa
situação, eles aceitavam o dinheiro agradecidos, ele o entregava de bom
grado, mas não houve mais nenhuma manifestação efusiva. Só a irmã
permaneceu próxima a Gregor, ela que diferente dele amava a música e
aprendera a tocar violino de um modo enternecedor, e ele planejava em
segredo matriculá"-la no conservatório no próximo ano, sem ligar para os
altos custos que isso devia acarretar, e que seriam compensados de uma ou
outra maneira. Várias vezes, nas conversas com a irmã, durante as curtas
estadias de Gregor na cidade, o conservatório era mencionado, porém sempre
como se fosse apenas um belo sonho, cuja realização era impensável, e os
pais não ouviam essas fantasias inocentes com muita satisfação; mas Gregor
havia refletido a sério sobre o assunto e pretendia anunciar seus planos
oficialmente na noite de Natal.

Esses pensamentos, inúteis na sua situação atual, passavam"-lhe pela cabeça
enquanto permanecia lá erguido e colado à porta, escutando. Às vezes,
devido ao cansaço de todo o corpo, não conseguia mais prestar atenção e
deixava sem querer a cabeça bater na porta, mas a endireitava no mesmo
instante, pois até o mínimo ruído que dessa forma provocava era ouvido do
lado de fora e fazia todos se calarem. “Que será que tanto se mexe”, dizia
o pai momentos depois, a voz alta dirigida para a porta, e só aos poucos a
conversa interrompida era retomada.

Gregor ficou então cansado de saber --- pois o pai fazia questão de repetir
sucessivas vezes suas explicações, em parte porque há muito que ele
próprio já não se ocupava dessas coisas, em parte também porque a mãe não
compreendia tudo logo na primeira vez --- que apesar de toda a desgraça
ainda tinham disponível uma sobra orçamentária dos velhos tempos, em todo
caso bastante pequena, mas que nesse ínterim os juros acumulados haviam
feito crescer um pouquinho. Além disso também havia o dinheiro que Gregor
todo mês deixava em casa --- retirava para si apenas algumas notas ---, que
não era integralmente gasto e acumulara"-se, originando um pequeno capital.
Atrás da porta, Gregor balançou a cabeça em sinal de aprovação, contente
com essa inesperada prudência e economia. Era bem verdade que, com esse
excedente de dinheiro, já poderia ter quitado outras parcelas da dívida
que o pai tinha com o chefe, e aquele dia em que poderia enfim se libertar
do emprego já estaria mais próximo, porém agora sem dúvida as coisas
estavam melhor assim, do jeito que o pai havia organizado.

Contudo, a quantia não era em absoluto suficiente para que se pudesse
viver de seus rendimentos; talvez bastasse para manter a família durante
um, no máximo dois anos, para mais não dava. Eram portanto apenas
economias de que não podiam dispor, e deviam conservar intactas para um
caso de maior necessidade; o dinheiro do dia"-a"-dia teria de ser ganho.
Acontece, no entanto, que o pai, embora ainda com saúde, era um homem
idoso, que estava há cinco anos totalmente afastado do trabalho e além do
mais já não podia fazer muito esforço; nesses cinco anos, que foram as
primeiras férias de sua vida diligente porém malsucedida, ele engordara
muito e por causa disso havia ficado mole e pesadão. E devia a velha mãe,
porventura, ir atrás do dinheiro, ela que sofria de asma e ficava cansada
só de andar pelo apartamento e que, dia sim dia não, passava deitada no
sofá, de janelas abertas, com crises respiratórias? Ou iria conseguir
dinheiro a irmã, ainda uma criança com seus dezessete anos e que até o
momento tivera uma vida invejável, nada mais que cuidar de se vestir,
dormir até tarde, ajudar um pouco na casa, uma ou outra atividade de lazer
modesta e tocar violino acima de tudo? Sempre que a conversa chegava na
necessidade de ganhar dinheiro, Gregor primeiro se soltava e em seguida se
jogava sobre o frio sofá de couro encostado ao lado da porta, pois ficava
inflamado diante de tanta humilhação e infortúnio.

Era frequente ele passar a noite inteira ali em cima, sem dormir um só
minuto, horas seguidas apenas arranhando o couro. Salvo quando se dispunha
a encarar o grande esforço de empurrar uma cadeira até a janela, depois
subir rastejando até alcançar o parapeito e, apoiado na cadeira,
inclinar"-se para frente, óbvio que apenas como uma espécie de recordação
da sensação de liberdade que antes tinha ao olhar pela janela. Pois era
patente que dia após dia via cada vez com menos nitidez as coisas que
estavam afastadas de si; não avistava mais o hospital do outro lado, cuja
visão cotidiana obrigatória ele antes amaldiçoava, e, se não tivesse
certeza absoluta de que morava na travessa Charlotte, uma viela tranquila,
não obstante sua total urbanidade, seria capaz de acreditar que sua janela
dava para um deserto no qual o cinza do céu e o cinza da terra se juntavam
e eram indistinguíveis. Bastou à atenciosa irmã ver em duas ocasiões a
cadeira encostada na janela para que todas as vezes, depois de arrumar o
quarto, voltasse a colocá"-la no mesmo lugar, e a partir de então até a
folha interna da janela era deixada aberta.

Se Gregor pudesse falar com a irmã, e agradecê"-la por tudo que estava
obrigada a fazer por ele, aceitaria os favores com mais facilidade; assim,
porém, ele sofria. A irmã decerto procurava disfarçar ao máximo o que
havia de penoso em tudo aquilo e, é claro, quanto mais o tempo passava,
tanto melhor ela o conseguia, entretanto, com o passar do tempo também
Gregor veio a perceber as coisas com maior exatidão. A entrada dela já lhe
parecia horrível. Mal havia entrado, sem sequer parar para fechar a porta,
ela que antes tanto se esforçava para esconder de todos a visão de Gregor,
disparava até a janela e a escancarava com as mãos apressadas, como se
estivesse sufocando, e ficava ali parada um instante, retomando o fôlego,
mesmo quando o frio era intenso. A correria e o barulho assustavam Gregor
duas vezes por dia; todos esses momentos passava tremendo embaixo do
canapé e todavia sabia muito bem que ela com certeza o dispensaria de tal,
se ao menos lhe fosse possível permanecer com as janelas fechadas num
quarto por ele ocupado.

Uma ocasião, passado já todo um mês desde a transformação, não havendo
mais nenhuma razão especial para a irmã se admirar com a aparência de
Gregor, ela veio um pouco antes do que costumava e o flagrou quando ele
ainda olhava pela janela, imóvel e dessa forma numa situação propensa a
provocar um susto. Se ela não entrasse, não teria sido uma atitude
inesperada para Gregor, uma vez que em sua posição ele a impedia de abrir
de imediato a janela, mas ela não apenas não entrou, como chegou a
retroceder e a trancar a porta; uma pessoa estranha teria o direito de
pensar que ele estava lá à espreita com a intenção de mordê"-la. Gregor,
logicamente, foi logo se esconder embaixo do canapé, porém teve de esperar
até o meio"-dia para que a irmã voltasse, e ela pareceu muito mais
desconfiada do que antes. Ele percebeu então que sua visão ainda era
insuportável para ela, e seguiria sendo insuportável no futuro, e que ela
devia se superar para não sair correndo ao avistar ainda que apenas uma ínfima parte
de seu corpo de sob o canapé. A fim de preservá"-la inclusive
dessa mínima visão, ele um dia carregou o lençol nas costas até o canapé ---
precisou de quatro horas para a operação --- e o arranjou de um modo tal que
todo o seu corpo ficaria encoberto, e a irmã, ainda que se abaixasse, não
conseguiria vê"-lo. Se em sua opinião o lençol não fosse necessário, então
ela mesma poderia retirá"-lo, pois estava claro o bastante que não era nada
do agrado dele isolar"-se assim tão completamente, porém ela deixou o
lençol do jeito que estava, e Gregor chegou a crer que divisara um olhar
de agradecimento quando uma vez cauteloso ergueu um pouquinho o pano com a
cabeça, para verificar como a irmã havia recebido o novo arranjo.

Nos primeiros quinze dias os pais não conseguiram superar o receio de
entrar no quarto dele, e Gregor muitas vezes ouviu como aprovavam sem
reservas o trabalho realizado pela irmã, apesar de até então terem vivido
meio aborrecidos com ela, que lhes parecia uma menina um tanto quanto
inútil. Mas agora os dois, pai e mãe, sempre esperavam diante da porta,
enquanto a irmã cuidava da arrumação, e mal ela saía tinha de descrever em
detalhes quais eram as condições do quarto, o que Gregor havia comido,
como agira dessa vez, e se dava para notar algum sinal de melhora. A mãe,
aliás, logo no começo teve vontade de ir vê"-lo, mas o pai e a irmã a
detiveram, a princípio com argumentos sensatos, aos quais Gregor prestou
muita atenção e com os quais estava inteiramente de acordo. Depois, porém,
precisaram retê"-la à força, e quando enfim ela gritou: “Me deixem entrar,
é o coitado do meu filho! Vocês não entendem que eu tenho que ver o meu
filho?”, Gregor então pensou que talvez fosse bom se a mãe viesse vê"-lo,
não todo dia, é lógico, mas podia ser uma vez por semana; ela com certeza
compreendia tudo muito melhor do que a irmã, que apesar de todo o seu
empenho era ainda só uma criança e em última hipótese devia ter assumido
um encargo tão pesado apenas por causa de sua insensatez juvenil.

Não demorou para Gregor satisfazer o seu desejo de ver a mãe. De dia, em
atenção aos pais, ele já não queria aparecer à janela, mas também não
tinha como rastejar muito nos parcos metros quadrados do chão, era difícil
ficar parado durante a noite, a comida logo deixou de lhe proporcionar o
menor prazer, e assim, como distração, ele adquiriu o hábito de rastejar
de um lado para o outro pelas paredes e pelo forro. Gostava em especial de
se pendurar lá em cima, no teto; era bem diferente de ficar deitado no
chão; respirava"-se com maior liberdade; uma vibração leve corria pelo
corpo; e no abandono quase feliz em que se encontrava lá no alto, podia
acontecer de se soltar, surpreendendo a si próprio, e se espatifar no
chão. Agora, porém, naturalmente, ele tinha sobre seu corpo um domínio
muito maior do que antes, e não se machucava, mesmo caindo de uma altura
dessas. A irmã percebeu de imediato o novo divertimento que Gregor havia
descoberto --- ele deixava um rastro de grude aqui e ali ---, e aí lhe veio à
cabeça dar a ele o máximo de espaço para rastejar, removendo os móveis que
o atrapalhavam, ou seja, principalmente o guarda"-roupa e a escrivaninha.
Ocorre que não seria capaz de fazer tudo isso sozinha; não ousava pedir a
colaboração do pai; a empregada com toda a certeza não a ajudaria pois,
para uma mocinha de mais ou menos dezesseis anos, esta até que resistia
com bravura desde a dispensa da antiga cozinheira, contudo havia
solicitado permissão para manter a cozinha o tempo todo trancada, e só ser
obrigada a abrir em caso de extrema necessidade; assim só restou à irmã
como opção ir atrás da mãe, em uma das ausências do pai. A mãe a seguiu
com arroubos de uma alegria incontida, mas ficou muda diante da porta.
Lógico que a irmã conferiu primeiro, para ver se estava tudo em ordem no
quarto; só depois é que a mãe pôde entrar. Gregor na pressa havia puxado o
lençol mais para baixo, deixando"-o mais amarrotado, o conjunto dava mesmo
a impressão de um lençol atirado ao acaso sobre o canapé. Gregor também,
nessa ocasião, absteve"-se de erguer o pano para espiar; renunciava a ver a
mãe logo dessa vez, e contentava"-se só por ela estar ali de verdade. “Pode
vir, ele não está à vista”, disse a irmã, ao que tudo indica puxando a mãe
pela mão. Gregor ouviu então como as duas mulheres magras empurraram o
guarda"-roupa, todavia pesado, de seu lugar, e como a irmã insistiu em
tomar para si a maior parte da tarefa, sem dar ouvidos às advertências da
mãe, receosa de que a filha se cansasse demais. Demorava muito. Após uns
bons quinze minutos, a mãe disse que o melhor seria deixar o móvel ali
mesmo, em primeiro lugar porque ele era muito pesado, não conseguiriam
terminar antes da chegada do pai e com o guarda"-roupa no meio do quarto
iriam bloquear toda a passagem a Gregor, em segundo lugar, porém, porque
não havia certeza de que a retirada dos móveis fosse do agrado dele. A ela
lhe parecia o inverso; a visão das paredes vazias era de oprimir o
coração; e por que Gregor não teria a mesma sensação, ele que já estava há
tanto tempo acostumado com os móveis e por isso se sentiria abandonado no
quarto vazio? “E não será o caso”, concluiu a mãe bem baixinho, já quase
num sussurro, como se quisesse impedir que Gregor, cujo esconderijo exato
ela desconhecia, ouvisse sequer o rumor de sua voz, pois que ele não
compreendia as palavras, disso estava convencida, “e não será o caso, ao
retirar os móveis, de demonstrar com isso que renunciamos a qualquer
esperança de recuperação e o deixamos, sem a menor consideração, entregue
à própria sorte? Eu penso que seria melhor tentar manter o quarto do
jeitinho que estava antes, para que Gregor, quando voltar de novo para a
gente, encontre tudo inalterado e possa ainda mais rápido esquecer esse
intervalo de tempo.”

Ao escutar essas palavras da mãe, Gregor admitiu que a privação total,
nesses últimos dois meses, de qualquer conversação humana direta,
associada ao convívio regular no seio da família, devia ter perturbado seu
juízo, pois de outro modo não saberia explicar como pôde sinceramente
desejar que seu quarto fosse esvaziado. Tinha de fato vontade de que esse
quarto aconchegante, com seus móveis antigos dispostos de forma tão
cômoda, fosse transformado numa toca onde ele poderia rastejar livre e
despreocupado em todas as direções, todavia sob o risco do esquecimento
paralelo, rápido e rasteiro, de seu passado humano? Já estava aliás bem
próximo desse esquecimento, e só mesmo a voz da mãe, que há muito não
ouvia, para alertá"-lo. Nada tinha de ser removido; tudo deveria ficar como
estava; ele não podia dispensar as influências benéficas dos móveis sobre
sua condição; e se estes o impediam de praticar o seu rastejar absurdo,
isso não era uma perda, e sim uma grande vantagem.

Infelizmente, porém, a irmã era de outra opinião; ela se acostumara, a
propósito não sem um certo direito, a se contrapor aos pais nas discussões
que tinham Gregor como tema, comportando"-se como uma especialista no
assunto, e assim naquela hora o conselho da mãe foi motivo suficiente para
que insistisse, não só na remoção do guarda"-roupa e da escrivaninha, como
a princípio pensara, mas também na retirada de todo o mobiliário, com
exceção do canapé, indispensável. O que a aferrava a essa exigência era,
claro, não apenas uma teimosia infantil e a autoconfiança tão inesperada,
adquirida a duras penas nos últimos tempos; ela com efeito também
observara que Gregor precisava de bastante espaço para rastejar, ao passo
que dos móveis, pelo contrário, tanto quanto se podia ver, não aproveitava
quase nada. Contudo, talvez colaborasse para sua atitude o espírito
entusiástico das jovens de sua idade, que não perde ocasião de se
manifestar e com o qual Grete era no momento instigada a querer tornar
ainda mais assustadora a situação de Gregor, para que então pudesse
realizar por ele mais do que fizera até agora. Pois em um local no qual
apenas Gregor imperasse entre as paredes vazias, é seguro que ninguém, a
não ser Grete, jamais se atreveria a entrar.

E assim ela não se deixou dissuadir de sua decisão pela mãe que, presa de
pura inquietação, se sentindo muito insegura dentro do quarto, logo se
calou e ajudou a irmã, na medida de suas forças, a levar o guarda"-roupa
para fora. Ora, em último caso Gregor ainda podia dispensar este móvel,
mas a escrivaninha, esta tinha que ficar. E as mulheres mal haviam deixado
o quarto com o guarda"-roupa, contra o qual se espremiam, gemendo, quando
Gregor projetou a cabeça por baixo do canapé, para ver como daria para
interferir com prudência e o máximo de delicadeza possível. Mas, por
infelicidade, foi bem a mãe quem retornou primeiro, enquanto Grete no
cômodo ao lado se atracava ao guarda"-roupa e tentava balançá"-lo de um lado
para o outro, é lógico que sem conseguir tirá"-lo do lugar. A mãe, porém,
não estava habituada à aparência de Gregor, poderia ter um ataque, e por
isso espavorido ele retrocedeu depressa até o fundo do canapé, entretanto
não pôde mais evitar que o lençol se movimentasse um pouco na parte da
frente. Foi o suficiente para chamar a atenção da mãe. Ela estacou, ficou
um instante imobilizada e então voltou para junto de Grete.

Apesar de Gregor repetir consigo mesmo, sucessivamente, que nada de
excepcional acontecia, era tão"-só a mudança de uns poucos móveis, esse vai
e vem das mulheres, as breves exortações entre elas, os riscos dos móveis
no assoalho, isso tudo o atingia, ele logo teve de reconhecer, como se
fosse um grande tumulto, com ruídos vindo de todos os lados, e mesmo
mantendo cabeça e pernas encolhidas e o corpo colado ao chão, ele se via
forçado a admitir que logo não suportaria mais aquilo. Estavam esvaziando
seu quarto; retiravam tudo o que tinha de mais caro; o guarda"-roupa, onde
guardava o arco de serra e as outras ferramentas, elas já haviam levado;
agora soltavam a escrivaninha que fora bem fixada no chão, e na qual ele
escrevera suas lições no tempo da escola de comércio, do colégio e até
mesmo do primário --- de modo que não sobrava muito tempo para que avaliasse
quão boas eram as intenções das duas mulheres, de cuja real existência
aliás ele já havia quase se esquecido, pois, de exaustas, elas agora
trabalhavam mudas, e ouviam"-se apenas as passadas pesadas de seus pés.

E assim, pois --- na hora em que as mulheres, já no cômodo ao lado,
apoiavam"-se à escrivaninha, procurando retomar o fôlego ---, ele avançou
para fora, mudou de posição quatro vezes, calculando a direção da corrida,
sem saber ao certo o que salvar primeiro, quando notou, em destaque na
parede já de todo nua, o quadro pendurado da dona vestida puramente de
peles, rastejou rápido até lá e apertou"-se contra o vidro, que o prendeu e
refrescou o calor de sua barriga. Pelo menos esse quadro, que agora Gregor
cobria por completo, com certeza ninguém levaria mais embora. Ele girou a
cabeça na direção da porta da sala, para prestar atenção nas mulheres, que
retornavam.

Elas não haviam se permitido muito tempo de descanso e logo estavam de
volta; Grete vinha com o braço ao redor da mãe e como que a arrastava. “E
agora, o que levamos?”, disse e olhou ao seu redor. Foi então que se
cruzaram, seu olhar com o de Gregor na parede. Por certo só em razão da
presença materna é que ela conservou a calma, abaixou o rosto para junto
da mãe, na tentativa de impedi"-la de olhar à sua volta, e falou, não
obstante trêmula e irrefletidamente: “Vamos voltar, não é melhor a gente
descansar mais um tempinho na sala?”. Para Gregor, a intenção de Grete era
clara, ela queria garantir a segurança da mãe para depois vir varrê"-lo da
parede. Pois ela que experimentasse! Ele estava unido ao seu quadro e não
o entregaria. Preferia pular na cara dela.

Mas as palavras de Grete a rigor só conseguiram sobressaltar a mãe, que
deu um passo para o lado, avistou a monstruosa mancha marrom sobre as
flores do papel de parede, gritou, com uma voz gutural e aflita, antes
mesmo de tomar consciência de que aquilo que via era Gregor: “Meu Deus,
meu Deus!”, e caiu por cima do canapé, com os braços estendidos, como se
largasse mão de tudo, e não voltou a se mexer. “Você, hein, Gregor!”,
bradou a irmã de punho erguido e com o olhar severo. Eram as primeiras
palavras diretas que lhe dirigia desde a transformação. Ela correu para o
cômodo ao lado, em busca de alguma essência com que pudesse despertar a
mãe desfalecida; Gregor também queria ajudar --- a defesa do quadro podia
ser adiada ---, mas estava tão grudado ao vidro que precisou usar da força
para se soltar; em seguida correu também até o aposento vizinho,
acreditando poder dar à irmã algum conselho, como nos velhos tempos; não
conseguiu fazer nada, porém, além de ficar imóvel atrás dela; enquanto
ainda remexia em diferentes frasquinhos, ela se virou e tomou um baita
susto; um vidro caiu no chão e se espatifou; um caco feriu Gregor no
rosto, uma substância corrosiva escorreu sobre ele; Grete, sem mais perda
de tempo, simplesmente pegou tantos frascos quanto podia carregar e
disparou com eles para junto da mãe; ao sair bateu a porta com o pé.
Gregor foi assim isolado da mãe, que por culpa dele talvez estivesse a
ponto de morrer; a porta ele não devia abrir, se não quisesse espantar a
irmã, que precisava ficar ao lado da mãe; não tinha no momento nada a
fazer, a não ser esperar; então, presa de remorsos e preocupação, ele
começou a rastejar, passando por cima de tudo, paredes, móveis, teto, e
por fim, em seu desespero, logo que todo o aposento começou a girar em
torno de si, caiu bem no centro, em cima da mesa grande.

Um breve intervalo de tempo transcorreu, Gregor continuava ali deitado,
prostrado, ao redor o silêncio, talvez isso fosse um bom sinal. Então a
campainha tocou. A mocinha, logicamente, estava trancada na cozinha e por
isso Grete teve de ir abrir. O pai estava de volta. “O que aconteceu?”,
foram suas primeiras palavras; a expressão da filha por certo lhe revelara
tudo. Grete respondeu com a voz abafada, o rosto apertado contra o peito
do pai: “Mamãe teve um desmaio, mas já está melhor. Gregor escapuliu”. “Eu
já esperava por isso”, disse o pai, “era o que eu sempre dizia, mas vocês
mulheres não quiseram me dar ouvidos.” Para Gregor ficou nítido que o pai
entendera mal o relato demasiado conciso de Grete e supunha que ele
tivesse cometido alguma brutalidade. Por isso devia agora tentar
apaziguá"-lo, pois para esclarecer as coisas não havia tempo, tampouco
possibilidade. De modo que se afastou até a porta do seu quarto e ficou
bem junto dela, para que o pai, assim que entrasse pela antessala, pudesse
ver que ele tinha toda a intenção de regressar imediatamente aos seus
aposentos, e que não seria preciso tocá"-lo de volta, pelo contrário,
bastava apenas abrir a porta e na mesma hora ele desapareceria.

Mas o pai não estava em condições de perceber tais sutilezas; “Arrá!”, ele
gritou logo ao entrar, em um tom que parecia ao mesmo tempo satisfeito e
ameaçador. Gregor tirou a cabeça da porta e a suspendeu na direção do pai.
Na verdade nunca havia imaginado o pai assim, como se apresentava agora;
além disso, nos últimos tempos, por conta da empolgação com o rastejar,
deixara de se preocupar como antes com os acontecimentos no resto do
apartamento, e portanto deveria estar pronto para encontrar muita coisa
mudada. Apesar disso, apesar de tudo, ainda era o pai? O mesmo homem que
antigamente, nas manhãs em que Gregor saía para uma viagem de negócios,
permanecia enterrado na cama, esgotado; que nas noites de regresso o
recebia no cadeirão, de pijama; incapaz de se erguer direito, podendo
apenas levantar os braços como sinal de alegria, e que nos raros passeios
da família, em um ou outro domingo do ano e nas datas mais importantes,
entre Gregor e a mãe, que em atenção a ele andavam devagar, caminhava
sempre um pouco mais devagar ainda, embrulhado em seu sobretudo gasto,
avançando a custo com a maior atenção ao apoiar a bengala, e quando queria
falar quase sempre se detinha e obrigava os outros a se reunir a seu
redor? Agora, porém, ele estava mesmo bem empertigado; metido em um
uniforme muito justo, azul com botões dourados, igual ao usado pelos
contínuos dos bancos; sobre o alto colarinho engomado do casaco
desdobrava"-se sua papada dupla; sob as sobrancelhas espessas aflorava o
brilho radiante e atento dos olhos escuros; o cabelo branco, outrora todo
espetado, fora emplastrado e repartido rigorosamente ao meio em um
penteado meticuloso e luzidio. O quepe, no qual despontava um monograma
dourado, decerto de uma casa bancária, ele jogou em cima do canapé, num
lançamento em curva que cruzou todo o cômodo, depois marchou em direção a
Gregor com a cara amarrada, as abas do casaco do uniforme postas para
trás, as mãos nos bolsos da calça. Ele próprio não sabia direito como
proceder; ainda assim, erguia os pés a uma altura fora do comum, e Gregor
se espantou com as proporções colossais do solado de suas botas. Claro que
não ficaria apenas nisso, sempre soube, desde o primeiro dia de sua nova
vida, que o pai, para lidar com ele, só considerava apropriado o máximo de
rigor. Então saiu da sua frente correndo, estacando quando o pai ficava
parado e voltando a se apressar mal o pai se mexia. Desse modo deram mais
de uma vez a volta pelo cômodo, sem que algo de decisivo acontecesse,
inclusive sem que, no geral, em consequência do lento andamento, houvesse
a aparência de uma perseguição. Em vista disso, Gregor por enquanto também
se mantinha no chão, tanto mais temendo que o pai pudesse tomar uma fuga
pelas paredes ou pelo teto como uma provocação maldosa. Contudo, foi
obrigado a reconhecer que não suportaria essas corridinhas por muito mais
tempo, porque, enquanto o pai dava um passo, ele tinha de realizar um
sem"-número de movimentos. A dificuldade de respirar logo começou a
aparecer, visto que, como nos velhos tempos, também não dispunha agora de
pulmões dignos de confiança. Por isso, ao se balançar para os lados, a fim
de reunir todas as forças para a corrida, nem abria os olhos; em sua
obtusidade, não conseguia pensar em outra solução a não ser correr; e
quase já havia esquecido que podia contar com as paredes, as quais,
todavia, eram aqui obstruídas por móveis finamente talhados, cheios de
pontas e quinas --- foi quando passou raspando ao lado dele alguma coisa
que, arremessada sem força, caiu e rolou à sua frente. Era uma maçã; logo
uma segunda voou em sua direção; Gregor ficou paralisado de medo; uma nova
corrida era inútil, pois o pai estava determinado a um bombardeio. Tinha
enchido os bolsos na fruteira do aparador e, sem caprichar na pontaria por
enquanto, atirava, maçã atrás de maçã. As pequenas frutas vermelhas
rolavam pelo chão, pareciam imantadas, e batiam umas contra as outras. Uma
maçã lançada de mansinho atingiu Gregor de leve, mas ricocheteou sem
maiores danos. A que veio voando logo em seguida, ao contrário, penetrou
firme em suas costas; Gregor quis ainda se arrastar como se a dor
extraordinariamente incrível fosse passar com a mudança de posição; porém
ele se sentia como se estivesse pregado e desmoronou, em completo colapso
de todos os sentidos. Só num último relance ainda pôde ver como a porta de
seu quarto se escancarou e a mãe, à frente da irmã que gritava, entrou
depressa, sem o vestido, pois a irmã a despira após o desmaio para
permitir"-lhe uma respiração mais livre, como em seguida a mãe correu ao
encontro do pai e suas anáguas desapertadas escorregaram uma após a outra
até o chão, e como ela, tropeçando nas vestes caídas, atirando"-se sobre o
pai e o abraçando, em completa conjunção com ele --- mas nessa hora a visão
de Gregor já vacilava ---, cruzando as mãos em torno de seu pescoço, pediu
clemência pela vida de Gregor. 

\sectionitem

O ferimento sério, com o qual padeceu mais de um mês --- a maçã continuou,
uma vez que ninguém se atreveu a retirar, enfiada na carne, como uma
recordação exposta ---, pareceu fazer até mesmo o pai se lembrar de que
Gregor, apesar de suas feições asquerosas e deprimentes, era um membro da
família, e não merecia ser tratado que nem um inimigo, pelo contrário, no
seu caso a lei das obrigações familiares mandava engolir a repulsa e
tolerar, nada além de tolerar.

E se então, devido ao ferimento, Gregor também havia perdido certa
mobilidade, talvez para sempre, e necessitava no momento, como um veterano
mutilado, de longos e longos minutos para cruzar seu quarto --- rastejar no
alto, nem pensar ---, agora, em troca desse sensível declínio em suas
condições, recebia uma compensação em sua opinião bastante satisfatória,
dado que à noitinha a porta da sala, que uma ou duas horas antes ele já
tratava de observar com atenção, era em geral aberta para que, deitado no
escuro do seu quarto, não visível da sala, ele pudesse ver a família
reunida na mesa iluminada e escutar suas conversas, de certo modo com o
consentimento de todos, e portanto bem diferente de antes.

Decerto não eram mais aquelas reuniões calorosas dos velhos tempos, que
nos minúsculos quartos de hotel Gregor imaginara, nunca sem uma ponta de
inveja, quando, exausto, era obrigado a se enfiar nos lençóis gelados.

Tudo agora transcorria quase sempre na maior monotonia. Depois do jantar o
pai logo pegava no sono em sua cadeira; a mãe e a irmã recomendavam
silêncio uma à outra; a mãe, bastante curvada embaixo da luz, costurava
fina roupa branca para uma loja de confecções; a irmã, que arranjara um
emprego de vendedora, aprendia taquigrafia e francês à noite, para quem
sabe mais tarde obter um cargo melhor. Às vezes o pai despertava e, sem
nem se dar conta de que estivera dormindo, dizia para a mãe: “Mas que
tanto você fica aí só costurando!”, e voltava a adormecer no instante
seguinte, enquanto mãe e irmã trocavam entre si um sorriso fatigado.

Com uma birra particular, o pai nem em casa aceitava tirar seu uniforme de
serviço; e enquanto o pijama pendia inútil no cabide ele cochilava no seu
canto completamente vestido, como se estivesse sempre a postos para o
trabalho e mesmo aqui esperasse pelas ordens do supervisor. Em
consequência disso, o uniforme, que já não era novo no começo, não parava
limpo, apesar de todo o cuidado da mãe e da irmã, e Gregor olhava várias
vezes a noite inteira para aquela roupa vistosa, com seus botões dourados
superlustrados, e a cada dia mais cheia de nódoas, com a qual o velho no
maior desconforto dormia, entretanto, bem tranquilo.

Na hora em que o relógio chegava às dez, a mãe, fingindo bronquear,
chamava o pai e em seguida tentava persuadi"-lo a ir para a cama, pois ali não
era lugar de dormir e uma boa noite de sono seria imprescindível para ele,
que devia se apresentar às seis da manhã em seu emprego. Mas, com a teima
que tinha desde que estava empregado, o pai sempre insistia em permanecer
à mesa, embora invariavelmente adormecesse, e então, além de tudo, era a
maior mão"-de"-obra induzi"-lo a trocar a cadeira pela cama. Por mais que a
mãe e a irmã o animassem com breves incentivos, ele ficava uns quinze
minutos balançando devagar a cabeça, mantinha os olhos fechados, e não se
levantava. A mãe o puxava pela manga, sussurrava palavras ternas ao pé do
seu ouvido, a irmã deixava as lições para ajudá"-la, mas com o pai isso de
nada valia. Ele só afundava ainda mais na cadeira. Apenas quando as
mulheres o tomavam pelos braços é que ele entreabria os olhos, olhava
alternado para a mãe e para a irmã, fazendo questão de dizer: “Isto é que
é vida. Este é o sossego da minha velhice”. E apoiado nas duas se
levantava, com muita dificuldade, como se fosse ele mesmo o que mais lhe
pesava, deixava"-se conduzir pelas mulheres até a porta, lá as dispensava e
prosseguia então sozinho, enquanto a mãe largava às pressas o estojo de
costura, a irmã a pena, para irem correndo atrás dele e continuar ajudando.

Nessa família esfalfada e moída pelo trabalho, quem teria tempo para se
ocupar com Gregor além do estritamente necessário? O orçamento doméstico
era cada vez mais apertado; a empregada já fora inclusive dispensada; uma
faxineira enorme e robusta, com uma cabeleira branca esvoaçando pela
cabeça, vinha agora pela manhã e no final da tarde para dar conta do
serviço pesado; o restante sobrava para a mãe, junto com suas muitas
obrigações de costura. Até aconteceu de diversas joias da família, que no
passado a mãe e a irmã chegaram a ostentar radiantes em recepções e
cerimônias, terem de ser vendidas, como Gregor veio a saber à noite pela
falação geral em torno do montante obtido. A maior reclamação, porém, era
sempre a de que não podiam abandonar aquele apartamento, grande demais nas
atuais circunstâncias, por não saberem como incluir Gregor na mudança. Mas
Gregor percebia muito bem que não era apenas a preocupação com ele que
impedia uma mudança, pois seria simples transportá"-lo em uma caixa
apropriada, com alguns furos para a entrada de ar; o que realmente
refreava a troca de apartamento era muito mais a total falta de esperanças
da família e o pensamento de que uma desgraça tal os fustigava, como nunca
a ninguém em todo o seu círculo de parentes e conhecidos. O que o mundo
demandava aos
%Iuri:cobravam dos
pobres, eles ofereciam ao máximo,
%Iuri:cumpriam totalmente
o pai ia buscar café para
os funcionários do baixo escalão do banco, a mãe se sacrificava pela roupa
íntima de pessoas estranhas, a irmã corria de um lado para o outro do
balcão atrás das exigências dos fregueses, mas as forças da família já
estavam no limite. E a ferida aberta nas costas voltava a doer em Gregor
como na primeira vez, quando a mãe e a irmã, depois de colocarem o pai na
cama, regressavam, deixavam a ocupação de lado, aproximavam"-se uma da
outra e sentavam"-se já de rosto colado; quando a mãe, apontando para
o quarto de Gregor, dizia: “Fecha aquela porta, Grete”, e quando enfim
Gregor voltava à escuridão, enquanto do outro lado as mulheres misturavam
suas lágrimas ou então, sem lacrimejar, contemplavam a mesa desconsoladas.

Gregor desperdiçava as noites e os dias e já quase não dormia. Algumas
vezes pensava em retomar as rédeas dos assuntos da família, igualzinho
como antes, na próxima vez que a porta fosse aberta; em seus pensamentos
voltavam a aparecer, depois de muito tempo, o chefe e o gerente, os
balconistas e os aprendizes, o imbecil do menino de recados, dois ou três
colegas de outras lojas, uma camareira de um hotel no interior,
lembrança querida e fugaz, uma atendente de caixa de uma loja de chapéus,
a quem cortejara a sério porém de modo muito tímido --- todos eles apareciam
misturados com pessoas estranhas ou já esquecidas, só que não serviam de
ajuda a ele e à sua família, ao contrário, eram indistintamente
inapeláveis, e Gregor ficava contente quando desapareciam. Mas daí não
tinha mais nenhuma gana de se preocupar com a família, sobrevinha apenas a
raiva pelo desleixo e, apesar de não poder imaginar nada que atiçasse o
seu apetite, planejava um modo de alcançar a despensa para retirar de lá
ao menos o que lhe era devido, ainda que não estivesse com fome. A irmã
agora, de manhã e ao meio"-dia, antes de sair correndo para a loja, sem nem
cogitar uma forma de fazer um agrado especial a ele, empurrava para dentro
do quarto às pressas, com o pé, algum resto de comida qualquer, que à
noitinha ela botava para fora com uma vassourada, nem reparando se a
comida por acaso havia sido provada ou --- caso mais frequente --- continuava
perfeitamente intacta. A arrumação do quarto, de que ela agora só se
ocupava à noite, não poderia ser feita de modo mais rápido. Trilhas de
sujeira se estendiam pelas paredes, em toda parte cresciam montinhos de pó
e lixo. Nas primeiras vezes, assim que a irmã entrava, Gregor se
posicionava de maneira a deixar patente a sua insatisfação. Mas poderia ficar ali parado uma
semana, sem que a irmã se emendasse; ela enxergava a imundície tanto
quanto ele, porém estava determinada a ignorá"-la. Ademais, com uma
suscetibilidade há pouco tempo adquirida, que a propósito contaminara toda
a família, ela zelava para que a arrumação do quarto de Gregor ficasse
exclusivamente por sua conta. Uma vez a mãe submeteu o quarto a uma faxina
geral, tarefa somente possível com o emprego de várias tinas de
água --- a umidade em excesso, no fim, também fez mal a Gregor e ele foi
deitar em cima do canapé, meio grogue, amargurado e incapaz de se mover ---,
mas o castigo da mãe não tardou a chegar. Pois à noite a irmã, mal notara
a mudança no quarto, voltou correndo para a sala, ofendidíssima, e, não
obstante o gesto de súplica das mãos erguidas da mãe, irrompeu numa crise
de choro que os pais --- o pai naturalmente acordou de um pulo de sua
cadeira --- a princípio observaram aturdidos e sem ação; até que também se
condoeram; o pai repreendia a mãe à direita, por não ter deixado à irmã a
limpeza do quarto de Gregor; à esquerda, por outro lado, berrava com a
irmã, exigindo que ela nunca mais limpasse aquele quarto; enquanto a mãe
tentava levar para a cama o pai, que ficara fora de si nesse alvoroço, a
irmã, sacudida pelos soluços, martelava a mesa com seus punhos pequenos; e
Gregor silvava alto de raiva, porque ninguém havia se lembrado de fechar a
porta e poupá"-lo da cena e de toda aquela barulheira.

Mas mesmo se a irmã, exaurida por sua atividade profissional, estivesse
cheia de cuidar dele como cuidava antes, a mãe não teria de modo algum
obrigação de rendê"-la e Gregor ainda assim não precisava ficar ao
desamparo. Pois agora tinham a faxineira. Essa velha viúva, que em sua
longa vida devia ter superado as piores situações com a ajuda de sua
notável robustez, não nutria a rigor nenhuma aversão por Gregor. Não sendo
de modo algum curiosa, havia uma vez por acaso aberto a porta do quarto
dele, que pego de surpresa começou a correr de um lado para outro, embora
ninguém o perseguisse, e ao avistá"-lo ela continuou de pé onde estava, as
mãos cruzadas no peito, admirada. Desde então, não passava um dia sem
entreabrir a porta, de manhã e no final da tarde, para espiar Gregor um
minutinho. No começo, ela também o chamava com palavras que devia julgar
simpáticas, tais como “Vem cá, bichão!” ou “Cadê o velho besourão?!”.
Gregor não atendia a esses chamados, ficava mais é quieto no seu canto,
como se a porta nem tivesse sido aberta. Se ao menos fosse ordenado a essa
faxineira que, em vez de satisfazer seus caprichos indo perturbá"-lo para
nada, limpasse seu quarto todos os dias! Certa ocasião, de manhã cedinho ---
uma chuva pesada golpeava a vidraça, talvez já um sinal anunciando a
primavera ---, logo que a faxineira veio de novo com aquele jeito de falar,
Gregor ficou irritado de tal modo que se virou para ela, é certo que num
passo lento e debilitado, como se fosse partir para o ataque. A faxineira,
porém, em vez de se intimidar, simplesmente tomou uma cadeira que
encontrou ao lado da porta, ergueu"-a bem no alto e, do modo como ficou
parada lá com a bocona escancarada, era clara sua intenção de só fechá"-la
quando a cadeira em suas mãos tivesse baixado nas costas dele. “Não vai
avançar mais?”, ela perguntou, ao vê"-lo retroceder, e pôs de volta
tranquila a cadeira em seu lugar.

Gregor já não comia quase nada. Só quando por acaso topava com a comida
disponível é que abocanhava de brincadeira uma pequena porção, que
mantinha na boca horas a fio e depois cuspia fora, na maior parte das
vezes. A princípio pensou que era a tristeza com o estado de seu quarto
que o impedia de comer, mas foi justo com as mudanças do quarto que ele se
conformou mais depressa. Coisas que não podiam ser alojadas em outra parte
eram jogadas lá dentro, e agora havia muitas dessas coisas, dado que um
quarto do apartamento havia sido alugado para três inquilinos. Esses
senhores muito sérios --- todos os três usavam barba, como Gregor averiguou
certa vez por uma fresta da porta --- eram cheios de escrúpulos quanto à
ordem, não apenas no quarto que ocupavam, mas também, já que agora estavam
hospedados ali, em toda a casa, particularmente na cozinha. Não toleravam
trastes inúteis, muito menos sujos. Além disso haviam trazido consigo, em
grande parte, seus próprios móveis e objetos de decoração. Por esse motivo
muitas coisas se tornaram supérfluas, coisas que na verdade não eram
vendáveis, mas que também não se queria jogar fora. Todas elas rumaram
para o quarto de Gregor. Mesmo destino da caixa de cinzas e do cesto de
lixo da cozinha. Bastava que algo não tivesse utilidade no momento para
que a faxineira, sempre com muita pressa, simplesmente o atirasse para
dentro do quarto de Gregor; na maioria das vezes, por sorte, Gregor via só
o objeto em questão e a mão que o segurava. Pode ser que a faxineira
tivesse a intenção, havendo tempo e oportunidade, de pegar as coisas de
volta ou então de jogar tudo fora de uma vez, mas na prática elas ficavam
por lá, largadas no mesmo local onde haviam sido atiradas, a não ser
quando Gregor se retorcia no meio da tralha e começava a deslocá"-las, no
começo forçado, porque já não sobrava espaço livre para rastejar, mas
depois com uma alegria crescente, embora, ao final dessas excursões, morto
de cansaço e mágoa, ficasse horas sem poder se movimentar de novo.

Como os inquilinos às vezes jantavam em casa, na sala, que era de uso
comum, a porta que dava para lá permanecia fechada algumas noites, mas
Gregor abdicava de sua abertura sem dificuldade, já tendo inclusive
deixado de aproveitá"-la em várias ocasiões em que fora franqueada,
preferindo ficar encolhido, sem que a família percebesse, no canto mais
escuro do seu quarto. Uma vez, porém, a faxineira deixou essa porta um
tanto entreaberta, e assim ela ficou, mesmo quando os inquilinos entraram
à noite e a luz foi acesa. Eles foram se sentar à ponta da mesa, onde
antigamente o pai, a mãe e Gregor comiam, desdobraram os guardanapos e
empunharam garfo e faca. Na mesma hora a mãe apareceu na porta com uma
travessa de carne e bem atrás dela a irmã com outra travessa transbordando
de batatas. A comida fervia numa nuvem de vapor. Os inquilinos se curvaram
sobre as travessas colocadas à sua frente, como se quisessem dar uma
conferida antes de comer, e de fato o que estava sentado no meio, e
parecia ter mais autoridade que os outros dois, deu um talho na carne
ainda na travessa, averiguando de modo ostensivo se estava tenra o
suficiente e se não era o caso de devolvê"-la à cozinha. Ele ficou
satisfeito, e a mãe e a irmã, que observavam apreensivas, puderam suspirar
com um sorriso de alívio.

A família mesmo foi comer na cozinha. Apesar disso, antes de ir para lá o
pai veio até a sala e, com uma única mesura, o quepe na mão, deu a volta
em torno da mesa. Os inquilinos levantaram"-se juntos e murmuraram de má
vontade qualquer coisa lá com suas barbas. Quando então ficaram a sós,
comeram quase que em total silêncio. Gregor achou estranho que, de todos
os diversos ruídos que acompanham uma refeição, o mais perceptível fosse o
som de dentes mastigando, como se com isso devesse ser mostrado a ele que
para comer eram necessários dentes, e que com mandíbulas banguelas, mesmo
as mais bonitas, não se obtém sucesso. “Eu tenho sim vontade de comer”,
Gregor disse a si mesmo, muito preocupado, “só que não essas
coisas. Como comem esses inquilinos, e eu aqui definhando!”

Bem naquela noite --- Gregor não se lembrava de tê"-lo escutado em todo
aquele tempo --- veio da cozinha o som do violino. Os inquilinos já haviam
terminado sua refeição noturna, o do meio desdobrou um jornal, deu aos
outros dois uma folha cada, e agora eles liam recostados e fumavam. Quando
o violino começou a soar, tiveram a atenção despertada, ergueram"-se e
foram na ponta dos pés até a porta da antessala, onde ficaram parados, um
espremido contra o outro. Devem ter sido ouvidos da cozinha, pois o pai
gritou de lá: “A música não agrada aos senhores? Ela pode parar agora
mesmo”. “Pelo contrário”, disse o senhor do meio, “a menina não quer vir
até aqui tocar na sala, que é muito mais cômodo e agradável?” “Oh, como
não!”, gritou o pai, como se fosse ele o violinista. Os senhores voltaram
para a sala e esperaram. Logo chegou o pai com a estante, a mãe com a
partitura e a irmã com o violino. A irmã calmamente ajeitou tudo para
tocar; os pais, que nunca antes haviam alugado quartos e por isso se
excediam na cortesia com os inquilinos, nem se atreveram a sentar nas
próprias cadeiras; o pai encostou na porta, a mão direita enfiada entre
dois botões do casaco abotoado; a mãe, porém, aceitou a cadeira oferecida
por um dos senhores e, como não a moveu do local onde ele por acaso a
colocara, acabou sentando afastada.

A irmã começou a tocar; pai e mãe, cada qual do seu lado, seguiam atentos
os movimentos das mãos dela. Gregor, atraído pela música, arriscou"-se um
pouco mais à frente e logo estava com a cabeça na sala. Já nem se admirava
de que, nos últimos tempos, demonstrasse tão pouca consideração pelos
outros; antigamente esse tipo de respeito era para ele um motivo de
orgulho. E na verdade teria justo agora muito mais razões para não
aparecer, porque, por causa da poeira que tomava conta de seu quarto e que
ao menor movimento se espalhava no ar, ele também vivia coberto de pó;
arrastava consigo fiapos, pelos, restos grudados dos lados e nas costas;
sua apatia diante de tudo era muito grande para que se desse ao trabalho
de deitar de costas, como fazia antes várias vezes ao dia, e se esfregar
no tapete. E mesmo nesse estado ele não teve o menor pudor de invadir um
pedaço do piso imaculado da sala.

Em todo caso ninguém nem reparava nele. A família estava bastante absorta
pelo som do violino; ao contrário dos inquilinos, que a princípio, as mãos
nos bolsos das calças, haviam se posicionado bem atrás da estante, muito
próximos da irmã, de modo a que pudessem todos os três acompanhar as
notas, o que com certeza devia atrapalhá"-la, mas logo, conversando a meia
voz, as cabeças abaixadas, se afastaram para junto da janela, onde
permaneceram, observados pelo olhar apreensivo pai. Essa atitude realmente
tinha a aparência mais do que clara de que eles haviam sido frustrados em
sua intenção de ouvir ao violino uma música bela ou animada, que estavam
fartos daquela apresentação e que apenas por educação ainda admitiam ter
sua paz perturbada. Sobretudo o modo como todos eles sopravam para o alto
a fumaça de seus charutos, pelo nariz e pela boca, dava a entender um alto
grau de irritação. E todavia a irmã tocava tão bonito. Seu rosto pendia
para o lado, seus olhos aguçados e tristes acompanhavam as linhas da
pauta. Gregor rastejou outro tanto adiante e manteve a cabeça bem junto ao
chão, para assim, quem sabe, poder interceptar o olhar dela. Era por ser
um animal que a música o atraía tanto? A ele parecia como se tivesse à sua
frente o caminho para o ansiado alimento desconhecido. Estava decidido a
avançar até a irmã, puxá"-la pela saia, e desse modo fazê"-la entender que
ela poderia com prazer vir ao quarto dele com o violino, pois ninguém aqui
valorizava a música como ele quisera valorizar. Queria então não mais
deixá"-la sair do quarto, pelo menos não enquanto estivesse vivo; sua
figura medonha pela primeira vez lhe seria útil; queria estar ao mesmo
tempo diante de todas as portas, bufando feito uma fera contra os
opressores; a irmã, porém, não devia ficar com ele à força, e sim por
vontade própria; ela iria se sentar ao lado dele no canapé, inclinar o
ouvido em sua direção, e ele queria nesse momento confidenciar a ela que
tivera o firme propósito de mandá"-la para o conservatório, e que, se nesse
meio tempo a desgraça não houvesse ocorrido, no último Natal --- o Natal já
tinha passado, aliás? ---, ele teria anunciado o fato a todos, sem se
importar com qualquer tipo de objeção. Depois dessa explicação, a irmã
iria desatar num choro comovido, e Gregor, erguendo"-se até altura do ombro
dela, lhe daria um beijo no pescoço que, desde que entrara na loja, ela
trazia sem fita nem cordão.

“Senhor Samsa!”, gritou para o pai o senhor do meio e, sem desperdiçar
mais palavras, apontou o dedo indicador na direção de Gregor, que avançava
devagar. O violino ficou mudo, o inquilino intermediário, num único
movimento de cabeça, sorriu para seus amigos e então voltou a olhar para
Gregor. Ao pai pareceu que o mais urgente era, em vez de tocar Gregor,
primeiro acalmar os inquilinos, embora estes não estivessem nada nervosos
e Gregor parecesse animá"-los bem mais do que a música do violino. Ele
correu para o lado deles e com os braços abertos tentou conduzi"-los a seus
aposentos e ao mesmo tempo bloquear com o corpo a visão de Gregor. Aí eles
de fato se zangaram um pouco, não dava para saber se com o comportamento
do pai ou se com a descoberta que acabavam de fazer, de que, sem que
soubessem, tinham como vizinho de quarto um tipo igual a Gregor. Exigiram
explicações do pai, ergueram como ele os braços, beliscaram as barbas com
impaciência e só a custo foram aos poucos recuando aos seus aposentos.
Enquanto isso a irmã conseguiu sair do estado de alheamento em que havia
caído com a interrupção súbita da música, depois de sustentar mais um
instante o violino e o arco nas mãos que pendiam frouxas, e em seguida
olhar para a partitura como se ainda estivesse tocando, conseguiu se recompor
num átimo, colocou o instrumento no colo da mãe, que com dificuldades
respiratórias e intensa atividade pulmonar seguia sentada em sua cadeira,
e foi correndo para o quarto ao lado, do qual, empurrados pelo pai, os
inquilinos se aproximavam agora mais depressa. Era de ver como, sob as
mãos treinadas da irmã, cobertas e travesseiros voavam pelo ar e pousavam
já dispostos direitinho nas camas. Ainda antes dos senhores chegarem à
entrada do quarto, ela já dera conta da arrumação e saíra de fininho. O
pai pareceu de novo tomado por sua teimosia, de uma tal forma que olvidou
todo o respeito a que estava obrigado perante seus locatários. Ele só
insistia e voltava a empurrar, até que, na porta do quarto, o senhor do
meio bateu com força o pé no chão, e desse modo conseguiu deter o pai.
“Declaro para todos os fins”, ele falou, ergueu a mão e dirigiu o olhar
também para a mãe e a irmã, “que, em vista das condições repugnantes em
vigor nesta casa e nesta família” --- nesse ponto ele, decidido, deu uma
breve cusparada no chão ---, “dou por cancelada minha locação. É evidente
que não pagarei o mínimo que seja, nem pelos dias em que aqui estive
hospedado, muito pelo contrário, ainda hei de ponderar se não apresento
contra o senhor uma queixa formal que --- o senhor pode ter certeza --- seria
bem fácil de justificar.” Ele se calou e olhou fixo à sua frente, como se
esperasse alguma coisa. Ato contínuo, seus amigos vieram com as palavras:
“Nós também cancelamos a nossa”. A seguir ele agarrou a maçaneta e fechou
ruidosamente a porta.

O pai, tateando com as mãos, voltou cambaleando até sua cadeira e ali
desabou; dava a impressão de que se recostava para sua soneca noturna
cotidiana, mas os intensos meneios da cabeça, que parecia solta, mostravam
que de modo nenhum estava dormindo. Gregor ficou esse tempo todo quieto,
deitado no mesmo lugar em que fora surpreendido pelos inquilinos. O
desapontamento com o insucesso de seus planos mas talvez também a fraqueza
provocada pela fome extrema impossibilitavam que se movimentasse. Com uma
certa garantia, temia já para o instante seguinte a tempestade coletiva a
desabar em cima dele, e esperava. Não o perturbou nem mesmo o violino na
hora em que caiu do colo da mãe, depois de escorregar de seus dedos
trêmulos, e produziu um estampido retumbante.

“Meus pais”, falou a irmã e bateu a mão na mesa à guisa de introdução,
“não dá para continuar assim. Se vocês por acaso não enxergam desse jeito,
eu enxergo. Não quero usar o nome do meu irmão com esse monstro, então só
digo uma coisa: temos que nos livrar disso. Tentamos o que era humanamente
possível para criá"-lo e suportá"-lo, acho que ninguém tem o menor direito
de nos condenar.”

“Ela está coberta de razão”, o pai disse consigo mesmo. A mãe, que ainda
não conseguia respirar normalmente, levou a mão à boca e começou a tossir
para dentro, com uma expressão de insanidade no olhar.

A irmã correu para junto da mãe e amparou"-lhe a testa. O pai, levado pelas
palavras da irmã, pareceu ter encontrado ideias mais precisas,
endireitou"-se na cadeira, brincou com o quepe do uniforme em meio aos
pratos largados sobre a mesa, ainda do jantar dos inquilinos, e olhou uma
vez ou outra para Gregor, que permanecia quieto.

“Temos que dar um jeito de nos livrar dessa coisa”, a irmã falou, agora
exclusivamente para o pai, pois a mãe, com a tosse, não escutava nada,
“isso vai acabar matando vocês dois, só estou vendo a hora. Quem é
obrigado a trabalhar tanto quanto nós três não pode ainda ter que aturar
em casa esse tormento sem fim. Eu já não suporto mais.” E rompeu num choro
tão forte que suas lágrimas espirraram no rosto da mãe, que as enxugou com
movimentos mecânicos da mão.

“Filha”, falou o pai morrendo de pena e exagerando na compreensão, “mas o
que vamos fazer?”

A irmã apenas deu de ombros como sinal da sensação de impotência que a
acometera enquanto chorava, o que contrastava com sua segurança de antes.

“Se ele pudesse nos entender”, disse o pai com ar interrogativo; a irmã em
meio ao choro sacudiu a mão com veemência em sinal de que isso estava fora
de cogitação.

“Se ele pudesse nos entender”, repetiu o pai e fechou os olhos, para
assimilar a convicção da irmã quanto à impossibilidade do fato, “então
quem sabe fosse possível um acordo com ele. Mas assim ---”

“Que vá embora”, a irmã gritou, “é o único jeito, pai. Basta se livrar do
pensamento de que é o Gregor. Ter acreditado nisso durante tanto tempo,
essa no fundo é a nossa desgraça. Mas como essa coisa pode ser o Gregor?
Se fosse o Gregor, já teria entendido há muito tempo que é impossível a
convivência das pessoas com um bicho desses, e teria partido por vontade
própria. Então não haveria mais irmão, mas podíamos seguir tocando a vida
e preservar sua memória. Mas assim, fica esse bicho nos seguindo, expulsa
os inquilinos, com certeza quer tomar o apartamento e nos deixar dormindo
na rua. Olha aí, pai”, ela gritou de repente, “lá vem ele de novo!” E, num
pânico de todo incompreensível para Gregor, a irmã deixou até a mãe de
lado, chegou mesmo a empurrar sua cadeira, como se preferisse sacrificar a
mãe a ficar perto dele, e correu atrás do pai que também se levantou,
provocado única e exclusivamente pelo comportamento dela, e posicionou os
braços a meia altura à sua frente, na tentativa de protegê"-la.

Mas Gregor não pretendia de modo algum amedrontar quem quer que fosse,
muito menos a irmã. Ele havia tão"-só começado a se virar para regressar ao
seu quarto, o que em todo caso seria muito chamativo, dado que, devido às
suas condições lastimáveis, era difícil fazer a volta e ele precisava
ajudar com a cabeça que por causa disso diversas vezes levantava e
batia contra o chão. Parou um pouco e olhou ao seu redor. Parece que
reconheciam sua boa intenção; tinha sido só um susto passageiro. Agora
todos o observavam calados e entristecidos. A mãe largada em sua cadeira,
as pernas estendidas uma por cima da outra, os olhos quase se fechando de
fadiga; o pai e a irmã sentados lado a lado, ela apoiando a mão atrás do
pescoço dele.

“Agora talvez me deixem completar a volta”, Gregor pensou e retomou sua
ocupação. Não conseguia evitar o ruído de sua respiração ofegante do
esforço e era obrigado a fazer uma pausa a todo momento. No mais, não era
pressionado por ninguém, havia sido deixado por sua própria conta. Depois
de completar a volta ele deu início ao retorno em linha reta. Ficou
admirado com a grande distância que o separava de seu quarto, e não
entendia como, com a sua fraqueza, tinha ainda há pouco, quase sem
perceber, percorrido o mesmo caminho. O tempo todo concentrado apenas em
rastejar depressa, ele mal reparava que não era incomodado por nenhuma
palavra, nenhum grito de sua família. Só depois de chegar à porta é que
ele virou a cabeça, não muito porque na hora sentiu um torcicolo, em todo
caso ainda viu que atrás de si nada havia mudado, apenas a irmã se
levantara. Seu último olhar avistou de relance a mãe, a essa altura já
totalmente adormecida.

Mal acabara de entrar no quarto e a porta foi fechada às pressas, com o
trinco e à chave. Gregor levou um susto tão grande com o barulho
inesperado às suas costas que suas perninhas vacilaram. Era a irmã que
tinha toda essa pressa. Já estava lá levantada só esperando, avançou então
saltitando na ponta dos pés, Gregor nem a ouviu se aproximar, e ao girar a
chave na fechadura ela gritou para os pais: “Até que enfim!”.

“E agora?”, perguntou"-se Gregor e olhou a escuridão ao seu redor. Logo
veio a descobrir que não conseguia se mexer de jeito nenhum. Não ficou
surpreso, antes lhe parecia pouco natural que até então tivesse podido
andar de verdade com aquelas perninhas finas. De resto ele estava
relativamente bem. É certo que tinha dores por todo o corpo, mas para ele
era como se elas fossem ficando mais e mais fracas e estivessem a ponto de
sumir de uma vez por todas. Já mal sentia a maçã podre em suas costas e a
inflamação que a circundava, ambas cobertas por uma fina camada de poeira.
Lembrou"-se de sua família com intensa ternura e amor. Sua posição a
respeito da necessidade de seu desaparecimento era na medida do possível
ainda mais convicta do que a da irmã. Ficou nesse estado meditabundo,
vazio e tranquilo, até que as três horas da manhã soaram no relógio da
torre. Ainda presenciou o despontar da claridade matutina do lado de fora
da janela. Então, independente de sua vontade, sua cabeça pendeu toda para
baixo e suas narículas expeliram sem força seu último suspiro.

Quando, de manhã cedo, a faxineira veio --- sempre afobada e com muita
energia, apesar de já lhe ter sido pedido várias vezes que o evitasse, ela
batia todas as portas de uma tal maneira que com a sua chegada já não era
mais possível dormir sossegado em nenhum cômodo do apartamento ---, a
princípio não achou nada de novo em sua habitual visitinha a Gregor.
Pensou que ele estivesse ali deitado imóvel de propósito, bancando o
ofendido; ela o julgava capaz de compreender tudo. Como por acaso estava
com a comprida vassoura nas mãos, tentou fazer cócegas nele ali mesmo da
porta. Quando isso também não resultou em nada, ela ficou brava e cutucou
um pouco mais fundo, e só após empurrá"-lo de seu lugar sem encontrar
nenhuma resistência é que procurou ver melhor. Tão logo reconheceu a
realidade dos fatos, arregalou os olhos, soltou um assobio, mas não se
conteve muito tempo, e sim foi logo escancarar a porta do quarto dos pais,
gritando bem alto para dentro da escuridão: “Venham dar uma olhada, ele
bateu as botas; está lá, abotoou de vez o paletó!”.

Marido e mulher estavam sentados na cama de casal e tiveram primeiro que
se recuperar do susto com a faxineira, antes de virem a compreender o que
ela anunciava. Mas então, cada qual pelo seu lado, o senhor e a senhora
Samsa saíram depressa da cama, o senhor Samsa jogou a coberta por cima dos
ombros, a senhora Samsa apareceu só de camisola; assim entraram no quarto
de Gregor. Entrementes, fora igualmente aberta a porta da sala, onde Grete
dormia desde a chegada dos inquilinos; ela estava vestida dos pés à
cabeça, como se nem tivesse dormido, seu rosto pálido parecia demonstrar a
mesma coisa. “Morto?”, falou a senhora Samsa e olhou para a faxineira,
esperando uma resposta, embora pudesse confirmar por si só ou até
reconhecer o fato sem confirmação nenhuma. “É o que eu acho”, disse a
faxineira e como prova empurrou com a vassoura o cadáver de Gregor um bom
pedaço para o lado. A senhora Samsa reagiu como se fosse tentar parar a
vassoura, mas não fez nada. “Bom”, falou o senhor Samsa, “agora podemos
agradecer a Deus.” Ele fez o sinal da cruz, e as três mulheres seguiram o
seu exemplo. Grete, que não tirava os olhos do cadáver, disse: “Vejam como
ele estava magro. Já fazia um bom tempo que não comia nada. A comida
voltava do jeito que ia”. De fato o corpo de Gregor estava bem achatado e
seco, e só agora é que reparavam nisso, quando ele não mais se erguia
sobre suas perninhas e também nenhuma outra coisa distraía a atenção.

“Vamos, Grete, entra um pouquinho com a gente”, disse a senhora Samsa com
um sorriso tristonho, e Grete, não sem voltar a olhar o cadáver mais uma
vez, entrou atrás dos pais no quarto do casal. A faxineira fechou a porta
e abriu toda a janela. Embora ainda fosse cedo, uma certa tepidez
misturava"-se ao ar fresco da manhã. Já era sem dúvida o final de março.

Os três inquilinos saíram de seu quarto e atônitos procuraram com os olhos
pelo café da manhã; haviam sido esquecidos. “Cadê o café da manhã?”, o
senhor do meio perguntou mal"-humorado à faxineira. Esta porém pôs o
indicador na frente da boca e a seguir, em silêncio, fez apressada um
sinal indicando aos inquilinos que por favor passassem ao quarto de
Gregor. Eles passaram e ficaram lá de pé, as mãos nos bolsos de seus
paletós curtos e um tanto gastos, no quarto já totalmente iluminado, em
torno do cadáver.

Foi quando a porta do quarto dos pais se abriu, e o senhor Samsa apareceu
de uniforme, sua esposa apoiada em um braço, sua filha no outro. Todos com
cara de quem tinha chorado um pouco; Grete vez por outra pressionava o
rosto no braço do pai.

“Saiam agora mesmo da minha casa!”, disse o senhor Samsa e apontou para a
porta, sem soltar as mulheres. “O que significa isso?”, disse o senhor do
meio um pouco espantado e com um sorriso amarelo. Os outros dois mantinham
as mãos para trás e esfregavam uma na outra, na alegre expectativa de uma
grande peleja, cujo resultado entretanto deveria ser favorável a eles.
“Significa exatamente o que acabo de dizer”, respondeu o senhor Samsa e
avançou em coluna com suas duas companheiras para cima do inquilino. Este
a princípio ficou parado onde estava e olhou para baixo, como se as coisas
em sua cabeça assumissem uma nova configuração. “Se é assim, nós vamos
embora”, acabou dizendo e ergueu o olhar para o senhor Samsa, como se, num
acesso repentino de humildade, tivesse de pedir permissão até mesmo para
uma decisão dessas. O senhor Samsa, de olhos bem abertos, limitou"-se a
acenar brevemente com a cabeça uma e outra vez. O inquilino, com efeito,
em obediência a isso, de imediato se dirigiu a passos largos para a
antessala; seus dois amigos já estavam de sobreaviso há algum tempo, com
as mãos bem quietinhas, e em dois pulos seguiram no seu encalço, como se
temendo que o senhor Samsa pudesse alcançar a antessala antes deles e
perturbar o vínculo que mantinham com o líder. Na antessala, todos os três
pegaram seus chapéus no cabideiro, retiraram suas bengalas do
porta"-bengalas, curvaram"-se num cumprimento mudo e deixaram o apartamento.
Com uma desconfiança que se revelou infundada, o senhor Samsa saiu com as
duas mulheres no corredor; debruçados no corrimão, ficaram observando como
os três senhores, é certo que devagar, porém de modo ininterrupto, desciam
a comprida escadaria, a cada andar desapareciam em uma determinada curva
da escada e alguns segundos depois voltavam a aparecer; quanto mais para
baixo iam, mais se perdia o interesse da família Samsa por eles, e quando
um empregado do açougue surgiu, cruzou com eles e, todo aprumado e
orgulhoso, continuou subindo com um pacote em cima da cabeça, logo o
senhor Samsa abandonou o corrimão ao lado das mulheres e todos, parecendo
aliviados, regressaram ao apartamento.

Decidiram aproveitar o dia para descansar e passear; não só tinham direito
a essa folga no trabalho, ela também lhes era mais do que necessária. E
assim foram se sentar à mesa e redigir três justificativas, o senhor Samsa
à direção do banco, a senhora Samsa ao dono da loja e Grete ao seu patrão.
Estavam escrevendo quando a faxineira veio dizer que já ia, pois havia
acabado o serviço da manhã. Ocupados com a escrita, os três a princípio
apenas acenaram com a cabeça, sem erguer os olhos, só quando notaram que a
faxineira demorava a sair é que resolveram olhar para ela, contrariados.
“O que foi?”, perguntou o senhor Samsa. A faxineira estava parada na
porta, sorridente, como se tivesse uma notícia muito boa para dar à
família, porém só faria isso se lhe fosse solicitado com alguma
insistência. Fincada quase reta em seu chapéu, a pequena pluma de
avestruz, que tanto irritava o senhor Samsa durante todo o expediente
dela, oscilava de leve em todas as direções. “Mas o que a senhora quer
afinal?”, perguntou a senhora Samsa, por quem a faxineira ainda nutria o
máximo de respeito. “Está bem”, respondeu a faxineira e não conseguiu
falar muito mais por causa da risada amistosa, “o negócio aí do lado, a
senhora não vai precisar se incomodar em pôr para fora. Está tudo
resolvido.” A senhora Samsa e Grete inclinaram"-se sobre o papel, dando a
entender que queriam continuar escrevendo; o senhor Samsa, ao perceber que
a faxineira agora queria começar a descrever tudo em detalhes, cortou a
conversa levantando a mão de maneira decidida. Vendo que não tinha licença
para contar, lembrou que estava com pressa, falou bem alto, visivelmente
ofendida: “Passar bem”, deu meia volta enfurecida e saiu do apartamento
debaixo de um assombroso bater de portas.

“Hoje mesmo ela vai ser despedida”, disse o senhor Samsa, sem porém obter
resposta, nem de sua esposa, nem de sua filha, pois a faxineira parecia
ter perturbado o sossego que acabavam de alcançar. Elas se levantaram,
foram até a janela e lá ficaram, abraçadas uma à outra. O senhor Samsa
virou"-se para elas sem sair de sua cadeira e ficou observando quieto um
momento. Então reclamou: “Ora, venham até aqui. Deixem em paz as coisas
passadas. E tenham um pouco de consideração por mim”. As mulheres o
obedeceram no ato, voltaram depressa para ele, fizeram"-lhe festinhas, e
logo terminaram de escrever suas justificativas.

Então todos os três saíram juntos do apartamento, o que já não faziam há
meses, e seguiram de bonde rumo aos espaços abertos nos arredores da
cidade. O vagão em que se sentaram a sós reluzia com o brilho quente do
sol. Comodamente reclinados em seus assentos, eles avaliaram suas
perspectivas de futuro e acharam que, examinando mais de perto, elas não
seriam assim tão ruins, pois todos os três empregos eram, embora não
tivessem informações muito precisas um do outro a esse respeito,
sobremaneira oportunos e, a longo prazo, bastante promissores. A principal
melhoria em sua situação no momento deveria advir sem muita dificuldade
com a mudança de residência; queriam agora alugar um apartamento menor e
mais barato, porém mais bem localizado e acima de tudo mais prático do que
o atual, que havia sido ainda uma escolha de Gregor. Enquanto conversavam
sobre essas coisas, o senhor e a senhora Samsa, admirados com a vivacidade
crescente da filha, notaram praticamente no mesmo instante como ela nos
últimos tempos, apesar de todo o sofrimento responsável pela palidez em
suas faces, desabrochara e era agora uma moça bonita e cheia de viço. Cada
vez mais calados e entendendo"-se quase que por instinto só com a troca de
olhares, pensaram que já era tempo de arranjar um bom marido para ela. E
foi para eles como que uma confirmação de seus novos sonhos e belos
projetos quando, no final da viagem, a filha se levantou primeiro e
espreguiçou seu corpo jovem.
} %
%\renewcommand{\ParallelAtEnd}{\noindent{}\vspace{1cm}\Large{Notas}}
\end{Parallel}
\end{document}
