%%%MANTER AS NOTAS COMO ENDNOTES?? Ñ FAZ MUITO SENTIDO
%\let\footnote=\endnote

\chapter*{Introdução}
\addcontentsline{toc}{chapter}{Introdução, \emph{por Celso Donizete Cruz}}

\begin{flushright}
\emph{Celso Donizete Cruz}
\end{flushright}


\textsc{Eis uma nova} tradução brasileira da obra mais conhecida de Franz Kafka.
A princípio quis chamá"-la de \textit{A transformação}, modo de
recuperar a repetição sonora do substantivo alemão do título original,
“Verwandlung”, que ecoa na forma verbal “verwandelt” (“transformado”),
no fim da primeira frase da narrativa, considerada por muitos a
sentença de abertura mais célebre de toda a literatura.\footnote{ “Als 
Gregor Samsa eines Morgens aus unruhigen
Träumen erwachte, fande er sich in seinem Bett zu einem ungeheuren
Ungeziefer \textit{verwandelt}”. (A fim de manter a mesma
correspondência, Modesto Carone, responsável pela primeira tradução
brasileira diretamente do alemão, na década de 1980,
propõe solução inversa: não mexe no título, mas traduz o verbo por
“metamorfoseado”, opção desde então seguida por algumas traduções
posteriores.)} Dizem que o escritor argentino Jorge Luis Borges também
criticava o título consagrado nas traduções, argumentando que a língua
alemã possui a palavra “Metamorphose”, e Kafka a adotaria se sua
intenção fosse de fato privilegiar em sua narrativa a mutação
biológica, o que não é o caso. Na recepção em espanhol do século \textsc{xxi}, em
consonância com o reparo, propõe"-se uma nova tradução do título,
\textit{La Transformación}, nas edições da Editorial Funambulista, de
Madri, e da Debolsillo, de Barcelona, ambas de 2005. Em língua inglesa,
já no final do século \textsc{xx} surgia uma proposta conciliatória, \textit{The
transformation (Metamorphosis)}, na edição da Penguin Classics, de
1995.

Os exemplos não são muitos, afinal, e houve também argumentos contrários
à adoção de um novo título, todos no fundo receosos de afrontar
gratuitamente a tradição (em português, tradução e tradição é que dão
um trocadilho revelador das condições do campo). A experiência poderia
ser desastrosa em mais de um sentido. Poderia levar a perder leitores
interessados na obra, porém em busca do título tradicional, o que
autoriza quando muito um parêntese, como na edição inglesa. A
mudança de título poderia além do mais ser vista como tática meramente
novidadeira, sem maiores implicações para a fruição da obra. Ou quem
sabe angariasse para a tradução a pecha de enganadora, dando título
desconhecido a uma obra mais do que famosa (fosse entendida a proposta
só como brincadeira, e já estaria melhor). Mantenha"-se então
\textit{A metamorfose}, a tradução consagrada do título em língua
portuguesa. Não há por que polemizar, a questão é mesmo menor. O que
importa vem depois do título, com ou sem eco, e aí logo se percebe que
o foco não está na metamorfose, mas nas transformações que ela
acarreta.

No Brasil, \textit{A metamorfose} vem funcionando como o carro"-chefe
da recepção de Kafka, sobretudo de sua recepção popular. De 1956 a
2002, contam"-se no país pelo menos 21 edições diferentes da
obra.\footnote{ Cf. Celso Cruz. \textit{Metamorfoses de Kafka}, São
Paulo: Annablume, 2007.} É o livro que fisga o leitor e lhe abre as
portas para o universo kafkiano.\footnote{ Alguns colegas reivindicam
essa primazia para \textit{O processo}, às vezes até para \textit{O
castelo}, o que pode até acontecer entre o público mais intelectual.
Porém, o número de edições e traduções de \textit{A metamorfose} é bem
maior, o que confirma sua extensa popularização. \textit{O processo}
exige mais do leitor, e \textit{O castelo} ainda mais, daí a
dificuldade dessas obras de atingir o grande público. Elas tendem a
atrair o interesse desse público no caminho que \textit{A metamorfose}
pavimenta.} A mesma coisa deve se dar em outros países. A história do
homem que se transforma em inseto tem um forte apelo, nunca deixando de
inspirar novos lançamentos, cuja sucessão reafirma o notável sucesso da
obra entre leitores dos mais distintos estratos culturais. Difusão sem
dúvida louvável, quando se pensa no teor crítico do discurso kafkiano e
em seu poder desalienante. Contudo, vendo o que já se fez para sua
divulgação, pode"-se supor também um leitor leigo, seduzido pelo
título e por algumas capas, a julgar que se trata de uma história de
terror cujo protagonista é um homem que vira uma barata gigante e
ameaça sua família. Tal leitor não estará absolutamente errado, só que
se acompanhar a narrativa há de topar com um terror estranho e
inesperado, por vezes mais engraçado que aflitivo (pode achar que
levou gato por lebre: queria \textit{A metamorfose}, e recebeu
\textit{A transformação}). A hipótese não é de todo descabida, ainda
que seja difícil um leitor se aproximar da obra assim tão
desavisadamente. O adjetivo derivado do nome de seu autor é presença
certa nos dicionários, além do que Kafka e os títulos de suas obras
mais famosas já são verbetes obrigatórios das enciclopédias. Se o
leitor vai ao livro, é porque em geral soube de antemão alguma coisa.

Soube no mínimo do grande prestígio do escritor, um dos nossos maiores
ícones literários. Embora tenha escrito no começo do século \textsc{xx}, e
alcançado a glória póstuma após a metade desse mesmo século,
rapidamente ganhou posição ao lado dos clássicos imortais da literatura
de todos os tempos. O adjetivo “kafkiano” ultrapassou os círculos do
pensamento literário, vindo a servir para designar determinadas
situações de nossa vida prática. Tão entranhado assim ficou em nossa
cultura, que figura ao lado de outros adjetivos literários, como dantesco, quixotesco,
homérico. A uma tal altura no Olimpo das letras, não será difícil ao
leitor divisá"-lo ao adentrar o pátio principal da literatura do
Ocidente. Mas o que justifica tamanho destaque? De onde virá a força
que lhe assegura de saída um lugar no panteão dos gênios indisputáveis?
Na resposta a essas questões, há a considerar o que Kafka fez, e o que
dele foi feito.

\textit{A metamorfose} é um bom exemplo para tanto. Em parte por ser sua
obra mais popular e ter sido publicada com o autor ainda vivo. Não
foi muita coisa que ele deixou vir a público enquanto vivia.\footnote{
Há opinião diversa, como a de Osman Durrani, no \textit{\mbox{The Cambridge}
Companion to Kafka}, que procura desfazer o mito do autor tímido, avesso à
publicação de sua obra. Em relação ao mito, até que Kafka publicou
bastante. In \textit{The Cambridge
Companion to Kafka}, Org. Julian Preece, Cambridge University Press,
2002.} Três pequenos livros de narrativas curtas:
\textit{Betrachtung} (\textit{Contemplação}), de 1913; \textit{Ein
Landartz} (\textit{Um médico rural}), de 1919; e \textit{Ein
Hungerkünstler} (\textit{Um artista da fome}), publicado no ano de sua
morte, 1924. Três narrativas médias: \textit{Der Heizer} (\textit{O
foguista}), de 1913; \textit{Das Urteil} (\textit{O julgamento} ou
\textit{O veredito}), de 1916; e \textit{In der Strafkolonie}
(\textit{Na colônia penal}), de 1919. Além de \textit{Die Verwandlung}
(\textit{A transformação [metamorfose]}), uma narrativa
longa, que saiu inicialmente na revista \textit{Weiße Blätter}
(\textit{Folhas Brancas}) em 1915, depois em livro, em 1916, e alcançou
uma segunda edição em 1918. Em conjunto, os escritos publicados em vida
não ultrapassam quinhentas páginas.\footnote{ São 447, numa contagem
mais recente, incluindo textos não literários. Cf. Osman Durrani,
“Editions, translations, adaptations”, in \textit{The Cambridge
Companion to Kafka. Op. cit}. p. 208.} A economia narrativa e o rigor no acabamento
apresentam"-se desde já como parte do projeto literário de Kafka. De
fato, apenas essas obras talvez fossem suficientes para garantir sua
posição entre os grandes mestres. As principais características de sua
ficção estão praticamente todas presentes. Só não se saberia então que
o material publicado era apenas parte do edifício.

Ficou mais do que notável uma das últimas vontades de Kafka, a de que,
após a sua morte, seu espólio literário fosse destruído. O amigo Max
Brod, incumbido pelo autor da realização dessa vontade, evidentemente
não cumpriu a promessa. Esse episódio biográfico já deu o que falar e é
um dos que contribuem para a construção de uma visão romantizada da vida de
Kafka. Pode"-se imaginar o escritor em seu leito de morte, vencido
pela tuberculose, entre acessos de tosse e escarros de sangue,
encarecendo o amigo com a tarefa inglória; na cena seguinte o amigo, ao
abrir o baú, surpreso e maravilhado com a quantidade e a qualidade do
tesouro que encontra; no final feliz, o tesouro partilhado com os
próximos e os pósteros\ldots{} Deve ter sido mais ou menos isso o que
aconteceu, descontada a dose de má ficção. O já citado Jorge Luis
Borges é um dos que referem o episódio,\footnote{ Num conhecido prólogo
publicado no Brasil na abertura de uma edição da Ediouro de \textit{A
metamorfose}, tradução de Torrieri Guimarães, de 1998, coleção
“Biblioteca de Babel”, dedicada à literatura fantástica, homônima porém
não a mesma dirigida por Borges e Bioy Casares na Argentina. A
propósito de Borges, ainda, também se acredita que tenha traduzido
\textit{A metamorfose}, fato entretanto desmentido por Fernando
Sorrentino, em ``\,`La Metamorfosis’ que Borges jamás tradujo”, \textit{La
Nación}, Buenos Aires, 9 de marzo de 1997 (disponível \textit{on-line} em
<http://www.sololiteratura.com/sor/sorrenelkafkiano.htm>, acesso em
30/05/2008, com o título “El kafkiano caso de la Verwandlung que Borges
jamás tradujo”).} evocando para efeito de comparação o caso de Virgílio
(outro escritor cujo similar último desejo também não se realizou) e a
seguir argumentando que se essa fosse realmente a vontade desses
autores, eles mesmos se encarregariam de riscar o fósforo. Ironias à
parte, o pedido não atendido de Kafka pode ser a tradução sincera de
sua dúvida quanto ao valor de suas páginas inacabadas. O rigor de seus
critérios de acabamento aumenta na medida da desproporção entre o muito
que escreveu e o pouco que publicou. \textit{A metamorfose}, todavia,
não deixa dúvidas, pois passou pelo crivo do autor, não sofreu as
interferências da organização e edição póstumas de Max Brod, não
padecendo assim da desconfiança, algo desmedida, diga"-se, quanto à
autenticidade de alguns trechos de sua produção literária divulgada
\textit{post mortem}. Por isso vem a ser mesmo o livro ideal para um
contato inicial preciso com a mais pura ficção kafkiana.\footnote{
Entretanto, não se quer dizer que o que veio após sua morte deva ser
descartado, longe disso. Inclusive, acontece uma coisa interessante, a
partir da recepção das obras do espólio. O inacabado e o fragmentário
próprio desses papéis cuja redação não foi retomada, ou que não foram
revistos para publicação, são incorporados como matrizes da expressão
literária de Kafka, e revertem sobre suas produções anteriores.
Cumpre"-se de certa forma a perspicaz observação, de novo de Borges,
agora em “Kafka e seus precursores”, de que os autores que influenciam
Kafka só vêm a surgir depois de sua morte.}

Proponho a distinção entre as narrativas póstumas e as publicadas em
vida apenas como tentativa de destacar o cuidado do autor com seus
escritos, sua consciência literária, seu senso crítico apurado, seu
compromisso vital com a literatura. Trata"-se de um homem de letras,
que frequentou espaços sociais comuns a intelectuais e artistas, que
tinha uma visão particular da literatura, estava informado das
novidades de seu tempo, e certamente manifestaria suas opiniões em
encontros com os amigos nos cafés de Praga. Não corresponderia
unicamente à imagem do escritor desconhecido, enclausurado, sombrio,
gênio incompreendido e maldito --- visões românticas tantas vezes
propagadas nas biografias. Max Brod, conhecendo o amigo, por certo
estaria consciente de seu alto valor literário, e de antemão calcularia
a importância do que o aguardava no baú. Kafka não foi afinal o
escritor anônimo descoberto da noite para o dia, infelizmente quando
era tarde demais e já não podia desfrutar da fama. Não viu o sucesso de
nenhuma das obras que publicou, é certo, porém é igualmente correto que
alcançou de imediato com elas o reconhecimento de seus pares em Praga,
despertando reações positivas também em alguns círculos literários da
Alemanha.\footnote{ Luiz Costa Lima comprova em \textit{Limites da voz:
Kafka} (Rocco, 1993) que o escritor “não foi um
ignorado”, e que sua “recepção inteligente” soube lhe destacar o valor,
além de ser em alguns casos muito feliz na caracterização de suas
peculiaridades.} O seu talento de primeira grandeza não era popular, mas
foi notado. Consta que arrebatou em 1915 a terceira edição do Prêmio
Theodor Fontane de Arte e Literatura, instituído na Alemanha, embora
tenha sido uma vitória indireta: o vencedor oficial, o dramaturgo
alemão Carl Sternheim, repassou depois a premiação a Kafka. Episódio
emblemático de uma recepção restrita --- que tem o autor como escritor dos
escritores, conhecido apenas em pequenos círculos literários, condição
que após a sua morte sua obra superaria totalmente, chegando ao coração
das massas, o que é até espantoso, em face do desconforto inevitável
provocado por sua leitura.

Franz Kafka nasceu em 1883 em Praga, capital da então Boêmia, hoje
República Tcheca. À época, a Boêmia fazia parte do Império
Austro"-Húngaro, e seu idioma administrativo oficial era o alemão. A
submissão compulsória ao império obviamente não retirava aos tchecos o
sentimento de pertença à cultura de sua região, e logo os movimentos
nacionalistas desta e de outras regiões submetidas iriam dissolver o
império. Imagine"-se a aversão pelo imperialismo, e a desconfiança
para com todos que parecessem mais fiéis ao império do que à Boêmia.
Este em parte devia ser o caso de Kafka que, apesar de seu local de
nascimento, não possuía identificação muito evidente com a cultura
tcheca. Era filho de pais judeus emigrados da Áustria, praticamente sem
laços afetivos ou nacionalistas com a Boêmia. Sua família fazia parte
da comunidade judaica de Praga e ao mesmo tempo flertava com os
oficiais alemães, tanto é que colocaram Kafka para estudar numa escola
alemã. Seu caso, evidentemente, não seria único, contudo não será
também de admirar a crise identitária e o sentimento de perseguição
decorrentes da situação. Kafka era tcheco, mas escreveu em alemão e
acabou órfão das duas culturas. Um dos maiores nomes da literatura
alemã de todos os tempos não era alemão. E Praga em sua obra é nada
mais que a sombra de um cenário ocasional. Judeu, mas desgarrado e
descrente, tampouco pode"-se dizer que encontrasse sua identidade em
meio à comunidade judaica. Exilado das três pátrias, seria hoje cidadão
do mundo\ldots{} Mas a Europa era outra, e Kafka a viu antes, durante e logo
depois da Primeira Guerra. Foi testemunha desse acontecimento
traumático, que literalmente expôs as entranhas de uma sociedade
pretensamente racional. A condição marginal lhe possibilitaria observar
essa sociedade sem comprometimentos patrióticos. O que tinha para
dizer, e deixou por escrito, não se dirigia especificamente à cultura
tcheca, alemã ou judaica. A nenhuma das três em particular, mas a todas
a um só tempo --- ao humano em cada uma delas.

Sua posição à parte no conturbado cenário europeu de então deu"-lhe uma
compreensão inusitada dos problemas do homem de seu tempo, o homem
contemporâneo, este que veio a ser o que ainda hoje somos. Sua obra
resulta dessa compreensão, um dos motivos elementares da importância a
ela atribuída. Kafka parece ter dito uma vez que concebia a literatura
como uma “expedição à verdade”.\footnote{ “Dichtung ist immer nur eine
Expedition nach der Wahrheit”, frase atribuída a Kafka por Gustav
Janouch em seu livro \textit{Conversas com Kafka}.} Essa concepção
acentua outro tanto o interesse pelo que deixou. Seus textos literários
são nesse sentido uma contribuição à filosofia (em sentido lato), que
de direito se ocupa dos problemas da verdade. A literatura kafkiana
demonstra que a reserva filosófica não impede a progressão do método
literário na exploração de um mesmo território. A diferença é que, onde
a filosofia explica, a literatura mostra. Não se recorre às premissas
que permitirão a dedução de uma situação absurda na qual o ser humano,
“barateado”, reduz"-se à condição de inseto. Não se aciona o
pensamento lógico \textit{stricto sensu}. O absurdo é maior e mais
impactante com a eclosão inexplicável do inseto humano no seio de uma
típica família pequeno"-burguesa. O fenômeno é incomum, e visível
apenas pelas lentes literárias. Mas não desperta nenhuma dúvida nas
personagens, que em nenhum momento questionam a impossibilidade do
fato. Note"-se que não se trata de metáfora, o inseto está lá em toda
sua concretude, para quem quiser ver. Age como inseto: tem dificuldades
para se mover, não possui dentes, rasteja pelas paredes e pelo teto,
se alimenta de restos e, apesar de ainda raciocinar como humano e de
entender a língua dos humanos, estes não só não entendem o que ele
fala, como o julgam (com exceção talvez da faxineira) incapaz de
compreendê"-los. Chama"-se aqui a atenção para o irreal que afinal
aparece como a condição para que se enxergue a realidade, e aí temos um
método de desalienação. Didática de Kafka: os contrassensos não são
discutidos, são vistos, e é o impacto do que se vê que perturba o
entendimento sossegado do leitor.

Vale falar de um propósito na dedicação extrema de Kafka à literatura. A
julgar pelo que relatou em diários e cartas, sua vida só adquiria
sentido em função da literatura. Acredito que seja possível confiar na
sinceridade desses escritos pessoais, embora a relação da biografia do
autor (em boa parte inspirada por esses mesmos escritos) com as obras
que deixou dê margem a interpretações muitas vezes equivocadas ou
ingênuas. Não é que tenha retratado episódios de sua vida pessoal.
Estes, no máximo, iriam lhe servir de inspiração. A literatura, como a
concebia, seria mais uma forma de flagrar as contradições da cultura
ocidental no princípio do século \textsc{xx}, de um modo eficaz, no entanto nada
confortável nem óbvio. Seria uma tentativa de entender o que acontece
com os humanos numa sociedade cada vez menos humanizada, se é que algum
dia houvesse sido mais\ldots{} Escrever lhe era vital, provavelmente porque
o punha em contato com a \textit{verdadeira} vida. A procura da
verdade, se por um lado enfeixa suas produções na confluência da
literatura com a filosofia --- e não por acaso os maiores filósofos do
século se dispuseram a interpretá"-lo ---, por outro lado leva a
classificá"-las como realistas. De um realismo que não se reduz à
descrição pitoresca da superfície do real, antes corresponde à
percepção objetiva da realidade. Com efeito, seu realismo é de tipo
expressionista, à medida que dá vazão a uma realidade desfigurada pela
percepção interna do sujeito. Entretanto, o propósito de objetivar essa
realidade impede a expressão puramente subjetiva. Como explica 
Luis Costa Lima (\textit{op}. \textit{cit}., pp. 65--66), a ficção
de Kafka pressupõe uma mediação, “um meio interposto entre a
subjetividade e o mundo externo, que permita a objetivação daquela”:
“Sua questão é converter as tematizações pessoais de próprias ao espaço
interno em capazes de se mover no externo; i.e., transformá"-las de
fantasmas em objetos, cujos traços mostrariam a si e a seu tempo”. Se
bem entendo a lição, diviso uma metodologia nessa busca de conversão do
interior em exterior, de “fantasmas em objetos”, da subjetividade em
objetividade, enfim. Só faz sentido falar em método quando se quer
atingir um objetivo, no caso \textit{mostrar} “a si e a seu tempo”, o
que vem a ser a confirmação de uma intenção realista.

Aqui se cai de chofre na \textit{selva selvaggia} da fortuna crítica.
Missão impossível não recorrer ao paradoxo na descrição da
singularidade do autor. O subjetivo objetivo, a ação que é inação, o
estranho familiar\ldots{} Nomeia"-se pela contradição uma obra que se
realiza no limite, sempre na dúvida entre o que \textit{é} e o que
\textit{não é}. Tal indecisão retira as bases de qualquer juízo crítico
absoluto. E o mistério sempre se mantém um mistério, mesmo depois de
aberto com as diferentes chaves forjadas pela crítica. Ora, não será
demasiado supor que era esse precisamente o ponto visado pela
literatura de Kafka, a apresentação de situações numa perspectiva
ambígua, trágica e cômica ao mesmo tempo, próxima e distante, real e
fantástica (termos e contratermos se sucedem\ldots{}). Toda representação
kafkiana sustenta"-se na evocação de sua face contrária. O caso de Gregor
Samsa, personagem principal de \textit{A metamorfose}, é mais uma vez
exemplar. Ele só toma consciência de sua alienação ao ser alienado de
sua forma humana. Não é o fato de se transformar em inseto o que o
aliena, isso só lhe revela sua real alienação. Já era inseto quando
ainda era humano, se ainda é humano quando já é inseto? 
Se para descobrir sua humanidade é
preciso que a perca, a metamorfose é a condição de sua consciência. A
exposição do humano é levada ao extremo com a oposição do inseto. No
choque dos opostos é possível viver uma verdade.

Não devia mesmo ser fácil ao autor sustentar tal ponto de vista. Kafka
sempre se queixou da falta de espaço e tempo para se dedicar à
literatura como gostaria e, de acordo com seus critérios, deveria. A
biografia em quadrinhos de Robert Crumb e David Zane
Mairowitz\footnote{ \textit{Kafka de Crumb}, trad. José Gradel, Rio de
Janeiro: Relume"-Dumará, 2006.} retrata enfaticamente a ausência de
privacidade na casa dos pais, onde residiu durante quase toda a sua
vida, e também o grau de concentração exigido em sua prática literária.
No traço de Crumb, o escritor entra em transe ao escrever, os olhos
esbugalhados, como se transportado para um outro plano existencial. O
transe é ainda ambíguo, pois significa a necessidade tanto de superar
um entorno desfavorável à prática (situação do sujeito) quanto de
aceder ao plano de perseguição da verdade (condição do objeto). Não
admira que o esforço exaurisse o autor, solicitando"-lhe uma
disposição que somente teria se pudesse abandonar o trabalho e demais
compromissos sociais. Kafka se dizia um fraco. Para poder escrever, se
viu obrigado a abdicar de possíveis casamentos e a buscar a solidão.
Ainda assim, o mínimo exigido de vida social já lhe parecia muito e
roubava"-lhe as forças de que necessitava para completar suas obras.
Em que pese a fantasia \textit{underground}, a representação proposta
por Crumb sintetiza os apuros do escritor, que se sentia hábil e capaz
apenas para o trabalho literário que sua vida lhe dificultava exercer.
O transe místico é ainda o simulacro de sua obsessão com a literatura,
à qual sacrificava a vida.

Nesse ponto tocamos a esfera do mito. Não é fácil acreditar que um autor
se dispusesse a tanto, nem que a literatura exija pacto tão radical.
Mas fato é que o próprio Kafka cultivou a ideia do escritor abnegado. A
divulgação de suas obras póstumas também pôs em circulação seus
escritos pessoais, e é nesses que se acham declarações do autor sobre
seu envolvimento com a criação literária. Essas declarações alimentam o
mito. De acordo com elas, o escritor dedica"-se à literatura como a um
sacerdócio. A literatura seria sua religião mais cara, se de fato lhe
revelasse a verdade. Daí a enxergá"-lo como profeta é um passo. Isso
sem contar a simpatia despertada por sua situação pessoal precária.
Note"-se, porém, que Kafka fala de uma \textit{expedição} à verdade,
não de uma \textit{revelação}. Uma expedição é uma viagem, uma
aventura, e só se vive uma verdadeira aventura quando não se pode
prever o final. Para abandonar o mito, é preciso compreender o
compromisso com a verdade do ponto de vista ético, não religioso. Os
resultados das expedições nunca são conclusivos. Mas o rigoroso relato
do percurso é a prova da dedicação e da fidelidade ao compromisso
assumido. A literatura é, assim, seu instrumento de busca da verdade,
ao encontro da qual não é necessário ir com a alma pura dos crentes
inocentes. Também não é preciso deixar ao cinismo o papel principal.
Parece haver em Kafka, como nos grandes autores, um compromisso com a
sinceridade. A tortura ou o transe da criação podem, sim, se associar antes ao
rigor do que à mística. Essa reivindicação, contudo, no fundo também
obedece a imperativos associados a uma visão específica da arte,
entendida aqui mais como construção e cálculo do que como magia e
inspiração. Reclama"-se um escritor consciente de sua proposta
literária, antes que um “médium” da expressão de forças superiores.

Entretanto, a preferência parece tender ao místico. 
Com a popularização de sua recepção, Franz Kafka
passa a habitar uma outra dimensão, e se transforma em um personagem da
mitologia moderna, cujos círculos inevitavelmente se misturam à
mitologia dos tempos imemoriais. E faz pouco mais de oitenta anos que
veio a falecer. Morria quando nossos pais ou avós nasciam, há duas
gerações apenas. Como esse intervalo é relativamente pequeno, ainda é
possível testar, com base em documentação histórica, a verdade de
alguns relatos biográficos.\footnote{ Cf., por exemplo, Anthony
Northey, “Myths and realities in Kafka biography”, in \textit{The
Cambridge Companion to Kafka}. \textit{Op. cit}.} Mas a própria disputa pela verdade biográfica tende a confirmar
o mito. Algumas correções não perturbam a imagem geral, pelo contrário.
A voracidade do mito traga qualquer migalha de veracidade histórica.

Creio que o que acontece com a recepção de Kafka no Brasil repete em
escala doméstica, ressalvado o atraso, o movimento internacional de sua
popularização. De início, sua leitura é privilégio de pequenos círculos,
mas logo suas produções vêm a ser difundidas para todos os estratos
sociais. \textit{A metamorfose} é sua obra mais divulgada, porta de
entrada de sua recepção, como observado. Sua primeira tradução
brasileira, de Brenno Silveira, data de 1956, e foi feita a partir do inglês. É só
a partir dos anos 1960 que o autor passa a ser popularizado, já
então como clássico. Em muito concorre para sua popularização a
história do homem que se transforma em um inseto. Na época de sua
primeira publicação em livro, em 1916, Kafka instou para que o inseto
de modo algum fosse sugerido na capa. Logo na primeira edição
brasileira, contudo, já aparecem justapostos, de costas um para o
outro, os perfis de um homem e de uma barata, esta mais detalhada que
aquele. A tônica no inseto descortina uma estratégia de difusão que
potencializa em demasia certos apelos popularescos da história
original, e tende a preservar o mito. \textit{A metamorfose} é
normalmente divulgada como clássico da literatura moderna ou universal.
Só por isso já deveria ser lida. Mas a presença do homem"-inseto é um
incentivo a mais, considerada a curiosidade que o fantástico e o
sobrenatural em geral despertam no público. Em nome desse apelo é que
se coloca a metamorfose do homem em primeiro plano, quando na verdade a
narrativa trata é das transformações de seu entorno em face de uma
situação totalmente inesperada. Todas as nuanças possíveis decorrentes
desse evento inicial são exploradas. A metamorfose desperta reações em
cadeia, e são essas reações que a narrativa de Kafka acompanha. Daí, na
verdade pouco importa o inseto, basta frisar sua inadequação e a
repulsa que ele provoca. Se em lugar do inseto houver uma massa amorfa
e gosmenta, nada muda, a não ser a forma de suas pegadas.

Por isso a sugestão de deslocar o foco. Trocar a metamorfose pela
transformação. As reações ao asco são mais interessantes que o objeto
asqueroso. Aí é que está o humor, um humor não autorizado pelo horror
da situação descrita e que entretanto comparece como possível matriz do
modo de narração. O distanciamento narrativo é máximo, mesmo que
tenhamos acesso direto aos pensamentos das personagens. O narrador é
onisciente e não se compromete. Mantém a objetividade ainda que o
evento narrado seja o maior dos absurdos. Faz questão de chamar a
atenção para detalhes periféricos das situações principais, e tais
detalhes acabam sendo reveladores das reais motivações das personagens.
Ora, o absurdo, o inesperado, o grotesco, o ínfimo que se revela
fundamental, essas ocorrências são comuns ao reino do cômico, isso sem
falar que o distanciamento é a condição da comédia, pois são poucos os
que acham graça quando são os objetos de derrisão. Não admira pois que,
conforme reza a lenda, Kafka tenha chegado às gargalhadas ao ler a
narrativa em primeira mão para os amigos. Acredito que falte uma pitada
maior desse humor nas edições e traduções brasileiras, o qual todavia
está presente nas ilustrações do pintor Walter Levy para a primeira
edição brasileira de 1956. Esse veio interpretativo ficou meio esquecido nas
várias edições posteriores. A tônica foi mais para o horror ou para o
trágico, que fazem justiça à obra, mas não a esgotam.

Constata"-se, por mais incrível que pareça, apesar de toda a avalanche
interpretativa a que o autor esteve e continua sujeito, a existência de
espaços ainda a explorar, afora a necessidade de revisão de algumas
ideias prontas herdadas de recepções passadas. O maior desafio da
crítica kafkiana talvez seja escapar aos mitos e às múltiplas
interpretações preestabelecidas de sua obra. De qualquer modo, todo
clássico acaba se impondo por si só quando nos dispomos à sua leitura.
Uma nova tradução é só mais uma proposta de interpretação, sempre
possível porque o contexto de recepção nunca é estável. O clássico
atravessa as gerações, tendo sempre o que dizer a cada uma delas. Há
portanto sempre uma oportunidade de renovação a comprovar o seu vigor
atemporal.


\chapter*{}
\addcontentsline{toc}{part}{A Metamorfose}
\begin{center}
\begin{vplace}[0.3]
\Large
A Metamorfose\\
\emph{Die Verwandlung}
\end{vplace}
\end{center}
\thispagestyle{empty}